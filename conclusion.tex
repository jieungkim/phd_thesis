%
%\section{Trusted Computing Based in CertiKOS}
%\label{chapter:conclusion:sec:trusted-computing-based-in-certikos}

This thesis 
explains
the toolkit 
for concurrent program verification, 
that supports for users to build abstraction layers with machine-checked proof objects in them. 
The toolkit also provides 
the framework 
to combine local isolated proof instances in the concurrent environment 
to form the formal link between the local instance and
the formal behavior of the program on concurrent machine model. 
As case studies, 
$\certikos$ and $\wormspace$ 
have presented in this thesis
which are a concurrent operating system with fine-grained locking
and a distributed system API 
based on a collection of $\paxos$. 

In regarding the verification on distributed systems,
we also provide the generic toolkit to support the safety proof of multiple leader-based distributed systems. 
We form a generic form that those systems has to follow that can aids the safety proof of those systems. 
It is critical in the verification of distributed systems, 
because providing
the high-assurance of those systems 
is not only associated with the functional correctness
but also related to the protocol safety that usually
requires a sophisticated case analysis.

As a future direction, 
we plan 
to rule out assumptions that 
our framework and the proofs rely on
and improve 
the framework to extend its expressiveness, 
simplify the framework itself.

%
%We build concurrent operating systems using CCAL, but there are some potholes in the verification. 
%Our x86 machine model in the verification assumes sequential consistency instead of the acutal x86 machine mode, x86 TSO.
%
%\section{Timed Behaviors}
%\label{chapter:conclusion:sec:timed-behaviors}
%TImed behaviors are a basis of real-time software systems.
%
%\section{Better Concurrent Linking Framework}
%\label{chapter:conclusion:sec:better-concurrent-linking-framework}
%
%Concurrent linking in our framework is still extremely complex, expensive, and not general enough to applied into any layers as we have shown in Chapter~\ref{chapter:linking}.  
%
%
%\section{Applying Witness Passing to Other Systems}
%\label{chapter:conclusion:sec:applying-witness-passing-to-other-systems}
%
%%We presents witness-passing in Chapter~\ref{chapter:witness-passing}, and show how it can be applied to 
%%verify other consensus mechanisms such as multiple works in blockchain. 
%
%
%\section{Conclusion}
%\label{chapter:conclusion:sec:conclusion}
%
%In this dissertation, we have presented a novel, compositional, practical, and powerful toolkit that can be applied to verify multiple complex system software. 
