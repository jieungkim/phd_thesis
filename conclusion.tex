\section{Limitations on Concurrent Certified Abstraction Layers}

Our CCAL has several limitations.
The automation still needs improvement. Among the two common layer-building patterns discussed in 
Chapter~\ref{chapter:ccal},
the \textit{log-lift} pattern still requires a plethora of manual proofs that users have to write. 
We believe that providing better automation for several of their parts will dramatically improve the efficiency of our framework.

Our per-thread layer interfaces also cannot allocate any new blocks in the memory. Even though we provide a new memory 
model for composing multiple threads’ memories together 
(in  Section~\ref{chapter:linking:subsec:multithreaded-env-configuration}),
our per-thread layer interfaces cannot support this model properly due to other dependencies of our framework 
(\ie, the compiler and linking library). This restriction significantly limits the expressiveness of per-thread layer interfaces 
and increases the complexity of multithreaded linking by making its size  (\ie, the concrete definition of 
$\CSched$, $\TLink$, $\TSched$,  thread configurations, and auxiliary functions) huge. 

Another limitation is the lack of general parallel composition rules. As discussed in Chapter~\ref{chapter:linking} and Chapter~\ref{chapter:certikos}, 
our parallel composition is limited in specific places among the full layer stacks. Extending the framework to support 
parallel compositions for every layer is required to strengthen our framework.

\section{Relaxed Memory Model}

We assume sequential consistency for atomic operations. 
This implies that all atomic operations on the system appear to take place in some total order--the global order of atomic operations.
This assumption makes reasoning about concurrent programs easy, differing from most $\intelmachine$
 hardware models. Based on the previous work~\cite{sevcik13}, 
race-free programs on a TSO model behave as if they were executing on the sequential consistent machine. 
We believe this observation can be applied to our program because the programs on our push/pull model are race-free. 
However, we need more investigation to extend our work to support relaxed memory models.


\section{Timed Behaviors}

Timed behavior is crucial for systems that provide time-sensitive services, such as real-time operating systems (RTOSs), 
which serve real-time applications. However, our framework does not support the reasoning of real-time behaviors, 
and providing the precise metric for each assembly instruction is challenging. Extending the current work to support 
this is another future direction, and a separate line of work in our group is focusing on it.


\section{Trusted Computing Base in CertiKOS}

Our verified kernel assumes the correctness of the hardware, and we assume the top-level system call specifications must be trusted. 
Those shim layers are vulnerable points that possess bugs, and the previous work~\cite{shimlayer} investigates the weak parts via comparing 
multiple verification works. We minimize those vulnerable points as much as possible. 
$\compcert$ does not contain full specifications of $\intelmachine$ machine models.
It does not have some control registers (\eg, CR3) and 
instructions (\eg, xchg).
We focus on modeling hardware specifications that are only related to our kernel verification instead of specifying the entire hardware manual. 
Trusted system call specifications are parts of our top-level abstraction layer ($\TSyscall$) in our $\certikos$.
Because all implementation details of our kernel are hidden in the layer, system call specifications are reasonably small. 
This minimizes the possibility of errors and reduces the cost of the review process. However, we found multiple errors 
in our review, so we believe testing those trusted parts is necessary to reduce the possibility of mistakes.

\section{Extending CertiKOS}

Providing richer interfaces to users is a desirable extension of our verified concurrent kernel. Among them, network stacks are 
one of the more promising candidates for connecting the verified kernel and the verified distributed system. 
Our group adds a network module in the unverified 
version of CertiKOS by porting lwIP ~\cite{lwip},  which is a small implementation of the TCP/IP protocol stack. 
But its verification has not yet been addressed.

\section{Trusted Computing Base in WormSpace}

Similar to TCBs in $\certikos$, we trust bottom and top layers in our abstraction layers.
 The bottom layer contains minimal system call specifications that are necessary to verify distributed systems. 
 For example, send-and-receive primitives and a socket-initialization primitive are parts of the bottom layer. 
Extending $\certikos$ can reduce these trusted parts. 
The top layer interface it contains the API for clients. 
We minimize the size of those specifications as much as possible, as we did in   $\certikos$.
This reduces the possibility of errors, but testing for those specifications is desirable if we have a proper testing tool for them.

\section{Extending WormSpace}

Our $\wormspace$ has multiple possibilities for extension. Our work currently does not support membership changes 
that are common in practical distributed systems. It also does not investigate liveness or other progress properties. 
To support these things, we need to modify the current asynchronous network model into a semi-asynchronous 
network model that has a time-related behavior (\ie, time out) in it. 
Our network assumption is a non-Byzantine setting, which trusts all nodes in the network. Relaxing this condition is sometimes desirable,
 and it may be a future direction of  $\wormspace$ as well as our distributed system verification. 
 We believe that modeling them is feasible with the current framework, and we focus on parts of them as ongoing works.
 
 
\section{Conclusion}

This thesis explains the toolkit for concurrent program verification that provides support for users to build abstraction layers containing 
machine-checked proof objects. The toolkit also provides the framework to combine local isolated proof instances in the concurrent environment 
to form the  formal link between the local instance and formal behavior of the program on a concurrent machine model. 
As case studies, we presented $\certikos$ and $\wormspace$, 
which are a concurrent operating system with fine-grained locking and a distributed system API based on a collection of $\paxos$, respectively.

Regarding the verification of distributed systems, we also provide the generic toolkit to support the safety proof of multiple 
leader-based distributed systems. We create a generic form that those systems must follow that can aid their safety proof. 
This is critical in the verification of distributed systems because providing the high assurance of those systems is not only associated with 
functional correctness but also the protocol safety that usually requires a sophisticated case analysis.

As a future direction, we plan to rule out assumptions on which our framework and the proofs rely and improve the framework to
extend its expressiveness and automation. We also plan to simplify the framework itself.