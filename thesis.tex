\documentclass[letterpaper,11pt]{yalephd}
% remove draft option for final printing.
% font size must be between 10pt-12pt.

\usepackage{geometry} % you need this for yalephd.cls to work.
\usepackage{dcolumn}
\usepackage{amsfonts}
\usepackage{appendix}
\usepackage{makecell}

\usepackage[numbers,square]{natbib}
\usepackage{times}
\usepackage{courier}
\usepackage[scaled]{helvet}
\usepackage{url}
\usepackage[utf8]{inputenc} %for utf8 input
\usepackage[T1]{fontenc} %for accented characters
\usepackage{microtype} %better micro typing
\usepackage{datetime}
\usepackage{amssymb} %for shift symbol
\usepackage{amsmath}
\usepackage{mathrsfs} %for mathscr font
\usepackage{listings, multicol, lstcoq} %for code
\usepackage{multirow}
\usepackage{enumitem}      % adjust spacing in enums
\usepackage{stmaryrd} %for llbracket
\usepackage{mathabx} % for boxes
\usepackage{graphicx} %to include png images
\usepackage{subfig}
\usepackage[usenames,dvipsnames]{color} 
\usepackage{comment}
\usepackage{bussproofs} %for proof trees
\usepackage{flushend}
\usepackage[colorlinks=false,allcolors=black,breaklinks,draft=false]{hyperref} 
\newcommand{\doi}[1]{doi:~\href{http://dx.doi.org/#1}{\Hurl{#1}}} 
\usepackage{mathptmx}      % use Times fonts if available on your TeX system
\usepackage{latexsym}
\usepackage{morefloats}
\usepackage{tabu}
\usepackage{amsthm}
\usepackage[normalem]{ulem} % for strike out

\usepackage{prettyref}
%----------------------------------------------------

\usepackage{mathpartir}
\usepackage[scaled=0.85]{DejaVuSansMono}
\definecolor{mygreen}{rgb}{0,0.6,0}
\definecolor{mygray}{rgb}{0.5,0.5,0.5}
\definecolor{mymauve}{rgb}{0.58,0,0.82}
\definecolor{ltblue}{rgb}{0,0.4,0.4}
\definecolor{dkblue}{rgb}{0,0.1,0.6}
\definecolor{dkgreen}{rgb}{0,0.35,0}
\definecolor{dkviolet}{rgb}{0.3,0,0.5}
\definecolor{dkred}{rgb}{0.5,0,0}
%----------------------------------------------------

\lstset{ %
	backgroundcolor=\color{white},   % choose the background color; you must add \usepackage{color} or \usepackage{xcolor}
	basicstyle=\ttfamily,      % the size of the fonts that are used for the code
	breakatwhitespace=false,         % sets if automatic breaks should only happen at whitespace
	breaklines=true,                 % sets automatic line breaking
	%captionpos=b,                   % sets the caption-position to bottom
	commentstyle=\color{mygreen},    % comment style
	deletekeywords={...},            % if you want to delete keywords from the given language
	escapeinside={<@}{@>},          % if you want to add LaTeX within your code
	extendedchars=true,              % lets you use non-ASCII characters; for 8-bits encodings only, does not work with UTF-8
	frame=none,	                   % adds a frame around the code
	keepspaces=true,                 % keeps spaces in text, useful for keeping indentation of code (possibly needs columns=flexible)
	keywordstyle=\color{blue},       % keyword style
	language=Octave,                 % the language of the code
	otherkeywords={*, Inductive, Fixpoint, Record, typedef, struct, Function , Definition, Prop, then, true, false, ...},           % if you want to add more keywords to the set
	emph = {in, uint, match, end, with, let, ret, do, forall, exist, :=, =>, ->},
	emphstyle=\bf,%
	moredelim=[is][emphstyle]{|>}{<|},%
	numbers=left,                    % where to put the line-numbers; possible values are (none, left, right)
	numbersep=5pt,                   % how far the line-numbers are from the code
	numberstyle=\scriptsize\color{mygray}, % the style that is used for the line-numbers
	rulecolor=\color{black},         % if not set, the frame-color may be changed on line-breaks within not-black text (e.g. comments (green here))
	showspaces=false,                % show spaces everywhere adding particular underscores; it overrides 'showstringspaces'
	showstringspaces=false,          % underline spaces within strings only
	showtabs=false,                  % show tabs within strings adding particular underscores
	stepnumber=1,                    % the step between two line-numbers. If it's 1, each line will be numbered
	stringstyle=\color{mymauve},     % string literal style
	tabsize=2,	                   % sets default tabsize to 2 spaces
	%title=\lstname                   % show the filename of files included with \lstinputlisting; also try caption instead of title
}

%% Anonymization
\newif \ifanonymized \anonymizedfalse

\ifanonymized
\newcommand\CTOS{CTOS}
\else
\newcommand\CTOS{CertiKOS}
\fi
%% END Anonymization

\ifTRthen
\newcommand{\ifTR}[2]{#1}
\else
\newcommand{\ifTR}[2]{#2}
\fi

\newcommand\raw{x86}
\newcommand\sys{ker}
\newcommand\layer[3]{(#1,#2,#3)}


\newcommand\sem[2]{[\![#2]\!]_{#1}}
\newcommand\join{\!\bowtie\!}
\newcommand\Refrel{\sqsubseteq}
\newcommand\ClightX[1]{\textrm{ClightX(}#1\textrm{)}}
\newcommand\CompCertX[1]{\textrm{CompCertX(}#1\textrm{)}}
\usepackage{multirow}
\newtheorem{invariant}{Invariant}
\newcommand\ignore[1]{}

%% COMMENTS 
\newif \ifcomments \commentstrue

\ifcomments
\newcommand{\zhong}[1]{\textbf{\textcolor{red}{[ #1 --Zhong]}}}
\newcommand{\jiyong}[1]{\textbf{\textcolor{blue}{[ #1 -- Ji-Yong]}}}
\newcommand{\wolf}[1]{\textbf{\textcolor{green}{[ #1 -- Wolf]]}}}
\newcommand{\lucas}[1]{\textbf{\textcolor{cyan}{[ #1 -- Lucas]}}}
\newcommand{\jieung}[1]{\textbf{\textcolor{violet}{[ #1 -- Jieung]}}}
\else
\newcommand{\zhong}[1]{}
\newcommand{\jiyong}[1]{}
\newcommand{\wolf}[1]{}
\newcommand{\lucas}[1]{}
\newcommand{\jieung}[1]{}
\fi
%% END COMMENTS

\newif \iftopics \topicsfalse
\iftopics
\newcommand{\topic}[1]{{\textcolor{gray}{[#1]\\}}}
\else
\newcommand{\topic}[1]{}
\fi

% keyword
\newcommand{\coq}{Coq}
\newcommand{\needcite}{\textbf{\textcolor{red}{[cite]}}}
\newcommand{\citethese}[1]{\textbf{\textcolor{red}{[cite:#1]}}}
\newcommand{\checkterms}[1]{\textbf{\textcolor{blue}{#1}}}
\newcommand{\globalstate}{global model}
\newcommand{\Globalstate}{Global model}

% Orders on layers and behaviors: \lsim and \lle correspond to QuiverSim,
% whereas \lpath corresponds to CategorySim.
\newcommand{\lle}{\preceq}
\newcommand{\lsim}[1]{\prec_{#1}}
\newcommand{\lpath}[1]{\le_{#1}}
%\newcommand{\lhtap}[1]{\ge_{#1}}
\newcommand{\id}{\textbf{id}}

% Typing judgements
\usepackage{relsize}
\newcommand{\ltyp}[4]{#1 \vdash_{#2} #3 : #4}
\newcommand{\lltyp}[5]{#2 \vdash^{#1}_{#3} #4 : #5}
\newcommand{\rulename}[1]{\textsc{\smaller{}#1}}

% Useful shorthands used in layer.tex
\newcommand{\pcom}{\parallel}
\usepackage{scalerel}
\DeclareMathOperator*{\bigJoin}{\scalerel*{\Join}{\sum}}
\newcommand{\kw}[1]{{\mathsf{#1}}} % render as a keyword
\newcommand{\pset}[1]{{\mathcal{P}({#1})}}
\newcommand{\refines}{\sqsubseteq} % refinement relation


\newcommand{\eg}{{e.g.\xspace}}
\newcommand{\ie}{{i.e.\xspace}}
%\newcommand{\cf}{{\em cf. }}
\newcommand{\cf}{see }
\newcommand{\etal}{{\em et al.\xspace}}
\newcommand\step[1]{\overset{#1}{\rightarrow}}


\newcommand*{\LargerCdot}{\raisebox{0.25ex}{\scalebox{0.7}{$\bullet$}}}
\newcommand\cons{\LargerCdot}
\newcommand\comm[1]{\mathsf{\textcolor{black}{#1}}}
\newcommand\commb[1]{\mathsf{#1}}
%\newcommand\mtext[1]{\mathit{#1}}
\newcommand\set[1]{\{#1\}}
\newcommand\any{\cdot}
\newcommand\nonev{\_}
\newcommand\integer{\mathbb{N}}

% The specification #1 takes a step from (#2, #3) to (#4, #5)
% The second case is for when there's no return value
\newcommand{\sstepr}[5]{#1(#2, #3) \ni (#4, #5)}
\newcommand{\sstep}[4]{#1(#2, #3) \ni (#4)}
\newcommand{\para}[1]{\ifTR{}{\vspace{-4pt}}\paragraph{\textbf{#1}}}
\newcommand{\modulef}[1]{#1 \mapsto \kappa_{#1}}

\newcommand{\remove}[1]{\textcolor{red}{\sout{#1}}}
\newcommand{\added}[1]{\textcolor{blue}{#1}}

\hyphenation{Comp-Cert}
  % all the local macros used in the paper

\bibliographystyle{abbrvunsrt}

\begin{document}

% Need to define title before the abstract.
\title{Modular and Compositional Development of certified Concurrent Software Systems}
\author{Jieung Kim}
\advisor{Zhong Shao}
\date{March, 2019} 

\frontmatter

\begin{abstract}
Distributed systems are notoriously complex due to the many possible interleavings of their coarsely-connected instances as well as the possibility of errors in both  those instances and the network environment. For these reasons, verification of distributed systems is desirable to remove the possibility of bugs and guarantee their safety and correctness. However, much current verification work still requires a great deal of effort and sometimes has limitations.

We present a verification approach that uses \textit{write-witness-passing}, which is simple but novel in distributed system verification. It is a scalable, reusable, and extensible approach that can be directly linked with the low-level implementations of distributed protocols through contextual refinement. Write-witness-passing can capture the common behaviors of many distributed protocols, and provides both a simple way of understanding the protocols as well as an easy methodology for verifying them.

To demonstrate how write-witnesses work, we verify the functional correctness and safety of Paxos, one of the most famous consensus protocols. We implement the key routines of Paxos in C, and use Coq to verify both the functional correctness of the implementation as well as the safety properties of the protocol within less than 4 person-months. We also describe how we can apply our approach to other distributed protocols to illustrate its generality.

\end{abstract}


\maketitle
\makecopyright{2018} % change as needed.

\chapter{Acknowledgements} % this needs to be before \mainmatter.
\thispagestyle{empty}


\clearpage


\tableofcontents
\listoffigures % remove this if you have no figures.
\listoftables % remove this if you have no tables.


% Starts proper arabic numbering of pages and chapters.
\mainmatter

\chapter{Introduction}
\label{chapter:introduction}
The goal of this dissertation is to build formal methods for concurrent program verification and apply those techniques to 
multiple examples so we can guarantee to users that these systems are reliable and trustworthy not only in terms of functional correctness 
but also other high-level progress properties or the protocol correctness of them.


\section{Challenges in Concurrent Program Verification}
\label{chapter:introduction:sec:challenges-in-concurrent-program-verification}




The prevalence of concurrent environments brings enormous changes to the software. 
They allow for achieving higher performance or more functionality in a single software by using interactions among several instances 
(\ie, multiple threads, nodes, I/O devices, and networks) in the system, 
but they create whole new challenges in terms of providing correctness software. 
Obviously, they are well-known to be difficult to get right and debug because of numerous (usually unbounded)
 interleavings among their multiple components. Testing is necessary to avoid their possible faults, 
 but it is not a promising way to provide their high assurance. Reproducing a bug is unfeasible unless testers know their precise interleaving order.
In this sense, building reliable concurrent programs requires verification to formally shows that they 
correctly reflect the desired behavior (\ie, the behavior stated in their specifications) without missing any single interleaving cases.


In addition, modern software systems consist of multiple sub-modules, 
which are intimately connected with other modules in the system. This brings another source of complexity to the verification. 
These sub-modules highly depend on others, which makes the modular reasoning of each component difficult. 
Accordingly, software verification is considered painful work with prohibitive costs associated with the difficulty of achieving scalability, 
reusability,  and extensibility. 
For example, a famous previous by an seL4 team ~\cite{klein2009sel4} accomplished a breakthrough in software verification 
by providing the first verified (sequential) microkernel and connecting all proofs in a mechanized proof assistant, 
but it required considerable effort. The verification took 11 person years for 7,500 lines of C codes but still contained unverified parts
 (\ie, 1,300 lines of C, 500 lines of assembly, and the compiler to extract the executable code from the verified C codes).
%When combined with concurrency,
%some modules facilitate
%shared operations of other modules
%to form the new shared services.
%It also provides another challenge 
%in the concurrent program verification, 
%which has to resolve not only the interleaving among the concurrent instances as well as the dependencies among the sub-modules 
%that forms the entire system.


\begin{figure}

NEED TO ADD FIGURE
NEED TO ADD FIGURE
NEED TO ADD FIGURE
NEED TO ADD FIGURE
NEED TO ADD FIGURE
NEED TO ADD FIGURE
NEED TO ADD FIGURE
NEED TO ADD FIGURE
NEED TO ADD FIGURE

\label{fig:intro:challenges-concurrent-prog-verification}
\caption{Challenges in Concurrent Program Verification.}
\end{figure}

\jieung{Need to add figure}


%%%%%%%% isolation is required

Therefore, modular and compositional reasoning is indispensable for concurrent program verification. 
The verification should be able to decompose the entire system into a collection of instances (\ie, multiple threads or a set of nodes) and 
further into a stack of modules in each instance; then it needs to be achieved with each module separately without considering complex 
dependencies or interleavings with other modules and instances on the system. Of course, it needs to provide evidence about the system's 
consistency by verifying each module and the behavior of the entire system. 
This modular and compositional software reasoning not only provides an efficient tool for verification 
but also opens the possibility of proving the correct behavior of the system software that is usually parameterized by other programs running on them.
%
%This feature is crucial in some sorts 
%of concurrent programs such as 
%operating systems, libraries, or application interfaces because of the
%proof of them usually needs to be parameterized by 
%other programs running on them. 
%In those cases, composition and proof isolation 
%give enough power to state and prove the correctness property 
%of those programs upon any arbitrary context programs run with the targeted programs. 

%%%%%%%v other previous works 
In this sense, several previous works address modular and/or compositional reasoning with respect to programs--either with or without concurrency.
% need separation logic????
There are two traditional approaches--rely-guarantee ~\cite{jones83} and separation logic ~\cite{ishtiaq01}-and many others 
that stem from either or both of them 
~\cite{feng07:sagl,vafeiadis:marriage,LRG,fu10:roch,sergey15, lili16,Vafeiadis11mfps, Yang07relsep,
Liang14lics}.
Besides, some approaches not only focus on functional correctness but also shows liveness~\cite{lili16}.
However, most previous works do not provide a programming framework that addresses all the above challenges
in concurrent program verification and is capable of extracting the running code from the program written in low-level programming 
languages such as C and assembly.

\section{Verification Toolkit for Concurrent Programs}
\label{chapter:introduction:sec:verification-toolkit-for-concurrent-programs}
%
%\begin{figure}
%\includegraphics[width=\textwidth]{figs/intro}
%\caption{Verification Toolkit Structure}
%\label{chapter:intro:verific	tion-toolkit-structure} 
%\end{figure}
With those investigations, we present the toolkit that supports modular and compositional verification for concurrent programs. 
The toolkit consists of two parts: 1) propose the method to build local layer interfaces for concurrent program verification; and 
2) provide the concurrent-linking framework. All layers and linking parts are also connected with the traditional simulation 
technique~\cite{compcert, deepspec}. 
 
The first part of our toolkit is to construct certified concurrent abstraction layers: a new compositional model for concurrency, 
a program verifier for concurrent C and assembly, and a verified C compiler. 
Each layer interface in our framework functions as an executable machine which consists of state and multiple transition rules. 
To divide the program into fine-grained pieces, we follow the idea of abstraction layers, which are widely used in modern computer systems. 
Our layered approach also reduces the complex dependency among the sub-modules in the entire system.
We use the verified compiler for this layered approach to minimize the trusted parts between the verified program and the executable code 
on a bare machine. Programs written and verified with our toolkit use a subset of C language, 
which can be compiled into $\intelmachine$ assembly language via $\compcertx$, which is the variance of verified C compiler $\compcert$.

This first part of our verification toolkit has several distinctive features which stem from the requirements of 
a sizeable concurrent system. First, it is suitable for dealing with the low-level code. To make the proofs tractable, 
we mainly work at the C level (relying on the $\compcert$ verified compiler~\cite{compcert}), 
but we sometimes need to go lower. 
For example, the MCS algorithm needs to use atomic CPU instructions (fetch-and-store and compare-and-swap), 
so we need a way to mix C and assembly code. At the same time, C itself is too low-level to reason about conveniently, 
so we need data abstraction to hide the details about representation in memory. 
Our toolkit provides an efficient way to verify C and assembly programs as well as connect the data abstraction 
with detailed representations in memory. Second, to handle large developments, we need separation of responsibilities. 
In a small proof of a small concurrent program in isolation, 
you can state the specification as a single pre- and post-condition which specifies the shape and ownership of the data structure, 
the invariants of the liveness conditions, and even the behavior of the entire system. 
But such a proof is not modular and not reusable. In our development, 
these are done as separate refinement steps in separate modules with explicit interfaces, 
and they can even be the responsibility of different software developers. 
Finally, the layers approach is general purpose in the sense that the same semantic framework can be used for proving all kinds of properties. 
The model of program execution exposed to the programmer is simple—mostly the same as what you would use for sequential code, 
and with a notion of logs of events to model concurrency. We did not have to add any special features to the logic to show high-level 
properties (such as a liveness property) because we could directly reason in $\coq$ about how long an execution will take.

The other part of our toolkit is to provide the connection between multiple concurrent instances in the system. 
Our certified concurrent abstraction layers provide the way to build and verify concurrent programs by decoupling 
the complex interleaving from the correctness proof of each layer in the system. 
This compositional requires each layer to use assumptions on the environment of the layer interface--the behavior of other concurrent components 
in the system. 
Those assumptions inevitably need to be compatible with the properties that other concurrent components can guarantee; 
thus, proving this property is desirable for our concurrent program verification toolkit. To fulfill this requirement, 
we provide the concurrent linking library as a part of our toolkit, which includes defining concurrent machine 
models and providing generic proof methods to show the validity in the parallel composition of multiple instances as well as 
the formal connection between the program on concurrent machine models and the program on the local layer interface.

Our linking library also has several unique aspects to apply it to large concurrent system verification. 
First, our concurrent machine semantics are generic enough to be applied to arbitrary programs written in our layered framework. 
We separate the linking process from the concurrent layer building so users do not have to deal with the details of linking itself when they 
build concurrent layers using our framework. Second, it also can be linked with the assembly model for the verified compiler $\compcertx$, thus
 providing the full formal connection between the program written in C and assembly and the data abstraction for the detailed data 
 representation on the memory. Third, it provides complex dependencies among multiple data structures in large concurrent programs. 
 This is not only related to the dependencies among shared objects in the program but also the dependencies between 
 shared and private objects. Our toolkit handles all those issues with reasonable restrictions.

\section{Concurrent Program Verification Examples}
\label{chapter:introduction:sec:concurrent-program-verification-examples}


\begin{figure}

NEED TO ADD FIGURE
NEED TO ADD FIGURE
NEED TO ADD FIGURE
NEED TO ADD FIGURE
NEED TO ADD FIGURE
NEED TO ADD FIGURE
NEED TO ADD FIGURE
NEED TO ADD FIGURE
NEED TO ADD FIGURE

\label{fig:intro:certikos-structure}
\caption{Structure of Concurrent Operating System Verification.}
\end{figure}



As examples of the applicability of our framework, we verified two large-scale concurrent systems--$\certikos$ and $\wormspace$.
These examples also represent two models for concurrent programs: 
the shared-memory model of concurrency and the message-passing model.

A concurrent operating system is a well-known example of the shared-memory model for concurrency that multiple threads 
in the system share the same physical memory or a common 
file system. Operating system verification has been studied for a while with impressive results~\cite{klein2009sel4, xu16, hawblitzel10}.
However, these studies either lack supporting fine-grained lock control on shared resources or lack progress property proofs of their kernels.
 On the other hand, $\certikos$ is the first verified concurrent operating system kernel with fine-grained locking. 
 The kernel is written in C and assembly, and the extracted code of the kernel (via verified compositional compilation) 
 runs on an $\intelmachine$ multicore machine. It consists of 6,500 lines of C and assembly implementation and 200K+ lines of $\coq$ proofs.

To manage such a massive verification work, we divide the kernel into multiple sub-modules and link them together to form the 
correctness theorem of the kernel. This work not only facilitates the abstraction-layer approach in our framework 
but also uses our concurrent-linking framework to show the simulation relation between 
the program on each instance with its concurrent environment and the program runs on the full 
concurrent machine--the $\intelmachine$ multicore machine. 

We also built a spinlock module to support fine-grained lock services for multiple shared objects in the kernel 
(\ie, page allocation, atomic queue, \etc).
The MCS Lock--one of two lock algorithms we used--verification is described in detail in this thesis, 
focusing on how we divided the requirement of lock verification into multiple layers, proving
liveness of the lock and providing a simple but common interface of the verified lock for other shared resources.

The MCS algorithm is well-known in scalable fair inter-CPU mutex locks. 
Although the program is short, the proof is challenging. First, the implementation of a lock algorithm cannot itself use locks, 
so it must rely solely on atomic-memory instructions and be robust against any possible interleavings between CPUs. 
This is the most challenging type of concurrency--so-called lock-free programming.
Second, unlike algorithms which only promise mutual exclusion, the MCS algorithm also aims for fairness among CPUs.
To check that it got it right, our correctness theorem needs to guarantee not only mutual exclusion (a safety property) but also bounded waiting time (a liveness property). We show how we resolve these challenges in this thesis.


Our verification toolkit is inspired by the $\certikos$ project, but it is not limited to operating system verification. 
Distributed systems are well-known as the message-passing model of concurrency, where nodes in distributed systems are connected with 
others via network communication. They serve millions of users in important applications
 these days (\ie, banking, communications, and social networking), but they are difficult programs to be correctly implemented 
 due to their concurrency and their failures. Local nodes may crash at arbitrary moments, and communications may possess failures 
 such as packet reordering, loss, and duplication. In this sense, distributed systems are a desirable target for verification to 
 show the applicability of our framework.

To build a trustworthy distributed system, we first introduce the WOR abstraction inherent in many distributed systems and present a simple, 
data-centric WOR API as a first-class programming abstraction.  
Next, we implement three distributed applications over this API; for each, our modular design easily allows new configurations 
with different performance and availability properties while matching or surpassing the performance of an existing monolithic implementation 
in a similar configuration. Finally, we apply our verification toolkit into distributed systems by adding a 
non-Byzantine band asynchronous network model (which allows packet duplication, delay, and loss). We built WormSpace, 
which is a distributed system API that provides the abstraction of the common interfaces that many distributed systems can use. 
The system is implemented via a collection of Paxos, and we prove its functional correctness as well as the key safety property of the 
protocol--immutability.


\section{Toolkit for Leader-based Distributed Protocols and Systems}
\label{chapter:introduction:sec:toolkit-for-leader-based-distributed-protocols-and-systems}


%The verification of distributed systems reveals another challenge beyond showing their functional correctness of those programs:  the safety proofs of their underlying protocols. 
While even verifying a single distributed system is challenging, in practice, distributed applications 
rely on several distributed systems. An application might employ different distributed systems for distinct functionalities (\eg,
consensus~\cite{vivaladifference}, distributed transactions~\cite{gray:2006},
and distributed locks~\cite{chubby, zookeeper} as part of a high-reliability distributed database),
or it might use systems that achieve the same goal 
 (\eg,
multi-Paxos~\cite{paxosmadesimple, rvrpaxos} vs. Raft~\cite{raft}) in different parts, depending on performance considerations or simple preference.
 Therefore, to realize a verified distributed system environment, methodologies to cover multiple distributed systems are necessary.

We find that distributed systems that realize strong semantics are typically designed under a common pattern: 
they exploit a leader node (or a centralized coordinator) explicitly or implicitly to coordinate distributed state changes. Indeed, 
for simplicity of management and understanding, this leader-based scheme is commonly used to implement critical distributed functions. 
For example, multi-Paxos and Raft elect a leader to replicate states across multiple nodes, two-phase commit employs a transaction manager to
 coordinate transactions over various resource managers, and coordination services grant a lock to a requester to allow for mutually exclusive 
 access to a distributed shared state. 
 To account for this, our dissertation proposes an idea that can be used in the distributed system 
 verification--especially leader-based distributed systems.

\section{Contributions by Collaborators}
\label{chapter:introduction:sec:contributions-by-collaborators}

The works in chapters~\ref{chapter:ccal},~\ref{chapter:mcs-lock},~\ref{chapter:linking}, and \ref{chapter:certikos} are parts of 
the $\certikos$ project, and were jointly done with various members in our group.
The author collaborated with Ronghui Gu on developing concurrent certified abstraction layers, concurrent linking libraries, and the verification on the verified concurrent OS kernel ($\certikos$).  
The author wrote almost all proofs in linking libraries and concurrent-linking proofs for $\certikos$ with some help from 
J{\'e}r{\'e}me Koenig.
Among the entire $\certikos$, 
the author developed the whole MCS Lock module with Vilhelm Sj{\"o}berg,
who also contributed to many other parts. The automation engine for proving the C source program were developed solely by Xiongnan (Newman) Wu,
 but its details are out of scope in this thesis. Wu’s thesis illustrates in-depth explanations about the automation engine. 
 For the case study on distributed system verification in Chapter~\ref{chapter:wormspace} ($\wormspace$), Ji-Yong Shin and Wolf Honore 
 worked together to build and verify the system, and the author had a leading role in designing the network models, 
 layers for the target system, and safety proof of the system.
 The author also worked with Ji-Yong Shin and Wolf Honore to provide a verification toolkit for leader-based distributed systems 
 in Chapter~\ref{chapter:witness-passing}.

\section{Contents of the Chapters}
\label{chapter:introduction:sec:contents-of-the-chapters}

The rest of this dissertation is organized as follows. Chapter~\ref{chapter:ccal} focuses on the framework to build local layer 
interfaces of concurrent programs--an abridged version of the related parts of our work in~\cite{concurrency}.
Chapter~\ref{chapter:mcs-lock} is an abridged version of our work from~\cite{mcslock},
which is a case study that uses the framework in Chapter~\ref{chapter:ccal}. 
Chapter~\ref{chapter:linking} provides the details for our concurrent-linking, multicore linking, and multithreaded-linking frameworks, 
of which the high-level idea is part of our work in~\cite{concurrency}.
However, this chapter differs from our previous publications by providing in-depth explanations for the parts related to concurrent linking, 
thus addressing how to use our linking framework by presenting formal rules, proofs, and its true capability, 
which are also addressed in~\cite{concurrency}.
Chapter~\ref{chapter:certikos} shows our work on $\certikos$,
which is closely related to our work in~\cite{certikos:osdi16}. 
It provides an interesting case study that uses all the ingredients of our concurrent verification
 framework as well as shows its full power. The verification work on distributed systems, $\wormspace$,
is discussed in Chapter~\ref{chapter:wormspace}.
It is an abridged version of our work in~\cite{wormspace},
which shows the applicability of our framework to distributed systems. 
Chapter~\ref{chapter:witness-passing} 
explains our idea for how to provide a generic toolkit to verify multiple leader-based distributed systems, 
which is inspired by the verification work on $\wormspace$.
Chapter~\ref{chapter:related}
offers an in-depth discussion of related work, and  Chapter~\ref{chapter:conclusion}  
mentions limitations and future directions as well as summarizes this thesis.


\chapter{Overview}
\label{chapter:overview}

\chapter{Concurrent Certified Abstraction Layer}
\label{chapter:ccal}


\chapter{Case Study: MCS Lock}
\label{chapter:mcs lock}
Distributed systems are notoriously complex due to the many possible interleavings of their coarsely-connected instances as well as the possibility of errors in both  those instances and the network environment. For these reasons, verification of distributed systems is desirable to remove the possibility of bugs and guarantee their safety and correctness. However, much current verification work still requires a great deal of effort and sometimes has limitations.

We present a verification approach that uses \textit{write-witness-passing}, which is simple but novel in distributed system verification. It is a scalable, reusable, and extensible approach that can be directly linked with the low-level implementations of distributed protocols through contextual refinement. Write-witness-passing can capture the common behaviors of many distributed protocols, and provides both a simple way of understanding the protocols as well as an easy methodology for verifying them.

To demonstrate how write-witnesses work, we verify the functional correctness and safety of Paxos, one of the most famous consensus protocols. We implement the key routines of Paxos in C, and use Coq to verify both the functional correctness of the implementation as well as the safety properties of the protocol within less than 4 person-months. We also describe how we can apply our approach to other distributed protocols to illustrate its generality.

\section{Introduction}
\label{sec:intro}

The MCS algorithm for scalable fair inter-CPU mutex locks makes for an interesting case study in program verification.
Although the program is short, the proof is challenging.
First, the implementation of a lock algorithm can not itself use locks, so it has to rely solely on atomic memory instructions and be robust against any possible interleavings between CPUs. This is the most challenging type of concurrency, so-called lock-free programming.
Second, unlike algorithms which only promise mutual exclusion, the MCS algorithm also aims for fairness among CPUs. To check that it got it right, our correctness theorem needs to guarantee not only mutual exclusion (a safety property) but also bounded waiting time (a liveness property).

Previous work~\cite{liang:lili,ogata:mcs-lock} has studied the
correctness of the algorithm itself, but those verification efforts
did not produce executable code, and did not explore how to integrate
the proof of the algorithm into a larger system. We have created a
fully verified implementation and added it as part of the CertiKOS
kernel~\cite{certikos16}, which consists of 6500 lines of C and
assembly implementation and 135K lines of Coq proofs.

In order to manage such a large verification effort, the CertiKOS team developed a methodology known as \emph{certified (concurrent) abstraction layers}, as well as a set of libraries and theorems to support it. Previous papers~\cite{dscal15,ccal16}
described this framework, but many readers found them  dense and hard to follow because they immediately present the formalism at its most abstract and general.
This paper aims to be a complement: by zooming in on the implementation of one small part of the kernel (the MCS Lock module), we illustrate  what it is like to \emph{use} the framework, how to write specifications in the ``layers'' style, and what the corresponding proof obligations are. We hope this paper will be an easier entry point for understanding our verification framework.

As we will see, CertiKOS-style verification has several distinctive features which stem from the requirements of a large kernel. First, it is suitable for {\bf dealing with low-level code}. To make the proofs tractable we mainly work at the C level (relying on the CompCert verified compiler~\cite{Leroy-Compcert-CACM}), but sometimes we need to go lower. For example the MCS algorithm needs to use atomic CPU instructions ({\em fetch-and-store} and {\em compare-and-swap}), so we need a way to  mix C and assembly code, while stating precisely what semantics we assume that the assembly code has. At the same time, C itself is too low-level to conveniently reason about, so we need {\bf data abstraction} to hide the details about representation in memory.

Second, in order to handle large developments we need {\bf separation of responsibilities}. In a small proof of an algorithm in isolation, you can state the specification as a single pre- and post-condition which specifies the shape and ownership of the data structure, the invariants (e.g. mutual exclusion), the liveness conditions, and even the behavior of the lock's client code (the critical section code). But such a proof is not modular and not re-usable. In our development, these are done as separate refinement steps, in separate modules with explicit interfaces, and can even be the responsibility of different software developers. 

Finally, the layers approach is {\bf general purpose}, in the sense
that the same semantic framework can be used for proving all kinds of
properties. The model of program execution exposed to the programmer
is simple, mostly the same as for sequential code and with a notion of logs of events to model concurrency.
Unlike working in a special-purpose program logic, we 
did not have to add any features to show a liveness property, because we can directly reason in Coq about \textbf{how long} an execution will take. 

In the remaining parts of the paper, we first explain the C code that we will be verifying (Sec.~\ref{sec:overview}).
Then in the bulk of the paper, we explain our proof strategy by going through each abstraction layer in turn, concluding with the safety and starvation freedom properties (Sec.~\ref{sec:verification}). Finally we explain how our proofs fit as a part of the larger CertiKOS development (Sec.~\ref{sec:evaluation}) and discuss related and future work (Sec.~\ref{sec:related}).

We revisit several points that have been mentioned in previous publications:

\begin{itemize}
\item We show how to customize the machine model by adding a trusted specification of particular instructions that we need. (Sec.~\ref{subsec:lowestmachinemodel}.)

\item We locally verify the execution of a single CPU, and treat the rest of the system as an opaque \emph{concurrent context}. (Sec.~\ref{subsec:abstractoperationlayer}.)

\item We illustrate how to abstract away from a C implementation, by refining it into a functional specification which can be conveniently reasoned about. (Sec.~\ref{subsec:atomicoperation}.)

\item Similarly, we show how the same type of refinement can be used to gradually add ghost state to a specification while hiding un-needed details. (Sec.~\ref{sec:representation-ghost}.)

\end{itemize}

We also make novel contributions:


\begin{itemize}
\item It provides a concrete example of CertiKOS-style verification; in particular we can see how to customize the machine model (Sec.~\ref{subsec:lowestmachinemodel}) and how to split the verification effort into CPU-local reasoning (Sec.~\ref{subsec:eventlogandoracle} and \ref{subsec:abstractoperationlayer}).
  
\item We show a way to prove that an atomic specification refines a concurrent implementation, while still using downward rather than upward simulations. The trick is to provide a \emph{function} from low-level to high-level logs of events. (Sec.~\ref{sec:liveness-atomicity}--\ref{sec:downwards-to-upwards}.)

\item We propose a new way to specify the desired---atomic---behavior of the lock/unlock methods. To ensure liveness, the specification of the lock method itself includes a promise to later call unlock; we do this using a bounding counter. (Sec.~\ref{sec:liveness-atomicity}.)

\item And of course, we provide the first implementation of the MCS algorithm that has been both rigorously verified (with a mechanized proof) and at the same time realized (as part of a running kernel).
\end{itemize}



\section{The MCS Algorithm}
\label{chapter:mcslock:sec:overview}

The $\mcsname$ algorithm~\cite{mcs91} is a list-based queuing lock that provides a \textit{fair} and \textit{scalable} mutex mechanism for multi-CPU computers. Fairness means that CPUs competing for the lock are guaranteed to receive it in the same order as they asked for it (FIFO order).
In  an unfair lock, CPUs
that attempt to take the lock can become nondeterministically passed over (even a million times in a row~\cite{lwn:ticketlocks}), thereby creating unpredictable latency.

Fairness is also an important factor to its verification, as it is possible that one particular CPU is continuously passed over; 
as such, it loops indefinitely without it. 
Therefore, unless the lock guarantees fairness, there is no method of proving a termination-sensitive refinement between the implementation and a simple (terminating) specification. 
With a non-fair lock, we would have to settle for either an ugly specification that allowed non-termination or for a weaker notion of correctness such as termination-insensitive refinement.
\begin{figure}
\lstinputlisting [language = C, multicols=2] {source_code/mcslock/mcs_struct.h}
\vspace{1em}

\lstinputlisting [language = C, multicols=2] {source_code/mcslock/mcs_lock_acquire.c}
\caption{Data Structure and Implementation of $\mcsname$ Lock (Written in C).}
\label{fig:chapter:mcslock:mcs_lock}
\end{figure}

Figure~\ref{fig:chapter:mcslock:mcs_lock} shows the implementation of the $\mcsname$ Lock algorithm written in C. 
The data structure  (Figure~\ref{fig:chapter:mcslock:mcs_lock})
has one global field pointing to the final node of the queue structure, and the 
per-CPU nodes forming each node in the queue. This is similar to an ordinary queue data structure (note that it only has a pointer to the tail, and not the head of the queue).
If the queue is empty, we set $\codeinmath{last}$ to the value $\invalidmcsval$, which acts as a null value (we could also have used \eg, $-1$). 
The queue is used to order the waiting CPUs in order to ensure that lock acquisition is FIFO. 
The structs also include padding to take up full cache lines and avoid false sharing. 
Each node is owned by one particular CPU (the array is indexed by CPU ID). 
To make the lock scalable, the CPU continues looping and waiting for the lock will only read its own busy flag; 
therefore, there is no cache-line bouncing. 
Simpler lock algorithms make all the CPUs read the same memory location, which does not scale past 10-40 CPUs~\cite{Boyd-wickizer12}.

We follow the programming style of $\ccalname$  presented in Chapter~\ref{chapter:ccal}; thus, the verified version we created has a property that all lock node and variables are statically assigned. 
However, this does not restrict the functionality of our verified $\mcsname$ Lock. 
Instead of creating the new node upon acquiring the lock and removing the node when releasing the lock, 
each process can pick up one node and use it as its node during the lock operation. 
All nodes are uniquely distinguished by lock ID and CPU ID, but these can be reused for the same purpose.


Figure~\ref{fig:chapter:mcslock:mcs_lock} 
also presents the code for acquire lock and release lock operations. 
The acquire lock function uses an atomic {\em fetch-and-store} expression to {\em fetch} the current $\codeinmath{last}$ value and  {\em store} its CPU ID as the $\codeinmath{last}$ value of the lock in a single action (line 6). 
Then, if the previous $\codeinmath{last}$  value was $\invalidmcsval$, the CPU can directly acquire the lock and enter the critical section (line 7). 
If the previous $\codeinmath{last}$ value was not $\invalidmcsval$, it implies that some other CPUs are in the critical section or in the queue waiting to enter it (line 9 to line 10). 
In this case, the current CPU will wait until the previous node in the queue sets the current CPU’s busy flag as $\codeinmath{FREE}$ during the lock release.

Release lock also has two execution paths based on the result of an atomic operation,  {\em  compare and swap} (line 15). 
This operation stores $\invalidmcsval$ into $\codeinmath{last}$ only if the old value of $\codeinmath{last}$ is equal to $\codeinmath{cpuid}$, 
and returns {\em true} if the update succeeds and {\em false} otherwise. 
The $\CAS$  operation succeeds, it immediately releases the lock if the current CPU is the only one in the queue.
 If the $\CAS$  fails, it implies that some other CPU has already performed the {\em fetch-and-store} operation (line 6). 
 Thus, the current CPU busy-waits until the other CPU sets the $\codeinmath{next}$ field (line 8), 
 and then passes the lock to the head of the waiting queue by assigning $\codeinmath{busy}$.


\begin{figure}
\begin{center}
\includegraphics[width=0.9\linewidth]{figs/mcslock/mcsex}
\end{center}
\caption{Possible Execution Sequence for  $\mcsname$ Lock.}
\label{fig:chapter:mcslock:mcs-example}
\end{figure}

Figure~\ref{fig:chapter:mcslock:mcs-example} 
illustrates a possible sequence of states taken by the algorithm. 
At the beginning (a),
the lock is free, and CPU 1 can take it in a single atomic $\FAS$ operation (b). 
Since CPU 1 did not have to wait for the lock, it did not need to update its \textit{next}-pointer. 
After that, CPUs 2 and 3 attempt to take the lock ((c) and (d)). 
The final value will be updated correctly due to the property of the atomic expression.
 However, there can be some delay between the CPU updating the \textit{tail} pointer and adjusting the \textit{next}-pointer of the previous node in the queue; as the example illustrates, this means that although there are three nodes which logically makes up the queue of waiting CPUs, 
 any subset of the next-pointers may be unset. 
 At (e), although CPU 1 wants to release the lock, 
 the $\CAS$ call will return false (because $\codeinmath{tail}$ is 3, not 1). 
 In this case,  CPU 1 must wait in a busy-loop until CPU 2 has set its next-pointer (f). 
 Thereafter, CPU 1 can set the busy flag to $\codeinmath{FREE}$ for the next node in the queue--CPU 2’s node--which releases the lock (g).






Since the algorithm is fair, it satisfies a \textit{liveness} property:\begin{quote}
``Suppose all clients of the lock are well-behaved, \ie, whenever they acquire a lock they release it again after some finite time, and suppose the scheduling of operations from different CPUs is fair. Then  whenever $\mcsacquire$ or $\mcsrelease$ are called they will succeed within some finite time.''
\end{quote}
A major part of our formal development is devoted to stating and proving the liveness property. 
As such, an informal proof sketch is provided here. Consider the CPU that starts executing 
 $\mcsacquire$.
At this time, the queue contains a finite number of CPUs already waiting for the lock.
By fairness of scheduling, the CPU at the head of the queue will get scheduled periodically, say every $F$ step. Each time it is scheduled, it will go through three phases. 
First, it will execute the code in $\mcsacquire$; also, since it is at the head of the queue, the loop will terminate immediately. 
Then, it executes the code in the critical section; 
by assumption, this completes after some finite number $k$ of operations. 
Finally, it executes $\mcsrelease$; this either completes immediately or enters the waiting loop, in which case it will complete as soon as the next CPU in the queue is scheduled.




\section{Verification---Layer by layer}
\label{chapter:mcslock:sec:verification}%

\begin{figure}
\begin{center}
\includegraphics[width=\textwidth]{figs/mcslock/layer_overview}
\end{center}
\caption{$\mcsname$ Lock Layers.}
\label{fig:chapter:mcslock:layeroverview}
\end{figure}


We build five layers, starting from a base layer which represents the machine model that our compiled code will
run on.
Figure~\ref{fig:chapter:mcslock:layeroverview} shows the overall structure of our development.
For simplicity the figure only includes lock primitives, and
primitives passed through from below are hidden in it.
We also omit the current focused set ($cid$ -- CPU ID) and the environmental context ($\oracle$) in the figure
 as well as the rest (except in lemmas and theorems) of this section for simplification.
The CPU ID will not be changed at all during the whole layers (note that all  layer interfaces in this $\mcsname$ Lock module 
are local layer interfaces), but there is a place that we need a simplification of the global log to guarantee the atomicity of 
acquire/release locks as mentioned in Section~\ref{chapter:ccal:subsec:local-layer-interface}.
We will clearly state that which layer is related to a \textit{log-lift} pattern; 
thus require the change in an environmental context.
Each big and outer rectangle in the figure means each layer in the $\mcsname$ Lock module, 
and small and inner rectangles in each layer implies primitives defined in the layer.
Arrows show dependencies between adjacent layers,
for example the implementation of $\waitlockfunc$ in  $\mmcslockopfull$
uses three primitives ($\mcsswaptailabs$,
$\mcssetnextabs$, and $\mcsgetbusyabs$) from  $\mmcslockabsintrofull$.

Most layers in the figure are related to the \textit{fun-lift} pattern in Section~\ref{chapter:ccal:subsec:local-layer-interface}.
The layers from  $\mmcsbootfull$  through the $\mmcslockabsintro$
introduces getter and setter functions for accessing the memory in the machine
(Section~\ref{chapter:mcslock:subsec:lowestmachinemodel} and
\ref{chapter:mcslock:subsec:abstractoperationlayer}). These layers also
contain logical primitives which record events to the log; we are in
effect manually implementing a model of concurrent execution by
extending a sequential operational semantics for C. 
The layer $\mmcslockopfull$ contains the C code from 
Figure~\ref{fig:chapter:mcslock:mcs_lock}. This layer proves low-level
functional correctness, \ie, it reasons about the C code and
abstracts away details about memory accesses, integer overflows, etc,
to expose an equivalent specification written as a $\coq$
function (Section~\ref{chapter:mcslock:subsec:atomicoperation}).
The two top layers, $\mqmcslockopfull$ and $\mhmcslockopfull$ , do not introduce any new primitives.
They simplify the specifications of 
the release- and acquire lock functions ($\passlockfunc$ and
$\waitlockfunc$), \ie, each layer ascribes a different
specification (with a different log replay function and a set of events)
to the same C function. Those specification names are notated inside the square bracket in Figure~\ref{fig:chapter:mcslock:layeroverview}.

On the other hand, two layers are closely related to the  \textit{log-lift} pattern in Section~\ref{chapter:ccal:subsec:local-layer-interface}.
The layer $\mqmcslockopfull$ adds ghost states, keeping track of a
queue of waiting CPUs.
(Section~\ref{chapter:mcslock:sec:representation-ghost}). This queue is key to the liveness proof but is not explicitly represented in the C implementation.
The top layer $\mhmcslockopfull$ proves starvation freedom and liveness
(Section~\ref{chapter:mcslock:sec:liveness-atomicity}). This lets us ascribe atomic
specifications where taking or releasing a lock generates just a
single event to the log.

\subsection{Memory operations layers}
\label{chapter:mcslock:subsec:lowestmachinemodel}


Although we glossed over this in Figure~\ref{fig:chapter:mcslock:mcs_lock}, our
actual C implementations of $\mcsacquire$ and
$\mcsrelease$ do not access the system's memory directly.  Instead, they call
a collection of helper functions with names like
$\mcssetnext$. The lowest two layers in our proofs
are devoted to implementing these helper functions.
The key concern is to make sure that the events that get appended to the log correspond to the actual actions to the memory. Some of that can be proven, 
but some parts of this layer are trusted as part of the machine model.

We first describe the first and the lowest tuple in our proofs,

\begin{center}
\begin{tabular}{P{0.95\textwidth}}
$\ltyp{\mmcsboot}{R_{(\mmcslockintro, \mmcsboot)}}{\codeinmath{M}_{\mmcslockintro}}{\mmcslockintro}$\\
(precisely, ``$\ltyp{\PLayer{\mmcsboot}{cid}{\oracle^{mcs}_{cid}}}{R_{(\mmcslockintro, \mmcsboot)}}{\codeinmath{M}_{\mmcslockintro}}{\PLayer{\mmcslockintro}{cid}{\oracle^{mcs}_{cid}}}$'' \\
when $cid$ is the focused current CPU ID and\\
 $\oracle^{mcs}_{cid}$ is an instance of environmental contexts for our program.)\\
\end{tabular}
\end{center}

The interface ($\mmcsbootfull$) represents the machine model that our compiled code will run on.
All primitives defined in $\mmcslockintrofull$ are parts of the trusted computing base and correspond to empty functions in our compiled code.

Eight of  primitives in $\mmcsbootfull$  are firmly related to the  $\mcsname$ Lock verification:

\begin{center}
\begin{tabular}{P{0.95\textwidth}}
$\{\atomicmcslog,\  \atomicmcsswap,\ \atomicmcscas,\ \mcsinitnodelog,$\\
$\mcsgetnextlog,\ \mcssetnextlog,\ \mcsgetbusylog,\ \mcssetbusylog\} $\\
\end{tabular}
\end{center}

Two primitives, $\atomicmcsswap$ and $\atomicmcscas$ are for  two atomic instructions {\em fetch-and-store} and {\em compare-and-swap}, and will be further discussed below.

\begin{figure}
\begin{center}
\lstinputlisting[language = Caml]{source_code/mcslock/mcs_set_next_log.v} 
\end{center}
\caption{$\mcssetnextlog$ Specification.}
\label{fig:chapter:mcslock:specification-of-mcssetnextlog}
\end{figure}

The other six are used to update the log.  
Ordinary assembly instructions only modify a physical memory, not
an abstract state, so in order for programs to be able to append events to
the log, we include these six primitives in $\mmcsbootfull$. 
For example, the specification of $\mcssetnextlog$ (in Figure~\ref{fig:chapter:mcslock:specification-of-mcssetnextlog}\newfootnote{Our implementation divides the global log discussed in Chapter~\ref{chapter:ccal} into a collection of logs, which is a partial form the lock identifier to the log designated with the lock ID by using projection functions for each lock ID. Besides, our logs are defined as a part ($\lockmultilogpool$ in the Figure) of our abstract data in our implementation, which is a little bit different from the local layer state definition in Section~\ref{chapter:ccal:subsec:local-layer-interface}, but which is exactly same in its meaning. We show how they are connected together in the next chapter  (Chapter~\ref{chapter:linking}.)})
updates the log by adding one ($\setnext{prev\_id}$) event.
In the compiled code, these primitives appear as empty functions that do nothing, they are only used to modify the logical state.

The code $\codeinmath{M}_{\mmcslockintro}$ in the layer contains  
functions which actually modifies the memory in the way the event announces.
Each function in $\codeinmath{M}_{\mmcslockintro}$ calls the corresponding primitive from
$\mmcsbootfull$ inside the function to add the event to the log.
For example, $\mcssetnext$, one function in $\codeinmath{M}_{\mmcslockintro}$, writes
to $\codeinmath{next}$ and also calls
the empty function $\mcssetnextlog$:
\lstinputlisting [language = Caml]{source_code/mcslock/mcs_set_next.c}


\begin{figure}
\begin{center}
\includegraphics[width=\linewidth]{figs/mcslock/getsetrefinement}
\end{center}
\caption{Structure of $\mcsname$ Lock Memory Operations Layer.}
\label{fig:chapter:mcslock:layer-struct-mcs-verification}
\end{figure}

The interface $\mmcslockintrofull$ contains the high level specification for each function defined in $\codeinmath{M}_ {\mmcslockintro}$. 
Those high-level specifications work on the log instead of the exact memory slot $\lockmemloc$.
Therefore, after proving the {\em refinement} between the memory ($\lockmemloc$ in Figure~\ref{fig:chapter:mcslock:layer-struct-mcs-verification})
and the abstract state ($\lockabsloc$ that is calculated by the \emph{log}\newfootnote{$\lockmultilogpool$: a partial map form a lock identifier to the log with the type $\lockmultilog$ associated with the identifier.}
in Figure~\ref{fig:chapter:mcslock:layer-struct-mcs-verification}), we only need to care about the abstract state.

For the refinement proof, we need two more ingredients.
The first one is a \emph{log replay function}.
A log is merely a list of events, but specifications need to know about the log is what the state of the system will look like after those events have executed,
and a replay function calculates that. 
Different layers may define different replay functions in order to interpret the same log in a way that suits their proofs.
Therefore, we have introduced the proper log replay function in several layers, and prove the relationship between the result of them when we introduced the new one.
In $\mmcslockintrofull$, we define $\calmcslock$, which has the following type:\newline
\begin{tabular}{P{0.95\textwidth}}
$ \calmcslock :\ \lockmultilog\ \rightarrow\ \optiondef\ \lockabsloc$
\end{tabular}\newline
where
\lstinputlisting [language = Caml] {source_code/mcslock/lowmcsstruct.v}
The return type of this log replay function closely corresponds to C data structures, which makes it easy to prove the refinement ($\zmapfunc$ is a finite map from $\ztype$ to $\booltype$*$\ztype$.)
The second ingredient is a relation $R_{\mmcslockintro}$ which shows the relationship between the concrete memory in an underlay $\mmcsbootfull$ and the abstract state in an overlay $\mmcslockintrofull$.
As a part of $R$, we define $\matchmcslock$, which is a part related to 
 $\mcsname$ Lock in $R_{(\mmcslockintro, \mmcsboot)}$,  as follows:

\begin{definition}[$\matchmcslock$]
Suppose that `$loc$' is among the proper field accessors for  $\mcsname$ Lock, which are `$\codeinmath{last}$', `$\codeinmath{ndpool[}i\codeinmath{].next}$', or  `$\codeinmath{ndpool[}i\codeinmath{].busy}$' when `$0 \leq i < \invalidmcsval$'.
 And, assuming that `$\codeinmath{lk\_id}$' is a lock identifier satisfies `$0 \leq \codeinmath{lk\_id} < \codeinmath{lock\_range}$' and $\codeinmath{l}$ is a shared log. Then define \newline
  \begin{tabular}{P{0.90\textwidth}}
    $\matchmcslock\ \codeinmath{(l:Log) (}\lockmemloc\codeinmath{ :block) loc}$\\
      iff ($\exists \codeinmath{val},\ \memloadsome{\memintregulartype}{m}{\lockmemloc}{loc}{\codeinmath{val}}$\\
        $\wedge\  \memaccess{m}{\lockmemloc}{loc}$\\
      $\wedge\ \calmcslock \codeinmath{(l) =}\Some\codeinmath{(mcsval)}\rightarrow\codeinmath{loc}_{a}\codeinmath{@mcsval = val)}$\\
\end{tabular}\newline
    when `$\codeinmath{loc}_{a}\codeinmath{@mcsval}$' represents the corresponding 
    value to the `$\codeinmath{loc}_{a}$' in the `$\codeinmath{mcsval}$' 
    and `$\codeinmath{loc}_{a}$' corresponds to the value of `$\codeinmath{loc}$'.
\end{definition}

Intuitively, the definition says that the value that
$\calmcslock$ calculates from the log always corresponds to the value 
in the memory with the same identifiers. The memory access functions $\memloadkwd$ and $\memaccesskwd$ are
from $\compcert$'s operational semantics for C.
Using the definition, we prove one theorem for each primitive, which
shows that the memory refines the shared log. \eg, for $\mcssetnext$ we prove:

\begin{theorem}[Simulation for $\mcssetnext$]
    \label{thm:chapter:mcslock:machine-state-refinement} Let $R_{(\mmcslockintro, \mmcsboot)}$ be the relation defined as $\matchmcslock$
    over $\codeinmath{LK@}mem$ and $\lockabsloc\codeinmath{@}A_{\mmcslockintro}$, 
identity relation for other parts of $mem$, $A_{\mmcsboot}$ and $A_{\mmcslockintro}$. Then\newline
 \begin{tabular}{P{0.95\textwidth}}
$ \forall (\codeinmath{{m}}_{1} \ \codeinmath{{m}}_{1}'\ \codeinmath{{m}}_{0} : mem)\  (\codeinmath{{d}}_{0} \ : A_{\mmcsboot})\ (\codeinmath{{d}}_{1} \ \codeinmath{{d}}_{1}' : A_{\mmcslockintro}). $ \\
$ \mcssetnext_{L_1}(v, \codeinmath{{m}}_1, \codeinmath{{d}}_1) = \Some (\codeinmath{{m}}'_1, \codeinmath{{d}}'_1) \ \rightarrow \
  R_{(\mmcslockintro, \mmcsboot)}\ (\codeinmath{{m}}_1, \codeinmath{{d}}_1)\ (\codeinmath{{m}}_0, \codeinmath{{d}}_0) \ \rightarrow $\\
  $ \exists (\codeinmath{{m}}_{0}' : mem)\ (\codeinmath{{d}}_{0}' : A_0).$ \\
  $  \mcssetnext_{L_0}(v, \codeinmath{{m}}_0, \codeinmath{{d}}_0) = \Some(\codeinmath{{m}}'_0, \codeinmath{{d}}'_0) \ \wedge \
  R_{(\mmcslockintro, \mmcsboot)}\ (\codeinmath{{m}}'_1, \codeinmath{{d}}'_1)\ (\codeinmath{{m}}'_0, \codeinmath{{d}}'_0).$ 
   \end{tabular}
\end{theorem}
Similar to  Theorem~\ref{thm:chapter:mcslock:machine-state-refinement},
proving the refinement property for other primitives between two layers 
are possible, and the generalized theorem for the refinement is as follows:

 \begin{theorem}[Machine State Refinement]
 \label{thm:chapter:mcslock:machine-state-refinement-full} 
Assuming that 
 1) ${\mmcsboot}[cid, \oracle^{mcs}_{cid}]$ and ${\mmcslockintro}[cid, \oracle^{mcs}_{cid}]$ are underlay and overlay layers;
2) and, $A_{\mmcsboot}$ and $A_{\mmcslockintro}$ are abstract datum for $L_{\mmcsboot}$ and $L_{\mmcslockintro}$, respectively.
    With the given $R_{(\mmcslockintro, \mmcsboot)}$, defined as $\matchmcslock$
     over the $\lockmemloc$ block in the memory  (\ie,$\lockmemloc\codeinmath{@}mem$) and 
 $\mcsname$ Lock related data structure in the abstract data of the overlay layer (\ie, $\lockabsloc\codeinmath{@}A_{\mmcslockintro}$), 
 identity relation for other parts of $A_{\mmcsboot}$ and $A_{\mmcslockintro}$, 
 The specification for the function $f$, $\sigma_f$, in $L_{\mmcslockintro}$ refines that in $L_{\mmcsboot}$ when:
 \begin{center}
 \begin{tabular}{P{0.95\textwidth}}
$ \forall (\codeinmath{{m}}_{1} \ \codeinmath{{m}}_{1}'\ \codeinmath{{m}}_{0} : mem)\  (\codeinmath{{d}}_{0} \ : A_{\mmcsboot})\ (\codeinmath{{d}}_{1} \ \codeinmath{{d}}_{1}' : A_{\mmcslockintro}). $ \\
 $ (\mmcslockintro[cid, \oracle^{mcs}_{cid}] \vdash \sigma_f : (\_, \codeinmath{m}_1, \codeinmath{d}_1) \rightarrow (\_, \codeinmath{m}_1', \codeinmath{d}_1'))  \ \rightarrow \
  R_{(\mmcslockintro, \mmcsboot)}\ (\codeinmath{{m}}_1, \codeinmath{{d}}_1)\ (\codeinmath{{m}}_0, \codeinmath{{d}}_0) \ \rightarrow $\\
  $ \exists (\codeinmath{{m}}_{0}' : mem)\ (\codeinmath{{d}}_{0}' : A_0).$ \\
$({\mmcsboot}[cid, \oracle^{mcs}_{cid}]  \vdash \sigma_f : (\_, \codeinmath{m}_0, \codeinmath{d}_0) \rightarrow (\_, \codeinmath{m}_0', \codeinmath{d}_0')) \ \wedge \
  R_{(\mmcslockintro, \mmcsboot)}\ (\codeinmath{{m}}'_1, \codeinmath{{d}}'_1)\ (\codeinmath{{m}}'_0, \codeinmath{{d}}'_0).$  \\
 \end{tabular}
 \end{center}
 \end{theorem}



One interesting variation is the semantics
for fetch-and-store and compare-and-swap. These instructions are not
formalized in the x86 assembly semantics we use, so we cannot prove
that replay function is correctly defined. Instead, we modify the last
(``pretty-printing'') phase of the compiler so that these primitive calls map to assembly
instructions, and one has to trust that they match the specification.


\subsection{Event interleaving layer}
\label{chapter:mcslock:subsec:abstractoperationlayer}

After abstracting memory accesses into the operation on the log, we
then need to model possible interleaving among multiple CPUs. In
our approach, this is done through a new layer which adds \emph{context queries}.

To recall what the environmental (concurrent context) is,
the concurrent context $\oracle^{mcs}_{cid}$ (sometimes called the ``oracle'') is
a function that gets  a CPU ID and a log and returns a log.
It has the type\newline
\begin{tabular}{P{0.95\textwidth}}
    $\oracle^{mcs}_{cid}:\  \codeinmath{Z}\ \rightarrow \ \codeinmath{Z}\ \rightarrow\ \codeinmath{list event}\ \rightarrow\ \codeinmath{list event}.$\\
\end{tabular}\newline
when the first argument is a lock ID, the second argument is a CPU ID, and the third is a current log\newfootnote{Similar to a global log, we divide a single concurrent context into multiple ones for optimization; thus the first parameter (lock ID) is necessary for querying the proper concurrent context. We show how this set of concurrent contexts is connected with a single concurrent context which is discussed in Chapter~\ref{chapter:ccal} in the next Chapter (in Chapter~\ref{chapter:linking}.)} 
It is one component of the abstract state in our implementation, and it represents the specific behavior of \emph{all
the other CPUs}, from the perspective of code running on the current
CPU.  Any time a program does an operation which reads or writes
shared memory, it should first query $\oracle^{mcs}_{cid}$ by giving it the
current log. The oracle will reply with a list of events that other
CPUs have generated since then, and we update the log by appending
those new events to it.

Primitive specifications are provided read-only access to a context
$\oracle^{mcs}_{cid}$ by the verification framework, and the framework also
guarantees that two properties are true of $\oracle^{mcs}_{cid}$: 1) the returned
partial log from the oracle query does not contain any events
generated by the given CPU ID; and 2) if we query the oracle with the
well-formed shared log, the updated log after the oracle query will
be well-formed.
The first assumption is straightforward because the purpose of the oracle is to represent the behavior of others' operation on the shared object.
The second one is also trivial when we prove 1) the initial shared log satisfy the well-formed condition, and 2) all the operations on the shared object with the given well-formed log return a well-formed shared log.
Those two assumptions, however, do not reduce the generality of the oracle, and the oracle can capture the proper interleaving that we hope to achieve in the $\mcsname$ Lock verification.

Similar to Section~\ref{chapter:mcslock:subsec:lowestmachinemodel}, 
we provide primitives in $\mmcsbootfull$ which query $\oracle^{mcs}_{cid}$ and extend the log.
Then in this second layer, we can model abstract operations with interleaving.
For example, $\mcssetnext$ can be re-written as
\lstinputlisting [language = Caml] {source_code/mcslock/mcs_set_next_low_charac.c}
by using the logical primitive which corresponds to the oracle query
(The function $\mcslog$ refines the semantics of $\atomicmcslog$ in the lowest layer by the $\matchmcslock$ relation).
To model the interleaving, all  setter and getter functions defined
in Section~\ref{chapter:mcslock:subsec:lowestmachinemodel} should be combined with the
oracle query.

\subsubsection{Trust in the machine model}
Some of design decisions in  memory access
layers have to be trusted, so the division between the machine model and
the implementation is unfortunately slightly blurred.
Ideally, we would have a generic machine model as proposed in Section~\ref{chapter:ccal:sec:interface-calculus}, where memory is partitioned into a thread-local
memory (no events), a lock-protected memory (accesses generate $\push$/$\pull$
events), and atomic memories (each access
generates one $\codeinmath{READ}$/$\codeinmath{WRITE}$/$\codeinmath{SWAP}$/etc event).  However, our starting point
is  $\compcert$ $\intelmachine$ semantics, which was designed for single-threaded
programs, and does not come with a log, so we add a log and memory access
primitives ourselves.
But because the spinlock module is the only code in the OS that uses
atomic memory, we do not add a generic operation called
read\_word etc. Instead, we take a short-cut and specify the particular
6 memory accesses that the lock code uses: $\mcssetnext$ etc.
For these procedures to correctly express the intended semantics,
there are two trusted parts we must take care to get right. First,
each access to non-thread-local memory must generate an event, so we
must not forget the call to
$\mcssetnextlog$.
Second, to account for
interleavings between CPUs (and not accidentally assume that consecutive
operations execute atomically) we must not forget the call to
$\mcslog$ after each access.



\subsection{Low-level functional specification}
\label{chapter:mcslock:subsec:atomicoperation}

Using the primitives that we have defined in lower layers, we prove the correctness of lock acquire, $\mcsacquire$, and release, $\mcsrelease$.
The target code in this layer is identical to the code in Figure~\ref{fig:chapter:mcslock:mcs_lock} except two aspects. 
First, we replaced all operations on memory with the getters and setters described in Section~\ref{chapter:mcslock:subsec:abstractoperationlayer}.
Second, $\mcsacquire$ has one more
 argument, which is a bound number for the client code of the lock.

Since functions defined in
Section~\ref{chapter:mcslock:subsec:abstractoperationlayer} already abstract interleaving
of multiple CPUs, the proofs in this layers work just like sequential
code verification. We find out the machine state after the function
call by applying the C operational semantics to our function
implementation, and check that it is equal to the desired state
defined in our specification.

However, writing specifications for these functions is slightly subtle, 
because they contain
while-loops without any obviously decreasing numbers. Since our
specifications are $\coq$ functions we need to model this by structural
recursion, in some way that later will let us show the loop is terminating.
So to define the semantics of $\mcswaitlockfunc$,
we define an auxiliary function
$\calmcsacqwait$ which describes the
behavior of the first $n$ iterations of the loop: each iteration
queries the the environment context $\oracle^{mcs(lkid)}_{\mathrm{i}}$ (\ie,  $lkid$ is a lock identifier and $\mathrm{i}$ is a CPU ID), replays a log to see if if $\codeinmath{busy}$ is now $\bfalse$ and appends a $\GETBUSY$ event.
If we do not succeed within $n$ iterations the function is undefined ($\coq$ $\None$).
Then, in  part of the  specification for the  acquire lock 
function ($\calmcsacqwait$ in Figure~\ref{fig:chapter:mcslock:calmcsacqwait}) where we need to talk about the while loop,
we say that it loops for some ``sufficiently large'' 
number of iterations $\CalWaitLockTime$  $\codeinmath{tq}$. 
\begin{figure}
\lstinputlisting[language = Caml]{source_code/mcslock/waitlockloop.v}
\caption{$\calmcsacqwait$ Definition.}
\label{fig:chapter:mcslock:calmcsacqwait}
\end{figure}
The function $\CalWaitLockTime$ computes a suitable 
number of loop iterations based on $\codeinmath{tq}$, time-bounds  which each of the queuing CPUs promised to respect.
We will show how it is defined in Section~\ref{chapter:mcslock:sec:liveness-atomicity}. 
However, in \emph{this part} of the proof, the definition doesn't matter. 
Computations where $n$ reaches 0 are considered crashing, and our
ultimate theorem is about safe programs, so when proving that the C
code matches the specification we only need to 
consider cases when $\calmcsacqwait$ returned $\codeinmath{(}\Some\codeinmath{ l)}$.
It is easy to show in a downward simulation that the C loop can match any such finite run, 
since the C loop can run any number of times.

\subsection{Data representation and ghost state}
\label{chapter:mcslock:sec:representation-ghost}

From here on, we never have to think about C programs again.  All the
subsequent reasoning is done on $\coq$ functions manipulating ordinary
$\coq$ data types, such as lists, finite maps, and unbounded integers.
Verifying functional programs written in $\coq$'s Gallina is exactly the
situation $\coq$ was designed to deal with. However, the data computed
by the replay function in in the previous layer still corresponds
exactly to the array-of-structs that represents the state of the lock
in memory.
In particular, the intuitive reason that the algorithm is fair is that
each CPU has to wait in a queue, but this conceptual queue is not identical with
the linked-list in memory, because the next-pointers may not be set.

In order to keep the data-representation and liveness concerns separate,
we introduce an intermediate layer, which keeps the same sequence of operations and same log of events, 
but manipulates an \emph{abstracted data representation}.
We provide a different replay function ($\QSCalLock$) with the type \newline
\begin{tabular}{P{0.95\textwidth}}
$\QSCalLock :\ \lockmultilog\ \rightarrow\ \optiondef\ (\nattype * \nattype * \codeinmath{head\_status} * \codeinmath{list} \ \ztype *  \set{\ztype} * \codeinmath{list} \ \nattype)$\\
\end{tabular}\newline
The tuple returned by this $\QSCalLock$ replay function provides the information we
need to prove liveness, 
similar to the concepts used in the informal
proof in Section~\ref{chapter:mcslock:sec:overview}. 
The meaning of a tuple $\codeinmath{(c1,c2,b,q,slow,t)}$ is as follows:
\begin{itemize}
\item  $\codeinmath{c1}$ and $\codeinmath{c2}$ are upper bounds on how many more operations 
the CPU which currently holds the lock will generate as part of the critical section and of 
releasing the lock, respectively. They are purely logical ghost states but can be deduced from the complete
history of events in the system.

\item $\codeinmath{b}$ is either  $\mcslempty$ or  $\mcslhold$, 
the lock status of the head of the queue.
They are closely related to $\push$/$\pull$ operations in our memory model discussed in Section~\ref{chapter:ccal:sec:interface-calculus}.
When the $\codeinmath{b}$ value is $\mcslhold$, this implies that one CPU already pulls the value for that shared memory block.
On the other hand, when $\codeinmath{b}$ value is $\mcslempty$, no CPUs pull the value from the block, so the next CPU in the wait list 
can pull the block to enter the critical section.

\item $\codeinmath{q}$ is a list of  CPUs who are currently waiting for the lock, 
and $\codeinmath{t}$ is a list of bound numbers that 
correspond to the related elements in $\codeinmath{q}$.

\item $\codeinmath{slow}$ is a finite set which represents a subset of CPUs in $\codeinmath{q}$ that have not yet executed their \emph{set next} operation.  
\end{itemize}
Our liveness proof is based on the fact that each CPU only needs to wait for CPUs that are ahead of it in $\codeinmath{q}$, which is the waiting queue that guarantees FIFO ordering.

Some of the information are implicit in the state of the memory, while some of it (for example $\codeinmath{c1}$ and $\codeinmath{c2}$) are purely ghost states. But in any case, it can be deduced from the complete history of events in the system, which is what the replay function $\QSCalLock$ does. We define it by recursion on the list $\codeinmath{l}$, computing the new state after each event. A few representative cases of the function are shown in Figure~\ref{fig:chapter:mcslock:QS_CalLock}.  For example, the event
$\setbusy$ indicates that a thread releases the lock. If the CPU $i$ is already the  front of the queue $\codeinmath{q}$, it currently holds the lock ($\mcslhold$), and the bound $\codeinmath{c2}$ has not yet reached zero, and $i$ is not slow, then generating this event will reset the lock status to $\mcslempty$ and remove the head element ($i$) from $\codeinmath{q}$ and $\codeinmath{t}$. In any of those side conditions are not satisfied, on the other hand, the replay function is undefined ($\None$). Similar
considerations hold executing memory operations (you must be in the critical section, and it decrements $\codeinmath{c1}$) and querying the busy flag (you must have executed $\SETNEXT$ first).

\begin{figure}
\lstinputlisting [language = Caml, firstline=1] {source_code/mcslock/midlogreplay_short.v}
    \caption{$\QSCalLock$ Definition.}
\label{fig:chapter:mcslock:QS_CalLock}
\end{figure}


\subsubsection{Invariant} 

The replay function plays two different roles. When it returns $\Some\codeinmath{ v}$, for some tuple $\codeinmath{v}$, it describes what the current state of the system is, which lets us write  specifications for those primitives. At the same time, the cases where the function is defined to return $\None$ are also important, because this can be read as a description of events that are \emph{not} possible. For example, from inspecting the program, we know that each CPU will create
exactly one $\SETNEXT$  event before it starts generating $\GETBUSY$ events, and this fact will be needed when doing proofs in the later layers (Section~\ref{chapter:mcslock:sec:liveness-atomicity}). By taking advantage of those  side conditions in the replay function, we can express all  invariants about the log in a single statement, ``the replay function is defined'':\newline
\begin{tabular}{P{0.95\textwidth}}
    $\exists \codeinmath{c1}\ \codeinmath{c2}\ \codeinmath{b}\ \codeinmath{q}\ \codeinmath{s} \codeinmath{t}.\ \QSCalLock\codeinmath{(l)=Some(c1,c2,b,q,s,t)}$\\
\end{tabular}

This type for the replay function is optimized to only expose exactly the information needed by the subsequent liveness proof. 
We need to expose the queue and the set of slow CPUs in order to define the termination measure $\mcsmeasure$
 (Section \ref{chapter:mcslock:sec:liveness-atomicity}). 
 On the other hand, this is not enough information to bridge the gap from the low-level functional specification. 
 In order to show that the memory cells for a valid linked-list and therefore respects the queue
ordering, we need to track exactly what the valid state transitions are. So inside the ghost state layer, 
we also introduce a different relation  $\QCalMCSLock$
which is mostly the same as $\QSCalLock$ but written as an (functional) inductive relation 
in $\coq$ instead of a recursive function, and which has even more preconditions for when it is defined. 
We then add one more condition in the layer invariant saying that $\QCalMCSLock$ and $\QSCalLock$ output the same
result. Most of the proofs inside the ghost layer are done using the relation instead of the function.
For simplicity, we will ignore the distinction in the rest of the paper, 
and write the lemma statements about $\QSCalLock$ even if they used $\QCalMCSLock$ in the actual $\coq$ code.


To show that the ghost layer refines the previous layer, we show a
one-step forward-downward refinement: if the method from the higher
layer returns, then method in the lower layer returns a related
value. For this particular layer the log doesn't change, so the
relation in the refinement is just equality (including the relation for the global log and the environmental context ($\oracle^{mcs}_{cid}$) -- \ie, the relation for strategies in the $\oracle^{mcs}_{cid}$), 
and the programmer just
has to show that  lower-level methods are at least as defined and
that they return equal results for equal arguments.


As we prove this, we need lemmas to show that we can satisfy preconditions for operations in the lower layer, by relating the data in $\codeinmath{la}$ to the abstract queue.  For example, when trying to take the lock, the high level specification checks if the current CPU is at the head of $\codeinmath{q}$, which the low specification tests if the $\codeinmath{busy}$ field is true, so we need Lemma~\ref{lem:chapter:mcslock:Q_CalMCSLock_tail_is_busy} to show that they will follow the same path of code. 


\begin{lemma}[tail is busy]
\label{lem:chapter:mcslock:Q_CalMCSLock_tail_is_busy}

    If $CalMCSLock\ \codeinmath{l=}\Some\codeinmath{(tl,la,tq)}$ and 
    $\QSCalLock\ \codeinmath{l=}\Some\codeinmath{(c1,c2,i::q,s,t)}$ and $j \in\codeinmath{q}$, then $\codeinmath{lock\_array[}j\codeinmath{] = (true, \_)}$.
\end{lemma}

\begin{theorem}[simulation for the ghost layer] 
    Let's assume that the abstract data $\codeinmath{d}$ satisfies the invariant of two layers,
    $\mmcslockop$ and $\mqmcslockop$. When 
$\waitqslockspec\codeinmath{(d)=}\Some\codeinmath{d')}$ then 
$\mcsacquirespec\codeinmath{(d)=}\Some\codeinmath{(d')}$.
\end{theorem}


\subsection{Liveness and atomicity}
\label{chapter:mcslock:sec:liveness-atomicity}

The specification in the previous section is still too low-level and
complex to be usable by client code in the rest of the system.  First,
the specification of the $\mcsacquire$ and
$\mcsrelease$ primitives contain loops, with complicated
bounds on the number of iterations, which clients certainly will not
want to reason directly about.  More importantly, since the
specifications generate multiple events, clients would have to show
all interleavings generate equivalent results.

To solve this we use the \textit{log-lift} design
pattern in Section~\ref{chapter:ccal:subsec:local-layer-interface}; build a new layer with \emph{atomic specifications},
\ie, each primitive is specified to generate  a single event.
For an atomic layer there is a
therefore a one-to-one mapping between events and primitives, and the global log
can be seen as a record of which primitives were invoked in which
order. Thus, the refinement proof which ascribes an atomic
specification proves once and for all that overlapping and interleaved
primitive invocations give correct results.
In this layer, the specifications only use three kinds 
of events: taking the lock ($\waitlock{n}$),
releasing it ($\rellock$), and modifications of the shared
memory that the lock protects ($\mcstshared{\ignorechar}$)



\begin{figure}
\lstinputlisting [language = Caml, firstline=1] {source_code/mcslock/highlogreplay.v}
\lstinputlisting [language = Caml, firstline=1] {source_code/mcslock/hswaitlockspec.v}
\lstinputlisting [language = Caml, firstline=1] {source_code/mcslock/hspasslockspec.v}
\caption{Final Atomic Specification of Aquire/Release Lock Functions.}
\label{fig:chapter:mcslock:hswaitlockspec}
\end{figure}

Figure~\ref{fig:chapter:mcslock:hswaitlockspec} shows  final specifications for 
wait and pass primitives. We show them in full details, with no elisions,
because they are the interfaces that clients use. First, the
specification for the lock acquire function itself
($\mcswaithlockspec$) takes  function arguments
$\codeinmath{bound}$, $\codeinmath{index}$, $\codeinmath{ofs}$, and maps an
abstract state ($\mcsabsdata$) to another. When writing this
specification we chose to use two components in the abstract state, a
log ($\lockmultilogpool$) and also a field ($\mcslockabsfield$) which
records for each numbered lock if it is free ($\mcsockfalse$)
or owned by a CPU ($\mcslockown \ \bfalse$). 
The boolean value in ($\mcslockown\ \_$) indicates whether copied/flushed the shared memory value to/from the local memory for the CPU or not.
The $\bfalse$ value in $\mcswaithlockspec$ implies that the CPU already hold the lock but does not pull the shared memory by using the $\pull$ operation yet. 
On the other hand, the  $\bfalse$ value   in $\mcspasshlockspec$ implies that the $\push$ primitive has been invoked already 
to flush the local change to the shared memory, but the CPU still owns the lock.
The $\codeinmath{lock}$ field is
not very important, because the same information can also be computed
from the log, but exposing it directly to clients is sometimes more
convenient.

The specification returns $\None$ in some
cases, and it is the
responsibility of the client to ensure  that does not
happen. So clients must ensure that: the CPU is in a kernel/host
mode (for the memory accesses to work); an index and an offset (used to
compute the lock id) are in range; the CPU did not already hold the
lock ($\mcsockfalse$); and the log is well-formed
($\HCalLock\ \codeinmath{l'}$ is defined, which will always be the case if
$\HCalLock\ \codeinmath{l}$ is defined).  When all these preconditions are
satisfied, the specification queries the context once, and appends a
single new $\WAITLOCK$ event to the log.
Figure~\ref{fig:chapter:mcslock:hswaitlockspec} also shows the replay function
$\HCalLock$.
It has a much simpler type than $\QSCalLock$in the
previous layer, because we have abstracted the internal state of the lock
to just whether it is free ($\LEMPTY$),
held ($\LHOLD$), and if taken, the CPU id ($\Some\ i$)
of the holder of the lock. Unlike the three bound numbers in the
previous layer, here we omit the numbers for the internal lock
operations and only keep the bound $\codeinmath{self\_c}$ for the number
of events generated during the critical section. Again, it's the
client's responsibility to avoid the cases when $\HCalLock$
returns $\None$. In particular, it is only allowed to release
the lock or to generate memory events if it already holds the lock
($\zeqop{i}{i0}$), and each memory event decrements the counter,
which must not reach zero. The client calling $\waitlockfunc$
specifies the initial value $n$ of the counter, promising to take at
most $n$ actions within the critical section.


In the rest of the section, we show how to prove that the function
does in fact satisfy this high-level atomic specification.
Unlike the previous layers we considered, in this case the log in the
upper layer differs from the one in the lower layer. For example, when
a CPU takes the lock, the log in the upper layer just has the one
atomic event ($\waitlock{n}$), while the log in the underlay
has a flurry of activity (swap the tail pointer, set the next-pointer,
repeatedly query the busy-flag).
Because the log represents shared data, our framework requires a
slightly strengthened refinement theorem for the log-component of the
state. Usually a refinement simulation works by specifying some
relation $R_{(\mhmcslockop, \mqmcslockop)}$ between machine state and abstract state, and then
proving that the state transitions preserve the relation. Indeed, for
thread-local data this is exactly what $\certikos$ does also.

Let's recall the contextual refinement property in Definition~\ref{def:chapter:ccal:contextual-refinement}. 
This relation   $R_{(\mhmcslockop, \mqmcslockop)}$ must includes the relation in between 
two strategies of those two layers are relying on; 
thus the relation for two different environmental contexts, which are $\oracle^{lk}_{cid}$ for $\mhmcslockopfull$ and $\oracle^{mcs}_{cid}$ for $\mqmcslockop$.
For example, suppose one particular
execution of the system generates the log $\codeinmath{l}_{\mqmcslockop}$.  A normal simulation
theorem for a CPU 1 would tell us that there \emph{exists} a log $\codeinmath{l}_{\mhmcslockop}$
that meets CPU 1's local specifications and satisfies the relation
($R_{({\mhmcslockop}, {\mqmcslockop})}$). 
Similarly, the local proof for a CPU 2 would say there
exists a some log $\codeinmath{l'}_{\mhmcslockop}$
related to  $\codeinmath{l}_{\mqmcslockop}$. 
But in order to derive a simulation for the
entire system, we need the constraint that that $\codeinmath{l}_{\mhmcslockop}$  is equal to
$\codeinmath{l}_{\mhmcslockop}$. 
The actual definition of the relation is by defining the proper function $f_{(lk, mcs)}$ between two logs.
In other words, when proving the simulation,
we find a function $f_{(lk, mcs)}$ for the logs, such that $f_{(lk, mcs)}(\codeinmath{l}_{\mqmcslockop}) = \codeinmath{l}_{\mhmcslockop}$.


As for  $\mcsname$ Lock, we define a function $\relatemcslogkwd$ from the
implementation log to the atomic log. Figure~\ref{fig:chapter:mcslock:logsequence}
shows by example what it does. It keeps the shared memory events as
they are, discards the events that are generated while a CPU wait for
the lock, and maps just the event that finally takes or releases the
lock into $\WAITLOCK$ and $\rellock$.

\begin{figure}
\includegraphics[width=\textwidth]{figs/mcslock/logsequence}
\caption{Log Sequence and Log Refinement Example.}
\label{fig:chapter:mcslock:logsequence}
\end{figure}

We then prove a one-step refinement theorem from the atomic specification 
to the implementation, in other words, that if a call to the atomic primitive returns a 
value, then a call to its implementation also returns with a related log:

\begin{theorem}[$\mcsname$ Wait Lock Exist]
  \label{thm:chapter:mcslock:mcs_wait_lock_exist}

  Suppose $\codeinmath{d}_{\mhmcslockop}$ and $\codeinmath{d}_{\mqmcslockop}$, abstract datum for two layers ($\mhmcslockop$ and $\mqmcslockop$, respectively),
   satisfy the layer
  invariants and are related by
   $\relatemcslog{\codeinmath{d}_{\mqmcslockop}}{\codeinmath{d}_{\mhmcslockop}}$ for their global logs and identity relations for others.
If $\mcswaithlockspec(\codeinmath{d}_{\mhmcslockop}) = \Some\ \codeinmath{d'}_{\mhmcslockop}$, then there exists some 
$\codeinmath{d'}_{\mqmcslockop}$
  which is $\waitqslockspec(\codeinmath{d}_{\mqmcslockop}) = \codeinmath{d'}_{\mqmcslockop}$ and is related with $\codeinmath{d'}_{\mhmcslockop}$  by
   $\relatemcslog{\codeinmath{d'}_{\mqmcslockop}}{\codeinmath{d'}_{\mhmcslockop}}$.
\end{theorem}


The proof requires a \emph{fairness assumption}.
A CPU cannot take the lock until the previous CPU releases it, 
and the previous CPU cannot release it if it never gets to run. 
At its most fundamental, the $\certikos$ machine model is a non-deterministic 
transition system (which is subsequently viewed as a log of events), 
and there is nothing in the basic model that ensures fairness, 
so we have to add an extra assumption somewhere. In principle, it would be 
possible to modify the machine model itself, and then pass the fairness assumptions 
along in the specification of each layer until we reach the layers related to mutex locks, 
but in our development, we choose a more expedient solution and express
the fairness assumption as an extra axiom talking about the logs 
in the data representation layer (Section~\ref{chapter:mcslock:sec:representation-ghost}). 
By doing that, our framework can use the previous machine 
model as it is, and can reuse most previous proofs.

Specifically, we assume that there exists some constant $F$ (for ``fairness'') such that no CPU that enters the queue has to wait for more than $F$ events until it runs again. 
In $\coq$ we provide a function, $\CalBound:\ \ztype\ \rightarrow\ \lockmultilog\ \rightarrow \ \nattype$,
which ``counts down'' 
until CPU $i$ gets a chance to 
execute. 
The full definition of $\CalBound$ is in Figure~\ref{fig:chapter:mcslock:calbound-definition}.


\begin{figure}
\begin{center}
\lstinputlisting [language = Caml, firstline=1] {source_code/mcslock/calbound.v}
\end{center}
\caption{$\CalBound$ Definition.}
\label{fig:chapter:mcslock:calbound-definition}
\end{figure}

The fairness assumption, then is that for all logs $\codeinmath{l}$, 
when the low level log replay function returns a 
value ($\QSCalLock\codeinmath{(l)=}\Some\codeinmath{(c1,c2,h,q,s,t)}$) and $j$ is 
in the waiting queue ($j \in \codeinmath{q}$), then we can conclude that $\CalBound\ j\ \codeinmath{l}>0$. 
We then define a natural-number valued termination measure $\mcsmeasure_i\codeinmath{(c1,c2,h,q,s,l)}$.  
This is a bound on how many events the CPU $i$ will
have to wait for in a state represented by the log $\codeinmath{l}$, and where
$\QSCalLock\codeinmath{(l)=}\Some\codeinmath{(c1,c2,h,q++}i\codeinmath{::q}_0\codeinmath{,s,t++n::t}_0)$.
Note that
we partition the waiting queue into two parts $\codeinmath{q}$ 
and $i\codeinmath{::q}_0$, where $\codeinmath{q}$
represents the waiting CPUs that were ahead of $i$ in the queue.
The function $\mcsmeasure$ has two cases that depend on the head status.
\begin{center}
\begin{tabular}{p{0.95\linewidth}}
$\mcsmeasure_i\codeinmath{(c1,c2,}\LEMPTY\codeinmath{,q,s,l)=}\CalBound_{\mathsf{hd}(q)}\codeinmath{(l)+(}K_1\codeinmath{(}\Sigma\codeinmath{t)+}|\codeinmath{q}\cup\codeinmath{s}|\codeinmath{)}\times K_2\codeinmath{)}$\\
$\mcsmeasure_i\codeinmath{(c1,c2,}\LHOLD\codeinmath{,q,s,l)=}\CalBound_{\mathsf{hd}(q)}\codeinmath{(l)+}\BoundValAux\times K_2$ \\
\hfill	 where $\BoundValAux\codeinmath{=(c1+c2+(}\Sigma\codeinmath{(}\mathsf{tl}\codeinmath{(t))}\times K_1\codeinmath{+}|\mathsf{tl}\codeinmath{(q)}\cup\codeinmath{s}|\codeinmath{)}$\\
\end{tabular}
\end{center}

In short, if the lock is not taken, the bound $\mcsmeasure$ is the sum of the
maximum time until the first thread in the queue gets scheduled again
($\CalBound_{\mathsf{hd}(q)}(\codeinmath{l})$), plus a constant times
the sum of the number of operations to be done
by the CPUs ahead of $i$ in the queue ($\Sigma\codeinmath{t}$) 
and the number of CPUs ahead of $i$ which has
yet to execute $\SETNEXT$ operation 
$\codeinmath{(}|\codeinmath{q}\cup\codeinmath{s}|\codeinmath{)}$. If the lock is currently
held, then $\codeinmath{{c1+c2}}$ is a bound of the number of operations it will
do
(and we can ignore the first element of $\codeinmath{q}$ and $\codeinmath{t}$, since they are
accounted for).
The constants and fairness assumption is general enough to handle the cases which takes a slightly longer execution than it is expected to.
The constants ($K_1 = F+5$ and $K_2 = F+4$) are chosen somewhat
arbitrary, and certainly $\mcsmeasure$ is not the tightest possible bound. It
does not need to be, since it does not occur in our final theorem
statement.

The definition of $\mcsmeasure$ is justified by the following two
lemmas. First, we prove that M decreases if CPU $i$ is waiting and some other CPU
$j$ executes an event $\codeinmath{e}_{j}$.

\begin{lemma}[Decreasing measure for other CPUs]
\label{lem:chapter:mcslock:MCS_CalLock_progress_onestep}
Assuming that $\QSCalLock\codeinmath{(l)=}\Some\codeinmath{(c1,c2,h,q}_1\codeinmath{++}i\codeinmath{::q}_2\codeinmath{,s,t}_1\codeinmath{++c::t}_2\codeinmath{)}$, where
$|\codeinmath{q}_1|\codeinmath{=}|\codeinmath{t}_1|$ and $\QSCalLock\codeinmath{(e}_j\codeinmath{::l)=}\Some\codeinmath{(c1',c2',h',q',s',t')} $
for some $j\neq i$ and 
 $\CalBound\codeinmath{(e}_j\codeinmath{::l)}\codeinmath{>}0$.
Then we can split the queue $\codeinmath{q'}$ with satisfying the following properties:
$\codeinmath{q'=q}_{1}'\codeinmath{++}i\codeinmath{::q}_{2}'$ as well as
$\mcsmeasure_i\codeinmath{(c1',c2',h',q}_{1}'\codeinmath{,s',t}_{1}'\codeinmath{,e}_j\codeinmath{::l)}<\mcsmeasure_i\codeinmath{(c1,c2,h,q}_1\codeinmath{,s,t}_1\codeinmath{,l)}$.
\end{lemma}

\begin{proof}
 We consider all possible events
$\codeinmath{e}_j$ which could make $\QSCalLock$ returns $\Some$. If $j$ is not the 
CPU at the head of the queue gets scheduled, it will not be
able to make any progress, so the abstract state of the queue remains the same,
but the counter $\CalBound$ decreases.
Otherwise, the counter $\CalBound$ will reset to the upper bound we assumed on fairness, $F$. 
However, in this case the algorithm will make some progresses that change $\codeinmath{c1}$, $\codeinmath{c2}$, $\codeinmath{q}$, or $\codeinmath{s}$.
For example, a CPU $j$ may execute  $\SETNEXT$ (which decreases the size of
$\codeinmath{s}$), it may enter the critical section (which moves some measure from
the head of $\codeinmath{q}$ to the counters $\codeinmath{c1+c2}$) or it may exit the section
(and that event will decrement $\codeinmath{c2}$).
\end{proof}


The second lemma ensures that the waiting loop will eventually
terminate (The preconditions that $i$ is somewhere in the waiting queue,
and that it has already left the set $\codeinmath{s}$, correspond the set-up
which $\waitlockfunc$ does before it starts looping).

\begin{lemma}[Loop termination]
\label{lem:chapter:mcslock:CalWaitGet_exist'}
Suppose that $\QSCalLock\codeinmath{(l)=}\Some\codeinmath{(c1,c2,h,q}_1\codeinmath{++}i\codeinmath{::q}_2\codeinmath{,s,t}_1\codeinmath{++c::t}_2\codeinmath{)}$, where
$|\codeinmath{q}_1|\codeinmath{=}|\codeinmath{t}_1|$, with $i \not\in\codeinmath{q}_1$ and $i \not\in\codeinmath{s}$ and suppose $\oracle^{mcs}_{cid}$ and $\oracle^{lk}_{cid}$  are valid
contexts (which are matched by the relation based on the function $f_{(l, mcs)}$.) Then, if $k>\mcsmeasure_i\codeinmath{(c1,c2,h,q}_1\codeinmath{,s,t}_1\codeinmath{)}$; thus there exists $\codeinmath{l'}$ such
that $CalWaitGet\codeinmath{(}k\codeinmath{,}i\codeinmath{,l)=}\Some\codeinmath{(l')}$.
\end{lemma}


\begin{proof}
The proof is by induction on $k$, the number of loop iterations. The
most interesting part of the proof is to show that each event
generated by the function will decrease the measure.
As it pulls more event to the log form the context, we appeal to
Lemma~\ref{lem:chapter:mcslock:MCS_CalLock_progress_onestep}, which says that the metric decreases. 
Then, there are two cases in the proof depending on whether $i$ has
arrived at the head of the queue (so $\codeinmath{q=}\nil$) or not. If it has,
$\waitqslockspec$ will generate a $\getbusy{false}$
event and return, so we are good. 
Otherwise, it will generate a $\getbusy{true}$ event, and
start another loop iteration. That event does not change the state of
the lock, but it does decrement the $\CalBound$ on when the head CPU
will get scheduled next, so the measure decreases as required.

The proof (for the termination of the loop in $\waitqslockspec$) also relies on the termination property of the busy-loop in $\passqslockspec$.
That proof, on the contrary, is easier. A CPU holding the lock will set
the next pointer before it does anything else, so we are only waiting
for the CPU at the head of the queue to get scheduled at all.
Now, to prove that the loop in $\mcsacquire$ specification
is defined, we just have to pick the function $\CalWaitLockTime$
so that $\CalWaitLockTime\codeinmath{(t)}$ is greater than $mcsmeasure$ at that
point. The rest of the simulation proof for Theorem~\ref{thm:chapter:mcslock:mcs_wait_lock_exist} is straightforward.
Except the waiting loop, other operations in the wait lock function are deterministic and finite. 
\end{proof}

With Lemma~\ref{lem:chapter:mcslock:CalWaitGet_exist'}, 
the simulation proof between $\mcswaithlockspec$ and $\waitqslockspec$ are straightforward.
\begin{theorem}
There is a one-step simulation from $\mcswaithlockspec$ to
$\waitqslockspec$, with the simulation on logs given by $\relatemcslogkwd$.
\end{theorem}


\subsection{From downwards- to upwards-simulation}
\label{chapter:mcslock:sec:downwards-to-upwards}

When moving from sequential to concurrent programs we must
re-visit some fundamental facts about refinement proofs.  Ultimately,
the correctness theorem we want to prove is ``all behaviors of the
machine satisfy the specification''. If we model the machine and the
specification as two transition systems $\codeinmath{M}$ and $\codeinmath{S}$, then this
corresponds to \emph{upwards simulation}: if $\codeinmath{S} \sim \codeinmath{M}$ and 
$\codeinmath{M} \Longrightarrow^* \codeinmath{M}'$, then $\exists \codeinmath{S}'. \codeinmath{S}' \sim \codeinmath{M}'$ and
 $\codeinmath{S} \Longrightarrow^* \codeinmath{S}'$, and if $\codeinmath{M}$ is stuck then $\codeinmath{S}$ is stuck also.
But directly proving an upwards simulation is difficult. You are given
a long sequence of low-level steps, and have to somehow reconstruct
the high-level steps and high-level ghost state corresponding to
it. One of the insights that made the $\compcert$ project
possible~\cite{Leroy-backend} is that as long as $\codeinmath{M}$ is deterministic
and $\codeinmath{S}$ is not stuck, it suffices to prove a \emph{downward
  simulation}: if $\codeinmath{S} \sim \codeinmath{M}$ and $\codeinmath{S} \Longrightarrow \codeinmath{S}'$, then $\exists
\codeinmath{M}'. \codeinmath{S}' \sim \codeinmath{M}'$ and $\codeinmath{M} \Longrightarrow^* \codeinmath{M}'$. (The assumption that $\codeinmath{S}$
is not stuck is standard, it corresponds to only proving refinement
for ``safe'' clients regarding to the specifications.)

Unfortunately, concurrent programs are \emph{not} deterministic: we
want to prove that every interleaving of operations from
different CPUs in the low-level machine results in correct
behavior. So if we had directly modeled the implementation as a
non-deterministic transition system, then we would have to work
directly with upwards simulations, which would be intractable when
reasoning about the low-level details of C programs.

In our approach, all the non-determinism is isolated to the concurrent
context $\oracle$. Any possible interleavings of  threads can be
modelled by initializing the abstract state with a particular
$\oracle^{mcs}_{cid}$, and the execution proceeds deterministically from
there. Therefore we can still use the $\compcert$ method of first
proving a downward simulation and then concluding the existence of a
upward simulation as a corollary.
The context-formalism is also helpful because $\oracle^{mcs}_{cid}$ contains
the entire execution of all other threads, both past and future, so we
have enough information to directly prove a \emph{forward}
simulation. Otherwise it may not be clear if a given low-level
operation can really ``commit'' (and generate a high-level event)
until we see what the other cores do, so proofs about fine-grained
concurrency can require a difficult backwards-simulation
from the end-state of the program.~\cite{DGLMQueue}

There is still an obligation to show that for every $\oracle^{mcs}_{cid}$, there
in fact exists an $\oracle^{lk}_{cid}$ with the right
properties. (Specifically, it should the always output logs which
respect the program invariants, \ie, the replay function is defined,
and also it should respect the refinement relation $f_{(lk, mcs)}$.) But this can
be managed by the framework in a generic way, which we discussed in Chapter~\ref{chapter:ccal}. When
verifying a particular layer, the programmer only needs to define $f_{(lk, mcs)}$.



\section{Evaluation}
\label{chapter:mcslock:sec:evaluation}

\begin{figure}
\begin{minipage}{\linewidth}
\noindent
\begin{multicols}{2}
\lstinputlisting[numbers = left, language = C]{source_code/mcslock/palloc_example.c}
\lstinputlisting[numbers = left]{source_code/mcslock/sharedeventtype.v}
\end{multicols}
\end{minipage}
\caption{$\pallocfunc$ Example.}
\label{fig:chapter:mcslock:palloc-example}
\end{figure}

\subsection{Clients}

The verified $\mcsname$ lock code is used by multiple clients in the $\certikos$ (in Chapter~\ref{chapter:certikos})
system. To be practical the design should require as little extra work
as possible compared to verifying non-concurrent programs, both to
better match the programmer's mental model, and to allow code-reuse
from the earlier, single-processor version of $\certikos$.

For this reason, we don't want our machine model to generate an event
for every single memory access to shared memory. Instead we use what
we call a \emph{push/pull memory model} in Section~\ref{chapter:ccal:sec:interface-calculus}.
To recall it,
 A CPU that wants to access shared memory first generates a ``$\pull$''
event, which declares that that CPU now owns a particular block of
memory. After it is done it generates a ``$\push$'' event, which
publishes the CPU's local view of memory to the rest of the system. In
this way, individual memory reads and writes are treated by the same
standard operational semantics as in sequential programs, but the
state of the shared memory can still be replayed from the log.  The
$\push$/$\pull$ operations are logical (generate no machine code) but
because the replay function is undefined if two different CPUs try to
pull at the same time, they force the programmer to prove that
programs are well-synchronized and race-free. Like we did for atomic
memory operations, we extend the machine model at the lowest layer by
adding logical primitives, e.g. $\releaseshared$ which takes a
memory block identifier as an argument and adds a
$\mcsomeme{\codeinmath{(l:list Integers.Byte.int)}}$ event to the log, where the byte list is a
copy of the contents of the shared memory block when the primitive was
called.

When we use $\acquirereleaselock$ we
need a lock to make sure that only one CPU pulls, so we begin
by defining combined functions $\acquirelockfunc$ which
takes the lock (with a bound of 10) and then pulls, and
$\releaselockfunc$ which pushes and then releases the
lock. The specification is similar to $\mcspasshlockspec$,
except it appends \emph{two} events.

Similar to Section~\ref{chapter:mcslock:sec:liveness-atomicity}, logs for different
layers can use different types of pull and push events.
Figure~\ref{fig:chapter:mcslock:palloc-example} (right) shows the events for the
$\pallocfunc$ function (which uses a lock to protect the page
allocation table). The lowest layer in the palloc-verification adds
$\mcsOMEME$ events, while higher layers instead add
($\mcsoateev{(a:\ \mcsalloctabletype)}$) events, where the relation between logs
uses the same relation as between raw memory and abstract
$\mcsalloctabletype$ data. Therefore, we write wrapper functions
$\acquirereleaselockATspec$, where the
implementation just calls $\acquirereleaselock$
with the particular memory block that contains the allocation table,
but the specification adds an $\mcsOATEev$ event.
This refinement step, which changes the log replay function to compute
allocation tables instead of byte lists, is
specific to the $\pallocfunc$ function.

\begin{figure}

\lstinputlisting{source_code/mcslock/release_lock_AT_spec_short.v}
\lstinputlisting{source_code/mcslock/palloc_spec_short.v}

\caption{$\pallocfunc$ Specification.}
\label{fig:chapter:mcslock:palloc-spec}
\end{figure}

We can then ascribe a low-level functional specification
$\pallocpspec$ to the $\pallocfunc$ function. As shown
in Figure~\ref{fig:chapter:mcslock:palloc-spec}, this is decomposed into three parts, the
acquire/release lock, and the specification for the critical
section. The critical section spec is exactly the same in a sequential
program: it does not modify the log, but instead only affects the $\mcsalloctable$ field in the abstract data.

Then in a final, pure refinement step, we ascribe a high-level atomic
specification $\lpallocspec$ to the $\pallocfunc$
function. In this layer we no longer have any lock-related events at
all, a call to $\pallocfunc$ appends a single
$\mcsopalloce$ event to the log. This is when we see the
proof obligations related to liveness of the locks.
Specifically, in order to prove the downwards refinement, we need to
show that the call to $\pallocpspec$ doesn't return
$\None$, so we need to show that $\HCalLock\codeinmath{ l'}$ is
defined, so in particular the bound counter must not hit zero.
By expanding out the definitions, we see that
$\pallocpspec$ takes a log $\codeinmath{l}$ as its initial global log
and generates the result,
$\rellock\codeinmath{::}\mcsoateev{(\mcsalloctable\ adt)}\codeinmath{::} \mcstshared{\mcsopullev}\codeinmath{::}\waitlock{10}\codeinmath{::l}.$
The initial bound is 10, and there are two shared memory events, so the
count never goes lower than 8. If a function modified more than one
memory block there would be additional push- and pull-events, which
could be handled by a larger initial bound.

Like all kernel-mode primitives in $\certikos$, the $\pallocfunc$ function is
total: if its preconditions are satisfied it always returns. So
when verifying it, we show that all loops inside the critical section
terminate. Through the machinery of bound numbers, this guarantee is
propagated to the the while-loops inside the lock implementation:
because all functions terminate, they can know that other CPUs will
make progress and add more events to the log, and because of the
bound number, they cannot add push/pull events forever. On the other
hand, the framework completely abstract away how long time (in microseconds) elapses
between any two events in the log.

\subsection{Code reuse} 
The same
$\acquirereleaselock$ specifications can be
used for all clients of the lock. The only proofs that need to be done
for a given client is the refinement into abstracted primitives like
$\releaselockATspec$ (easy if we already have a sequential
proof for the critical section), and the refinement proof for the
atomic primitive like $\lpallocspec$ (which is very
short). We never need to duplicate the thousands of lines of proof
related to the lock algorithm itself.

\subsection{Using more than one lock}

The layers approach is particularly nice when verifying code that uses more than one
lock. To avoid deadlock, all functions must acquire the locks in the
same order, and to prove the correctness the ordering must be
included in the program invariant. We \emph{could} do such a
verification in a single layer, by having a single log with different
events for the two locks, with the replay function being undefined if
the events are out of order. But the layers approach provides a
better way. Once we have ascribed an atomic specification to
$\pallocfunc$, as above, all higher layers can use it
freely without even knowing that the $\pallocfunc$ implementation
involves a lock (Note that the lock is not re-exported from the
$\pallocfunc$ layer, and if it was the proof of the atomic
specification would not go through.)  For example, some function in a
higher layer could acquire a lock, allocate a page, and release the
lock; in such an example the the order of the layers provides an order
on the locks implicitly.

\subsection{Proof Effort}


As an evaluation, we do not count the total lines of code in $\coq$ for our entire 
$\mcsname$ Lock module due to the two following reasons. First, our $\mcsname$ Lock implementation 
is a part of $\certikos$ (in Chapter~\ref{chapter:certikos}). Therefore, our $\mcsname$ Lock module also contains several definitions 
and proofs that are totally irrelevant to $\mcsname$ Lock verification. 
This implies that counting the total lines of code for $\mcsname$ Lock module has a 
high possibility of misinterpretation due to the lines of code for those definitions and proofs.
Second, we intensively use contextual refinement approach to 
build the whole system rather than focusing on verifying the correctness and 
liveness of $\mcsname$ Lock. Therefore, our proof efforts are mainly focus on proving 
$\mcsname$ Lock that is able to be easily combined with multiple client codes 
rather than the efficient lock verification itself.  

Among the whole proofs, the most challenging parts are the proofs for starvation 
freedom theorems like Theorem~\ref{thm:chapter:mcslock:mcs_wait_lock_exist}, 
and the functional correctness proofs for $\mcsacquire$ 
and $\mcsrelease$ functions
in Section~\ref{chapter:mcslock:subsec:atomicoperation}.
The total lines of codes for starvation freedom is 2.5K lines, 0.6K lines for specifications, 
and 1.9k lines for proofs. This is because of the subtlety of those proofs. 
To prove the starvation freedom theorems and show the evidence of loop termination,
lots of lemmas are required to express
state changes by replaying the log. 
When $\QSCalLock\codeinmath{(l)=}\Some\codeinmath{(c1,c2,b,q,s,t)}$
and $\codeinmath{q=}\nil$, 
for instance, the mechanized proof for $\codeinmath{s=}\emptyset$ 
and $\codeinmath{t=}\nil$ is necessary. It looks trivial in the hand-written proofs, 
but requires multiple lines of codes in the mechanized proof. 

The total lines of codes for the low-level functional correctness
of two main C functions, $\mcsacquire$ and $\mcsrelease$, are 3.2K lines,  
0.7K lines for specifications, and 2.5K lines for proofs.
It is much bigger than other code correctness proofs for while-loops in $\certikos$, which we will
discuss in Chapter~\ref{chapter:certikos},
because these loops do not have any explicit decreasing value.
One another big part in our $\mcsname$ Lock proofs is the proofs for 
Theorem~\ref{thm:chapter:mcslock:machine-state-refinement} and the lines of code for this part is 
approximately 5K lines. The log replay function ($\calmcslock$) always 
return the whole $\mcsname$ Lock values ($\lockabsloc$) related 
to the  $\codeinmath{mcs\_lock}$ structure defined in Figure~\ref{fig:chapter:mcslock:mcs_lock}. 
In this sense, we always have to give the exact values for all memory 
chunks and prove the correspondence between the memory and the abstract 
data even the event associated with reading values (\eg, $\getnext$).
Hence, those proofs contain a lot of duplicate proofs for the memory access. 
However, they are quite straightforward and easy to produce. 
On top of that, we strongly believe 
that they can be easily reduced by introducing mechanized user-defined tactics later. 

As can be seen from these line counts, proofs about concurrent programs
have a huge ratio of lines of proof to lines of C code.
If we tried to directly verify shared objects that use locks to 
perform more complex operations, like thread scheduling
and inter-process communication, a monolithic proof  
would become much bigger than the current one, and would be quite
unmanageable. The modular lock specification is essential here.

By contrast, the proofs for them in $\certikos$ are quite tractable, 
because the proofs for the locks are modular, re-usable, and can 
be combined with other client-part proofs like we have briefly 
mentioned earlier in this Section.
Therefore, we believe that our approach is a promising way to 
show the correctness of large systems that use shared objects with mutex protection. 


\section{Related Work}
\label{sec:related}

Dijkstra~\cite{dijkstra68a,Dijkstra72} proposed to ``realize'' a
complex program by decomposing it into a hierarchy of linearly ordered
abstract machines.  Based on this idea, the PSOS team at
SRI~\cite{psos80} developed the Hierarchical Development Methodology
(HDM) and applied it to design and specify an OS using 20
hierarchically organized modules. HDM was later also used for the KSOS
system~\cite{ksos84}.
\citet{dscal15} developed new languages and tools for building
certified abstraction layers with {\em deep} specifications, and
showed how to apply the layered methodology to construct fully
certified (sequential) OS kernels in Coq.

\citet{costanzo16} showed how to prove sophisticated global properties
(e.g., information-flow security) over a deep specification of a
certified OS kernel and then transfer these properties from the
specification level to its correct assembly-level implementation.
\citet{chen16} extended the layer methodology to build certified
kernels and device drivers running on multiple {\em logical}
CPUs. They treat the driver stack for each device as if it were
running on a logical CPU dedicated to that device. Logical CPUs do not
share any memory, and are all eventually mapped onto a single physical
CPU.
%%%%
None of these systems, however, can support shared-memory concurrency
with fine-grained locking.

\ignore{Our new \CTOS\ framework adds several
significant novelties (e.g., new models and refinement proofs for
concurrent layer machines, new layer design with environment context),
but it still connects back to the previous
work~\cite{dscal15,chen16,costanzo16} really nicely.  A concurrent
layer with a specific environment context can be composed freely just
as sequential layers~\cite{dscal15}.  The invariants over the
environment contexts (i.e., the ``rely'' conditions) are used to
guarantee that per-CPU or per-thread reasoning can be soundly composed
(when their ``rely'' conditions are compatible with each other).
}

The seL4 team~\cite{klein2009sel4,klein14} was the first to verify the
functional correctness and security properties of a high-performance
L4-family microkernel. The seL4 microkernel, however, does not support
multicore concurrency with fine-grained locking.  \citet{peters15}
and \citet{vontessin13} argued that for an seL4-like microkernel,
concurrent data accesses across multiple CPUs can be reduced to a
minimum, so a single {\em big kernel lock (BKL)} might be good enough
for achieving good performance on multicore machines.
\citet{vontessin13} further showed how to convert the single-core seL4
proofs into proofs for a BKL-based clustered multikernel.

\ignore{
One high-level difference between seL4 and \CTOS\ is that the seL4
team~\cite{klein14} focused on verifying a particular microkernel. The
designers of the L4-family kernels~\cite{liedtke95,heiser13} advocated
the {\em minimality principle}: a concept is tolerated inside the
microkernel only if moving it outside the kernel would prevent the
implementation of the system's required functionality.  This is a
reasonable principle but its interpretation of the ``kernel-user''
boundary (as the hardware-enforced ``red-line'') is quite narrow.  Our
new \CTOS\ architecture advocates replacing the traditional ``red
line'' with a large number of certified abstraction layers enforced by
formal specification and proofs; hardware mechanism (such as address
protection) is just one (quick) way of ensuring that a specific
process will not violate the invariants required by a particular
kernel abstraction layer.
}

The Verisoft team~\cite{verisoft07,leinenbach09,alkassar10} applied
the VCC framework~\cite{vcc09} to formally verify Hyper-V, which is a
widely deployed multiprocessor hypervisor by Microsoft consisting of
100 kLOC of concurrent C code and 5 kLOC of assembly. However, only
20\% of the code is verified~\cite{vcc09}; it is also only verified
for function contracts and type invariants, not the full functional
correctness property.  There is a large body of other
work~\cite{bevier89,hawblitzel10,ironclad14,fscq15,ironfleet15,verdi15,cogent16,uberspark16}
showing how to build verified OS kernels, hypervisors, file systems, device
drivers, and distributed systems, but they do not address the issues
on concurrency.

\citet{xu16} developed a new verification framework by combining
rely-guarantee-based simulation~\cite{RGSim} with Feng~{et~al.}'s
program logic for reasoning about interrupts~\cite{feng08:aim}.
They have successfully verified key modules in the $\mu$C/OS-II
kernel~\cite{ucosii}. Their work supports preemption but only on a
single-core machine. They have not verified any assembly code nor
connected their verified C-like source programs to any certified
compiler so there is no end-to-end theorem about the entire
kernel. They have not proved any progress properties so even their
verified kernel modules or interrupt handlers could still diverge.


\ignore{
\citet{lili16} presented the first program logic (Lili) that can apply
contextual refinement techniques to prove both linearizability and a
progress property for various concurrent objects (including ticket
locks and MCS locks). Their assertion language does not allow
assertions on event traces, so temporal invariants must be described
using special predicates (called {\em definite actions}).  Our new
\CTOS\ framework, on the other hand, directly reasons about the
environment contexts so temporal properties can be expressed uniformly
as other invariants. The Lili language also only supports the
high-level parallel composition construct, so it is unclear how their
logic can be used to verify the yield/sleep/wakeup primitives in \mCTOS.
}


\ignore{

Bevier~\cite{bevier89} developed a full correctness proof
for a highly idealized kernel in an automated theorem prover. The
Verisoft team~\cite{verisoft07} has done a large body of work aiming
to verify OS kernels and
hypervisors~\cite{leinenbach09,alkassar10}. The Verve
project~\cite{hawblitzel10} managed to prove the type safety of an
entire kernel by combining the partial correctness proof of a nucleus
and the type-safety guarantee from a certifying C\# compiler (for the
rest of the kernel); by using powerful automated proving tools (\eg,
Boogie and Z3), Verve managed to certify the nucleus in 9
person-months.


Hawblitzel~{\em et al}~\cite{ironclad14} has recently developed a set
of new tools based on the Dafny verifier~\cite{dafny10} and Z3 SMT
solver~\cite{moura08}, and applied them to build their Ironclad system
which includes a verified kernel (based on Verve~\cite{hawblitzel10}),
verified drivers, verified system and crypto libraries, and several
applications.  This is another impressive effort that advances the
frontier of system software verification. Ironclad, however, only
proves the partial correctness property (at the assembly level), which
is weaker than the total correctness properties proved by seL4 and
\CTOS. All properties proved by Ironclad are not ``contextual'' so it
is unclear how properties proved on Ironclad apps would still hold
when new extensions are added into their system. Ironclad also differs
from seL4 and \CTOS\ in that its proofs are all done by an SMT solver
which does not produce any machine-checkable proof objects.


\citet{filipovic10} showed that proving linearizability for
concurrent objects is precisely equivalent to proving
termination-insensitive contextual refinement for a simple
object-based concurrent language. \citet{liang13,li} extended this
result


\paragraph*{Comparison with seL4}
As mentioned in Section~\ref{sec:intro}, the seL4 team only proved the
{\em refinement} property but not the {\em contextual refinement}
property, so the global properties (\eg,
security~\cite{murray13,sewell11}) proved at the abstract
specification level cannot be transferred to the C-implementation
level.\david{they do transfer security to the C level; need to reword this} 
The root cause of this problem is their rather simplistic
C-level state machine which they used to verify their 7500 lines of C
code. This machine is too high level to model
several key OS features (e.g, kernel initialization,
context switches, address translation, and page-fault
handling). Indeed, these features happen to coincide with the
unverified C and assembly code in their kernel.

Sewell {\em et al.}~\cite{sewell13} used translation validation to
build a refinement proof between the semantics of the verified C
source code and the corresponding binary (compiled by GCC).  This
proof is not as high quality as the rest of the seL4 effort because
it was not done in a proof assistant (thus it has no machine-checkable
proof) and the translation validator itself still has not been
verified.

Even with this work by Sewell {\em et al.}~\cite{sewell13}, the
previously unverified C code (1200 lines) and assembly code (600
lines) in seL4 still remain unverified. These are actually quite {\em
  major} assumptions for a verified kernel because they include the
correctness of context switches, kernel initialization, address
translation, and linking between verified C and assembly; all of which
were considered as major challenge problems by many researchers
working in this
field~\cite{verisoft06,ni07,feng08:vstte,BedrockPLDI11,vaynberg12}.

Using \CTOS, we have successfully tackled all of these challenges:
context switches, kernel initialization, address translation, and page
fault handling are all certified. All kernel components (in C and
assembly) are correctly linked together to form a complete system in
an assembly machine and all our proofs are machine-checkable in Coq.

Much of the implementation complexity of the seL4 kernel lies on its
support of capability-based access control. Capabilities are important
in seL4 as they are used to prevent unwanted interference between
different kernel components. However, they significantly increase the
complexity of the seL4 kernel.  In contrast, the \CTOS-family kernels
we have built so far rely on the CompCert memory
model~\cite{leroy12} to enforce isolation and prove contextual
refinement.



Vaynberg {\em et al.}~\cite{vaynberg12} also advocated a layered approach
and used it to verify a small virtual memory manager. Their layers
are not linearly ordered; instead, their seven abstract machines
form a DAG with potential upcalls (i.e., calls from a lower layer to
upper ones). As a result, their initialization function (an upcall)
was much harder to verify. Their refinement proofs between layers are
insensitive to termination, from which they can only prove partial
correctness but not the strong contextual refinement property which we
prove in our current work.

}





\chapter{Multicore and Multithreaded Linking}
\label{chapter:linking}


\chapter{Case Study: Concurrent CertiKOS}
\label{chapter:concurrent-certikos}



\chapter{Case Study: Multi-registred Paxos}
\label{chapter:wormspace}


\chapter{Beyond Functional Correctness of Distributed Systems}
\label{chapter:witness-passing}


\chapter{Related Work}
\label{chapter:related}

\chapter{Limitations, Future Work, and Conclusions}
\label{chapter:conclusion}

\backmatter

\bibliography{refs}
% for your own sake, use a bibtex file, so all of the numbering of references will be done
% automatically.

\end{document}
