\documentclass[letterpaper,11pt]{yalephd}
% remove draft option for final printing.
% font size must be between 10pt-12pt.

\usepackage{geometry} % you need this for yalephd.cls to work.
\usepackage{dcolumn}
\usepackage{amsfonts}
\usepackage{appendix}
\usepackage{makecell}

\usepackage[numbers,square]{natbib}
\usepackage{times}
\usepackage{courier}
\usepackage[scaled]{helvet}
\usepackage{url}
\usepackage[utf8]{inputenc} %for utf8 input
\usepackage[T1]{fontenc} %for accented characters
\usepackage{microtype} %better micro typing
\usepackage{datetime}
\usepackage{amssymb} %for shift symbol
\usepackage{amsmath}
\usepackage{mathrsfs} %for mathscr font
\usepackage{listings, multicol, lstcoq} %for code
\usepackage{multirow}
\usepackage{enumitem}      % adjust spacing in enums
\usepackage{stmaryrd} %for llbracket
\usepackage{mathabx} % for boxes
\usepackage{graphicx} %to include png images
\usepackage{subfig}
\usepackage[usenames,dvipsnames]{color} 
\usepackage{comment}
\usepackage{bussproofs} %for proof trees
\usepackage{flushend}
\usepackage[colorlinks=false,allcolors=black,breaklinks,draft=false]{hyperref} 
\newcommand{\doi}[1]{doi:~\href{http://dx.doi.org/#1}{\Hurl{#1}}} 
\usepackage{mathptmx}      % use Times fonts if available on your TeX system
\usepackage{latexsym}
\usepackage{morefloats}
\usepackage{tabu}
\usepackage{amsthm}
\usepackage[normalem]{ulem} % for strike out
\usepackage{tcolorbox}
\usepackage{tikz}
\usepackage{flushend}

\usetikzlibrary{arrows.meta, decorations.pathmorphing}

\usepackage{prettyref}
%----------------------------------------------------

\usepackage{mathpartir}
\usepackage[scaled=0.85]{DejaVuSansMono}
\definecolor{mygreen}{rgb}{0,0.6,0}
\definecolor{mygray}{rgb}{0.5,0.5,0.5}
\definecolor{mymauve}{rgb}{0.58,0,0.82}
\definecolor{ltblue}{rgb}{0,0.4,0.4}
\definecolor{dkblue}{rgb}{0,0.1,0.6}
\definecolor{dkgreen}{rgb}{0,0.35,0}
\definecolor{dkviolet}{rgb}{0.3,0,0.5}
\definecolor{dkred}{rgb}{0.5,0,0}
%----------------------------------------------------

\lstset{ %
	backgroundcolor=\color{white},   % choose the background color; you must add \usepackage{color} or \usepackage{xcolor}
	basicstyle=\ttfamily,      % the size of the fonts that are used for the code
	breakatwhitespace=false,         % sets if automatic breaks should only happen at whitespace
	breaklines=true,                 % sets automatic line breaking
	%captionpos=b,                   % sets the caption-position to bottom
	commentstyle=\color{mygreen},    % comment style
	deletekeywords={...},            % if you want to delete keywords from the given language
	escapeinside={<@}{@>},          % if you want to add LaTeX within your code
	extendedchars=true,              % lets you use non-ASCII characters; for 8-bits encodings only, does not work with UTF-8
	frame=none,	                   % adds a frame around the code
	keepspaces=true,                 % keeps spaces in text, useful for keeping indentation of code (possibly needs columns=flexible)
	keywordstyle=\color{blue},       % keyword style
	language=Octave,                 % the language of the code
	otherkeywords={*, Inductive, Fixpoint, Record, typedef, struct, Function , Definition, Prop, then, true, false, ...},           % if you want to add more keywords to the set
	emph = {in, uint, match, end, with, let, ret, do, forall, exist, :=, =>, ->},
	emphstyle=\bf,%
	moredelim=[is][emphstyle]{|>}{<|},%
	numbers=left,                    % where to put the line-numbers; possible values are (none, left, right)
	numbersep=5pt,                   % how far the line-numbers are from the code
	numberstyle=\scriptsize\color{mygray}, % the style that is used for the line-numbers
	rulecolor=\color{black},         % if not set, the frame-color may be changed on line-breaks within not-black text (e.g. comments (green here))
	showspaces=false,                % show spaces everywhere adding particular underscores; it overrides 'showstringspaces'
	showstringspaces=false,          % underline spaces within strings only
	showtabs=false,                  % show tabs within strings adding particular underscores
	stepnumber=1,                    % the step between two line-numbers. If it's 1, each line will be numbered
	stringstyle=\color{mymauve},     % string literal style
	tabsize=2,	                   % sets default tabsize to 2 spaces
	%title=\lstname                   % show the filename of files included with \lstinputlisting; also try caption instead of title
}

%% Anonymization
\newif \ifanonymized \anonymizedfalse

\ifanonymized
\newcommand\CTOS{CTOS}
\else
\newcommand\CTOS{CertiKOS}
\fi
%% END Anonymization

\ifTRthen
\newcommand{\ifTR}[2]{#1}
\else
\newcommand{\ifTR}[2]{#2}
\fi

\newcommand\raw{x86}
\newcommand\sys{ker}
\newcommand\layer[3]{(#1,#2,#3)}


\newcommand\sem[2]{[\![#2]\!]_{#1}}
\newcommand\join{\!\bowtie\!}
\newcommand\Refrel{\sqsubseteq}
\newcommand\ClightX[1]{\textrm{ClightX(}#1\textrm{)}}
\newcommand\CompCertX[1]{\textrm{CompCertX(}#1\textrm{)}}
\usepackage{multirow}
\newtheorem{invariant}{Invariant}
\newcommand\ignore[1]{}

%% COMMENTS 
\newif \ifcomments \commentstrue

\ifcomments
\newcommand{\zhong}[1]{\textbf{\textcolor{red}{[ #1 --Zhong]}}}
\newcommand{\jiyong}[1]{\textbf{\textcolor{blue}{[ #1 -- Ji-Yong]}}}
\newcommand{\wolf}[1]{\textbf{\textcolor{green}{[ #1 -- Wolf]]}}}
\newcommand{\lucas}[1]{\textbf{\textcolor{cyan}{[ #1 -- Lucas]}}}
\newcommand{\jieung}[1]{\textbf{\textcolor{violet}{[ #1 -- Jieung]}}}
\else
\newcommand{\zhong}[1]{}
\newcommand{\jiyong}[1]{}
\newcommand{\wolf}[1]{}
\newcommand{\lucas}[1]{}
\newcommand{\jieung}[1]{}
\fi
%% END COMMENTS

\newif \iftopics \topicsfalse
\iftopics
\newcommand{\topic}[1]{{\textcolor{gray}{[#1]\\}}}
\else
\newcommand{\topic}[1]{}
\fi

% keyword
\newcommand{\coq}{Coq}
\newcommand{\needcite}{\textbf{\textcolor{red}{[cite]}}}
\newcommand{\citethese}[1]{\textbf{\textcolor{red}{[cite:#1]}}}
\newcommand{\checkterms}[1]{\textbf{\textcolor{blue}{#1}}}
\newcommand{\globalstate}{global model}
\newcommand{\Globalstate}{Global model}

% Orders on layers and behaviors: \lsim and \lle correspond to QuiverSim,
% whereas \lpath corresponds to CategorySim.
\newcommand{\lle}{\preceq}
\newcommand{\lsim}[1]{\prec_{#1}}
\newcommand{\lpath}[1]{\le_{#1}}
%\newcommand{\lhtap}[1]{\ge_{#1}}
\newcommand{\id}{\textbf{id}}

% Typing judgements
\usepackage{relsize}
\newcommand{\ltyp}[4]{#1 \vdash_{#2} #3 : #4}
\newcommand{\lltyp}[5]{#2 \vdash^{#1}_{#3} #4 : #5}
\newcommand{\rulename}[1]{\textsc{\smaller{}#1}}

% Useful shorthands used in layer.tex
\newcommand{\pcom}{\parallel}
\usepackage{scalerel}
\DeclareMathOperator*{\bigJoin}{\scalerel*{\Join}{\sum}}
\newcommand{\kw}[1]{{\mathsf{#1}}} % render as a keyword
\newcommand{\pset}[1]{{\mathcal{P}({#1})}}
\newcommand{\refines}{\sqsubseteq} % refinement relation


\newcommand{\eg}{{e.g.\xspace}}
\newcommand{\ie}{{i.e.\xspace}}
%\newcommand{\cf}{{\em cf. }}
\newcommand{\cf}{see }
\newcommand{\etal}{{\em et al.\xspace}}
\newcommand\step[1]{\overset{#1}{\rightarrow}}


\newcommand*{\LargerCdot}{\raisebox{0.25ex}{\scalebox{0.7}{$\bullet$}}}
\newcommand\cons{\LargerCdot}
\newcommand\comm[1]{\mathsf{\textcolor{black}{#1}}}
\newcommand\commb[1]{\mathsf{#1}}
%\newcommand\mtext[1]{\mathit{#1}}
\newcommand\set[1]{\{#1\}}
\newcommand\any{\cdot}
\newcommand\nonev{\_}
\newcommand\integer{\mathbb{N}}

% The specification #1 takes a step from (#2, #3) to (#4, #5)
% The second case is for when there's no return value
\newcommand{\sstepr}[5]{#1(#2, #3) \ni (#4, #5)}
\newcommand{\sstep}[4]{#1(#2, #3) \ni (#4)}
\newcommand{\para}[1]{\ifTR{}{\vspace{-4pt}}\paragraph{\textbf{#1}}}
\newcommand{\modulef}[1]{#1 \mapsto \kappa_{#1}}

\newcommand{\remove}[1]{\textcolor{red}{\sout{#1}}}
\newcommand{\added}[1]{\textcolor{blue}{#1}}

\hyphenation{Comp-Cert}
  % all the local macros used in the paper

\bibliographystyle{abbrvunsrt}

\begin{document}

% Need to define title before the abstract.
\title{Modular and Compositional Development of Certified Concurrent Software Systems}
\author{Jieung Kim}
\advisor{Zhong Shao}
\date{March, 2019} 

\frontmatter

\begin{abstract}
% ADD CONTENTS
% Distributed systems are notoriously complex due to the many possible interleavings of their coarsely-connected instances as well as the possibility of errors in both  those instances and the network environment. For these reasons, verification of distributed systems is desirable to remove the possibility of bugs and guarantee their safety and correctness. However, much current verification work still requires a great deal of effort and sometimes has limitations.

We present a verification approach that uses \textit{write-witness-passing}, which is simple but novel in distributed system verification. It is a scalable, reusable, and extensible approach that can be directly linked with the low-level implementations of distributed protocols through contextual refinement. Write-witness-passing can capture the common behaviors of many distributed protocols, and provides both a simple way of understanding the protocols as well as an easy methodology for verifying them.

To demonstrate how write-witnesses work, we verify the functional correctness and safety of Paxos, one of the most famous consensus protocols. We implement the key routines of Paxos in C, and use Coq to verify both the functional correctness of the implementation as well as the safety properties of the protocol within less than 4 person-months. We also describe how we can apply our approach to other distributed protocols to illustrate its generality.

% END CONTENTS
\end{abstract}


\maketitle
\makecopyright{2019} % change as needed.

\chapter{Acknowledgements} % this needs to be before \mainmatter.
\thispagestyle{empty}


\clearpage


\tableofcontents
\listoffigures % remove this if you have no figures.
\listoftables % remove this if you have no tables.


% Starts proper arabic numbering of pages and chapters.
\mainmatter

\chapter{Introduction}
\label{chapter:introduction}
% ADD CONTENTS
% The goal of this dissertation is to build formal methods for concurrent program verification and apply those techniques to 
multiple examples so we can guarantee to users that these systems are reliable and trustworthy not only in terms of functional correctness 
but also other high-level progress properties or the protocol correctness of them.


\section{Challenges in Concurrent Program Verification}
\label{chapter:introduction:sec:challenges-in-concurrent-program-verification}




The prevalence of concurrent environments brings enormous changes to the software. 
They allow for achieving higher performance or more functionality in a single software by using interactions among several instances 
(\ie, multiple threads, nodes, I/O devices, and networks) in the system, 
but they create whole new challenges in terms of providing correctness software. 
Obviously, they are well-known to be difficult to get right and debug because of numerous (usually unbounded)
 interleavings among their multiple components. Testing is necessary to avoid their possible faults, 
 but it is not a promising way to provide their high assurance. Reproducing a bug is unfeasible unless testers know their precise interleaving order.
In this sense, building reliable concurrent programs requires verification to formally shows that they 
correctly reflect the desired behavior (\ie, the behavior stated in their specifications) without missing any single interleaving cases.


In addition, modern software systems consist of multiple sub-modules, 
which are intimately connected with other modules in the system. This brings another source of complexity to the verification. 
These sub-modules highly depend on others, which makes the modular reasoning of each component difficult. 
Accordingly, software verification is considered painful work with prohibitive costs associated with the difficulty of achieving scalability, 
reusability,  and extensibility. 
For example, a famous previous by an seL4 team ~\cite{klein2009sel4} accomplished a breakthrough in software verification 
by providing the first verified (sequential) microkernel and connecting all proofs in a mechanized proof assistant, 
but it required considerable effort. The verification took 11 person years for 7,500 lines of C codes but still contained unverified parts
 (\ie, 1,300 lines of C, 500 lines of assembly, and the compiler to extract the executable code from the verified C codes).
%When combined with concurrency,
%some modules facilitate
%shared operations of other modules
%to form the new shared services.
%It also provides another challenge 
%in the concurrent program verification, 
%which has to resolve not only the interleaving among the concurrent instances as well as the dependencies among the sub-modules 
%that forms the entire system.


\begin{figure}

NEED TO ADD FIGURE
NEED TO ADD FIGURE
NEED TO ADD FIGURE
NEED TO ADD FIGURE
NEED TO ADD FIGURE
NEED TO ADD FIGURE
NEED TO ADD FIGURE
NEED TO ADD FIGURE
NEED TO ADD FIGURE

\label{fig:intro:challenges-concurrent-prog-verification}
\caption{Challenges in Concurrent Program Verification.}
\end{figure}

\jieung{Need to add figure}


%%%%%%%% isolation is required

Therefore, modular and compositional reasoning is indispensable for concurrent program verification. 
The verification should be able to decompose the entire system into a collection of instances (\ie, multiple threads or a set of nodes) and 
further into a stack of modules in each instance; then it needs to be achieved with each module separately without considering complex 
dependencies or interleavings with other modules and instances on the system. Of course, it needs to provide evidence about the system's 
consistency by verifying each module and the behavior of the entire system. 
This modular and compositional software reasoning not only provides an efficient tool for verification 
but also opens the possibility of proving the correct behavior of the system software that is usually parameterized by other programs running on them.
%
%This feature is crucial in some sorts 
%of concurrent programs such as 
%operating systems, libraries, or application interfaces because of the
%proof of them usually needs to be parameterized by 
%other programs running on them. 
%In those cases, composition and proof isolation 
%give enough power to state and prove the correctness property 
%of those programs upon any arbitrary context programs run with the targeted programs. 

%%%%%%%v other previous works 
In this sense, several previous works address modular and/or compositional reasoning with respect to programs--either with or without concurrency.
% need separation logic????
There are two traditional approaches--rely-guarantee ~\cite{jones83} and separation logic ~\cite{ishtiaq01}-and many others 
that stem from either or both of them 
~\cite{feng07:sagl,vafeiadis:marriage,LRG,fu10:roch,sergey15, lili16,Vafeiadis11mfps, Yang07relsep,
Liang14lics}.
Besides, some approaches not only focus on functional correctness but also shows liveness~\cite{lili16}.
However, most previous works do not provide a programming framework that addresses all the above challenges
in concurrent program verification and is capable of extracting the running code from the program written in low-level programming 
languages such as C and assembly.

\section{Verification Toolkit for Concurrent Programs}
\label{chapter:introduction:sec:verification-toolkit-for-concurrent-programs}
%
%\begin{figure}
%\includegraphics[width=\textwidth]{figs/intro}
%\caption{Verification Toolkit Structure}
%\label{chapter:intro:verific	tion-toolkit-structure} 
%\end{figure}
With those investigations, we present the toolkit that supports modular and compositional verification for concurrent programs. 
The toolkit consists of two parts: 1) propose the method to build local layer interfaces for concurrent program verification; and 
2) provide the concurrent-linking framework. All layers and linking parts are also connected with the traditional simulation 
technique~\cite{compcert, deepspec}. 
 
The first part of our toolkit is to construct certified concurrent abstraction layers: a new compositional model for concurrency, 
a program verifier for concurrent C and assembly, and a verified C compiler. 
Each layer interface in our framework functions as an executable machine which consists of state and multiple transition rules. 
To divide the program into fine-grained pieces, we follow the idea of abstraction layers, which are widely used in modern computer systems. 
Our layered approach also reduces the complex dependency among the sub-modules in the entire system.
We use the verified compiler for this layered approach to minimize the trusted parts between the verified program and the executable code 
on a bare machine. Programs written and verified with our toolkit use a subset of C language, 
which can be compiled into $\intelmachine$ assembly language via $\compcertx$, which is the variance of verified C compiler $\compcert$.

This first part of our verification toolkit has several distinctive features which stem from the requirements of 
a sizeable concurrent system. First, it is suitable for dealing with the low-level code. To make the proofs tractable, 
we mainly work at the C level (relying on the $\compcert$ verified compiler~\cite{compcert}), 
but we sometimes need to go lower. 
For example, the MCS algorithm needs to use atomic CPU instructions (fetch-and-store and compare-and-swap), 
so we need a way to mix C and assembly code. At the same time, C itself is too low-level to reason about conveniently, 
so we need data abstraction to hide the details about representation in memory. 
Our toolkit provides an efficient way to verify C and assembly programs as well as connect the data abstraction 
with detailed representations in memory. Second, to handle large developments, we need separation of responsibilities. 
In a small proof of a small concurrent program in isolation, 
you can state the specification as a single pre- and post-condition which specifies the shape and ownership of the data structure, 
the invariants of the liveness conditions, and even the behavior of the entire system. 
But such a proof is not modular and not reusable. In our development, 
these are done as separate refinement steps in separate modules with explicit interfaces, 
and they can even be the responsibility of different software developers. 
Finally, the layers approach is general purpose in the sense that the same semantic framework can be used for proving all kinds of properties. 
The model of program execution exposed to the programmer is simple—mostly the same as what you would use for sequential code, 
and with a notion of logs of events to model concurrency. We did not have to add any special features to the logic to show high-level 
properties (such as a liveness property) because we could directly reason in $\coq$ about how long an execution will take.

The other part of our toolkit is to provide the connection between multiple concurrent instances in the system. 
Our certified concurrent abstraction layers provide the way to build and verify concurrent programs by decoupling 
the complex interleaving from the correctness proof of each layer in the system. 
This compositional requires each layer to use assumptions on the environment of the layer interface--the behavior of other concurrent components 
in the system. 
Those assumptions inevitably need to be compatible with the properties that other concurrent components can guarantee; 
thus, proving this property is desirable for our concurrent program verification toolkit. To fulfill this requirement, 
we provide the concurrent linking library as a part of our toolkit, which includes defining concurrent machine 
models and providing generic proof methods to show the validity in the parallel composition of multiple instances as well as 
the formal connection between the program on concurrent machine models and the program on the local layer interface.

Our linking library also has several unique aspects to apply it to large concurrent system verification. 
First, our concurrent machine semantics are generic enough to be applied to arbitrary programs written in our layered framework. 
We separate the linking process from the concurrent layer building so users do not have to deal with the details of linking itself when they 
build concurrent layers using our framework. Second, it also can be linked with the assembly model for the verified compiler $\compcertx$, thus
 providing the full formal connection between the program written in C and assembly and the data abstraction for the detailed data 
 representation on the memory. Third, it provides complex dependencies among multiple data structures in large concurrent programs. 
 This is not only related to the dependencies among shared objects in the program but also the dependencies between 
 shared and private objects. Our toolkit handles all those issues with reasonable restrictions.

\section{Concurrent Program Verification Examples}
\label{chapter:introduction:sec:concurrent-program-verification-examples}


\begin{figure}

NEED TO ADD FIGURE
NEED TO ADD FIGURE
NEED TO ADD FIGURE
NEED TO ADD FIGURE
NEED TO ADD FIGURE
NEED TO ADD FIGURE
NEED TO ADD FIGURE
NEED TO ADD FIGURE
NEED TO ADD FIGURE

\label{fig:intro:certikos-structure}
\caption{Structure of Concurrent Operating System Verification.}
\end{figure}



As examples of the applicability of our framework, we verified two large-scale concurrent systems--$\certikos$ and $\wormspace$.
These examples also represent two models for concurrent programs: 
the shared-memory model of concurrency and the message-passing model.

A concurrent operating system is a well-known example of the shared-memory model for concurrency that multiple threads 
in the system share the same physical memory or a common 
file system. Operating system verification has been studied for a while with impressive results~\cite{klein2009sel4, xu16, hawblitzel10}.
However, these studies either lack supporting fine-grained lock control on shared resources or lack progress property proofs of their kernels.
 On the other hand, $\certikos$ is the first verified concurrent operating system kernel with fine-grained locking. 
 The kernel is written in C and assembly, and the extracted code of the kernel (via verified compositional compilation) 
 runs on an $\intelmachine$ multicore machine. It consists of 6,500 lines of C and assembly implementation and 200K+ lines of $\coq$ proofs.

To manage such a massive verification work, we divide the kernel into multiple sub-modules and link them together to form the 
correctness theorem of the kernel. This work not only facilitates the abstraction-layer approach in our framework 
but also uses our concurrent-linking framework to show the simulation relation between 
the program on each instance with its concurrent environment and the program runs on the full 
concurrent machine--the $\intelmachine$ multicore machine. 

We also built a spinlock module to support fine-grained lock services for multiple shared objects in the kernel 
(\ie, page allocation, atomic queue, \etc).
The MCS Lock--one of two lock algorithms we used--verification is described in detail in this thesis, 
focusing on how we divided the requirement of lock verification into multiple layers, proving
liveness of the lock and providing a simple but common interface of the verified lock for other shared resources.

The MCS algorithm is well-known in scalable fair inter-CPU mutex locks. 
Although the program is short, the proof is challenging. First, the implementation of a lock algorithm cannot itself use locks, 
so it must rely solely on atomic-memory instructions and be robust against any possible interleavings between CPUs. 
This is the most challenging type of concurrency--so-called lock-free programming.
Second, unlike algorithms which only promise mutual exclusion, the MCS algorithm also aims for fairness among CPUs.
To check that it got it right, our correctness theorem needs to guarantee not only mutual exclusion (a safety property) but also bounded waiting time (a liveness property). We show how we resolve these challenges in this thesis.


Our verification toolkit is inspired by the $\certikos$ project, but it is not limited to operating system verification. 
Distributed systems are well-known as the message-passing model of concurrency, where nodes in distributed systems are connected with 
others via network communication. They serve millions of users in important applications
 these days (\ie, banking, communications, and social networking), but they are difficult programs to be correctly implemented 
 due to their concurrency and their failures. Local nodes may crash at arbitrary moments, and communications may possess failures 
 such as packet reordering, loss, and duplication. In this sense, distributed systems are a desirable target for verification to 
 show the applicability of our framework.

To build a trustworthy distributed system, we first introduce the WOR abstraction inherent in many distributed systems and present a simple, 
data-centric WOR API as a first-class programming abstraction.  
Next, we implement three distributed applications over this API; for each, our modular design easily allows new configurations 
with different performance and availability properties while matching or surpassing the performance of an existing monolithic implementation 
in a similar configuration. Finally, we apply our verification toolkit into distributed systems by adding a 
non-Byzantine band asynchronous network model (which allows packet duplication, delay, and loss). We built WormSpace, 
which is a distributed system API that provides the abstraction of the common interfaces that many distributed systems can use. 
The system is implemented via a collection of Paxos, and we prove its functional correctness as well as the key safety property of the 
protocol--immutability.


\section{Toolkit for Leader-based Distributed Protocols and Systems}
\label{chapter:introduction:sec:toolkit-for-leader-based-distributed-protocols-and-systems}


%The verification of distributed systems reveals another challenge beyond showing their functional correctness of those programs:  the safety proofs of their underlying protocols. 
While even verifying a single distributed system is challenging, in practice, distributed applications 
rely on several distributed systems. An application might employ different distributed systems for distinct functionalities (\eg,
consensus~\cite{vivaladifference}, distributed transactions~\cite{gray:2006},
and distributed locks~\cite{chubby, zookeeper} as part of a high-reliability distributed database),
or it might use systems that achieve the same goal 
 (\eg,
multi-Paxos~\cite{paxosmadesimple, rvrpaxos} vs. Raft~\cite{raft}) in different parts, depending on performance considerations or simple preference.
 Therefore, to realize a verified distributed system environment, methodologies to cover multiple distributed systems are necessary.

We find that distributed systems that realize strong semantics are typically designed under a common pattern: 
they exploit a leader node (or a centralized coordinator) explicitly or implicitly to coordinate distributed state changes. Indeed, 
for simplicity of management and understanding, this leader-based scheme is commonly used to implement critical distributed functions. 
For example, multi-Paxos and Raft elect a leader to replicate states across multiple nodes, two-phase commit employs a transaction manager to
 coordinate transactions over various resource managers, and coordination services grant a lock to a requester to allow for mutually exclusive 
 access to a distributed shared state. 
 To account for this, our dissertation proposes an idea that can be used in the distributed system 
 verification--especially leader-based distributed systems.

\section{Contributions by Collaborators}
\label{chapter:introduction:sec:contributions-by-collaborators}

The works in chapters~\ref{chapter:ccal},~\ref{chapter:mcs-lock},~\ref{chapter:linking}, and \ref{chapter:certikos} are parts of 
the $\certikos$ project, and were jointly done with various members in our group.
The author collaborated with Ronghui Gu on developing concurrent certified abstraction layers, concurrent linking libraries, and the verification on the verified concurrent OS kernel ($\certikos$).  
The author wrote almost all proofs in linking libraries and concurrent-linking proofs for $\certikos$ with some help from 
J{\'e}r{\'e}me Koenig.
Among the entire $\certikos$, 
the author developed the whole MCS Lock module with Vilhelm Sj{\"o}berg,
who also contributed to many other parts. The automation engine for proving the C source program were developed solely by Xiongnan (Newman) Wu,
 but its details are out of scope in this thesis. Wu’s thesis illustrates in-depth explanations about the automation engine. 
 For the case study on distributed system verification in Chapter~\ref{chapter:wormspace} ($\wormspace$), Ji-Yong Shin and Wolf Honore 
 worked together to build and verify the system, and the author had a leading role in designing the network models, 
 layers for the target system, and safety proof of the system.
 The author also worked with Ji-Yong Shin and Wolf Honore to provide a verification toolkit for leader-based distributed systems 
 in Chapter~\ref{chapter:witness-passing}.

\section{Contents of the Chapters}
\label{chapter:introduction:sec:contents-of-the-chapters}

The rest of this dissertation is organized as follows. Chapter~\ref{chapter:ccal} focuses on the framework to build local layer 
interfaces of concurrent programs--an abridged version of the related parts of our work in~\cite{concurrency}.
Chapter~\ref{chapter:mcs-lock} is an abridged version of our work from~\cite{mcslock},
which is a case study that uses the framework in Chapter~\ref{chapter:ccal}. 
Chapter~\ref{chapter:linking} provides the details for our concurrent-linking, multicore linking, and multithreaded-linking frameworks, 
of which the high-level idea is part of our work in~\cite{concurrency}.
However, this chapter differs from our previous publications by providing in-depth explanations for the parts related to concurrent linking, 
thus addressing how to use our linking framework by presenting formal rules, proofs, and its true capability, 
which are also addressed in~\cite{concurrency}.
Chapter~\ref{chapter:certikos} shows our work on $\certikos$,
which is closely related to our work in~\cite{certikos:osdi16}. 
It provides an interesting case study that uses all the ingredients of our concurrent verification
 framework as well as shows its full power. The verification work on distributed systems, $\wormspace$,
is discussed in Chapter~\ref{chapter:wormspace}.
It is an abridged version of our work in~\cite{wormspace},
which shows the applicability of our framework to distributed systems. 
Chapter~\ref{chapter:witness-passing} 
explains our idea for how to provide a generic toolkit to verify multiple leader-based distributed systems, 
which is inspired by the verification work on $\wormspace$.
Chapter~\ref{chapter:related}
offers an in-depth discussion of related work, and  Chapter~\ref{chapter:conclusion}  
mentions limitations and future directions as well as summarizes this thesis.

% END CONTENTS

\chapter{Concurrent Certified Abstraction Layer}
\label{chapter:ccal}
%
%\section{Introduction}
\label{sec:intro}

%%% Outline
%% structure 
%% 1. concurrent verification is done in several works 
%% 2. how about showing the non-deterministic full machine model refines  ... 
%% 3. For example CCAL provide a useful tool for building concurrent abstraction layer 
%% 3-1. building layers is feasible 
%% 3-2. However proving the refinement between concurrent machine model and the per-instance machine model 
%%
%% 3-3. Based on the CCAL, we show how we build the linking for them 
%% 3-3-1. Multicore Linking 
%% 3-3-1-1. t provides the universal abstract semantics for multicore non-deterministic machine (with sequential consistency)
%% 3-3-1-2. it provides detailed refinement between those abstract functions 
%% 3-3-1-3. it provides the concrete instance of those proofs by connecting them with the lowest layer of CompCertX layer 
%% 3-3-2. Multithreaded Linking 
%% 3-3-3-1. It provides the CompCert Assembly machine models for CompCertX to build per-thread machine models 
%% 3-3-3-2. it provides the refinement between those machine models (parameterized by any kinds of Layers with the guarantee about the certain properties) 
%%                 - that allows us to allocate the proper dynamic initial state for each thread / invariant preserving in the initial state / using the same compiler with 
%%                    CompCertX                    
%% 3-3-3-3. it provides the actual proofs using the example in the certified layers (the language and the proofs are parameterized by the concrete layer definition)
%%                  - shows the identity of the private state change while  sleep and yield 
%%                  - mutual exclusion of user memory regions 
%%                  - mutual exclusion of other private states  






%
%Dependencies due to shared data
%•
%Subtle effects of synchronizations
%•
%Often manually parallelized
%–
%Difficult to debug
%•
%too many 
%interleavings
%of threads
%•
%hard to reproduce bugs
%
%
%
%

%%% concurrent program verification is necessary 
The prevalence of shared-memory multicore machine 
brings the eminent changes in the  software. 
With the machine, achieving higher performance on a single computer than before 
becomes possible, 
but it requires us to facilitate 
concurrency, running multiple threads on multiple cores.
Concurrency, however, 
brings the whole new challenges in terms of software correctness. 
They are well known 
to be difficult to get right and to debug because 
of their intrinsic characteristic, numerous number (usually unbounded) of interleavings among multiple components of the system. 
Testing is also not a promising way to provide the high-assurance of those programs. 
Due to a plethora of possible interleavings, 
reproducing a bug is unfeasible unless testers knows the 
precise interleaving order of them. 
In this sense, 
Building reliable concurrent programs 
needs verification of them, which formally shows that those programs correct reflects the 
desirable behavior (\textit{i.e.,} are stated in their specifications) 
without missing any single interleaving cases. 

%%% Composition is required
The concurrent program verification requires compositional reasoning in its essence,
since it provides an isolation of each instance of concurrent program
(on a single core or a single thread) separately  
 in its verification
without directly considering complex interleaving 
with other components in the system. 
This feature is crucial in some sorts 
of concurrent programs such as 
operating systems, libraries, or application interfaces
because the
proof of them 
are usually need to be parameterized by 
other programs running on them. 
In those cases, composition and proof isolation 
give  an enough power 
to state and prove the correctness property 
of those programs upon any arbitrary context programs run with the targeted programs. 

%%%% several previous works and machine checkable proof  

In this sense, 
multiple previous works handle compositional reasoning about concurrent programs.
There are two traditional different approaches,
rely-guarantee~\jieung{cite rely guarantee} and separation logic~\jieung{CSL cite separation logic  - need to refer View for citation},
and many other approaches that stem from either or both of them
\jieung{SAGL (2007) / Bornat-at (2005) RGSep (2007) Gotsman-al (2007) RSL (2013) Deny Guarantee (2009) LRG (2009) RGSim (2012) Liang-Feng (2013) 
Lili (2016) / Iris (2015) Iris 2.0 (2016) FCSL (2014) (SCSL (2013) FTCSL (2015) CoLoSL (2015) CAP (2010)   View paper / CCAL paper / CSpec (MIT)
- Please refer the specification of POSIX File Systems slide}.
In addition, some of them are not only focusing on the functional correctness but also 
shows liveness~\jieung{LiLi}. 
Some, CSpec and CCAL, also provides a verified layered structure to build modular verification, an another important 
feature to build a large scaled program verification in a modular ways.


%%%% several previous works and machine checkable proof  
Bsed on them, few works \jieung{verifying concurrent software using movrs in CSPEC / preemtive kernel verification (Xinyu Feng - CAV), CertiKOS, MCSLock CCAL} 
organizes machine checked proofs 
about concurrent execution. 
Among them, both CSPEC and CertiKOS facilitates layered structures 
for scalable and modular verification and formally connect top level operations into bottom-layer operations.

%%%% CCAL - what is missing 
They, however, overlook the difficulty in one another piece of machine checked concurrent program verification, 
provide the evidence of concurrent linking.
The concurrent linking shows 
the precise evidence of the composition that the underlying logics provide. 
In this sense, 
it requires the definition of 
concurrent machine model that can run multiple instances of concurrent program together (\textit{e.g.,} multicore and multithreaded machine) 
as well as 
the linking proofs between the program runs on top of concurrent machine and the composition of multiple single instances together. 
It also requires the proof that 
shows the single instance of the concurrent program correctly reflects
the program run on the multicore machine model. 

They are necessary to show the full correctness of the program, 
but providing concurrent machine model is bothersome, especially when the model is close to that of bare machines, 
and the proof between it wiith the machine that runs the single instance is also a subtle work.
To handle those challenges,
CCAL slightly mentioned these issues,
but it only carries out
a key idea of
linking without exposing underlying multiple obstacles.  
In this sense, 
providing the information about which steps are necessary for concurrent linking and what kind of things that 
the users have to fill out is desired.
In this sense, the idea in the paper is far from 
the enough idea to achieve how 
concurrent linking can be worked in such 
a large scaled concurrent program. 

\jieung{need to add sentence about CompCertX}


%%%% The contribution of this paper

Therefore, our paper aim to deliver all necessary 
and important ides for concurrent linking,
which includes modeling the generic concurrent machine model, 
necessary information to prove refinements between them, 
and how to connect those concurrent linking with the 
proof layers of concurrent programs in a generic way. 
It is definitely not able to be achieved in a single shot.
We introduce multiple intermediate languages and 
context that users has a responsibility to 
connect the generic concurrent linking proof with 
their one verified programs.
We, in this paper, handle all of them in detail. 
In short, he key contribution of this paper is as follows: 

\begin{itemize}
\item We formally define non-deterministic multicore semantics and multiple intermediate languages that are independent from specific machines (such as x86 or ARM). 
\item We provide the refinement proofs between them that can be used for \compcertkwd-style backward simulation. 
\item We connect those intermediate languages and proofs with the CPU local CCAL layer, that uses \compcertkwd-like sequential x86 assembly model with 
environment context.
\item We provide multithreaded machine model with minimal assumptions about a certain CPU local CCA layer, which implies that the machine model does not stick to the specific layer definition.
\item We provide intermediate languages to introduce per thread machines and refinement proofs among them. 
\item We connect those intermediate languages and refinement proofs with the specific layer definition in CertiKOS, which fully link the layer on per-thread machine with the layer on per-CPU machine.
\end{itemize}

The structure of remaining paper is as follows:
Section~\ref{sec:overview} shows a brief high level idea of CCAL as well as how our linking works. Section~\ref{sec:multicore} shows the details of multicore linking,
and Sect.~\ref{sec:multithreaded} shows the implementation of our intermediate machine models for our multithreaded environment.
Section~\ref{sec:multithreaded-linking-impl} shows how are framework 
can be fitted into the actual concurrent kernel implementations.
Evaluations about our implementation can be found in Sect.~\ref{sec:evaluation} 
and the related work and conclusion is in Sect.~\ref{sec:related}.


%
%
%\begin{figure}
%\caption{Requirements in Concurrent Program Verification}
%\label{fig:concurrent-verification-challenge}
%\end{figure}
%
%However, even with the importance of concurrent program verification and 
%a large body of recent work on shared-memory concurrency verification ~\jieung{cite},
%there are few certified programming tools for a large scale software due to the requirement of multiple challenges described in Fig.~\ref{fig:concurrent-verification-challenge}.
%
%\jieung{ need to site ESOP papers too}
%
%They first have to 
%provide a way to build the software in multiple layers
%that enable us to build a large scale program as a modular way. 
%For example, 
%operating systems can be divided into multiple parts, 
%memory management, process management, and so on.
%
%They also have to provide \jieung{need different word} a methodology to 
%represent the behavior of other components in the concurrent environment. 
%For the program running on multicore environment, 
%the single instance of the program, which is a program runs on top of 
%a single CPU, has to correctly capture the 
%environmental behavior (the behavior of programs on other CPUs). 
%
%In addition to that, 
%providing the end-to-end theorem also requires us 
%to link the multiple proof instances to 
%form a single proof that is based on
%the concurrent environment itself which does not have 
%any environmental contexts at all. 
%In the example of the operating system on multicore environment,
%the end-to-end theorem 
%has to prove that 
%the program running on the single CPU is correctly refined by 
%the whole thread programs running on the multicore machine. 
%
%Previous works, CertiKOS~\jieung{need cite} and Certified Concurrent Abstraction Layer~\jieung{need cite}, 
%tackles all the above examples.  
%CCAL is a tool to build a certified concurrent layers, which provides 
%a way to build concurrent abstraction layers, 
%
%
%
%However, the paper does not handle how the linking process works with the concrete machine models. 
%It briefly mentions the high level idea of linking and the memory extension for linking framework. 
%
%Therefore, this paper aims the gap between the high level perspective of CCAL and the 
%low level details of concurrent proof linking. 
%This low level details contains two parts. 
%First, it requires us to define and build multiple intermediate languages to connect
%the x86 multicoro machine model with the LAsm, which is the machine model for one single CPU. 
%In addition to that, 
%the framework also needs to show the refinement 
%between layers on those intermediate machine models to formally link
%all those proofs together. 
%CCAL also briefly provide the idea of how they implement the practical machine models that can be used with CompCertX.
%However, only providing few details does not provide 
%the  useful information to show how it works with the actual running large scale software.
%Thus, our paper tackles the issues that CCAL overlooked in the paper 
%by providing the formal rules and proofs.
%The key contribution of this paper is as follows: 
%
%\begin{itemize}
%\item We provide the detailed intermediate language semantics for multicore machine model based on CCAL, 
%and instantiate all those intermediate language semantics and refinement proofs 
%to link them with CompCertX with environmental context 
%\item We provide the intermediate machine models to build single threaded machine model from a single CPU machine model. 
%Based on the machine models, we provide the linking theorem in between 
%two abstraction layers, which contains different semantics for software schedulers. 
%\end{itemize}
%
%The structure of remaining paper is as follows:
%Section~\ref{sec:overview} shows a brief high level idea of CCAL as well as how our linking works. Section~\ref{sec:multicore} shows the details of multicore linking,
%and Sect.~\ref{sec:multithreaded} shows the implementation of our intermediate machine models for our multithreaded environment.
%Section~\ref{sec:multithreaded-linking-impl} shows how are framework 
%can be fitted into the actual concurrent kernel implementations.
%Evaluations about our implementation can be found in Sect.~\ref{sec:evaluation} 
%and the related work and conclusion is in Sect.~\ref{sec:related}.
%
%
%
%\ignore{
%Despite the importance of concurrent layers and a large body of recent work on 
%shared-memory concurrency verification, 
%
%
%there are no certified programming tools that can specify, compose, and compile concurrent layers to form a whole system [6]. Formal reasoning across multiple concurrent layers is challenging because different layers often exhibit different interleaving semantics and have a different set of observable events. For example, the spinlock module in Fig. 1 assumes a multicore model with an overlapped execution of instruction streams from different CPUs. This model differs significantly from the multithreading model for building high-level synchro- nization libraries: each thread will block instead of spinning if a queuing lock or a CV event is not available; and it must count on other threads to wake it up to ensure liveness.
%
%
%
%
%many of these abstraction layers also become concurrent in nature. Their interfaces not only hide the concrete data representations and algorithmic de- tails, but also create an illusion of atomicity for all of their methods: each method call is viewed as if it completes in a single step, even though its implementation contains com- plex interleavings with operations done by other threads. Herlihy et al. [19, 20] advocated using layers of these atomic objects to construct large-scale concurrent software systems.
%
%
%The importance of software systems' accuracy is growing rapidly these days. 
%In addition to that, 
%the concurrent environment, including multicore and device drivers, are ubiquitous in modern periods. 
%Therefore, 
%the verification methodology for concurrent programs is critical now. 
%
%In this sense, several previous works propose
%proof logics and tools for that purpose \jieung{need cite}.
%
%However, few of them are working on the linking multiple instances of 
%verified concurrent programs with concrete machine models that can be run 
%on the bare machines. 
%
%One tool, Certified Concurrent Abstraction Layers, 
%provides the tool that can be used for building a practical concurrent programs 
%such as a small operating system or distributed system. 
%It also provides the tool to link the 
%}
%
%
%
%\section{Introduction}
\label{sec:intro}

Abstraction layers (e.g., circuits, ISA, device drivers, OS kernels,
and hypervisors) are widely used in modern computer
systems to help reduce the complex interdependencies among components
at different levels of abstraction~\cite{salzer09,baldwin00}.  An
abstraction layer defines an interface that hides the implementation
details of its underlying software or hardware components. Client
programs built on top of each layer are understood solely based on
the interface, independent of the layer implementation.

As multicore hardware and multithreaded programming become more
pervasive, many of these abstraction layers also become {\em
  concurrent} in nature.  Their interfaces not only hide the concrete
data representations and algorithmic details, but also create an
illusion of {\em atomicity} for all of their methods: each method call
is viewed as if it completes in a single step, even though its
implementation contains complex interleavings with operations done by
other threads.  Herlihy~{et~al.}~\cite{herlihy90,Herlihy08book}
advocated using layers of these atomic objects to 
construct large-scale concurrent software systems.


%%%%%%%%%%%%%%%%%%%%%%%%%%%%%%%%%%%%%%%%%%%%%%%%%%%%%%%%%%%%%%%%%%%%%%%%%%%
\begin{figure}
\centering
\includegraphics[scale=.5]{figs/ccal/sys_arch}
\caption{An overview of concurrent abstraction layers in a modern
  multithreaded and multicore environment (arrow means
  possible function call from one component to another).}
\label{fig:arch}
\end{figure}
%%%%%%%%%%%%%%%%%%%%%%%%%%%%%%%%%%%%%%%%%%%%%%%%%%%%%%%%%%%%%%%%%%%%%%%%%%%

%%%%%%%%%%%%%%%%%%%%%%%%%%%%%%%%%%%%%%%%%%%%%%%%%%%%%%%%%%%%%%%%%%%%%%%%%%%
\begin{figure*}[t]
\centering
%\ifTR{}{\vspace*{-7pt}}
\includegraphics[scale=.5]{figs/ccal/tool_chain}
%\ifTR{}{\vspace*{-7pt}}
\caption{System architecture of the CCAL programming toolkit.}
\label{fig:toolchain}
%\ifTR{}{\vspace*{-5pt}}
\end{figure*}
%%%%%%%%%%%%%%%%%%%%%%%%%%%%%%%%%%%%%%%%%%%%%%%%%%%%%%%%%%%%%%%%%%%%%%%%%%%

Figure~\ref{fig:arch} presents a few common concurrent layer objects
in a modern multicore runtime. Here we
use the light gray color to stand for thread-local (or CPU-local)
objects, blue (also with round dots in their top-right corner) for objects shared between CPU cores, green for objects exported and shared between threads, and orange for threads
themselves. Above the hardware layers, we must first build an efficient
and starvation-free spinlock implementation~\cite{mcs91}. 
With spinlocks, we can implement shared objects for sleep and pending thread
queues, which are then used to implement the thread schedulers,
and the primitives yield, sleep, and wakeup. 
On top of them,
we can then implement high-level synchronization libraries such as
queuing locks, condition variables (CV), and message-passing
primitives~\cite{ospp11}.

Despite the importance of concurrent layers and a large body of recent
work on shared-memory concurrency
verification~\cite{ohearn:concur04,brookes:concur04,feng07:sagl,vafeiadis:marriage,LRG,gotsman13,Turon13popl,nanevski13,sergey15pldi,iris15,pinto16,lili16,xu16},
there are no certified programming tools that can
specify, compose, and compile concurrent layers to form a whole
system~\cite{sfm16}.
Formal reasoning across multiple concurrent
layers is challenging because different layers often exhibit different
interleaving semantics and have a different set of observable
events. For example, the spinlock module in Fig.~\ref{fig:arch}
assumes a multicore model with an overlapped execution of instruction
streams from different CPUs. This model differs significantly
from the multithreading model for building high-level synchronization
libraries: each thread will block instead of  spinning if a queuing
lock or a CV event
is not available; and
it must count on other threads to wake it up to ensure liveness.


Reasoning across these different abstraction levels requires a
general, unified compositional semantic model that can cover all of
these concurrent layers. It must also support a general ``parallel
layer composition rule'' that can handle explicit thread control
primitives (e.g., sleep and wakeup). It must also support
vertical composition~\cite{ospp11} of these concurrent layer
objects~\cite{Herlihy08book} while preserving both the
linearizability and progress (e.g., starvation-freedom) properties.


\para{Contributions.} 
In this paper, we present CCAL---a fully mechanized
programming toolkit implemented in
Coq~\cite{coq} and developed under the CertiKOS project \cite{certikos-osdi16} for building certified {\em concurrent}
abstraction layers.  
As shown in Fig.~\ref{fig:toolchain}, CCAL consists of a novel
compositional semantic model for concurrency, a collection of C and
assembly program verifiers, a library for building layered refinement
proofs, a thread-safe verified C compiler based on
CompCertX~\cite{dscal15}, and a set of certified linking tools for
composing multithreaded or multicore layers.

We define a certified  concurrent abstraction layer
as a triple $(L_1[A],M,L_2[A])$ plus a mechanized proof object showing that the layer implementation $M$, running on behalf of a thread set $A$ over the interface $L_1$, indeed faithfully implements the desirable interface $L_2$
above. Our compositional semantics model  is based upon
ideas from game semantics~\cite{gsinvite}. It
enables local reasoning such that  the implementation can be first verified over a single thread $t$ by building $(L_1[\set{t}], M,L_2[\set{t}])$ without worrying too much about the concurrency
and the guarantees can then be propagated to the whole concurrent machine by parallel compositions. 

Following \citet{dscal15}, certified concurrent layers enforce {\em
  termination-sensitive} contextual correctness property. In the
concurrent setting, this means that every certified concurrent object
satisfies not only a safety property (e.g., {\em
  linearizability})~\cite{herlihy90,filipovic10} but also a progress
property (e.g., {\em starvation-freedom})~\cite{liang13}.

The CCAL toolkit has already been used in multiple
large-scale verification projects under CertiKOS: \citet{certikos-osdi16} have
successfully used CCAL to build the world's first fully certified
concurrent OS kernel; \citet{cpp-mcs-lock} used CCAL to verify the
safety and liveness of a complex MCS lock
implementation~\cite{mcs91}. Neither of these two
papers~\cite{certikos-osdi16,cpp-mcs-lock} explained the internals of
CCAL and how and why it can work so effectively.

This paper, rather than focusing on the applications of  CCAL, gives an in-depth exploration of the CCAL toolkit itself and how it can be used for building various certified concurrent objects. Over \citet{certikos-osdi16}, this paper presents the following three  technical contributions:
\begin{itemize}[leftmargin=15pt] \itemsep1em

\item We introduce a new compositional semantic model for shared-memory
  concurrent abstract machines and prove a general parallel layer
  composition rule. We show how our new framework is
  used to specify, verify, and compose various concurrent objects at different levels of abstraction (see Fig.~\ref{fig:arch}) 
  such as multicore
  machine hardware, implementations of fine-grained
  synchronization primitives (e.g., spinlocks, queuing locks,
  condition variables), low-level thread primitives (e.g., $\yield$,
  $\sleep$, $\wakeup$), and multi-threading in the presence of a software 
  scheduler and I/O interrupts.
\item We show how to apply standard {\em simulation}
  techniques~\cite{compcert,dscal15} to verify
  the safety and liveness of concurrent objects in a unified setting.
Following RGSim~\cite{RGSim}, we can impose
  {\em invariants} both over the environment contexts (i.e., the ``rely'')
  and also over the active threads themselves (i.e., the ``guarantee'').
  However, unlike RGSim,
  because our environment context
  specifies not just the environment's past events
  but also {\em future events}, we can readily impose temporal
  invariants such as fairness requirements (for schedulers) or
  {\em definite actions}~\cite{lili16} (for releasing locks). This allows
  us to give full specifications for lock primitives and support vertical
  composition of starvation-free atomic objects, none of which have ever
  been possible before~\cite{lili16}.
%%%%%%%%%%%%%
\item
  We have also developed a new {\em thread-safe} version of the
  CompCertX compiler~\cite{dscal15} that can compile certified concurrent C
  layers into assembly layers.
  To support certified multithreaded linking, we have developed a new
   extended algebraic memory model (for CompCertX) whereby stack frames
  allocated for each thread are combined to form a single coherent
  CompCert-style memory.
%%%%%%%%%%%%%
\end{itemize}
%
%%\vspace{-5pt}
%\para{Scope and Paper Outline.} While the notion of certified
%concurrent layer can potentially be applied to a more general
%setting~\cite{lynch96,sangiorgi03}, in this paper, we focus on
%shared-memory concurrent program modules as described in
%\citet{ospp11} and \citet{Herlihy08book}, which are sufficient to
%verify layers as shown in Fig.~\ref{fig:arch}.  Section~\ref{sec:related}
%discusses related work and puts our work in broader perspective. Both
%the CCAL toolkit and all our assembly (or C) machines assume strong
%sequential consistency for  shared primitives. Adding support for
%relaxed memory models is left as future work.
%
%%\para{Supplemental Material}
%More technical details are found in 
% the Coq implementation (submitted as the supplemental material)
%of our CCAL toolkit, including the layered specifications, simulation
%proofs, linking theorems, 
%and the thread-safe CompCertX compiler.


  
%\input{oldcontents/ccal/new-overview} 
%\input{oldcontents/ccal/mach-multicore}    
%
%
\para{Conclusions}
We have developed write-witness-passing, a simple but powerful methodology for
verifying the key safety properties of distributed systems.
This approach can also be easily combined with
the concurrent verification framework~\cite{concurrency}.
Together they provide the tools
to build and verify distributed systems in a scalable and extensible way with 
less human effort required.
As far as we know, our example of Paxos verification
is the first work to mechanically prove the functional correctness
and safety of a distributed protocol written in
a low-level programming language such as C or Assembly.
We believe this is a promising approach that can be applied to many other distributed systems.
In the future we hope to use write-witness-passing to create mechanized proofs of some of the systems we
describe in Section \ref{sec:witness-passing-semantics-with-paxos-variants} as well as other
complicated protocols such as PBFT \cite{pbft}.



\chapter{Case Study: MCS Lock}
\label{chapter:mcs lock}

%\section{Introduction}
\label{sec:intro}

%%% Outline
%% structure 
%% 1. concurrent verification is done in several works 
%% 2. how about showing the non-deterministic full machine model refines  ... 
%% 3. For example CCAL provide a useful tool for building concurrent abstraction layer 
%% 3-1. building layers is feasible 
%% 3-2. However proving the refinement between concurrent machine model and the per-instance machine model 
%%
%% 3-3. Based on the CCAL, we show how we build the linking for them 
%% 3-3-1. Multicore Linking 
%% 3-3-1-1. t provides the universal abstract semantics for multicore non-deterministic machine (with sequential consistency)
%% 3-3-1-2. it provides detailed refinement between those abstract functions 
%% 3-3-1-3. it provides the concrete instance of those proofs by connecting them with the lowest layer of CompCertX layer 
%% 3-3-2. Multithreaded Linking 
%% 3-3-3-1. It provides the CompCert Assembly machine models for CompCertX to build per-thread machine models 
%% 3-3-3-2. it provides the refinement between those machine models (parameterized by any kinds of Layers with the guarantee about the certain properties) 
%%                 - that allows us to allocate the proper dynamic initial state for each thread / invariant preserving in the initial state / using the same compiler with 
%%                    CompCertX                    
%% 3-3-3-3. it provides the actual proofs using the example in the certified layers (the language and the proofs are parameterized by the concrete layer definition)
%%                  - shows the identity of the private state change while  sleep and yield 
%%                  - mutual exclusion of user memory regions 
%%                  - mutual exclusion of other private states  






%
%Dependencies due to shared data
%•
%Subtle effects of synchronizations
%•
%Often manually parallelized
%–
%Difficult to debug
%•
%too many 
%interleavings
%of threads
%•
%hard to reproduce bugs
%
%
%
%

%%% concurrent program verification is necessary 
The prevalence of shared-memory multicore machine 
brings the eminent changes in the  software. 
With the machine, achieving higher performance on a single computer than before 
becomes possible, 
but it requires us to facilitate 
concurrency, running multiple threads on multiple cores.
Concurrency, however, 
brings the whole new challenges in terms of software correctness. 
They are well known 
to be difficult to get right and to debug because 
of their intrinsic characteristic, numerous number (usually unbounded) of interleavings among multiple components of the system. 
Testing is also not a promising way to provide the high-assurance of those programs. 
Due to a plethora of possible interleavings, 
reproducing a bug is unfeasible unless testers knows the 
precise interleaving order of them. 
In this sense, 
Building reliable concurrent programs 
needs verification of them, which formally shows that those programs correct reflects the 
desirable behavior (\textit{i.e.,} are stated in their specifications) 
without missing any single interleaving cases. 

%%% Composition is required
The concurrent program verification requires compositional reasoning in its essence,
since it provides an isolation of each instance of concurrent program
(on a single core or a single thread) separately  
 in its verification
without directly considering complex interleaving 
with other components in the system. 
This feature is crucial in some sorts 
of concurrent programs such as 
operating systems, libraries, or application interfaces
because the
proof of them 
are usually need to be parameterized by 
other programs running on them. 
In those cases, composition and proof isolation 
give  an enough power 
to state and prove the correctness property 
of those programs upon any arbitrary context programs run with the targeted programs. 

%%%% several previous works and machine checkable proof  

In this sense, 
multiple previous works handle compositional reasoning about concurrent programs.
There are two traditional different approaches,
rely-guarantee~\jieung{cite rely guarantee} and separation logic~\jieung{CSL cite separation logic  - need to refer View for citation},
and many other approaches that stem from either or both of them
\jieung{SAGL (2007) / Bornat-at (2005) RGSep (2007) Gotsman-al (2007) RSL (2013) Deny Guarantee (2009) LRG (2009) RGSim (2012) Liang-Feng (2013) 
Lili (2016) / Iris (2015) Iris 2.0 (2016) FCSL (2014) (SCSL (2013) FTCSL (2015) CoLoSL (2015) CAP (2010)   View paper / CCAL paper / CSpec (MIT)
- Please refer the specification of POSIX File Systems slide}.
In addition, some of them are not only focusing on the functional correctness but also 
shows liveness~\jieung{LiLi}. 
Some, CSpec and CCAL, also provides a verified layered structure to build modular verification, an another important 
feature to build a large scaled program verification in a modular ways.


%%%% several previous works and machine checkable proof  
Bsed on them, few works \jieung{verifying concurrent software using movrs in CSPEC / preemtive kernel verification (Xinyu Feng - CAV), CertiKOS, MCSLock CCAL} 
organizes machine checked proofs 
about concurrent execution. 
Among them, both CSPEC and CertiKOS facilitates layered structures 
for scalable and modular verification and formally connect top level operations into bottom-layer operations.

%%%% CCAL - what is missing 
They, however, overlook the difficulty in one another piece of machine checked concurrent program verification, 
provide the evidence of concurrent linking.
The concurrent linking shows 
the precise evidence of the composition that the underlying logics provide. 
In this sense, 
it requires the definition of 
concurrent machine model that can run multiple instances of concurrent program together (\textit{e.g.,} multicore and multithreaded machine) 
as well as 
the linking proofs between the program runs on top of concurrent machine and the composition of multiple single instances together. 
It also requires the proof that 
shows the single instance of the concurrent program correctly reflects
the program run on the multicore machine model. 

They are necessary to show the full correctness of the program, 
but providing concurrent machine model is bothersome, especially when the model is close to that of bare machines, 
and the proof between it wiith the machine that runs the single instance is also a subtle work.
To handle those challenges,
CCAL slightly mentioned these issues,
but it only carries out
a key idea of
linking without exposing underlying multiple obstacles.  
In this sense, 
providing the information about which steps are necessary for concurrent linking and what kind of things that 
the users have to fill out is desired.
In this sense, the idea in the paper is far from 
the enough idea to achieve how 
concurrent linking can be worked in such 
a large scaled concurrent program. 

\jieung{need to add sentence about CompCertX}


%%%% The contribution of this paper

Therefore, our paper aim to deliver all necessary 
and important ides for concurrent linking,
which includes modeling the generic concurrent machine model, 
necessary information to prove refinements between them, 
and how to connect those concurrent linking with the 
proof layers of concurrent programs in a generic way. 
It is definitely not able to be achieved in a single shot.
We introduce multiple intermediate languages and 
context that users has a responsibility to 
connect the generic concurrent linking proof with 
their one verified programs.
We, in this paper, handle all of them in detail. 
In short, he key contribution of this paper is as follows: 

\begin{itemize}
\item We formally define non-deterministic multicore semantics and multiple intermediate languages that are independent from specific machines (such as x86 or ARM). 
\item We provide the refinement proofs between them that can be used for \compcertkwd-style backward simulation. 
\item We connect those intermediate languages and proofs with the CPU local CCAL layer, that uses \compcertkwd-like sequential x86 assembly model with 
environment context.
\item We provide multithreaded machine model with minimal assumptions about a certain CPU local CCA layer, which implies that the machine model does not stick to the specific layer definition.
\item We provide intermediate languages to introduce per thread machines and refinement proofs among them. 
\item We connect those intermediate languages and refinement proofs with the specific layer definition in CertiKOS, which fully link the layer on per-thread machine with the layer on per-CPU machine.
\end{itemize}

The structure of remaining paper is as follows:
Section~\ref{sec:overview} shows a brief high level idea of CCAL as well as how our linking works. Section~\ref{sec:multicore} shows the details of multicore linking,
and Sect.~\ref{sec:multithreaded} shows the implementation of our intermediate machine models for our multithreaded environment.
Section~\ref{sec:multithreaded-linking-impl} shows how are framework 
can be fitted into the actual concurrent kernel implementations.
Evaluations about our implementation can be found in Sect.~\ref{sec:evaluation} 
and the related work and conclusion is in Sect.~\ref{sec:related}.


%
%
%\begin{figure}
%\caption{Requirements in Concurrent Program Verification}
%\label{fig:concurrent-verification-challenge}
%\end{figure}
%
%However, even with the importance of concurrent program verification and 
%a large body of recent work on shared-memory concurrency verification ~\jieung{cite},
%there are few certified programming tools for a large scale software due to the requirement of multiple challenges described in Fig.~\ref{fig:concurrent-verification-challenge}.
%
%\jieung{ need to site ESOP papers too}
%
%They first have to 
%provide a way to build the software in multiple layers
%that enable us to build a large scale program as a modular way. 
%For example, 
%operating systems can be divided into multiple parts, 
%memory management, process management, and so on.
%
%They also have to provide \jieung{need different word} a methodology to 
%represent the behavior of other components in the concurrent environment. 
%For the program running on multicore environment, 
%the single instance of the program, which is a program runs on top of 
%a single CPU, has to correctly capture the 
%environmental behavior (the behavior of programs on other CPUs). 
%
%In addition to that, 
%providing the end-to-end theorem also requires us 
%to link the multiple proof instances to 
%form a single proof that is based on
%the concurrent environment itself which does not have 
%any environmental contexts at all. 
%In the example of the operating system on multicore environment,
%the end-to-end theorem 
%has to prove that 
%the program running on the single CPU is correctly refined by 
%the whole thread programs running on the multicore machine. 
%
%Previous works, CertiKOS~\jieung{need cite} and Certified Concurrent Abstraction Layer~\jieung{need cite}, 
%tackles all the above examples.  
%CCAL is a tool to build a certified concurrent layers, which provides 
%a way to build concurrent abstraction layers, 
%
%
%
%However, the paper does not handle how the linking process works with the concrete machine models. 
%It briefly mentions the high level idea of linking and the memory extension for linking framework. 
%
%Therefore, this paper aims the gap between the high level perspective of CCAL and the 
%low level details of concurrent proof linking. 
%This low level details contains two parts. 
%First, it requires us to define and build multiple intermediate languages to connect
%the x86 multicoro machine model with the LAsm, which is the machine model for one single CPU. 
%In addition to that, 
%the framework also needs to show the refinement 
%between layers on those intermediate machine models to formally link
%all those proofs together. 
%CCAL also briefly provide the idea of how they implement the practical machine models that can be used with CompCertX.
%However, only providing few details does not provide 
%the  useful information to show how it works with the actual running large scale software.
%Thus, our paper tackles the issues that CCAL overlooked in the paper 
%by providing the formal rules and proofs.
%The key contribution of this paper is as follows: 
%
%\begin{itemize}
%\item We provide the detailed intermediate language semantics for multicore machine model based on CCAL, 
%and instantiate all those intermediate language semantics and refinement proofs 
%to link them with CompCertX with environmental context 
%\item We provide the intermediate machine models to build single threaded machine model from a single CPU machine model. 
%Based on the machine models, we provide the linking theorem in between 
%two abstraction layers, which contains different semantics for software schedulers. 
%\end{itemize}
%
%The structure of remaining paper is as follows:
%Section~\ref{sec:overview} shows a brief high level idea of CCAL as well as how our linking works. Section~\ref{sec:multicore} shows the details of multicore linking,
%and Sect.~\ref{sec:multithreaded} shows the implementation of our intermediate machine models for our multithreaded environment.
%Section~\ref{sec:multithreaded-linking-impl} shows how are framework 
%can be fitted into the actual concurrent kernel implementations.
%Evaluations about our implementation can be found in Sect.~\ref{sec:evaluation} 
%and the related work and conclusion is in Sect.~\ref{sec:related}.
%
%
%
%\ignore{
%Despite the importance of concurrent layers and a large body of recent work on 
%shared-memory concurrency verification, 
%
%
%there are no certified programming tools that can specify, compose, and compile concurrent layers to form a whole system [6]. Formal reasoning across multiple concurrent layers is challenging because different layers often exhibit different interleaving semantics and have a different set of observable events. For example, the spinlock module in Fig. 1 assumes a multicore model with an overlapped execution of instruction streams from different CPUs. This model differs significantly from the multithreading model for building high-level synchro- nization libraries: each thread will block instead of spinning if a queuing lock or a CV event is not available; and it must count on other threads to wake it up to ensure liveness.
%
%
%
%
%many of these abstraction layers also become concurrent in nature. Their interfaces not only hide the concrete data representations and algorithmic de- tails, but also create an illusion of atomicity for all of their methods: each method call is viewed as if it completes in a single step, even though its implementation contains com- plex interleavings with operations done by other threads. Herlihy et al. [19, 20] advocated using layers of these atomic objects to construct large-scale concurrent software systems.
%
%
%The importance of software systems' accuracy is growing rapidly these days. 
%In addition to that, 
%the concurrent environment, including multicore and device drivers, are ubiquitous in modern periods. 
%Therefore, 
%the verification methodology for concurrent programs is critical now. 
%
%In this sense, several previous works propose
%proof logics and tools for that purpose \jieung{need cite}.
%
%However, few of them are working on the linking multiple instances of 
%verified concurrent programs with concrete machine models that can be run 
%on the bare machines. 
%
%One tool, Certified Concurrent Abstraction Layers, 
%provides the tool that can be used for building a practical concurrent programs 
%such as a small operating system or distributed system. 
%It also provides the tool to link the 
%}


%The MCS Lock, a small but complex piece of low-level software, is a standard algorithm for providing inter-CPU locks with FIFO ordering guarantee and scalability.
It is an interesting target for verification---short and subtle, involving both liveness and safety properties. 
We implemented and verified the MCS Lock algorithm as part of the CertiKOS kernel~\cite{certikos16}, showing that the C/assembly implementation
 {\em contextually refines} atomic specifications of the acquire and release lock methods.
Our development follows the methodology of \emph{certified concurrent abstraction layers}~\cite{dscal15,ccal16}. 
By splitting the proof into layers, we can modularize it into separate parts for the low-level machine model, data abstraction, and reasoning about concurrent interleavings.  This separation of concerns makes the layered methodology suitable for verified programming in the large, and our MCS Lock can be composed with other shared objects in CertiKOS kernel.

%\section{Introduction}
\label{sec:intro}

%%% Outline
%% structure 
%% 1. concurrent verification is done in several works 
%% 2. how about showing the non-deterministic full machine model refines  ... 
%% 3. For example CCAL provide a useful tool for building concurrent abstraction layer 
%% 3-1. building layers is feasible 
%% 3-2. However proving the refinement between concurrent machine model and the per-instance machine model 
%%
%% 3-3. Based on the CCAL, we show how we build the linking for them 
%% 3-3-1. Multicore Linking 
%% 3-3-1-1. t provides the universal abstract semantics for multicore non-deterministic machine (with sequential consistency)
%% 3-3-1-2. it provides detailed refinement between those abstract functions 
%% 3-3-1-3. it provides the concrete instance of those proofs by connecting them with the lowest layer of CompCertX layer 
%% 3-3-2. Multithreaded Linking 
%% 3-3-3-1. It provides the CompCert Assembly machine models for CompCertX to build per-thread machine models 
%% 3-3-3-2. it provides the refinement between those machine models (parameterized by any kinds of Layers with the guarantee about the certain properties) 
%%                 - that allows us to allocate the proper dynamic initial state for each thread / invariant preserving in the initial state / using the same compiler with 
%%                    CompCertX                    
%% 3-3-3-3. it provides the actual proofs using the example in the certified layers (the language and the proofs are parameterized by the concrete layer definition)
%%                  - shows the identity of the private state change while  sleep and yield 
%%                  - mutual exclusion of user memory regions 
%%                  - mutual exclusion of other private states  






%
%Dependencies due to shared data
%•
%Subtle effects of synchronizations
%•
%Often manually parallelized
%–
%Difficult to debug
%•
%too many 
%interleavings
%of threads
%•
%hard to reproduce bugs
%
%
%
%

%%% concurrent program verification is necessary 
The prevalence of shared-memory multicore machine 
brings the eminent changes in the  software. 
With the machine, achieving higher performance on a single computer than before 
becomes possible, 
but it requires us to facilitate 
concurrency, running multiple threads on multiple cores.
Concurrency, however, 
brings the whole new challenges in terms of software correctness. 
They are well known 
to be difficult to get right and to debug because 
of their intrinsic characteristic, numerous number (usually unbounded) of interleavings among multiple components of the system. 
Testing is also not a promising way to provide the high-assurance of those programs. 
Due to a plethora of possible interleavings, 
reproducing a bug is unfeasible unless testers knows the 
precise interleaving order of them. 
In this sense, 
Building reliable concurrent programs 
needs verification of them, which formally shows that those programs correct reflects the 
desirable behavior (\textit{i.e.,} are stated in their specifications) 
without missing any single interleaving cases. 

%%% Composition is required
The concurrent program verification requires compositional reasoning in its essence,
since it provides an isolation of each instance of concurrent program
(on a single core or a single thread) separately  
 in its verification
without directly considering complex interleaving 
with other components in the system. 
This feature is crucial in some sorts 
of concurrent programs such as 
operating systems, libraries, or application interfaces
because the
proof of them 
are usually need to be parameterized by 
other programs running on them. 
In those cases, composition and proof isolation 
give  an enough power 
to state and prove the correctness property 
of those programs upon any arbitrary context programs run with the targeted programs. 

%%%% several previous works and machine checkable proof  

In this sense, 
multiple previous works handle compositional reasoning about concurrent programs.
There are two traditional different approaches,
rely-guarantee~\jieung{cite rely guarantee} and separation logic~\jieung{CSL cite separation logic  - need to refer View for citation},
and many other approaches that stem from either or both of them
\jieung{SAGL (2007) / Bornat-at (2005) RGSep (2007) Gotsman-al (2007) RSL (2013) Deny Guarantee (2009) LRG (2009) RGSim (2012) Liang-Feng (2013) 
Lili (2016) / Iris (2015) Iris 2.0 (2016) FCSL (2014) (SCSL (2013) FTCSL (2015) CoLoSL (2015) CAP (2010)   View paper / CCAL paper / CSpec (MIT)
- Please refer the specification of POSIX File Systems slide}.
In addition, some of them are not only focusing on the functional correctness but also 
shows liveness~\jieung{LiLi}. 
Some, CSpec and CCAL, also provides a verified layered structure to build modular verification, an another important 
feature to build a large scaled program verification in a modular ways.


%%%% several previous works and machine checkable proof  
Bsed on them, few works \jieung{verifying concurrent software using movrs in CSPEC / preemtive kernel verification (Xinyu Feng - CAV), CertiKOS, MCSLock CCAL} 
organizes machine checked proofs 
about concurrent execution. 
Among them, both CSPEC and CertiKOS facilitates layered structures 
for scalable and modular verification and formally connect top level operations into bottom-layer operations.

%%%% CCAL - what is missing 
They, however, overlook the difficulty in one another piece of machine checked concurrent program verification, 
provide the evidence of concurrent linking.
The concurrent linking shows 
the precise evidence of the composition that the underlying logics provide. 
In this sense, 
it requires the definition of 
concurrent machine model that can run multiple instances of concurrent program together (\textit{e.g.,} multicore and multithreaded machine) 
as well as 
the linking proofs between the program runs on top of concurrent machine and the composition of multiple single instances together. 
It also requires the proof that 
shows the single instance of the concurrent program correctly reflects
the program run on the multicore machine model. 

They are necessary to show the full correctness of the program, 
but providing concurrent machine model is bothersome, especially when the model is close to that of bare machines, 
and the proof between it wiith the machine that runs the single instance is also a subtle work.
To handle those challenges,
CCAL slightly mentioned these issues,
but it only carries out
a key idea of
linking without exposing underlying multiple obstacles.  
In this sense, 
providing the information about which steps are necessary for concurrent linking and what kind of things that 
the users have to fill out is desired.
In this sense, the idea in the paper is far from 
the enough idea to achieve how 
concurrent linking can be worked in such 
a large scaled concurrent program. 

\jieung{need to add sentence about CompCertX}


%%%% The contribution of this paper

Therefore, our paper aim to deliver all necessary 
and important ides for concurrent linking,
which includes modeling the generic concurrent machine model, 
necessary information to prove refinements between them, 
and how to connect those concurrent linking with the 
proof layers of concurrent programs in a generic way. 
It is definitely not able to be achieved in a single shot.
We introduce multiple intermediate languages and 
context that users has a responsibility to 
connect the generic concurrent linking proof with 
their one verified programs.
We, in this paper, handle all of them in detail. 
In short, he key contribution of this paper is as follows: 

\begin{itemize}
\item We formally define non-deterministic multicore semantics and multiple intermediate languages that are independent from specific machines (such as x86 or ARM). 
\item We provide the refinement proofs between them that can be used for \compcertkwd-style backward simulation. 
\item We connect those intermediate languages and proofs with the CPU local CCAL layer, that uses \compcertkwd-like sequential x86 assembly model with 
environment context.
\item We provide multithreaded machine model with minimal assumptions about a certain CPU local CCA layer, which implies that the machine model does not stick to the specific layer definition.
\item We provide intermediate languages to introduce per thread machines and refinement proofs among them. 
\item We connect those intermediate languages and refinement proofs with the specific layer definition in CertiKOS, which fully link the layer on per-thread machine with the layer on per-CPU machine.
\end{itemize}

The structure of remaining paper is as follows:
Section~\ref{sec:overview} shows a brief high level idea of CCAL as well as how our linking works. Section~\ref{sec:multicore} shows the details of multicore linking,
and Sect.~\ref{sec:multithreaded} shows the implementation of our intermediate machine models for our multithreaded environment.
Section~\ref{sec:multithreaded-linking-impl} shows how are framework 
can be fitted into the actual concurrent kernel implementations.
Evaluations about our implementation can be found in Sect.~\ref{sec:evaluation} 
and the related work and conclusion is in Sect.~\ref{sec:related}.


%
%
%\begin{figure}
%\caption{Requirements in Concurrent Program Verification}
%\label{fig:concurrent-verification-challenge}
%\end{figure}
%
%However, even with the importance of concurrent program verification and 
%a large body of recent work on shared-memory concurrency verification ~\jieung{cite},
%there are few certified programming tools for a large scale software due to the requirement of multiple challenges described in Fig.~\ref{fig:concurrent-verification-challenge}.
%
%\jieung{ need to site ESOP papers too}
%
%They first have to 
%provide a way to build the software in multiple layers
%that enable us to build a large scale program as a modular way. 
%For example, 
%operating systems can be divided into multiple parts, 
%memory management, process management, and so on.
%
%They also have to provide \jieung{need different word} a methodology to 
%represent the behavior of other components in the concurrent environment. 
%For the program running on multicore environment, 
%the single instance of the program, which is a program runs on top of 
%a single CPU, has to correctly capture the 
%environmental behavior (the behavior of programs on other CPUs). 
%
%In addition to that, 
%providing the end-to-end theorem also requires us 
%to link the multiple proof instances to 
%form a single proof that is based on
%the concurrent environment itself which does not have 
%any environmental contexts at all. 
%In the example of the operating system on multicore environment,
%the end-to-end theorem 
%has to prove that 
%the program running on the single CPU is correctly refined by 
%the whole thread programs running on the multicore machine. 
%
%Previous works, CertiKOS~\jieung{need cite} and Certified Concurrent Abstraction Layer~\jieung{need cite}, 
%tackles all the above examples.  
%CCAL is a tool to build a certified concurrent layers, which provides 
%a way to build concurrent abstraction layers, 
%
%
%
%However, the paper does not handle how the linking process works with the concrete machine models. 
%It briefly mentions the high level idea of linking and the memory extension for linking framework. 
%
%Therefore, this paper aims the gap between the high level perspective of CCAL and the 
%low level details of concurrent proof linking. 
%This low level details contains two parts. 
%First, it requires us to define and build multiple intermediate languages to connect
%the x86 multicoro machine model with the LAsm, which is the machine model for one single CPU. 
%In addition to that, 
%the framework also needs to show the refinement 
%between layers on those intermediate machine models to formally link
%all those proofs together. 
%CCAL also briefly provide the idea of how they implement the practical machine models that can be used with CompCertX.
%However, only providing few details does not provide 
%the  useful information to show how it works with the actual running large scale software.
%Thus, our paper tackles the issues that CCAL overlooked in the paper 
%by providing the formal rules and proofs.
%The key contribution of this paper is as follows: 
%
%\begin{itemize}
%\item We provide the detailed intermediate language semantics for multicore machine model based on CCAL, 
%and instantiate all those intermediate language semantics and refinement proofs 
%to link them with CompCertX with environmental context 
%\item We provide the intermediate machine models to build single threaded machine model from a single CPU machine model. 
%Based on the machine models, we provide the linking theorem in between 
%two abstraction layers, which contains different semantics for software schedulers. 
%\end{itemize}
%
%The structure of remaining paper is as follows:
%Section~\ref{sec:overview} shows a brief high level idea of CCAL as well as how our linking works. Section~\ref{sec:multicore} shows the details of multicore linking,
%and Sect.~\ref{sec:multithreaded} shows the implementation of our intermediate machine models for our multithreaded environment.
%Section~\ref{sec:multithreaded-linking-impl} shows how are framework 
%can be fitted into the actual concurrent kernel implementations.
%Evaluations about our implementation can be found in Sect.~\ref{sec:evaluation} 
%and the related work and conclusion is in Sect.~\ref{sec:related}.
%
%
%
%\ignore{
%Despite the importance of concurrent layers and a large body of recent work on 
%shared-memory concurrency verification, 
%
%
%there are no certified programming tools that can specify, compose, and compile concurrent layers to form a whole system [6]. Formal reasoning across multiple concurrent layers is challenging because different layers often exhibit different interleaving semantics and have a different set of observable events. For example, the spinlock module in Fig. 1 assumes a multicore model with an overlapped execution of instruction streams from different CPUs. This model differs significantly from the multithreading model for building high-level synchro- nization libraries: each thread will block instead of spinning if a queuing lock or a CV event is not available; and it must count on other threads to wake it up to ensure liveness.
%
%
%
%
%many of these abstraction layers also become concurrent in nature. Their interfaces not only hide the concrete data representations and algorithmic de- tails, but also create an illusion of atomicity for all of their methods: each method call is viewed as if it completes in a single step, even though its implementation contains com- plex interleavings with operations done by other threads. Herlihy et al. [19, 20] advocated using layers of these atomic objects to construct large-scale concurrent software systems.
%
%
%The importance of software systems' accuracy is growing rapidly these days. 
%In addition to that, 
%the concurrent environment, including multicore and device drivers, are ubiquitous in modern periods. 
%Therefore, 
%the verification methodology for concurrent programs is critical now. 
%
%In this sense, several previous works propose
%proof logics and tools for that purpose \jieung{need cite}.
%
%However, few of them are working on the linking multiple instances of 
%verified concurrent programs with concrete machine models that can be run 
%on the bare machines. 
%
%One tool, Certified Concurrent Abstraction Layers, 
%provides the tool that can be used for building a practical concurrent programs 
%such as a small operating system or distributed system. 
%It also provides the tool to link the 
%}
%\section{The MCS Algorithm}
\label{chapter:mcslock:sec:overview}

The $\mcsname$ algorithm~\cite{mcs91} is a list-based queuing lock that provides a \textit{fair} and \textit{scalable} mutex mechanism for multi-CPU computers. Fairness means that CPUs competing for the lock are guaranteed to receive it in the same order as they asked for it (FIFO order).
In  an unfair lock, CPUs
that attempt to take the lock can become nondeterministically passed over (even a million times in a row~\cite{lwn:ticketlocks}), thereby creating unpredictable latency.

Fairness is also an important factor to its verification, as it is possible that one particular CPU is continuously passed over; 
as such, it loops indefinitely without it. 
Therefore, unless the lock guarantees fairness, there is no method of proving a termination-sensitive refinement between the implementation and a simple (terminating) specification. 
With a non-fair lock, we would have to settle for either an ugly specification that allowed non-termination or for a weaker notion of correctness such as termination-insensitive refinement.
\begin{figure}
\lstinputlisting [language = C, multicols=2] {source_code/mcslock/mcs_struct.h}
\vspace{1em}

\lstinputlisting [language = C, multicols=2] {source_code/mcslock/mcs_lock_acquire.c}
\caption{Data Structure and Implementation of $\mcsname$ Lock (Written in C).}
\label{fig:chapter:mcslock:mcs_lock}
\end{figure}

Figure~\ref{fig:chapter:mcslock:mcs_lock} shows the implementation of the $\mcsname$ Lock algorithm written in C. 
The data structure  (Figure~\ref{fig:chapter:mcslock:mcs_lock})
has one global field pointing to the final node of the queue structure, and the 
per-CPU nodes forming each node in the queue. This is similar to an ordinary queue data structure (note that it only has a pointer to the tail, and not the head of the queue).
If the queue is empty, we set $\codeinmath{last}$ to the value $\invalidmcsval$, which acts as a null value (we could also have used \eg, $-1$). 
The queue is used to order the waiting CPUs in order to ensure that lock acquisition is FIFO. 
The structs also include padding to take up full cache lines and avoid false sharing. 
Each node is owned by one particular CPU (the array is indexed by CPU ID). 
To make the lock scalable, the CPU continues looping and waiting for the lock will only read its own busy flag; 
therefore, there is no cache-line bouncing. 
Simpler lock algorithms make all the CPUs read the same memory location, which does not scale past 10-40 CPUs~\cite{Boyd-wickizer12}.

We follow the programming style of $\ccalname$  presented in Chapter~\ref{chapter:ccal}; thus, the verified version we created has a property that all lock node and variables are statically assigned. 
However, this does not restrict the functionality of our verified $\mcsname$ Lock. 
Instead of creating the new node upon acquiring the lock and removing the node when releasing the lock, 
each process can pick up one node and use it as its node during the lock operation. 
All nodes are uniquely distinguished by lock ID and CPU ID, but these can be reused for the same purpose.


Figure~\ref{fig:chapter:mcslock:mcs_lock} 
also presents the code for acquire lock and release lock operations. 
The acquire lock function uses an atomic {\em fetch-and-store} expression to {\em fetch} the current $\codeinmath{last}$ value and  {\em store} its CPU ID as the $\codeinmath{last}$ value of the lock in a single action (line 6). 
Then, if the previous $\codeinmath{last}$  value was $\invalidmcsval$, the CPU can directly acquire the lock and enter the critical section (line 7). 
If the previous $\codeinmath{last}$ value was not $\invalidmcsval$, it implies that some other CPUs are in the critical section or in the queue waiting to enter it (line 9 to line 10). 
In this case, the current CPU will wait until the previous node in the queue sets the current CPU’s busy flag as $\codeinmath{FREE}$ during the lock release.

Release lock also has two execution paths based on the result of an atomic operation,  {\em  compare and swap} (line 15). 
This operation stores $\invalidmcsval$ into $\codeinmath{last}$ only if the old value of $\codeinmath{last}$ is equal to $\codeinmath{cpuid}$, 
and returns {\em true} if the update succeeds and {\em false} otherwise. 
The $\CAS$  operation succeeds, it immediately releases the lock if the current CPU is the only one in the queue.
 If the $\CAS$  fails, it implies that some other CPU has already performed the {\em fetch-and-store} operation (line 6). 
 Thus, the current CPU busy-waits until the other CPU sets the $\codeinmath{next}$ field (line 8), 
 and then passes the lock to the head of the waiting queue by assigning $\codeinmath{busy}$.


\begin{figure}
\begin{center}
\includegraphics[width=0.9\linewidth]{figs/mcslock/mcsex}
\end{center}
\caption{Possible Execution Sequence for  $\mcsname$ Lock.}
\label{fig:chapter:mcslock:mcs-example}
\end{figure}

Figure~\ref{fig:chapter:mcslock:mcs-example} 
illustrates a possible sequence of states taken by the algorithm. 
At the beginning (a),
the lock is free, and CPU 1 can take it in a single atomic $\FAS$ operation (b). 
Since CPU 1 did not have to wait for the lock, it did not need to update its \textit{next}-pointer. 
After that, CPUs 2 and 3 attempt to take the lock ((c) and (d)). 
The final value will be updated correctly due to the property of the atomic expression.
 However, there can be some delay between the CPU updating the \textit{tail} pointer and adjusting the \textit{next}-pointer of the previous node in the queue; as the example illustrates, this means that although there are three nodes which logically makes up the queue of waiting CPUs, 
 any subset of the next-pointers may be unset. 
 At (e), although CPU 1 wants to release the lock, 
 the $\CAS$ call will return false (because $\codeinmath{tail}$ is 3, not 1). 
 In this case,  CPU 1 must wait in a busy-loop until CPU 2 has set its next-pointer (f). 
 Thereafter, CPU 1 can set the busy flag to $\codeinmath{FREE}$ for the next node in the queue--CPU 2’s node--which releases the lock (g).






Since the algorithm is fair, it satisfies a \textit{liveness} property:\begin{quote}
``Suppose all clients of the lock are well-behaved, \ie, whenever they acquire a lock they release it again after some finite time, and suppose the scheduling of operations from different CPUs is fair. Then  whenever $\mcsacquire$ or $\mcsrelease$ are called they will succeed within some finite time.''
\end{quote}
A major part of our formal development is devoted to stating and proving the liveness property. 
As such, an informal proof sketch is provided here. Consider the CPU that starts executing 
 $\mcsacquire$.
At this time, the queue contains a finite number of CPUs already waiting for the lock.
By fairness of scheduling, the CPU at the head of the queue will get scheduled periodically, say every $F$ step. Each time it is scheduled, it will go through three phases. 
First, it will execute the code in $\mcsacquire$; also, since it is at the head of the queue, the loop will terminate immediately. 
Then, it executes the code in the critical section; 
by assumption, this completes after some finite number $k$ of operations. 
Finally, it executes $\mcsrelease$; this either completes immediately or enters the waiting loop, in which case it will complete as soon as the next CPU in the queue is scheduled.




%%\section{Abstraction Layers}
%\label{sec:layers}
%
%The most distinctive thing about CertiKOS-style verification is the
%notion of \emph{abstraction layers}~\cite{dscal15}. Of course, any
%large-scale programming or verification project uses layers of
%abstraction, but typically these are merely an informal organization
%that the programmer has in mind when writing the program. In CertiKOS,
%we formalize layers as objects defined in Coq, these layers are
%treated as first-class objects, and we use the framework to vertically
%compose them. We split the MCS Lock verification into
%five layers, each building on the interface exposed by the layer
%below.
%
%Our notion of a ``layer interface'' is a particular style of
%state machine where the transitions correspond to function calls, while a
%``layer'' in our development is a proof of refinement between
%interfaces. More formally,  an \emph{abstraction layer} is a
%tuple ($L_1$, $M$, $L_2$), together with a refinement proof showing
%that the code $M$, when run on top of a system specified by the
%interface $L_1$, faithfully implements the interface $L_2$.
%Then  another layer ($L_2$, $M'$, $L_3$) can run on top of
%the first one. Functions in $M'$ can call functions in $M$,
%but we only need to look at the specification $L_2$ to prove them correct.
%
%The code $M$ is a set of functions written in C or assembly, and the
%entire stack of layers can be compiled to executable code using a
%modified version of CompCert called CompCertX~\cite{dscal15}.  It is
%also possible to have a layer with no code at all. Such a ``pure
%refinement'' layer represents a proof that the interfaces $L_1$ and
%$L_2$ are equivalent. The last two layers in our development are pure
%refinements.
%
%Each layer interface $L$ is a pair $L = (A,P)$, where $A$ is Coq data
%type (usually a record type) which we call the \emph{abstract state
%  type},  and $P$ is a set of named \emph{primitive specifications}
%which describe the behavior of C/assembly functions.
%Each primitive
%specification $\sigma \in P$ is written as a Coq function of type
%$\sigma : (\mathit{val}^* \times \mathit{mem} \times A) \rightarrow
%\mathsf{option}\ (\mathit{val} \times \mathit{mem} \times A)$.
%The types $\mathit{val}$ and $\mathit{mem}$ are borrowed from
%CompCert's operational semantics for C; $\mathit{val}$ and
%$\mathit{val}^*$ are the type of C values and lists of values (for the
%function return value and arguments), and $\mathit{mem}$ is the type of C
%memory states. 
%
%The idea is that a pair $(m,d) : \mathit{mem} \times A$ represents the
%state of the computer. A typical refinement proof for a layer
%$((A_1,P_1),M,(A_2,P_2))$  will give a relation
%$R$ saying that the fields in $A_2$ represent certain objects stored
%in memory. Then the high-level specifications in $P_2$ can refer to
%the abstract value $d$ when specifications in $P_1$ had to talk about
%the memory state $m$.
%For example, in the simplest case when the
%program contains a single function $f$, and the state is a single
%thread-local integer variable, we
%might pick $A_1 = \mathrm{\texttt{unit}}$ and $A_2 = \mathrm{\texttt{nat}}$.
%The lower level specification in $P_1$ would say that invoking $f$
%modifies a location in the memory, while the high level specification in $P_2$
%will be a plain Coq function manipulating ordinary numbers.
%
%In particular, in Sec.~\ref{subsec:lowestmachinemodel} we will
%define a layer which proves a relation between the array \lstinline$LK$ (see
%Fig.~\ref{fig:exp:mcs_lock}) and an abstract state. The
%specifications in all layers about it never need to mention memory
%again, so they avoid all the side conditions to do with C memory accesses.

\section{Events, logs, and concurrent contexts}
\label{subsec:eventlogandoracle}

In order to handle concurrent programs, the verification framework
imposes some structure on the specifications~\cite{ccal16}.  Each
record type $A$ must include at least a \emph{log of events} (written
$l$) and a \emph{concurrent context} (written $\oracle$, further
explained in Sec.~\ref{subsec:abstractoperationlayer}). For almost
all of the MCS Lock development, these are the only two fields that
matter. Instead of representing the state of shared
memory by an arbitrary type $A$, it will be represented using the log.

%%%%%%%%%%%%%%%%%%%%%%%%%%%%%%
\begin{figure}
\begin{minipage}{\linewidth}
\lstinputlisting[numbers = left, language = C]{source_code/mcslock/lockeventtype.v}
\end{minipage}
\caption{Event set for MCS Lock}
\label{fig:lock_event_type}
\end{figure}
%%%%%%%%%%%%%%%%%%%%%%%%%%%%%%

An \emph{event} is any action which has observable consequences for
other CPUs. Each specification must define events for all the points
in the program where it reads or writes to shared memory (but not for
accesses to thread-local memory). The \emph{log} is a list of
events, representing all actions that have happened in the computer
since it began running. Actions from different CPUs are interleaved in
the list.
When we write a specification we can chose the set of events, as long
as it is fine-grained enough to capture all scheduling interleavings
that may happen.
Fig.~\ref{fig:lock_event_type} shows the event definition used to
model lock acquire and release. They correspond to the part of the MCS lock source code in Fig.~\ref{fig:exp:mcs_lock}
and acquiring/releasing the lock after we show starvation freedom. 

Because all CPUs see a single linear log, this model assumes that the
machine is sequentially consistent. Even with this assumption,
verifying the MCS algorithm is not easy (the other proofs we are aware
of assume sequential consistency too), so we leave weak memory models
to future work.

\begin{itemize}

\item \textbf{{\swaptail{bound}{success}}} event is for the
operations from line 5 to 7 in Fig.~\ref{fig:exp:mcs_lock} and takes
two arguments. The second argument is a boolean flag indicating
whether the previous ``last'' value of MCS lock was \invalidmcsval,
which means it records whether the if-statement at line 9 took the fast path or not.
The first argument is the \emph{bound number}, which is a key idea in
our development. Every client that invokes \lstinline$mcs_acquire$ has
to promise a bound for the critical section. This number
does not influence the compiled code in any way, but the
\emph{specification} says that it is invalid to hold the critical
section for longer than that (c.f. the counter $c_1$ in
Sec.~\ref{sec:representation-ghost}).
This bound number enables \emph{local} reasoning about liveness.
For the thread waiting for acquiring or releasing a lock,
its wait time can be estimated based on other threads' bound number. For the lock holder, it has to guarantee
to exit the critical section within its own bound. 
Thus, by showing that each thread follows this protocol,
we can derive the liveness property for the whole system.
(To be precise, the bound number is a limit on the number of events
that can get appended to the log, see the counter \lstinline$c1$ in
Sec.~\ref{sec:representation-ghost}.
Every CPU adds at least
one event every time it "does something", e.g. each loop iteration in \lstinline$mcs_release$ appends a GET\_NEXT
event, so
as we will see in
Sec.~\ref{sec:liveness-atomicity} this sufficies to give a bound of
the number of loop iterations in the lock acquire function. In the following we often speak of
``number of operations'', which does not mean single CPU instructions,
but instead whatever operation is represented by particular events.)


\item \textbf{\setnext{prev$\_$last}} event corresponds to the code at line 10. 
the \texttt{prev$\_$last} represents the \texttt{prev$\_$id} in the code.

\noindent\textbf{\getbusy{busy}} event shows the busy waiting in the acquire lock function.
The first argument will be true when the last value is same with the current CPU-id that calls the primitive which generates this event.
It will be false when the last value is not same with the current CPU-id that calls the primitive.
\end{itemize}

Next, other threes are enough to represent release lock in Fig.~\ref{fig:exp:mcs_lock}.

\begin{itemize}

\item \textbf{\castail{busy}} represents the atomic expression at line 21  in Fig.~\ref{fig:exp:mcs_lock}. 
In addition, the ``busy'' corresponds to the result of the \texttt{CAS} operation in Fig.~\ref{fig:exp:mcs_lock}.

\item  \textbf{\getnext} corresponds to the primitive that try to get the next value of the current CPU's node, and abstracts busy waiting in release lock function.

\item  \textbf{\setbusy} represents the last three lines in \texttt{mcs$\_$release}.
\end{itemize}

Those six events are used to show the functional correctness of
an MCS Lock. However, for clients that use the MCS Lock to build shared
objects they expose too much implementation details.
In Sec.~\ref{sec:liveness-atomicity} we will prove linearizability and
starvation freedom,  to replace them
with just two events.


\begin{itemize} 
\item \textbf{\waitlock{bound}} corresponds to lock acquire. The ``bound'' number in here is exactly same with the ``bound'' number in \swaptail{bound}{success} event.

\item \textbf{\rellock} corresponds to the lock release.
\end{itemize}

In addition to the above eight events, which are generated by the lock
acquire and release functions, the clients of the lock will also
generate events while they are in the critical section. Mutex locks in
CertiKOS are used to protect blocks of shared memory, so we call the
events generated by the client code \textbf{shared memory events}. The
final specification we prove will entail that a shared memory event
from CPU $i$ can only happen in the interval between a lock acquire
event for $i$ and a lock release event for $i$, which is how we
express the mutual exclusion property.

\section{Verification---Layer by layer}
\label{sec:verification}

\begin{figure}
\begin{center}
\includegraphics[width=\linewidth]{figs/mcslock/layer_overview}
\end{center}
\caption{MCS Lock Layers}
\label{fig:layeroverview}
\end{figure}

We build five layers, starting from a base
layer which represents the machine model that our compiled code will
run on.
Fig.~\ref{fig:layeroverview} shows the overall structure of our development.
For simplicity the figure only includes lock primitives, and not
primitives passed through from below.
In the figure, each big and outer rectangle means each layer in the MCS Lock Module, 
and small and inner rectangles in each layer implies the primitives defined in the layer.
The arrows show dependencies between adjacent layers,
for example the definition of \texttt{wait$\_$lock} in \texttt{MMCSLockOp}
uses three primitives (\texttt{mcs$\_$swap$\_$tail},
$\texttt{mcs$\_$set$\_$next}$, and $\texttt{mcs$\_$get$\_$busy}$) from the \texttt{MMCSLockAbsIntro} layer.

The layers \texttt{MCSMCurID} through \texttt{MMCSLockAbsIntro}
introduces getter and setter functions for accessing memory
(Sec.~\ref{subsec:lowestmachinemodel} and
\ref{subsec:abstractoperationlayer}). These layers also
contain logical primitives which record events to the log; we are in
effect manually implementing a model of concurrent execution by
extending a sequential operational semantics for C. 

The layer \texttt{MMCSLockOp} contains the C code from 
Fig.~\ref{fig:exp:mcs_lock}. This layer proves low-level
functional correctness, i.e. it reasons about the C code and
abstracts away details about memory accesses, integer overflows, etc,
to expose an equivalent specification written as a Coq
function (Sec.~\ref{subsec:atomicoperation}).

The two top layers, \texttt{MQCSLockOp} and \texttt{MHMCSLockOp}, do not introduce any new primitives.
They simplify the specifications of 
the release- and acquire lock functions (\texttt{pass$\_$lock} and
\texttt{wait$\_$lock}), i.e. each layer ascribes a different
specification (with a different log replay function and a set of events)
to the same C function. Those specification names are notated inside the square bracket in Fig.~\ref{fig:layeroverview}.

The layer \texttt{MQMCSLockOp} adds ghost state, keeping track of a
queue of waiting CPUs.
(Sec~\ref{sec:representation-ghost}). This queue is key to the liveness proof but is not explicitly represented in the C implementation.
The top layer \texttt{MHMCSLockOp} proves starvation freedom and liveness
(Sec~\ref{sec:liveness-atomicity}). This lets us ascribe atomic
specifications where taking or releasing a lock generates just a
single event to the log.




\subsection{Memory operations layers}
\label{subsec:lowestmachinemodel}

Although we glossed over this in Fig.~\ref{fig:exp:mcs_lock}, our
actual C implementations of \lstinline$msc_acquire$ and
\lstinline$msc_release$ do not access memory directly.  Instead, they call
a collection of helper functions with names like
\lstinline$mcs_set_next$. The lowest two layers in our proofs
are devoted to implementing these helper functions.
The key concern is to make sure that the events that get appended to the log
correspond to the actual actions to the memory. Some of that can be proven, 
but some parts of this layer are trusted as a part of the machine model.

We first describe the first and the lowest tuple in our proofs, ($L_0$, $M$, $L_1$).
The interface ($L_0$) represents the machine model that our compiled code will run on.
All primitives defined in $L_0$ are part of the trusted computing base, and correspond to empty functions in our compiled code.

Eight of the primitives in $L_0$  are closely related to the MCS Lock verification:
$$
\begin{small}
\begin{array}{c}
\{\mathrm{atomic\_mcs\_log},\ \mathrm{atomic\_mcs\_SWAP},\ \mathrm{atomic\_mcs\_CAS},\ \mathrm{mcs\_init\_node\_log},\\
\mathrm{mcs\_GET\_NEXT\_log},\ \mathrm{mcs\_SET\_NEXT\_log},\ \mathrm{mcs\_GET\_BUSY\_log},\ \mathrm{mcs\_SET\_BUSY\_log}\} \\
\end{array}
\end{small}
$$
Two primitives, \texttt{atomic$\_$mcs$\_$SWAP} and \texttt{atomic$\_$mcs$\_$CAS} are for the two atomic instructions {\em fetch-and-store} and {\em compare-and-swap}, and will be further discussed below.

The other six are used to update the log.  As we noted in
Sec.~\ref{subsec:eventlogandoracle}, the log is part of the abstract
state. Ordinary assembly instructions only modify physical memory, not
abstract state, so in order for programs to be able to append events to
the log we include these six primitives in $L_0$. 
For example, the specification of \texttt{mcs$\_$SET$\_$NEXT$\_$log} updates the log by adding one (\setnext{prev$\_$id}) event as follows:
\lstinputlisting[language = Caml]{source_code/mcslock/mcs_set_next_log.v} 
In the compiled code, these primitives appear as empty functions that do nothing, they are only used to modify the logical state.


The code $M$ in the layer contains the 
functions which actually modifies the memory in the way the event announces.
Each function in $M$ calls the corresponding primitive from
\texttt{MCSMCurID} inside the function to add the event to the log.
For example, \texttt{mcs$\_$SET$\_$NEXT}, one function in $M$, writes
to \texttt{next} and also calls
the empty function \texttt{mcs$\_$SET$\_$NEXT$\_$log}:
\lstinputlisting [language = Caml]{source_code/mcslock/mcs_set_next.c}

\begin{figure}
\begin{center}
\includegraphics[width=\linewidth]{figs/mcslock/layer3}
\end{center}
\caption{The structure of the memory operations layer}
\label{fig:layer-struct-mcs-verification}
\end{figure}

The interface \texttt{MMCSLockIntro} contains the high level specification for each function defined in $M$. 
The high level specifications work on the log instead of the exact memory slot \lstinline$LK$.
Therefore, after proving the {\em refinement} between the memory (\lstinline$LK$ in Fig.~\ref{fig:layer-struct-mcs-verification})
and the abstract state (\emph{log} in Fig.~\ref{fig:layer-struct-mcs-verification}), we only need to care about the abstract state.

For the refinement proof, we need two more ingredients.
The first one is a \emph{log replay function}.
A log is merely a list of events, but what specifications need to know is what the state of the system will look like after those events have executed,
and a replay function calculates that. 
Different layers may define different replay functions in order to interpret the same log in a way that suits their proofs.
Therefore, we have introduced the proper log replay function in several layers, and prove the relationship between the result of them when we introduced the new one.
In $L_1$, we define \texttt{CalMCSLock}, which has the following type:
\lstinputlisting [language = Caml] {source_code/mcslock/lowreplaytype.v}
where
\lstinputlisting [language = Caml] {source_code/mcslock/lowmcsstruct.v}
The return type of this log replay function closely corresponds to C data structures, which makes it easy to prove the refinement. (\lstinline$ZMap$ is a finite map from \lstinline$Z$ to \lstinline$bool*Z$.)
The second ingredient is a relation $R$ which shows the relationship between the concrete memory in underlay $L_0$ and the abstract state in overlay $L_1$.
As a part of $R$, we define \texttt{match$\_$MCSLOCK} as follows:


\begin{definition}[\texttt{match$\_$MCSLOCK}]
Suppose that `loc' is among the proper field accessors for the MCS Lock (i.e. `\lstinline$last$', `\lstinline$ndpool[i].next$', or  `\lstinline$ndpool[i].busy$' when `$0 \leq i <$  \lstinline$TOTAL_CPU$'). And, assuming that `\lstinline$lk_id$' is a lock identifier satisfies `$0 \leq$ \lstinline$lk_id$ $< \mathrm{lock\_range}$' and \lstinline$l$ is a shared log. Then define 

\begin{center}
  \begin{tabular}{c}
    \lstinline$match_MCSLOCK (l: Log) (b: block) loc$\\
%$\forall$ \lstinline$(l: Log) (b: block) lk_id loc,$\\
iff \lstinline$($$\exists$ \lstinline$val, Mem.load Mint32 m b loc = Some(val)$ $\wedge$  \lstinline$Mem.valid_access m b loc$\\
$\wedge$ \lstinline$(CalMCSLock(l) = Some(mcsval) -> loc$$_{a}$\lstinline$@mcsval = val))$
\end{tabular}

\end{center}
when `\lstinline$loc$$_{a}$\lstinline$@mcsval$' represents the corresponding 
value to the `\lstinline$loc$$_{a}$' in the `\lstinline$mcsval$' 
and `\lstinline$loc$$_{a}$' corresponds to the value of `$loc$'.
\end{definition}

Intuitively, the definition says that the value that
\lstinline$CalMCSLock$ calculates from the log always corresponds to the value 
in the memory with the same identifiers. The memory access functions \lstinline$Mem.load$ and \lstinline$Mem.valid_access$ are
from CompCert's operational semantics for C.
Using the definition, we prove one theorem for each primitive, which
shows that the memory refines the shared log. E.g., for \lstinline$mcs_SET_NEXT$ we prove:

\begin{theorem}[Simulation for $\mathrm{\texttt{mcs}}\_\mathrm{\texttt{SET}}\_\mathrm{\texttt{NEXT}}$]
  \label{thm:machine-state-refinement} Let $R$ be the relation defined as \lstinline$match_MCSLOCK$ 
over \lstinline$LK@$$mem$ and \lstinline$LOG@$$A_1$, 
identity relation for other parts of $mem$, $A_0$ and $A_1$. Then
 \begin{center}
 \begin{tabular}{c}
$ \forall (\mathrm{\texttt{m}}_{1} \ \mathrm{\texttt{m}}_{1}'\ \mathrm{\texttt{m}}_{0} : mem)\  (\mathrm{\texttt{d}}_{0} \ : A_0)\ (\mathrm{\texttt{d}}_{1} \ \mathrm{\texttt{d}}_{1}' : A_1),$ \\
$ \mbox{if } \mathrm{\texttt{mcs\_SET\_NEXT}}_{L_1}(v, \mathrm{\texttt{m}}_1, \mathrm{\texttt{d}}_1) = \mathrm{\texttt{Some}}(\mathrm{\texttt{m}}'_1, \mathrm{\texttt{d}}'_1) \mbox{ and }
  R\ (\mathrm{\texttt{m}}_1, \mathrm{\texttt{d}}_1)\ (\mathrm{\texttt{m}}_0, \mathrm{\texttt{d}}_0),$\\
  $ \mbox{ then there exists } (\mathrm{\texttt{m}}_{0}' : mem)\ (\mathrm{\texttt{d}}_{0}' : A_0), \mbox{ such that}$\\
$  \mathrm{\texttt{mcs\_SET\_NEXT}}_{L_0}(v, \mathrm{\texttt{m}}_0, \mathrm{\texttt{d}}_0) = \mathrm{\texttt{Some}}(\mathrm{\texttt{m}}'_0, \mathrm{\texttt{d}}'_0) \mbox{ and}
  R\ (\mathrm{\texttt{m}}'_1, \mathrm{\texttt{d}}'_1)\ (\mathrm{\texttt{m}}'_0, \mathrm{\texttt{d}}'_0).$ 
   \end{tabular}
 \end{center}
\end{theorem}


 \begin{theorem}[Machine State Refinement]
 \label{thm:machine-state-refinement} Let's assume the following conditions:
 1) $L_0$ and $L_1$ are underlay and overlay layers;
 2) $mem_0$ and $mem_1$ are memories for $L_0$ and $L_1$;
 3) and, $A_0$ and $A_1$ are abstract datum for $L_0$ and $L_1$, respectively.
 With the given $R$, defined as \lstinline$match_MCSLOCK$ 
 over \lstinline$MCS_LOC@$$mem_0$ and \lstinline$LOG@$$A_1$, 
 identity relation for other parts of $mem_0$ and $mem_1$, and $A_0$ and $A_1$, 
 The specification for the function $f$, $\sigma_f$, in $L_1$ refines that in $L_0$ when:
 \begin{center}
 \begin{tabular}{c}
 $\forall ($\lstinline$m$$_{0} \ $\lstinline$m$$_{0}' : mem_0)\ ($\lstinline$m$$_{1} \ $\lstinline$m$$_{1}' : mem_1)\ ($\lstinline$s$$_{0} \ $\lstinline$s$$_{0}' : A_0)\ ($\lstinline$s$$_{1} \ $\lstinline$s$$_{1}' : A_1),$\\
 $(L_0 \vdash \sigma_f : (\_, $\lstinline$m$$_0, $\lstinline$s$$_0) \rightarrow (\_, $\lstinline$m$$_0', $\lstinline$s$$_0')) \wedge\
 (L_1 \vdash \sigma_f : (\_, $\lstinline$m$$_1, $\lstinline$s$$_1) \rightarrow (\_, $\lstinline$m$$_1', $\lstinline$s$$_1')) \wedge$\\
 $(R\ ($\lstinline$m$$_1, $\lstinline$s$$_1)\ ($\lstinline$m$$_0, $\lstinline$s$$_0) \rightarrow R\ ($\lstinline$m$$_1', $\lstinline$s$$_1')\ ($\lstinline$m$$_0', $\lstinline$s$$_0'))$\\
 \end{tabular}
 \end{center}
 \end{theorem}

%
%The log in $adt$, however, requires the way to infer the current state about the shared object. 
%To do that, log replay functions are introduced to interpret the current state with the given shared log.
%Those functions gets the well-formed shared log and returns the value of the defined data structure that can represent the current state.
%
%For several purpose, we have introduced three log replay functions during the verification. 
%First, \texttt{CalMCSLock} is the replay function that we have used until we show the functional correctness. 
%The next one, \texttt{QS$\_$CalLock} evaluates the shared log and establish the queue data structure to show the FIFO and starvation freedom properties of the MCS Lock. 
%The last log replay function is \texttt{H$\_$CalLock}, which can be used after we wrapped the all the lock acquire and release events as one events. 
%Those functions essentially have the same meaning with the well-formed shared log inputs, but we always have to show that the result generated by those log replay functions are satisfied by refinement relationship. 
%To do them, we have introduced logical layers.
%Fig.~\ref{fig:layer-struct-mcs-verification} (b) shows the example of building a logical layer.
%In the layer, no additional primitives are introduced, but the we have proved that the new abstract state in the {\em overlay} layer generated by the new replay function refines the abstract state in the {\em underlay} layer with the previous replay function.

One interesting variation is the semantics
for fetch-and-store and compare-and-swap. These instructions are not
formalized in the x86 assembly semantics we use, so we cannot prove
that replay function is correctly defined. Instead we modify the last
(``pretty-printing'') phase of the compiler so that these primitive calls map to assembly
instructions, and one has to trust that they match the specification.

\subsection{Event interleaving layer}
\label{subsec:abstractoperationlayer}

After abstracting memory accesses into the operation on the log, we
then need to model possible interleaving among multiple CPUs. In
our approach, this is done through a new layer which adds \emph{context queries}.

The concurrent context $\oracle$ (sometimes called the ``oracle'') is
a function of the CPU-id and the log which has the type
% ``\texttt{Z} $\rightarrow$ \texttt{list event} $\rightarrow$ \texttt{list event}''.
$\oracle:$ \lstinline$Z -> list event -> list event$.
It is one component of the abstract state, and it represents \emph{all
the other CPUs}, from the perspective of code running on the current
CPU.  Any time a program does an operation which reads or writes
shared memory, it should first query $\oracle$ by giving it the
current log. The oracle will reply with a list of events that other
CPUs have generated since then, and we update the log by appending
those new events to it.

Primitive specifications are provided read-only access to a context
$\oracle$ by the verification framework, and the framework also
guarantees that two properties are true of $\oracle$: 1) the returned
partial log from the oracle query does not contain any events
generated by the given CPU-id; and 2) if we query the oracle with the
well-formed shared log, the updated log after the oracle query will
be well-formed.
The first assumption is straightforward because the purpose of the oracle is to represent the behaviour of others' operation on the shared object.
The second one is also trivial when we prove 1) the initial shared log satisfy the well-formed condition, and 2) all the operations on the shared object with the given well-formed log return a well-formed shared log.
Those two assumptions, however, do not reduce the generality of the oracle, and the oracle can capture the proper interleaving that we hope to achieve in the MCS Lock verification.

Similar to Sec.~\ref{subsec:lowestmachinemodel}, we provide primitives in $L_0$ which query $\oracle$ and extend the log.
Then in this second layer, we can model abstract operations with interleaving.
For example, \texttt{mcs$\_$SET$\_$NEXT} can be re-written as
\lstinputlisting [language = Caml] {source_code/mcslock/mcs_set_next_low_charac.c}
by using the logical primitive which corresponds to the oracle query
(The function \texttt{mcs$\_$log} refines the semantics of \texttt{atomic$\_$mcs$\_$log} in the lowest layer by the \texttt{match$\_$MCSLOCK} relation).
To model the interleaving, all the setter and getter functions defined
in Sec.~\ref{subsec:lowestmachinemodel} should be combined with the
oracle query.

\paragraph{Trust in the machine model}
Some of the design decisions in the memory access
layers have to be trusted, so the division between machine model and
implementation is unfortunately slightly blurred.
Ideally, we would have a generic machine model as proposed by Gu et
al~\cite{certikos16}, where memory is partitioned into thread-local
memory (no events), lock-protected memory (accesses generate PUSH/PULL
events), and atomic memories (each access
generates one READ/WRITE/SWAP/etc event).  However, our starting point
is the CompCert x86 semantics, which was designed for single-threaded
programs, and does not come with a log, so we add a log and memory access
primitives ourselves.
But because the spinlock module is the only code in the OS that uses
atomic memory, we do not add a generic operation called
read\_word etc. Instead we take a short-cut and specify the particular
6 memory accesses that the lock code uses: \lstinline$mcs_get_next$ etc.
For these procedures to correctly express the intended semantics,
there are two trusted parts we must take care to get right. First,
each access to non-thread-local memory must generate an event, so we
must not forget the call to
\texttt{mcs$\_$SET$\_$NEXT$\_$log}. 
Second, to account for
interleavings between CPUs (and not accidentally assume that consecutive
operations execute atomically) we must not forget the call to
\texttt{mcs$\_$log} after each access.


\subsection{Low-level functional specification}
\label{subsec:atomicoperation}

Using the primitives that we have defined in lower layers, we prove the correctness of lock acquire, \lstinline$mcs_acquire$, and release, \lstinline$mcs_release$.
The target code in this layer is identical to the code in Fig.~\ref{fig:exp:mcs_lock} except two aspects. 
First, we replaced all operations on memory with the getters and setters described in Sec.~\ref{subsec:abstractoperationlayer}.
Second, \lstinline$mcs_acquire$ has one more
 argument, which is the bound number for the client code of the lock.

Since the functions defined in
Sec.~\ref{subsec:abstractoperationlayer} already abstract interleaving
of multiple CPUs, the proofs in this layers work just like sequential
code verification. We find out the machine state after the function
call by applying the C operational semantics to our function
implementation, and check that it is equal to the desired state
defined in our specification.

However, writing the specifications for these functions is slightly subtle, 
because they contain
while-loops without any obviously decreasing numbers. Since our
specifications are Coq functions we need to model this by structural
recursion, in some way that later will let us show the loop is terminating.
So to define the semantics of \texttt{mcs$\_$wait$\_$lock},
we define an auxiliary function
\lstinline$CalMCS_AcqWait$ which describes the
behavior of the first $n$ iterations of the loop: each iteration
queries the the environment context $\oracle$, replays the log to see if if \lstinline$busy$ is now \lstinline$false$, and appends a \texttt{GET\_BUSY} event.
If we do not succeed within $n$ iterations the function is undefined (Coq \texttt{None}).
Then, in the part of the  specification for the  acquire lock 
function (\texttt{CalMCS$\_$AcqWait}) where we need to talk about the while loop,
we say that it loops for some ``sufficiently large'' 
number of iterations \lstinline$CalWaitLockTime tq$. 
\lstinputlisting[language = Caml]{source_code/mcslock/waitlockloop.v}
The function \lstinline$CalWaitLockTime$ computes a suitable 
number of loop iterations based on \lstinline$tq$, the time-bounds  which each of the queuing CPUs promised to respect.
We will show how it is defined in Sec.~\ref{sec:liveness-atomicity}. 
However, in \emph{this part} of the proof, the definition doesn't matter. 
Computations where $n$ reaches 0 are considered crashing, and our
ultimate theorem is about safe programs, so when proving that the C
code matches the specification we only need to 
consider cases when \texttt{CalMCS$\_$AcqWait} returned \texttt{(Some l)}.
It is easy to show in a downward simulation that the C loop can match any such finite run, 
since the C loop can run any number of times.
%
%\begin{theorem}{Acquire Lock Functional Correctness}
%Acquire lock function, `f$\_$mcs$\_$wait$\_$lock'', satisfies the specification, ``mcs$\_$wait$\_$lock$\_$spec$\_$low.
%\end{theorem}
%
%\begin{theorem}{Release Lock Functional Correctness}
%Release lock function, ``f$\_$mcs$\_$wait$\_$lock'', satisfies the specification, ``mcs$\_$wait$\_$lock$\_$spec$\_$low.
%\end{theorem}
%
%
%The verified primitives in this layer generate multiple events during the execution. 
%Thus, other steps of refinements need to be done to merge those multiple events into the single one for each function to make them as atomic primitives. 
%From the section, we will discuss how we merge those events into a single atomic lock acquire event, \waitlock{bound}, and a single atomic lock release event, \rellock. 
%
%
%We have defined \mmcslockop\ layer to show the functional correctness of acquire lock and release lock. 
%We named them as \texttt{mcs$\_$wait$\_$lock} and \texttt{mcs$\_$pass$\_$lock}.
%Since this layer is only focusing on the functional property of an MCS Lock, the layer does not contain any proofs related to starvation freedom yet. 
%We, however, have to show that the busy waiting will terminate within certain times. 
%With the fairness scheduling, which implies that all CPUs will have a fair chance in generating events, we could introduce two numbers for loop termination proofs.
%
%The \texttt{CalPassLockLoopTime} is the constant number for busy waiting termination proofs of  \texttt{mcs$\_$pass$\_$lock}. 
%Note that there are no arguments for the number because the loop in the  \texttt{mcs$\_$pass$\_$lock} is not relevant with other CPUs bound numbers. 
%
%The \texttt{CalPassLockLoopTime} is the constant number for busy waiting termination proofs of  \texttt{mcs$\_$pass$\_$lock}. 
%Note that there are no arguments for the number because the loop in the  \texttt{mcs$\_$pass$\_$lock} is not relevant with other CPUs bound numbers. 
%
%The \texttt{CalWaitLockTime} is the number for the termination proofs of \texttt{mcs$\_$wait$\_$lock}. 
%When we look at the \texttt{CalWaitLockTime} definition we need to calculate the number using \texttt{tq} which is the list of bound numbers for client codes that are waiting locks.
%This implies that we prove that the function will be terminated with a certain time that is related to the bound numbers in the lock waiting queues.
%It, however, does not have any proofs that those lock wait loops can be merged into the single events.
%Making them into the atomic one are the next steps in our verification processes.

\subsection{Data representation and ghost state}
\label{sec:representation-ghost}

From here on, we never have to think about C programs again.  All the
subsequent reasoning is done on Coq functions manipulating ordinary
Coq data types, such as lists, finite maps, and unbounded integers.
Verifying functional programs written in Coq's Gallina is exactly the
situation Coq was designed to deal with. However, the data computed
by the replay function in in the previous layer still corresponds
exactly to the array-of-structs that represents the state of the lock
in memory.
In particular, the intuitive reason that the algorithm is fair is that
each CPU has to wait in a queue, but this conceptual queue is not identical with
the linked-list in memory, because the next-pointers may not be set.

In order to keep the data-representation and liveness concerns separate,
we introduce an intermediate layer, which keeps the same sequence of operations and same log of events, 
but manipulates an \emph{abstracted data representation}.
We provide a different replay function with the type 

\begin{lstlisting}
QS_CalLock : Multi_Log -> option (nat * nat * head_status * list Z * ZSet.t * list nat)
\end{lstlisting}

The tuple returned by this replay function provides the information we
need to prove liveness, 
similar to the concepts used in the informal
proof in Sec.~\ref{sec:overview}. 
The meaning of a tuple
\lstinline$(c1, c2, b, q, slow, t)$ is as follows:
\lstinline$c1$ and \lstinline$c2$ are upper bounds on how many more operations 
the CPU which currently holds the lock will generate as part of the critical section and of 
releasing the lock, respectively. 
They are purely logical ghost state but can be deduced from the complete
history of events in the system.
\lstinline$b$ is either  \lstinline$LHOLD$ or \lstinline$LFREE$, 
the lock status of the head of the queue.
\lstinline$q$ is the list of the CPUs currently waiting for the lock, 
and \lstinline$t$ is the list of bound numbers that 
corresponds to each element in \lstinline$q$.
\lstinline$slow$ is a finite set which represents the subset of CPUs in \lstinline$q$ that have not yet executed their \emph{set next} operation.  
Our liveness proof is based on the fact that each CPU only needs to wait for CPUs that are ahead of it in \lstinline$q$.

Some of this information is implicit in the state of the memory, while some of it (for example \lstinline$c1$ and \lstinline$c2$) is purely ghost state. But in any case, it can be deduced from the complete history of events in the system, which is what the replay function \lstinline$QS_calLock$ does. We define it by recursion on the list $l$, computing the new state after each event. A few representative cases of the function are shown in Fig.~\ref{fig:QS_CalLock}.  For example, the event \lstinline$SET_BUSY$ indicates that a thread releases the lock. If the CPU $i$ is already the  front of the queue $q$, it currently holds the lock (\lstinline$LHOLD$), and the bound \lstinline$c2$ has not yet reached zero, and $i$ is not slow, then generating this event will reset the lock status to \lstinline$LEMPTY$ and remove the head element ($i$) from \lstinline$q$ and \lstinline$t$. In any of those side conditions are not satisfied, on the other hand, the replay function is undefined (\lstinline$None$). Similar considerations hold executing memory operations (you must be in the critical section, and it decrements \lstinline$c1$) and querying the busy flag (you must have executed \lstinline$SET_NEXT$ first).

\begin{figure}
\lstinputlisting [language = Caml, firstline=1] {source_code/mcslock/midlogreplay_short.v}
\caption{The replay function \lstinline$QS_CalLock$}
\label{fig:QS_CalLock}
\end{figure}


\paragraph*{Invariant} The replay function plays two different roles. When it returns \lstinline$Some v$, for some tuple \lstinline$v$, it describes what the current state of the system is, which lets us write the specifications for the primitives. At the same time, the cases where the function is defined to return \lstinline$None$ are also important, because this can be read as a description of events that are \emph{not} possible. For example, from inspecting the program, we know that each CPU will create exactly one \lstinline$SET_NEXT$ event before it starts generating \lstinline$GET_BUSY$ events, and this fact will be needed when doing the proofs in the later layers (Sec.~\ref{sec:liveness-atomicity}). By taking advantage of  the side conditions in the replay function, we can express all the invariants about the log in a single statement, ``the replay function is defined'':
\begin{center}
\begin{tabular}{c}
$\exists$ \lstinline$c1 c2 b q s t. QS_CalLock(l) = Some(c1, c2, b, q, s, t)$\\
\end{tabular}
\end{center}


This type for the replay function is optimized to only expose exactly the information needed by the subsequent liveness proof. We need to expose the queue and the set of slow CPUs in order to define the termination measure $M$ (Sec \ref{sec:liveness-atomicity}). On the other hand, this is not enough information to bridge the gap from the low-level functional specification. In order to show that the memory cells for a valid linked-list and therefore respects the queue ordering, we need to track exactly what the valid state transitions are. So inside the ghost state layer, we also introduce a different relation  \lstinline$Q_CalMCSLock$ which is mostly the same as \lstinline$QS_CalLock$ but written as an (functional) inductive relation in Coq instead of a recursive function, and which has even more preconditions for when it is defined. We then add one more condition in the layer invariant saying that \lstinline$Q_CalMCSLock$ and \lstinline$QS_CalLock$ output the same result. Most of the proofs inside the ghost layer are done using the relation instead of the function. For simplicity, we will ignore the distinction in the rest of the paper, and write the lemma statements about \lstinline$QS_CalLock$ even if they used \lstinline$Q_CalMCSLock$ in the actual Coq code.


To show that the ghost layer refines the previous layer, we show a
one-step forward-downward refinement: if the method from the higher
layer returns, then method in the lower layer returns a related
value. For this particular layer the log doesn't change, so the
relation in the refinement is just equality, and the programmer just
has to show that the lower-level methods are at least as defined and
that they return equal results for equal arguments.


As we prove this, we need lemmas to show that we can satisfy the preconditions for the operations in the lower layer, by relating the data in \lstinline$la$ to the abstract queue.  For example, when trying to take the lock, the high level specification checks if the current CPU is at the head of \lstinline$q$, which the low specification tests if the \lstinline$busy$ field is true, so we need Lemma~\ref{lem:Q_CalMCSLock_tail_is_busy} to show that they will follow the same path of code. 

\begin{lemma}[tail soundness]
If \lstinline$CalMCSLock l = Some (tl, la, tq)$ and $QS\_CalLock = Some (c1,c2,q,s,t)$, then \lstinline$tl$ is \texttt{NULL} if $q = \nil$, and \lstinline$tl$ the last element of \lstinline$q$ if $q \neq \nil$.
\end{lemma}

\begin{lemma}[next-correctness]
Let's assume that \lstinline$CalMCSLock l = Some (tl, la, tq)$ and \lstinline$QS_CalLock = Some (c1,c2,q_1++$$i$\lstinline$::$$j$\lstinline$::q_2,s,t)$, then \lstinline$lock_array[$$i$\lstinline$] = (_, TOTAL_CPU)$ if $j \in $ \lstinline$s$, and 
\lstinline$lock_array[$$i$\lstinline$] = (_, $$j$\lstinline$)$ if $j \not\in$ \lstinline$s$.
\end{lemma}

\begin{lemma}[tail is busy]
\label{lem:Q_CalMCSLock_tail_is_busy}
If \lstinline$CalMCSLock l = Some (tl, la, tq)$ and \lstinline$QS_CalLock = Some (c1,c2,$\lstinline$i$$::q,s,t)$ and $j \in$ \lstinline$q$, then \lstinline$lock_array[$$j$\lstinline$] = (true, _)$.
\end{lemma}

\begin{theorem}[simulation for the ghost layer] Suppose \lstinline$d$ satisfies the invariant and
\lstinline$wait_qslock_spec(d)= Some(d')$. Then \lstinline$mcs_acquire_spec(d)= Some(d')$.
\end{theorem}


\subsection{Liveness and atomicity}
\label{sec:liveness-atomicity}

The specification in the previous section is still too low-level and
complex to be usable by client code in the rest of the system.  First,
the specification of the \lstinline$mcs_acquire$ and
\lstinline$mcs_release$ primitives contain loops, with complicated
bounds on the number of iterations, which clients certainly will not
want to reason directly about.  More importantly, since the
specifications generate multiple events, clients would have to show
all interleavings generate equivalent results.

To solve this we propose a basic design
pattern: build a new layer with \emph{atomic specifications},
i.e. each primitive is specified to generate  a single event.
For an atomic layer there is a
therefore a one-to-one mapping between events and primitives, and the global log
can be seen as a record of which primitives were invoked in which
order. Thus, the refinement proof which ascribes an atomic
specification proves once and for all that overlapping and interleaved
primitive invocations give correct results.
In this layer, the specifications only use three kinds 
of events: taking the lock (\lstinline$WAIT_LOCK n$),
releasing it (\lstinline$PASS_LOCK$), and modifications of the shared
memory that the lock protects (\lstinline$TSHARED _$).

Fig.~\ref{fig:hswaitlockspec} shows the final specification for the
wait primitive. We show this one in full detail, with no elisions,
because this is the interface that clients use. First, the
specification for the lock acquire function itself
(\lstinline$mcs_wait_hlock_spec$) takes the function arguments
\lstinline$bound$, \lstinline$index$, \lstinline$ofs$, and maps an
abstract state (\lstinline$RData$) to another. When writing this
specification we chose to use two components in the abstract state, the
log (\lstinline$multi_log$) and also a field (\lstinline$lock$) which
records for each numbered lock if it is free (\lstinline$LockFalse$)
or owned by a CPU (\lstinline$LockOwn$). The \lstinline$lock$ field is
not very important, because the same information can also be computed
from the log, but exposing it directly to clients is sometimes more
convenient.

The specification returns \lstinline$None$ in some
cases, and it is the
responsibility of the client to ensure  that does not
happen. So clients must ensure that: the CPU is in kernel/host
mode (for the memory accesses to work); the index/offset (used to
compute the lock id) are in range; the CPU did not already hold the
lock (\lstinline$LockFalse$); and the log is well-formed
(\lstinline$H_CalLock l'$ is defined, which will always be the case if
\lstinline$H_CalLock l$ is defined).  When all these preconditions are
satisfied, the specification queries the context once, and appends a
single new \lstinline$WAIT_LOCK$ event to the log.
Fig.~\ref{fig:hswaitlockspec} also shows the replay function
\lstinline$H_CalLock$.
It has a much simpler type than \lstinline$QS_CalLock$ in the
previous layer, because we have abstracted the internal state of the lock
to just whether it is free (\lstinline$LEMPTY$),
held (\lstinline$LHOLD$), and if taken, the CPU id (\lstinline$Some i$)
of the holder of the lock. Unlike the three bound numbers in the
previous layer, here we omit the numbers for the internal lock
operations and only keep the bound \lstinline$self_c$ for the number
of events generated during the critical section. Again, it's the
client's responsibility to avoid the cases when \lstinline$H_CalLock$
returns \lstinline$None$. In particular, it is only allowed to release
the lock or to generate memory events if it already holds the lock
(\lstinline$zeq i i0$), and each memory event decrements the counter,
which must not reach zero. The client calling \lstinline$wait_lock$
specifies the initial value $n$ of the counter, promising to take at
most $n$ actions within the critical section.


\begin{figure}
\lstinputlisting [language = Caml, firstline=1] {source_code/mcslock/highlogreplay.v}
\lstinputlisting [language = Caml, firstline=1] {source_code/mcslock/hswaitlockspec.v}
\lstinputlisting [language = Caml, firstline=1] {source_code/mcslock/hspasslockspec.v}
\caption{The final, atomic, specification of the aquire lock function and the release lock function.}
\label{fig:hswaitlockspec}
\end{figure}


In the rest of the section, we show how to prove that the function
does in fact satisfy this high-level atomic specification.
Unlike the previous layers we considered, in this case the log in the
upper layer differs from the one in the lower layer. For example, when
a CPU takes the lock, the log in the upper layer just has the one
atomic event (\lstinline$WAIT_LOCK n$), while the log in the underlay
has a flurry of activity (swap the tail pointer, set the next-pointer,
repeatedly query the busy-flag).
Because the log represents shared data, our framework requires a
slightly strengthened refinement theorem for the log-component of the
state. Usually a refinement simulation works by specifying some
relation $R$ between machine state and abstract state, and then
proving that the state transitions preserve the relation. Indeed, for
thread-local data this is exactly what CertiKOS does also.

However, an arbitrary relation $R$ is not enough for local reasoning
about concurrent programs.  For example, suppose one particular
execution of the system generates the log \lstinline$l$$_L$.  A normal simulation
theorem for CPU 1 would tell us that there \emph{exists} a log \lstinline$l$$_H$
that meets CPU 1's local specification and satisfies the relation
($R$ \lstinline$l$$_H,$\lstinline$l$$_L$). Similarly, the local proof for CPU 2 would say there
exists some log \lstinline$l'$$_H$. But in order to derive a simulation for the
entire system, we need the constraint that that \lstinline$l$$_H$ is equal to
\lstinline$l'$$_H$. Our solution is to require the relation $R$ to be a function $f$. 
In other words, when proving the simulation,
we find a function $f$ for the logs, such that $f($\lstinline$l$$_L) = $\lstinline$l$$_H$.


As for the MCS Lock, we define a function \lstinline$relate_mcs_log$ from the
implementation log to the atomic log. Fig.~\ref{fig:logsequence}
shows by example what it does. It keeps the shared memory events as
they are, discards the events that are generated while a CPU wait for
the lock, and maps just the event that finally takes or releases the
lock into \lstinline$WAIT_LOCK$ and \lstinline$REL_LCOK$.
\begin{figure*}
\includegraphics[width=\textwidth]{figs/mcslock/logsequence}
\caption{Log Sequence and Log Refinement Example}
\label{fig:logsequence}
\end{figure*}

We then prove a one-step refinement theorem from the atomic specification 
to the implementation, in other words, that if a call to the atomic primitive returns a 
value, then a call to its implementation also returns with a related log:


\begin{theorem}[MCS Wait Lock Exist]
  \label{thm:mcs_wait_lock_exist}
  Suppose $\mathrm{\texttt{d}}_{\scriptsize\\mathrm{\texttt{MHMCSLockOp}}}$ and $\mathrm{\texttt{d}}_{\scriptsize\mathrm{\texttt{MQMCSLockOp}}}$ satisfy the layer
  invariants and are related by $\mathrm{\texttt{relate\_mcs\_log}}(\mathrm{\texttt{d}}_{\scriptsize\mathrm{\texttt{{MQMCSLockOp}}}}) = 
  \mathrm{\texttt{d}}_{\scriptsize\mathrm{\texttt{MHMCSLockOp}}}$. \\
If $\mathrm{\texttt{wait\_hlock\_spec}}(\mathrm{\texttt{d}}_{\scriptsize\mathrm{\texttt{MHMCSLockOp}}}) = \mathrm{\texttt{Some}}(\mathrm{\texttt{d}}'_{\scriptsize\mathrm{\texttt{MHMCSLockOp}}})$, then there exists some $\mathrm{\texttt{d}}'_{\scriptsize\mathrm{\texttt{MQMCSLockOp}}}$\\
  which is $\mathrm{\texttt{wait\_qslock\_spec}}(\mathrm{\texttt{d}}_{\scriptsize\mathrm{\texttt{MQMCSLockOp}}}) = \mathrm{\texttt{d}}'_{\scriptsize\mathrm{\texttt{MQMCSLockOp}}}$ and is related with $\mathrm{\texttt{d}}'_{\scriptsize\mathrm{\texttt{MHMCSLockOp}}}$\\
   by $\mathrm{\texttt{relate\_mcs\_log}}(\mathrm{\texttt{d}}'_{\scriptsize\mathrm{\texttt{MQMCSLockOp}}}) = \mathrm{\texttt{d}}'_{\scriptsize\mathrm{\texttt{MHMCSLockOp}}}$.
\end{theorem}

The proof  requires a \emph{fairness assumption}.
A CPU cannot take the lock until the previous CPU releases it, 
and the previous CPU cannot release it if it never gets to run. 
At its most fundamental, the CertiKOS machine model is a nondeterministic 
transition system (which is subsequently viewed as a log of events), 
and there is nothing in the basic model that ensures fairness, 
so we have to add an extra assumption somewhere. In principle it would be 
possible to modify the machine model itself, and then pass the fairness assumptions 
along in the specification of each layer until we reach the layers related to mutex locks, 
but in our development we choose a more expedient solution, and express
the fairness assumption as an extra axiom talking about the logs 
in the data representation layer (Sec.~\ref{sec:representation-ghost}). 
By doing that, our framework can use the previous machine 
model as it is, and can reuse most previous proofs.

Specifically, we assume that there exists some constant $F$ (for ``fairness'') such that no CPU that enters the queue has to wait for more than $F$ events until it runs again. 
In Coq we provide a function \lstinline$CalBound$ which ``counts down'' 
until CPU $i$ gets a chance to 
execute (\lstinline$CalBound : Z -> MultiLog -> nat$).
\lstinputlisting [language = Caml, firstline=1] {source_code/mcslock/calbound.v}

The fairness assumption, then is that for all logs \lstinline$l$, 
when the low level log replay function returns a 
value (\lstinline$QS_CalLock(l) = Some(c1,c2,h,q,s,t)$) and $j$ is 
in the waiting queue ($j \in$ \lstinline$q$), then \lstinline$CalBound$ $j$ \lstinline$l > 0$. 

We then define a natural-number valued termination measure $M_i$\lstinline$(c1,c2,h,q,s,l)$. 
This is a bound on how many events the CPU $i$ will
have to wait for in a state represented by the log \lstinline$l$, and where
\lstinline$QS_CalLock(l) = Some(c1,c2,h,q++$$i$\lstinline$::q$$_0$\lstinline$,s,t++n::t$$_0$\lstinline$)$. 
Note that
we partition the waiting queue into two parts \lstinline$q$ 
and $i$\lstinline$::q$$_0$, where \lstinline$q$
represents the waiting CPUs that were ahead of $i$ in the queue.
The function $M$ has two cases that depend on the head status.
\begin{center}
\begin{tabular}{p\textwidth}
$M_i$\lstinline$(c1,c2,LEMPTY,q,s,l)$ = \lstinline$CalBound$$_{\mathsf{hd}(q)}($\lstinline$l$$) + (K_1(\Sigma $\lstinline$t$$) + |$\lstinline$q$$\cup $\lstinline$s$$|)\times K_2)$\\
$M_i$\lstinline$(c1,c2,LHOLD,q,s,l)$ = \lstinline$CalBound$$_{\mathsf{hd}(q)}($\lstinline$l$$) + \texttt{BoundValAux}\times K_2$ \\
\hfill	 where $\texttt{BoundValAux} = ($\lstinline$c1$$+$\lstinline$c2$$ + (\Sigma (\mathsf{tl}($\lstinline$t$$)) \times K_1 + |\mathsf{tl}($\lstinline$q$$)\cup $\lstinline$s$$|)$\\
\end{tabular}
\end{center}

In short, if the lock is not taken, the bound $M$ is the sum of the
maximum time until the first thread in the queue gets scheduled again
(\lstinline$CalBound$$_{\mathsf{hd}(q)}($\lstinline$l$$)$), plus a constant times
the sum of the number of operations to be done
by the CPUs ahead of $i$ in the queue ($\Sigma $\lstinline$t$) 
and the number of CPUs ahead of $i$ which has
yet to execute $\SETNEXT$ operation 
($|$\lstinline$q$$ \cup $\lstinline$s$$|$). If the lock is currently
held, then \lstinline$c1 + c2$ is a bound of the number of operations it will
do
(and we can ignore the first element of \lstinline$q$ and \lstinline$t$, since they are
accounted for).
The constants and fairness assumption is general enough to handle the cases which takes a slightly longer execution than it is expected to.
The constants ($K_1 = F+5$ and $K_2 = F+4$) are chosen somewhat
arbitrary, and certainly $M$ is not the tightest possible bound. It
doesn't need to be, since it does not occur in our final theorem
statement.

The definition of $M$ is justified by the following two
lemmas. First, we prove that M decreases if CPU $i$ is waiting and some other CPU
$j$ executes an event \lstinline$e$$_j$.

\begin{lemma}[Decreasing measure for other CPUs]
\label{lem:MCS_CalLock_progress_onestep}
Assuming that \lstinline$QS_CalLock(l) = Some(c1,c2,h,q$$_1$\lstinline$++$$i$\lstinline$::q$$_2$\lstinline$,s,t$$_1$\lstinline$++c::t$$_2$\lstinline$)$, where
$|$\lstinline$q$$_1|=|$\lstinline$t$$_1|$ as well as \lstinline$QS_CalLock(e$$_j$\lstinline$::l) =Some(c1',c2',h',q',s',t')$
for some $j\neq i$ and \lstinline$CalBound(e$$_j$\lstinline$::l) > 0$ .
Then we can split \lstinline$q' = q$$_{1}$\lstinline$'$\lstinline$++$$i$\lstinline$::q$$_{2}$\lstinline$'$, and
$M_i$\lstinline$(c1',c2',h',q$$_{1}$\lstinline$'$\lstinline$,s',t$$_{1}$\lstinline$'$\lstinline$,e$$_j$\lstinline$::l)$ $< M_i$\lstinline$(c1,c2,h,q$$_1$\lstinline$,s,t$$_1$\lstinline$,l)$.
\end{lemma}

\begin{proof}
The proof follows the informal outline in
Sec.~\ref{sec:overview}.  We consider all possible events
\lstinline$e$$_j$ which could make \lstinline$QS_CalLock$ return \lstinline$Some$. If $j$ is not the 
CPU at the head of the queue gets scheduled, it will not be
able to make any progress, so the abstract state of the queue remains the same,
but the counter \lstinline$CalBound$ decreases.
Otherwise, the counter \lstinline$CalBound$ will reset to the upper bound we assumed on fairness, $F$. 
However, in this case the algorithm will make some progress that changes \lstinline$c1$, \lstinline$c2$, \lstinline$q$, or \lstinline$s$.
For example, CPU $j$ may execute a $\SETNEXT$ (which decreases the size of
\lstinline$s$), it may enter the critical section (which moves some measure from
the head of \lstinline$q$ to the counters \lstinline$c1+c2$) or it may exit the section
(and that event will decrement \lstinline$c2$).
\end{proof}


The second lemma ensures that the waiting loop will eventually
terminate (The preconditions that $i$ is somewhere in the waiting queue,
and that it has already left the set \lstinline$s$, correspond the set-up
which \lstinline$wait_lock$ does before it starts looping).

\begin{lemma}[Loop termination]
\label{lem:CalWaitGet_exist'}
Let's assume that \lstinline$QS_CalLock(l) = Some(c1,c2,h,q$$_1$\lstinline$++$$i$\lstinline$::q$$_2$\lstinline$,s,t$$_1$\lstinline$++c::t$$_2$\lstinline$)$, where
$|$\lstinline$q$$_1| = |$\lstinline$t$$_1|$, with $i$ $\not\in$ \lstinline$q$$_1$ and $i$ $\not\in$ \lstinline$s$ and suppose $\oracle$ is a valid
context. If $k$ $> M_i$\lstinline$(c1,c2,h,q$$_1$\lstinline$,s,t$$_1$\lstinline$)$, then there exists \lstinline$l'$ such
that \lstinline$CalWaitGet($$k$,$i$,\lstinline$l) = Some(l')$.
\end{lemma}


\begin{proof}
The proof is by induction on $k$, the number of loop iterations. The
most interesting part of the proof is to show that each event
generated by the function will decrease the measure.
As it pulls more event to the log form the context, we appeal to
Lemma~\ref{lem:MCS_CalLock_progress_onestep}, which says that the metric decreases. 
Then, there are two cases in the proof depending on whether $i$ has
arrived at the head of the queue (so \lstinline$q = nil$) or not. If it has,
\lstinline$wait_qslock_spec$ will generate a \lstinline$GET_BUSY false$
even and return, so we are good. 
Otherwise, it will generate a \lstinline$GET_BUSY true$ event, and
start another loop iteration. That event does not change the state of
the lock, but it does decrement the $\CalBound$ on when the head CPU
will get scheduled next, so the measure decreases as required.
\end{proof}

To prove the termination of the loop in \lstinline$wait_qslock_spec$, 
we also need to show that the busy-loop in \lstinline$pass_qslock_spec$ terminates, 
but that proof is easier. A CPU holding the lock will set
the next pointer before it does anything else, so we are only waiting
for the CPU at the head of the queue to get scheduled at all.
Now, to prove that the loop in \lstinline$mcs_acquire$ specification
is defined, we just have to pick the function \lstinline$CalWaitLockTime$
so that \lstinline$CalWaitLockTime(t)$ is greater than $M$ at that
point. The rest of the simulation proof for Theorem~\ref{thm:mcs_wait_lock_exist} is straightforward.
Except the waiting loop, other operations in the wait lock function are deterministic and finite. 


\begin{theorem}
There is a one-step simulation from \lstinline$mcs_wait_hslock_spec$ to
\lstinline$mcs_wait_qslock_spec$, with the simulation on logs given by \lstinline$relate_mcs_log$.
\end{theorem}


\subsection{From downwards- to upwards-simulation}
\label{sec:downwards-to-upwards}

When moving from sequential to concurrent programs we must
re-visit some fundamental facts about refinement proofs.  Ultimately,
the correctness theorem we want to prove is ``all behaviors of the
machine satisfy the specification''. If we model the machine and the
specification as two transition systems $M$ and $S$, then this
corresponds to \emph{upwards simulation}: if $S \sim M$ and $M
\Longrightarrow^* M'$, then $\exists S'. S' \sim M'$ and $S
\Longrightarrow^* S'$, and if $M$ is stuck then $S$ is stuck also.
But directly proving an upwards simulation is difficult. You are given
a long sequence of low-level steps, and have to somehow reconstruct
the high-level steps and high-level ghost state corresponding to
it. One of the insights that made the CompCert project
possible~\cite{Leroy-backend} is that as long as $M$ is deterministic
and $S$ is not stuck, it suffices to prove a \emph{downward
  simulation}: if $S \sim M$ and $S \Longrightarrow S'$, then $\exists
M'. S' \sim M'$ and $M \Longrightarrow^* M'$. (The assumption that $S$
is not stuck is standard, it corresponds to only proving refinement
for ``safe'' clients regarding to the specifications.)

Unfortunately, concurrent programs are \emph{not} deterministic: we
want to prove that every interleaving of operations from
different CPUs in the low-level machine results in correct
behavior. So if we had directly modeled the implementation as a
nondeterministic transition system, then we would have to work
directly with upwards simulations, which would be intractable when
reasoning about the low-level details of C programs.

In our approach, all the nondeterminism is isolated to the concurrent
context $\oracle$. Any possible interleaving of the threads can be
modelled by initializing the abstract state with a particular
$\oracle_L$, and the execution proceeds deterministically from
there. Therefore we can still use the Compcert/CertiKOS method of first
proving a downward simulation and then concluding the existence of a
upward simulation as a corollary.

The context-formalism is also helpful because $\oracle_L$ contains
the entire execution of the other threads, both past and future, so we
have enough information to directly prove a \emph{forward}
simulation. Otherwise it may not be clear if a given low-level
operation can really ``commit'' (and generate a high-level event)
until we see what the other cores do, so proofs about fine-grained
concurrency can require a difficult backwards-simulation
from the end-state of the program.~\cite{doherty:lock-free}

There is still an obligation to show that for every $\oracle_L$, there
in fact exists an $\oracle_H$ with the right
properties. (Specifically, it should the always output logs which
respect the program invariants, i.e. the replay function is defined,
and also it should respect the refinement relation $f$.) But this can
be managed by the framework in a generic way~\cite{ccal16}. When
verifying a particular layer, the programmer only needs to define $f$.


%
\section{Evaluation}
\label{sec:evaluation}

\begin{figure}
\begin{minipage}{\linewidth}
\noindent
\begin{multicols}{2}
\lstinputlisting[numbers = left, language = C]{source_code/mcslock/palloc_example.c}
\lstinputlisting[numbers = left]{source_code/mcslock/sharedeventtype.v}
\end{multicols}
\end{minipage}
\caption{\lstinline$palloc$ Example}
\label{fig:palloc-example}
\end{figure}

\paragraph{Clients}
The verified MCS lock code is used by multiple clients in the CertiKOS
system. To be practical the design should require as little extra work
as possible compared to verifying non-concurrent programs, both to
better match the programmer's mental model, and to allow code-reuse
from the earlier, single-processor version of CertiKOS.

For this reason, we don't want our machine model to generate an event
for every single memory access to shared memory. Instead we use what
we call a \emph{push/pull memory model}~\cite{certikos16,ccal16}. A
CPU that wants to access shared memory first generates a ``pull''
event, which declares that that CPU now owns a particular block of
memory. After it is done it generates a ``push'' event, which
publishes the CPU's local view of memory to the rest of the system. In
this way, individual memory reads and writes are treated by the same
standard operational semantics as in sequential programs, but the
state of the shared memory can still be replayed from the log.  The
push/pull operations are logical (generate no machine code) but
because the replay function is undefined if two different CPUs try to
pull at the same time, they force the programmer to prove that
programs are well-synchronized and race-free. Like we did for atomic
memory operations, we extend the machine model at the lowest layer by
adding logical primitives, e.g. \lstinline$release_shared$ which takes a
memory block identifier as an argument and adds a
\lstinline$OMEME (l:list Integers.Byte.int)$ event to the log, where the byte list is a
copy of the contents of the shared memory block when the primitive was
called.

When we use \lstinline$acquire$/\lstinline$release_shared$ we
need a lock to make sure that only one CPU pulls, so we begin
by defining combined functions \lstinline$acquire_lock$ which
takes the lock (with a bound of 10) and then pulls, and
\lstinline$release_lock$ which pushes and then releases the
lock. The specification is similar to \lstinline$pass_hlock_spec$,
except it appends \emph{two} events.

Similar to Sec.~\ref{sec:liveness-atomicity}, logs for different
layers can use different types of pull/push events.
Fig.~\ref{fig:palloc-example} (right) shows the events for the
\lstinline$palloc$ function (which uses a lock to protect the page
allocation table). The lowest layer in the palloc-verification adds
\lstinline$OMEME$ events, while higher layers instead add
(\lstinline$OATE (a: ATable)$) events, where the relation between logs
uses the same relation as between raw memory and abstract
\lstinline$ATable$ data. Therefore, we write wrapper functions
\lstinline$acquire$/\lstinline$release_lock_AT_spec$, where the
implementation just calls \lstinline$acquire$/\lstinline$release_lock$
with the particular memory block that contains the allocation table,
but the specification adds an \lstinline$OATE$ event.
This refinement step, which changes the log replay function to compute
allocation tables instead of byte lists, is
specific to the \lstinline$palloc$ function.

\begin{figure}

\lstinputlisting{source_code/mcslock/release_lock_AT_spec_short.v}
\lstinputlisting{source_code/mcslock/palloc_spec_short.v}

\caption{Specification for \lstinline$palloc$}
\label{fig:palloc-spec}
\end{figure}

We can then ascribe a low-level functional specification
\lstinline$palloc'_spec$ to the \lstinline$palloc$ function. As shown
in Fig~\ref{fig:palloc-spec}, this is decomposed into three parts, the
acquire/release lock, and the specification for the critical
section. The critical section spec is exactly the same in a sequential
program: it does not modify the log, but instead only affects the 
\lstinline$AT$ field in the abstract data.

Then in a final, pure refinement step, we ascribe a high-level atomic
specification \lstinline$lpalloc_spec$ to the \lstinline$palloc$
function. In this layer we no longer have any lock-related events at
all, a call to \lstinline$palloc$ appends a single
\lstinline$OPALLOCE$ event to the log. This is when we see the
proof obligations related to liveness of the locks.
Specifically, in order to prove the downwards refinement, we need to
show that the call to \lstinline$palloc'_spec$ doesn't return
\lstinline$None$, so we need to show that \lstinline$H_CalLock l'$ is
defined, so in particular the bound counter must not hit zero.
By expanding out the definitions, we see that
\lstinline$palloc'_spec$ takes a log \lstinline$l$ to
\lstinline$REL_LOCK :: (OATE (AT adt)) :: (TSHARED OPULL) :: (WAIT_LOCK 10) :: l$.
The initial bound is 10, and there are two shared memory events, so the
count never goes lower than 8. If a function modified more than one
memory block there would be additional push- and pull-events, which
could be handled by a larger initial bound.

Like all kernel-mode primitives in CertiKOS, the \lstinline$palloc$ function is
total: if its preconditions are satisfied it always returns. So
when verifying it, we show that all loops inside the critical section
terminate. Through the machinery of bound numbers, this guarantee is
propagated to the the while-loops inside the lock implementation:
because all functions terminate, they can know that other CPUs will
make progress and add more events to the log, and because of the
bound number, they cannot add push/pull events forever. On the other
hand, the framework completely abstract away how long time (in microseconds) elapses
between any two events in the log.

\paragraph{Code reuse} The same
\lstinline$acquire$/\lstinline$release_lock$ specifications can be
used for all clients of the lock. The only proofs that need to be done
for a given client is the refinement into abstracted primitives like
\lstinline$release_lock_AT_spec$ (easy if we already have a sequential
proof for the critical section), and the refinement proof for the
atomic primitive like \lstinline$lpalloc_spec$ (which is very
short). We never need to duplicate the thousands of lines of proof
related to the lock algorithm itself.

\paragraph{Using more than one lock}
The layers approach is particularly nice when verifying code that uses more than one
lock. To avoid deadlock, all functions must acquire the locks in the
same order, and to prove the correctness the ordering must be
included in the program invariant. We \emph{could} do such a
verification in a single layer, by having a single log with different
events for the two locks, with the replay function being undefined if
the events are out of order. But the layers approach provides a
better way. Once we have ascribed an atomic specification to
\lstinline$palloc$, as above, all higher layers can use it
freely without even knowing that the \lstinline$palloc$ implementation
involves a lock (Note that the lock is not re-exported from the
\lstinline$palloc$ layer, and if it was the proof of the atomic
specification would not go through.)  For example, some function in a
higher layer could acquire a lock, allocate a page, and release the
lock; in such an example the the order of the layers provides an order
on the locks implicitly.




\ignore{

Our verified MCS Lock is actually used by multiple clients in CertiKOS system. Fig.~\ref{fig:palloc-example} (left) is one example of client code. Each client code that uses the lock defines its own lock wrappers. In the example, \lstinline$acquire_lock_AT$ and \lstinline$release_lock_AT$ are the lock primitives that are defined using the atomic lock specifications (\lstinline$wait_hlock_spec$ and \lstinline$pass_hlock_spec$, respectively), and they are only designed to the page allocation table protection. 

Using the lock with client codes also requires a way to add shared memory events into the log. They are essential in our system because replaying the log should reflect the status of global shared memory as well as the lock status. 
Those events may be different in the different layers because of the similar reason in Sec.~\ref{sec:liveness-atomicity}. Fig.~\ref{fig:palloc-example} (right) shows the events that are for \lstinline$palloc$, and we can refine logs that are using different types of shared memory events using the approach that we mention in Sec.~\ref{sec:verification}.

\ignore{Our MCS Lock verification is actually used with several client codes in CeritKOS system, and we briefly discuss how our MCS Lock can be associated with them by using the simplest example, \lstinline$palloc$ in Fig.~\ref{fig:palloc-example}.}

First step is introducing the \textit{logical} machine-level primitive 
that can memorize the actual memory values associated with shared memory in the log. 
In this sense, we have introduced \lstinline$release_shared$, 
a logical primitive in the lowest layer in our system.
The primitive will add \lstinline$OMEME (l: list Integers.Byte.int)$ event to the log, 
which mirrors the current local copy of the shared memory. 
However, this does not mean that the operation on shared memories generate events at all, 
and memorizing the current shared memory status in the log when the system release the lock.
Thus, mapping this \lstinline$release_shared$ primitive with 
our MCS Lock verification is the second step to use the verified lock code with client codes.
After introducing the atomic lock specification (\lstinline$wait_hlock_spec$ 
and \lstinline$pass_hlock_spec$), we now can define lock acquire and release 
functions for each shared memory operation. 
To define \lstinline$release_lock_AT_spec$, 
the release lock specification designated to the page allocation table, 
we simply combine \lstinline$release_shared$ and \lstinline$pass_hlock_spec$, and refine the general 
purpose shared memory event to the allocation table type event (\lstinline$OATE (a: ATable)$).}

\ignore{Then, using this lock in  \lstinline$palloc$ function and verifying the correctness of the code is simple. 
We only need to encapsulate the  \lstinline$palloc$ code with \lstinline$acquire_lock_AT$, 
the acquire lock specification for the page allocation table,
and \lstinline$release_lock_AT$ primitives to protect the operation on the allocation table. 
Thanks to the proofs that we have done in MCS Lock module, our \lstinline$palloc$ proof also contains the mutual exclusiveness, linearizability, liveness, and functional correctness.
Note that other client codes also use this approach, which makes us to easily re-use the verified MCS Lock for different kinds of client codes. }

\paragraph{Proof Effort}


Among the whole proofs, the most challenging parts are the proofs for starvation 
freedom theorems like Thm.~\ref{thm:mcs_wait_lock_exist}, 
and the functional correctness proofs for \lstinline$mcs_acquire$ 
and \lstinline$mcs_release$ functions
in Sec.~\ref{subsec:atomicoperation}.
The total lines of codes for starvation freedom is 2.5K lines, 0.6K lines for specifications, 
and 1.9k lines for proofs. This is because of the subtlety of those proofs. 
To prove the starvation freedom theorems and show the evidence of loop termination,
lots of lemmas are required to express
state changes by replaying the log. For instance, 
when \lstinline$QS_CalLock(l) = Some(c1, c2, b, q, s, t) $ 
and \lstinline$q = nil$, \lstinline$s$ $=\emptyset$ 
and \lstinline$t = nil$. It looks trivial in the hand-written proofs, 
but requires multiple lines of codes in the mechanized proof. 

The total lines of codes for the low-level functional correctness 
of \lstinline$mcs_acquire$ and \lstinline$mcs_release$ are 3.2K lines,  
0.7K lines for specifications, and 2.5K lines for proofs.
It is much bigger than other code correctness proofs for while-loops in CertiKOS, 
because these loops do not have any explicit decreasing value.
One another big part in our MCS Lock proofs is the proofs for 
Thm.~\ref{thm:machine-state-refinement} and the lines of code for this part is 
approximately 5K lines. The log replay function (\lstinline$CalMCSLock$) always 
return the whole MCS Lock values (\lstinline$MCSLock$) related 
to the  \lstinline$mcs_lock$ structure defined in Fig.~\ref{fig:exp:mcs_lock}. 
In this sense, we always have to give the exact values for all memory 
chunks and prove the correspondence between the memory and the abstract 
data even the event associated with reading values (e.g. \lstinline$GET_NEXT$).
Hence, those proofs contain a lot of duplicate proofs for the memory access. 
However, they are quite straightforward and easy to produce. 
On top of that, we strongly believe 
that they can be easily reduced by introducing mechanized user-defined tactics later. 

As an evaluation, we do not count the total lines of code in Coq for our entire 
MCS Lock module due to the two following reasons. First, our MCS Lock implementation 
is a part of CertiKOS. Therefore, our MCS Lock module also contains several definitions 
and proofs that are totally irrelevant to MCS Lock verification. 
This implies that counting the total lines of code for MCS Lock module has a 
high possibility of misinterpretation due to the lines of code for those definitions and proofs.
Second, we intensively use contextual refinement approach to 
build the whole system rather than focusing on verifying the correctness and 
liveness of MCS Lock. Therefore, our proof efforts are mainly focus on proving 
MCS Lock that is able to be easily combined with multiple client codes 
rather than the efficient lock verification itself.  

As can be seen from these line counts, proofs about concurrent programs
have a huge ratio of lines of proof to lines of C code.
If we tried to directly verify shared objects that use locks to 
perform more complex operations, like thread scheduling
and inter-process communication, a monolithic proof  
would become much bigger than the current one, and would be quite
unmanageable. The modular lock specification is essential here.


By contrast, the proofs for them in CertiKOS are quite tractable, 
because the proofs for the locks are modular, re-usable, and can 
be combined with other client-part proofs like we have briefly 
mentioned earlier in this Section.
Therefore, we believe that our approach is a promising way to 
show the correctness of large systems that use shared objects with mutex protection. 


The total lines of code in Coq for the MCS Lock module is approximately 30k lines, 9k lines for specifications and 21k lines for proofs.
Compared to the lines of C code in Section~\ref{sec:overview}, the verification effort is quite huge, although
those proofs also contain a lot of parts unrelated to the MCS Lock verification because of the primitives that are used in other modules of CertiKOS.

However, even if we exclude the proofs for those other modules and look at the proofs which are directly related to the MCS Lock code, the proof size is very large.
The proofs related to the two highest layers, which deal with designing atomic MCS Lock specifications, contain is approximately 8k lines of Coq including 2.5k lines of starvation freedom proofs.
Evidently, proving correctness of concurrent code is not easy work even for this short and seemingly simple program.

All the result in here implies that concurrent program proofs has a huge ratio in terms of the lines of code written in C versus the lines of code for the verification.
If we tried to directly verify shared objects that use the lock to  perform more complex operations, such as \texttt{thread scheduling} and \texttt{inter process communication (IPC)}, a monolithic proof would become much bigger than the current one, and would be quite unmanageable.
By contrast, the proofs for those features in CertiKOS are quite tractable, because the proofs for the locks are modular, re-usable, and can be combined with other client-part proofs. 
Therefore, we believe that our approach is a promising way to show the correctness of large systems that use shared objects with mutex protection, and using interactive theorem provers like Coq is desirable for this concurrent object proofs, even if the code is quite simple like this MCS algorithm. 


%\section{Related work and conclusions} 
\label{sec:related}

\paragraph{Verified system software}
CertiKOS is an end-to-end verified concurrent system showing that its
assembly code indeed ``implements'' (contextually simulates) the
high-level specification.
Other verified systems~\cite{klein2009sel4,hawblitzel10,hawblitzel:ironclad},
are single-threaded, or use a per-core big kernel lock.
The Verisoft team used VCC~\cite{vcc09} to verify spinlocks in a
hypervisor by directly postulating a Hoare logic rather than building
on top of an operational semantics for C, and only proved properties
about the low-level primitives rather than the full functionality of
the hypervisor. By contrast, CertiKOS deals with the problem of
formulating a specification in a way that can be used as one layer
inside a large stack of proofs. As for CertiKOS itself, while we
discussed the ``local'' verification of a single module, other papers
explain how to relate the log and context to a more realistic
nondeterministic machine model~\cite{certikos16}, how to
``concurrently link'' the per-CPU proofs into a proof about the full
system~\cite{ccal16}, and how this extends to multiple threads per
CPU~\cite{ccal16}.

%
%\paragraph{Other papers on CertiKOS}
%Because this paper focused on a single module of the kernel,
%we discussed the ``local'' aspects: how to formulate the
%specifications of the OS primitives and how to ascribe
%an atomic specification to a concurrent implementation.
%
%As the starting point of this paper, we assumed a C-level model where
%concurrency is handled through an event log, and the log only has to
%be updated from the context at certain points in the program. Gu et
%al.~\cite{certikos16} explain how to relate that model to a more
%realistic machine model where executions of different CPUs can be
%nondeterministically interleaved anywhere (the proof uses a
%backwards-simulation once at the lowest level of verification). The
%different machine model also required support in the correctness
%proofs for the C compiler.~\cite{ccal16}.
%
%We also need to relate the local refinement proofs of one primitive to
%proofs about the whole system. First, the proofs in this paper handle
%code running on a single CPU in isolation, with all the other CPUs
%abstracted away as a concurrent context. We need a ``concurrent
%linking'' theorem saying that this suffices to prove the correctness
%when all CPUs run simultaneously producing a single global
%log. Second, we only proved the correctness of a single layer in
%isolation, but eventually this will be composed with higher
%abstraction layers containing the client code of the lock (``vertical
%composition''), and eventually with user-code running on top of the
%OS. We need a ``contextual refinement'' theorem talking about the
%entire system. And finally, the abstraction level used in this paper
%only talks about inter-CPU concurrency, but to support user-space
%programs, at higher abstraction levels the OS needs to provide
%threading.~\cite{ccal16}

\paragraph{Fine-grained concurrency}
The MCS algorithm uses low-level operations like CAS instead of
locks. There is much research about how to reason about such programs,
more than we have space to discuss here. One key choice is how much to
prove. At least all operations should be
linearizable~\cite{herlihy:linearizability} (a safety property). Some
authors have considered mechanized verification of linearizability
(e.g. \cite{doherty:lock-free,derrick:mechanical-linearizability}),
but on abstract transition system models, not directly on executable
code. The original definition of linearizability instrumented programs
to record a global history of method-invocation and method-return
events. However, that's not a convenient theorem statement when
verifying client code. Our formulation is closer to Derrick et
al~\cite{derrick:mechanical-linearizability}, who prove a simulation
to a history of single atomic actions modifying abstract state.  Going
beyond safety, one also wants to prove a progress property such as
wait-freedom~\cite{herlihy:wait-freedom} or (in our case)
starvation-freedom~\cite{Herlihy08book}.

% Formal proofs have included
% e.g. Jia et al.~\cite{jia:lock-freedom}.  Most such research deals
% with datastructures (sets, queues, etc), but mutexes present one extra
% complication because different clients will hold the lock for
% different amount of times, so we had to introduce bound-numbers to
% give a modular specification.

Liang {\em et al}~\cite{liang13} showed that the linearizability and
progress properties~\cite{Herlihy08book} for concurrent objects is
exactly equivalent to various termination-sensitive versions of the
contextual simulation property. Most modern separation-style
concurrent
logics~\cite{cap10,Turon13popl,sergey15pldi,pinto14,iris15,pinto16} do
not prove the same strong termination-sensitive contextual simulation
properties as our work does, so it is unclear how they can be used to
prove both the linearizability and starvation-freedom properties of
our MCS Lock module.  Total-TaDA~\cite{pinto16} can be used to prove
the total correctness of concurrent programs but it has not been
mechanized in any proof assistant and there is no formal proof that
its notion of liveness is precisely equivalent to Helihy's notion of
linearizability and progress properties for concurrent
objects~\cite{Herlihy08book}. FCSL~\cite{sergey15pldi} attempts to
build proofs of concurrent programs in a ``layered'' way, but it does
not address the liveness properties. Many of these program
logics~\cite{Turon13popl,iris15}, however, support 
higher-order functions which our work does not address.

\paragraph{Other work on the MCS algorithm}
We are aware of two other efforts to apply formal verification methods
to the MCS algorithm.  Ogata and Futatsugi developed a mechanized
proof using the UNITY program logic.~\cite{ogata:mcs-lock} They work
with an abstract transition system, not executable code. Like us,
their correctness proof works by refinement (between a fine-grained
and a more atomic spec) but they directly prove backward
simulation.

One difference is that Ogata and Futatsugi's proof is
done using a weaker fairness assumption. They assume ``every CPU gets
scheduled infinitely often'', while we require a maximum scheduling
period ($F$ in Section~\ref{sec:liveness-atomicity}).  This is because
we write our specification of \lstinline$wait_lock$ as a Coq function
defined by recursion on a natural number, and all Coq functions must
be total. So although our ultimate theorem only states that the method
terminates ``eventually'', as an intermediate lemma we need to prove
an explicit natural number bound on when a given call to
\lstinline$wait_lock$ will finish.  We could avoid this by e.g. using
Coq's facilities to define functions by well-founded recursion, and
making the termination measure $M_i$ take ordinal instead of number
values, but in practice assuming a fixed $F$ seems like a reasonable
model of multi-core concurrency.

The other MCS Lock verification we know of is by Liang and
Feng~\cite{liang:lili}, who define a program logic LiLi to prove
liveness and linearizability properties and verify the MCS algorithm
as one of their examples.  The LiLi proofs are done on paper, so they
can omit many ``obvious'' steps, and they work with a simple
while-loop language instead of C. Many of the concepts in our proof
are also recognizable in theirs. The state of their concrete
  programs includes a pointer $\mathrm{\texttt{tail}}$ and nodes
  $\mathrm{\texttt{Node}}(\mathrm{\texttt{busy}}, \mathrm{\texttt{next}}, \mathrm{\texttt{ThrdID}})$.
In their invariant and precondition they use specificational variables
$\mathrm{\textit{ta}}$ and $\mathrm{\textit{tb}}$ (like \texttt{la} in
Sec.~\ref{subsec:atomicoperation}), $\mathrm{\textit{tl}}$ and $S$ (like $q$
and $s$ in Sec.~\ref{sec:representation-ghost}). Their
  ``wellformed lock'' predicate $\mathrm{\textsf{lls}}$ includes our
  tail-soundness and next-correctness properties, so in order to prove
  that the invariant is preserved they need essentially the same
  lemmas as in Section~\ref{sec:representation-ghost}. and their
termination measure $f(\mathrm{G})$ includes the length of
$\mathit{tl}$ and the size of $S$ (like $M$ in
Sec.~\ref{sec:liveness-atomicity}. On the other hand, the fairness
constant makes no appearance in $f(\mathrm{G})$, because fairness
assumptions are implicit in their inference rules.

A big difference between our work and LiLi is our emphasis on
modularity.  Between every two lines of code of a program in LiLi, you
need to prove all the different invariants, down to low-level data
representation in memory. The specification takes the form of a single
pre- and post-condition which involves concepts at many level of
abstraction. For example, unfolding the definition of the measure $f$,
we find not only $\mathit{tl}$, but also the tail-pointer $p$, and
eventually the lock-array $\mathit{ta}$. In our development, these
concerns are in different modules which can be completed by different
programmers.  Similarly, we aim to produce a stand-alone specification
of the lock operations. In the LiLi example, the program being
verified is an entire ``increment'' operation, which takes a lock,
increments a variable and releases the lock. The pre/post-conditions
of the code in the critical section includes the low-level
implementation invariants of the lock, and the fact the lock will
eventually be released is proved for the ``increment'' operation as a
whole. Our locks are specified using \emph{bound} numbers, so they can
be used by many different methods.

Apart from modularity, one can see a more philosophical difference
between the CertiKOS approach and program logics such as LiLi.  Liang
and Feng are constructing a program logic which is tailor-made
precisely to reason about liveness properties under fair
scheduling. To get a complete mechanized proof for a program in that
setting would require mechanizing not only the proof of the program
itself, but also the soundness proof for the logic, which is a big
undertaking. Other parts of the program will favor other kinds of
reasoning, for example many researchers have studied program logics
with inference rules for reasoning about code \emph{using} locks. One
of the achievements of the CertiKOS style of specification is its
flexibility, because the same model---a transition system with data
abstraction and a log of events---works throughout the OS kernel. When
we encountered a feature that required thinking about liveness and
fairness, we were able to do that reasoning without changing the
underlying logical framework.

\paragraph{Conclusion and Future Work}
Using the ``layers'' framework by Gu et al.~\cite{dscal15} made our
MCS lock proofs modular and reusable.  It also lets us verify the code
from end to end and extract certified executable code.  Those proofs
are also combined with client code using MCS Locks, which shows they
can be used in a large scale system verification without increasing
the complexity dramatically. As far as we know, this is the
  first mechanized MCS Lock proof with modular properties.  In the
future, we are planning to devise generic methods for building
oracles, log replay functions, liveness proofs, and so on. We intend
to generalize the machine model to handle weak memory models instead
of assuming sequential consistency. And we also plan to apply this
approach to other concurrent algorithms.




\chapter{Multicore and Multithreaded Linking}
\label{chapter:linking}

%\section{Introduction}
\label{sec:intro}

%%% Outline
%% structure 
%% 1. concurrent verification is done in several works 
%% 2. how about showing the non-deterministic full machine model refines  ... 
%% 3. For example CCAL provide a useful tool for building concurrent abstraction layer 
%% 3-1. building layers is feasible 
%% 3-2. However proving the refinement between concurrent machine model and the per-instance machine model 
%%
%% 3-3. Based on the CCAL, we show how we build the linking for them 
%% 3-3-1. Multicore Linking 
%% 3-3-1-1. t provides the universal abstract semantics for multicore non-deterministic machine (with sequential consistency)
%% 3-3-1-2. it provides detailed refinement between those abstract functions 
%% 3-3-1-3. it provides the concrete instance of those proofs by connecting them with the lowest layer of CompCertX layer 
%% 3-3-2. Multithreaded Linking 
%% 3-3-3-1. It provides the CompCert Assembly machine models for CompCertX to build per-thread machine models 
%% 3-3-3-2. it provides the refinement between those machine models (parameterized by any kinds of Layers with the guarantee about the certain properties) 
%%                 - that allows us to allocate the proper dynamic initial state for each thread / invariant preserving in the initial state / using the same compiler with 
%%                    CompCertX                    
%% 3-3-3-3. it provides the actual proofs using the example in the certified layers (the language and the proofs are parameterized by the concrete layer definition)
%%                  - shows the identity of the private state change while  sleep and yield 
%%                  - mutual exclusion of user memory regions 
%%                  - mutual exclusion of other private states  






%
%Dependencies due to shared data
%•
%Subtle effects of synchronizations
%•
%Often manually parallelized
%–
%Difficult to debug
%•
%too many 
%interleavings
%of threads
%•
%hard to reproduce bugs
%
%
%
%

%%% concurrent program verification is necessary 
The prevalence of shared-memory multicore machine 
brings the eminent changes in the  software. 
With the machine, achieving higher performance on a single computer than before 
becomes possible, 
but it requires us to facilitate 
concurrency, running multiple threads on multiple cores.
Concurrency, however, 
brings the whole new challenges in terms of software correctness. 
They are well known 
to be difficult to get right and to debug because 
of their intrinsic characteristic, numerous number (usually unbounded) of interleavings among multiple components of the system. 
Testing is also not a promising way to provide the high-assurance of those programs. 
Due to a plethora of possible interleavings, 
reproducing a bug is unfeasible unless testers knows the 
precise interleaving order of them. 
In this sense, 
Building reliable concurrent programs 
needs verification of them, which formally shows that those programs correct reflects the 
desirable behavior (\textit{i.e.,} are stated in their specifications) 
without missing any single interleaving cases. 

%%% Composition is required
The concurrent program verification requires compositional reasoning in its essence,
since it provides an isolation of each instance of concurrent program
(on a single core or a single thread) separately  
 in its verification
without directly considering complex interleaving 
with other components in the system. 
This feature is crucial in some sorts 
of concurrent programs such as 
operating systems, libraries, or application interfaces
because the
proof of them 
are usually need to be parameterized by 
other programs running on them. 
In those cases, composition and proof isolation 
give  an enough power 
to state and prove the correctness property 
of those programs upon any arbitrary context programs run with the targeted programs. 

%%%% several previous works and machine checkable proof  

In this sense, 
multiple previous works handle compositional reasoning about concurrent programs.
There are two traditional different approaches,
rely-guarantee~\jieung{cite rely guarantee} and separation logic~\jieung{CSL cite separation logic  - need to refer View for citation},
and many other approaches that stem from either or both of them
\jieung{SAGL (2007) / Bornat-at (2005) RGSep (2007) Gotsman-al (2007) RSL (2013) Deny Guarantee (2009) LRG (2009) RGSim (2012) Liang-Feng (2013) 
Lili (2016) / Iris (2015) Iris 2.0 (2016) FCSL (2014) (SCSL (2013) FTCSL (2015) CoLoSL (2015) CAP (2010)   View paper / CCAL paper / CSpec (MIT)
- Please refer the specification of POSIX File Systems slide}.
In addition, some of them are not only focusing on the functional correctness but also 
shows liveness~\jieung{LiLi}. 
Some, CSpec and CCAL, also provides a verified layered structure to build modular verification, an another important 
feature to build a large scaled program verification in a modular ways.


%%%% several previous works and machine checkable proof  
Bsed on them, few works \jieung{verifying concurrent software using movrs in CSPEC / preemtive kernel verification (Xinyu Feng - CAV), CertiKOS, MCSLock CCAL} 
organizes machine checked proofs 
about concurrent execution. 
Among them, both CSPEC and CertiKOS facilitates layered structures 
for scalable and modular verification and formally connect top level operations into bottom-layer operations.

%%%% CCAL - what is missing 
They, however, overlook the difficulty in one another piece of machine checked concurrent program verification, 
provide the evidence of concurrent linking.
The concurrent linking shows 
the precise evidence of the composition that the underlying logics provide. 
In this sense, 
it requires the definition of 
concurrent machine model that can run multiple instances of concurrent program together (\textit{e.g.,} multicore and multithreaded machine) 
as well as 
the linking proofs between the program runs on top of concurrent machine and the composition of multiple single instances together. 
It also requires the proof that 
shows the single instance of the concurrent program correctly reflects
the program run on the multicore machine model. 

They are necessary to show the full correctness of the program, 
but providing concurrent machine model is bothersome, especially when the model is close to that of bare machines, 
and the proof between it wiith the machine that runs the single instance is also a subtle work.
To handle those challenges,
CCAL slightly mentioned these issues,
but it only carries out
a key idea of
linking without exposing underlying multiple obstacles.  
In this sense, 
providing the information about which steps are necessary for concurrent linking and what kind of things that 
the users have to fill out is desired.
In this sense, the idea in the paper is far from 
the enough idea to achieve how 
concurrent linking can be worked in such 
a large scaled concurrent program. 

\jieung{need to add sentence about CompCertX}


%%%% The contribution of this paper

Therefore, our paper aim to deliver all necessary 
and important ides for concurrent linking,
which includes modeling the generic concurrent machine model, 
necessary information to prove refinements between them, 
and how to connect those concurrent linking with the 
proof layers of concurrent programs in a generic way. 
It is definitely not able to be achieved in a single shot.
We introduce multiple intermediate languages and 
context that users has a responsibility to 
connect the generic concurrent linking proof with 
their one verified programs.
We, in this paper, handle all of them in detail. 
In short, he key contribution of this paper is as follows: 

\begin{itemize}
\item We formally define non-deterministic multicore semantics and multiple intermediate languages that are independent from specific machines (such as x86 or ARM). 
\item We provide the refinement proofs between them that can be used for \compcertkwd-style backward simulation. 
\item We connect those intermediate languages and proofs with the CPU local CCAL layer, that uses \compcertkwd-like sequential x86 assembly model with 
environment context.
\item We provide multithreaded machine model with minimal assumptions about a certain CPU local CCA layer, which implies that the machine model does not stick to the specific layer definition.
\item We provide intermediate languages to introduce per thread machines and refinement proofs among them. 
\item We connect those intermediate languages and refinement proofs with the specific layer definition in CertiKOS, which fully link the layer on per-thread machine with the layer on per-CPU machine.
\end{itemize}

The structure of remaining paper is as follows:
Section~\ref{sec:overview} shows a brief high level idea of CCAL as well as how our linking works. Section~\ref{sec:multicore} shows the details of multicore linking,
and Sect.~\ref{sec:multithreaded} shows the implementation of our intermediate machine models for our multithreaded environment.
Section~\ref{sec:multithreaded-linking-impl} shows how are framework 
can be fitted into the actual concurrent kernel implementations.
Evaluations about our implementation can be found in Sect.~\ref{sec:evaluation} 
and the related work and conclusion is in Sect.~\ref{sec:related}.


%
%
%\begin{figure}
%\caption{Requirements in Concurrent Program Verification}
%\label{fig:concurrent-verification-challenge}
%\end{figure}
%
%However, even with the importance of concurrent program verification and 
%a large body of recent work on shared-memory concurrency verification ~\jieung{cite},
%there are few certified programming tools for a large scale software due to the requirement of multiple challenges described in Fig.~\ref{fig:concurrent-verification-challenge}.
%
%\jieung{ need to site ESOP papers too}
%
%They first have to 
%provide a way to build the software in multiple layers
%that enable us to build a large scale program as a modular way. 
%For example, 
%operating systems can be divided into multiple parts, 
%memory management, process management, and so on.
%
%They also have to provide \jieung{need different word} a methodology to 
%represent the behavior of other components in the concurrent environment. 
%For the program running on multicore environment, 
%the single instance of the program, which is a program runs on top of 
%a single CPU, has to correctly capture the 
%environmental behavior (the behavior of programs on other CPUs). 
%
%In addition to that, 
%providing the end-to-end theorem also requires us 
%to link the multiple proof instances to 
%form a single proof that is based on
%the concurrent environment itself which does not have 
%any environmental contexts at all. 
%In the example of the operating system on multicore environment,
%the end-to-end theorem 
%has to prove that 
%the program running on the single CPU is correctly refined by 
%the whole thread programs running on the multicore machine. 
%
%Previous works, CertiKOS~\jieung{need cite} and Certified Concurrent Abstraction Layer~\jieung{need cite}, 
%tackles all the above examples.  
%CCAL is a tool to build a certified concurrent layers, which provides 
%a way to build concurrent abstraction layers, 
%
%
%
%However, the paper does not handle how the linking process works with the concrete machine models. 
%It briefly mentions the high level idea of linking and the memory extension for linking framework. 
%
%Therefore, this paper aims the gap between the high level perspective of CCAL and the 
%low level details of concurrent proof linking. 
%This low level details contains two parts. 
%First, it requires us to define and build multiple intermediate languages to connect
%the x86 multicoro machine model with the LAsm, which is the machine model for one single CPU. 
%In addition to that, 
%the framework also needs to show the refinement 
%between layers on those intermediate machine models to formally link
%all those proofs together. 
%CCAL also briefly provide the idea of how they implement the practical machine models that can be used with CompCertX.
%However, only providing few details does not provide 
%the  useful information to show how it works with the actual running large scale software.
%Thus, our paper tackles the issues that CCAL overlooked in the paper 
%by providing the formal rules and proofs.
%The key contribution of this paper is as follows: 
%
%\begin{itemize}
%\item We provide the detailed intermediate language semantics for multicore machine model based on CCAL, 
%and instantiate all those intermediate language semantics and refinement proofs 
%to link them with CompCertX with environmental context 
%\item We provide the intermediate machine models to build single threaded machine model from a single CPU machine model. 
%Based on the machine models, we provide the linking theorem in between 
%two abstraction layers, which contains different semantics for software schedulers. 
%\end{itemize}
%
%The structure of remaining paper is as follows:
%Section~\ref{sec:overview} shows a brief high level idea of CCAL as well as how our linking works. Section~\ref{sec:multicore} shows the details of multicore linking,
%and Sect.~\ref{sec:multithreaded} shows the implementation of our intermediate machine models for our multithreaded environment.
%Section~\ref{sec:multithreaded-linking-impl} shows how are framework 
%can be fitted into the actual concurrent kernel implementations.
%Evaluations about our implementation can be found in Sect.~\ref{sec:evaluation} 
%and the related work and conclusion is in Sect.~\ref{sec:related}.
%
%
%
%\ignore{
%Despite the importance of concurrent layers and a large body of recent work on 
%shared-memory concurrency verification, 
%
%
%there are no certified programming tools that can specify, compose, and compile concurrent layers to form a whole system [6]. Formal reasoning across multiple concurrent layers is challenging because different layers often exhibit different interleaving semantics and have a different set of observable events. For example, the spinlock module in Fig. 1 assumes a multicore model with an overlapped execution of instruction streams from different CPUs. This model differs significantly from the multithreading model for building high-level synchro- nization libraries: each thread will block instead of spinning if a queuing lock or a CV event is not available; and it must count on other threads to wake it up to ensure liveness.
%
%
%
%
%many of these abstraction layers also become concurrent in nature. Their interfaces not only hide the concrete data representations and algorithmic de- tails, but also create an illusion of atomicity for all of their methods: each method call is viewed as if it completes in a single step, even though its implementation contains com- plex interleavings with operations done by other threads. Herlihy et al. [19, 20] advocated using layers of these atomic objects to construct large-scale concurrent software systems.
%
%
%The importance of software systems' accuracy is growing rapidly these days. 
%In addition to that, 
%the concurrent environment, including multicore and device drivers, are ubiquitous in modern periods. 
%Therefore, 
%the verification methodology for concurrent programs is critical now. 
%
%In this sense, several previous works propose
%proof logics and tools for that purpose \jieung{need cite}.
%
%However, few of them are working on the linking multiple instances of 
%verified concurrent programs with concrete machine models that can be run 
%on the bare machines. 
%
%One tool, Certified Concurrent Abstraction Layers, 
%provides the tool that can be used for building a practical concurrent programs 
%such as a small operating system or distributed system. 
%It also provides the tool to link the 
%}

% ADD CONTENTS
%\section{Syntax, Semantics for CompCertX and CCAL}

Definitions in here is not the full definition of \compcertx \ (or \compcertkwd). 
They are the minimum definitions that may be required for us to understand the following sections. 


\subsection{Key Definitions for LAsm}
\label{subsec:key-def-for-lasm}




$
\begin{array}{llll}
%\listlengthkwd & : & \listconstructor{\toptype} \rightarrow \nattype & \mbox{(function for getting the lenght of list)}\\

\ccbyte & := & \set{ v~\vert~ v \in \nattype \wedge v < 2^8} & \mbox{(byte)}\\
\ccintval & := & \set{v ~\vert~ v \in  \ccbytes \wedge \listlength{v}{4}} \\

\ccval & := & \ccvundef~\vert~ \ccvalintkwd{(vi : \ccintval)} ~\vert~\ccvalptrkwd{(bid : \nattype)}{(ofs : \nattype)} & \mbox{(definition of value)} ~\vert~  \\
\cctracekwd &:=& \ccemptytrace~\vert~ \cdots &{\mbox{(\compcertkwd\ trace)}}\\
\primitiveid, \cpuident & : & \nattype & \mbox{(Identifier for pritimives and CPU respectively)} \\
\ccmemloc, \ccmemvallength   & := & \nattype & \mbox{(location  and the length of the value in memory, respectively)} \\
\ccmem & := & \ccmemloc \rightarrow \ccmemvallength \rightarrow \ccval & \mbox{(memory)} \\
\ccprivatemem, \ccsharedmem & : & \ccmem & \mbox{(private memory and  shared memory respectively)}\\
\ccregister & := & \ccregisterpc~\vert~\ccregistereax~\vert~\ccregisterebx~\vert~\ldots & \mbox{(registers)} \\
\ccregisterset & := & \ccregister \rightarrow \ccval & \mbox{(register set definition)}\\
\ccprivateabsstate, \ccsharedabsstate & : & \toptype & \mbox{(private and shared abstract states respectively)} \\
\cpueventkwd & := & \toptype & \mbox{(event for the layer associated with the \textit{n}th layer)}\\
\cpulogkwd & := &   \listconstructor{\cpueventkwd} & \mbox{(log for the layer associated with the \textit{i}th layer)} \\
\replayfunckwd & : & \cpulogkwd \rightarrow \ccsharedabsstate &\mbox{(replay function)} \\
\cpuprivatestate & := & \cpuprivatestatecon{\ccregisterset}{\ccprivatemem}{\ccprivateabsstate} & \mbox{(Private state for each core)}\\
%\cpuprivatestatepool & : & \cpuident \rightharpoonup \cpuprivatestate & \mbox{(Private state Pool)}\\
\cpustatekwd & := & \cpustatecon{\cpuident}{\cpuprivatestate}{\ccsharedmem}{\cpulogkwd} & \mbox{(State Definition)}\\
\cpuasminst & : & \toptype & \mbox{(assembly instructions including function calls and so on)}\\
\cpuasmfunc & : & \listconstructor{\cpuasminst} & \mbox{(Assembly function is a list of instructions)}\\
\cpuasmmodule & : & \primitiveid \rightharpoonup \cpuasmfunc & \mbox{(module is a partial set)}\\

\primitivespeckwd & : & \cpustatekwd \rightarrow \listconstructor{\ccval} \rightarrow \cpustatekwd \rightarrow \ccval \rightarrow \mcprop & \mbox{(specification of primtiives)}\\
\cpulayerprimdef & : & \primitiveid \rightharpoonup \primitivespeckwd & \mbox{(collection of functions)}\\
\cpulayerinvariantkwd & : & \cpustatekwd \rightarrow \mcprop & \mbox{(invariant)} \\

\cpulayerinvariantkwd & : & \cpustatekwd \rightarrow \mcprop & \mbox{(invariant)} \\
\cpurely{(cid : \cpuident)} & : & \cpulayerinvariantkwd & \mbox{(rely invariants)} \\
\cpuguarantee{(cid : \cpuident)} & : & \cpulayerinvariantkwd  & \mbox{(guarantee invariants)} \\
\cpuenvcontextkwd & : & \cpuident \rightarrow (\cpulogkwd \rightharpoonup \cpulogkwd) & \mbox{(environmental context for each layer)}\\
\cpulayerdefkwd{(cid : \cpuident)} & := & \cpulayerdefcon{\cpulayerprimdef}{\cpurely{cid}}{\cpuguarantee{cid}}{\cpuenvcontext{cid}}\\
\cpuglobalenv & : & \set{id~\vert~ id \in \ccmemloc \vee id \in \primitiveid} \rightarrow \mcprop & \mbox{(check validity of memory location or primitive or function id)}\\
\end{array}
$
%
%
%\noindent\fbox{property of those syntax}
%
%$
%\begin{array}{l}
%(disjoint)\\
%\end{array}
%$

\subsection{Semantic Rules for LAsm}
\label{subsec:semantics-for-lasm}

\noindent\fbox{step relation: ${}_{[(cid : \cpuident), (layer : \cpulayerdefkwd{cid})]} (genv: \cpuglobalenv) : \cpustatekwd \rightarrow \cctracekwd \rightarrow \cpustatekwd \rightarrow \mcprop$} 


\noindent\fbox{initial state:}

%\section{Multicore Linking Machines}
\label{sec:multicore-linking-machine}



\subsection{Hardware Variables, Abstract Syntax and Semantics, and Key Properties}
\label{subsec:hardware-variables-and-its-key-properties}


\noindent\fbox{abstract hardware setting, abstract command definitions:}

$
\begin{array}{lllr}
\privatestate &:& \toptype & \mbox{(private state)}\\
\sharedpiece &:& \toptype  & \mbox{(shared state: }\forall (s_1 \ s_2: \sharedpiece), \set{s_1 = s_2} + \set{s_1 \neq s_2}\mbox{)} \\
\atomicevent &:& \toptype & \mbox{(atomic event: }\forall (e_1 \ e_2: \atomicevent), \set{e_1 = e_2} + \set{e_1 \neq e_2}\mbox{)} \\
\atomiceventidentkwd & : & \atomicevent \rightarrow \primitiveid  & \mbox{(getting identifier of an atomic event)} \\
\coreset & : & \set{\ztype} & \mbox{(set of cores - full core set for the multicore machine)}\\
\schedid & : & \ztype & \mbox{(logical hardware scheduler CPU ID: } \schedid \notin \coreset\mbox{)}\\
\end{array}
$

 $
\begin{array}{lll}
\command &:=& \privatecmdkwd ~\vert~ \atomiccmdkwd\langle id : \ztype, e:\primitiveid\rangle\\
&& \vert~ \acqsharedcmdkwd\langle id : \ztype\rangle~\vert~\relsharedcmdkwd\langle id : \ztype\rangle\\
\mcevent &:=& \yieldevkwd\langle from : \ztype\rangle~\vert~ \yieldbackevkwd\langle to : \ztype\rangle~\vert~ \acqevkwd\langle from : \ztype, id : \ztype\rangle\\
&& \vert~\relevkwd\langle from :\ztype, id : \ztype, d:\sharedpiece\rangle\\
&&\vert~\atomicevkwd\langle from : \ztype, id : \ztype, e:\atomicevent\rangle\\
\mclog & := & \listconstructorkwd\ \mcevent\\
\mcoracle{S : \set{\ztype}}{log : \toptype}{ret: \toptype} & : &  \ztype \rightarrow log \rightarrow ret \\
\ogetkwd_{[S: \set{\ztype}, log : \toptype, ret : \toptype]} & : &   log \rightarrow \mcoracle{S}{log}{ret} \rightarrow ret \\

\end{array}
$


\noindent\fbox{abstract hardware semantics:}

$
\begin{array}{llll}
\programcounterkwd & : &  \privatestate \rightarrow \command \rightarrow \mcprop\\
\privatestepkwd & : & \ztype \rightarrow \privatestate \rightarrow \privatestate \rightarrow \mcprop \\
\getsharedstepkwd & : & \privatestate \rightarrow \sharedpiece \rightarrow \privatestate \rightarrow  \rightarrow \mcprop \\ 
\setsharedstepkwd & :  & \privatestate \rightarrow \optioncmd{\privatestate} \rightarrow \mclog \rightarrow \privatestate \rightarrow \atomicevent \rightarrow \mcprop\\
\atomicstepkwd & : &  \ztype \rightarrow \ztype \rightarrow \privatestate \rightarrow \mclog \rightarrow \privatestate \rightarrow \atomicevent \rightarrow \mcprop  \\
\end{array}
$


\noindent\fbox{properties of abstract hardware semantics:}

\begin{tabular}{l}
$
   \loggetatomkwd(l : \mclog) (eid : \primitiveid) : \mclog :=$\\
$    \left\{\begin{array}{lr}
        \nulllist & \text{for } l = \nulllist \\
       {\listcons{ev}{l'}} & \text{for } l = \listcons{ev}{l'} \wedge \loggetatom{l'}{id}{o}  \wedge eid = eid'\\
                    &   \wedge (ev = \atomicev{\_}{eid'}{\_} \vee ev = \acqev{\_}{eid'}   \vee ev = \acqev{\_}{eid'})  \\
      l' & \text{for } l = \listcons{ev}{l'} \wedge \loggetatom{l'}{id}{o}  \wedge eid = eid'\\
                   & \wedge ((ev = \atomicev{\_ }{eid'}{\_} \vee ev = \acqev{\_}{eid'} \\
                   &  \vee ev = \acqev{\_}{eid'} \wedge eid \neq eid') \vee ev = \_) \\     
        \end{array} \right.
$\\
\end{tabular}

\begin{mathpar}
\inferrule[PC Determ]
{\programcounter{ps}{c_1} \\
\programcounter{ps}{c_2}}{c_1 = c_2}

\inferrule[Private Step Determ]
{\privatestep{n}{ps}{ps_1} \\ 
\privatestep{n}{ps}{ps_2}}{ps_1 = ps_2}

\inferrule[Get Shared Determ]
{\getsharedstep{ps}{sp_1}{ps_1}\\
\getsharedstep{ps}{sp_2}{ps_2}}{sp_1 = ps_2 \wedge ps_1 = ps_2}

\inferrule[Set Shared Determ]
{\setsharedstep{ps}{sp}{ps_1} \\ 
\setsharedstep{ps}{sp}{ps_2}}{ps_1 = ps_2}

\inferrule[Atomic Determ]
{\atomicstep{curid}{id}{ps}{l}{ps_1}{ev_1}\\
\atomicstep{curid}{id}{ps}{l}{ps_2}{ev_2}}{ps_1 = ps_2 \wedge ev_1 = ev_2}

\inferrule[Atomic Valid]
{\atomicstep{i}{eid}{ps}{l}{ps'}{e}\\
\loggetatom{l}{eid}{l''} \\ 
\loggetatom{l'}{eid}{l''}}
{\atomicstep{i}{eid}{ps}{l'}{ps'}{e}}

\end{mathpar}




\subsection{Introduce Multicore Machine}
\label{subsec:multicoremachine}

\noindent\fbox{local view:}

$
\begin{array}{llll}
\ownstatekwd & := & \ofreekwd\langle s : \optioncmd~\sharedpiece\rangle~\vert~\ownkwd\langle i : \ztype\rangle \\
\localviewkwd & := & \localviewconkwd :~\privatestate \rightarrow \mclog \rightarrow \localviewkwd \\
\end{array}
$

\noindent\fbox{calculate owner:}

\begin{tabular}{l}
$
   \calownerkwd(l : \mclog) (i : \primitiveid) : \ownstatekwd :=$\\
   $ \left\{\begin{array}{lr}
        \nulllist & \text{for } l = \nulllist \\
       \ownstate{from} & \text{for } l = \listcons{ev}{l'} \wedge \calowner{l'}{id}{o} \\
                    &\wedge e = \acqev{from}{id'} \wedge id = id' \wedge o =  \ownstate{i}  \\
       \ofreestate{\optionsome \ d} & \text{for } l = \listcons{ev}{l'} \wedge \calowner{l'}{id}{o} \\
                    &\wedge e = \relev{from}{id'}{d}  \wedge id = id' \wedge o = \ownstate{i} \wedge  i = from    \\
       o  & \text{for }  l = \listcons{ev}{l'} \wedge \calowner{l'}{id}{o} \wedge \text{otherwise} \\
        \end{array} \right.
$ \\
\end{tabular}

\noindent\fbox{hardware local step: $\ztype \rightarrow \localviewkwd \rightarrow \localview \rightarrow \mcprop$}

\begin{mathpar}
\inferrule[acqrule]
{ \programcounter{ps}{\acqsharedcmd{id}}\\
\calowner{l}{id}{\ofreestate{d}}\\
\setsharedstep{ps}{d}{ps'}}
{ \hardwarelocalstep{curid}{(\localview{ps}{l})}{(\localview{ps'}{(\listcons{\acqev{curid}{id}}{\nulllist})})}}


\inferrule[relrule]
{  \programcounter{ps}{\relsharedcmd{id}}\\
\calowner{l}{id}{\ownstate{curid}}\\
\getsharedstep{ps}{d}{ps'}}
{ \hardwarelocalstep{curid}{(\localview{ps}{l})}{(\localview{ps'}{(\listcons{\relev{curid}{id}{d}}{\nulllist})})}}


\inferrule[private]
{  \programcounter{ps}{\privatecmdkwd}\\
\privatestep{curid}{ps}{ps'}}
{ \hardwarelocalstep{curid}{(\localview{ps}{l})}{(\localview{ps'}{l})} }

\inferrule[atomic]
{  \programcounter{ps}{\atomiccmd{id}{\atomiceventident{e}}}\\
\atomicstep{curid}{id}{ps}{l}{ps'}{e}}
{ \hardwarelocalstep{curid}{(\localview{ps}{l})}{(\localview{ps'}{(\listcons{\atomicev{curid}{id}{e}}{\nulllist}}))} }

\end{mathpar}

\noindent\fbox{hstate:} 

$
\begin{array}{lll}
%\hstatekwd& := & \hstateconkwd :~ \ztype \rightarrow \set{i \rightarrow \privatestate~\vert~ i \in \coreset} \rightarrow \mclog \rightarrow \hstatekwd\\
\privatestatepool{S : \set{\ztype}} & := &  \ztype \rightharpoonup \privatestate \\
& & \hfill  (\forall i . i \in S \rightarrow \exists . (ps : \privatestate) (i, ps) \in \privatestatepool{S}) \wedge   (\forall j . j \notin S  \rightarrow (j, \_) \in \privatestatepool{S}) \\

\hstatekwd& := & \hstateconkwd :~ \ztype \rightarrow \privatestatepool{\coreset} \rightarrow \mclog \rightarrow \hstatekwd\\
\end{array}
$

\noindent\fbox{getter and setter function:} 

\begin{tabular}{l}
$
   \getpstkwd_{[ltyp: \toptype]} (lsp : \ztype \rightharpoonup ltyp) (curid :\ztype) := \left\{\begin{array}{lr}
      \optionsome\ ps & \text{for } (curid, ps) \in lsp \\
      \optionnone & \text{for } (curid, \_) \notin lsp \\
        \end{array} \right.
$\\
$
   \setpstkwd_{[ltyp: \toptype]} (curid :\ztype) (ps : ltyp) (lsp : \ztype \rightharpoonup \privatestate) := (lsp - \set{(curid, \_)})  \cup \set{(curid, ps)}
$\\
\end{tabular}

\noindent\fbox{hardware step: $(start: \ztype) : \hstatekwd \rightarrow \cctracekwd \rightarrow \hstatekwd \rightarrow \mcprop$}

\begin{mathpar}
\inferrule
{ l_0 = \listcons{\yieldbackev{curid'}}{\listcons{\yieldev{curid}}{l}} \\
\getpstsome{\privatestate}{lsp}{curid'}{ps}\\
curid \in \coreset \\
\hardwarelocalstep{curid'}{(\localview{ps}{l0})}{(\localview{ps'}{l'})}\\
\setpst{\privatestate}{curid'}{ps'}{lsp}{lsp'} }
{ \hardwarestep{start}{(\hstate{curid}{lsp}{l})}{\ccemptytrace}{(\hstate{curid'}{lsp'}{(\listapp{l'}{ l_0})}))}}
\end{mathpar}

\subsection{Introduce Hardware Oracle}
\label{subsec:hworacle}

\noindent\fbox{state: }

$
\begin{array}{lll}
\localstatekwd &:=& \localstateconkwd : \privatestate \rightarrow \booltype \rightarrow \localstatekwd\\
\localstatepool{S : \set{\ztype}} & := &  \ztype \rightharpoonup \localstatekwd\\
& & \hfill  (\forall i . i \in S \rightarrow \exists . (ps : \localstatekwd) (i, ps) \in \privatestatepool{S}) \wedge   (\forall j . j \notin S  \rightarrow (j, \_) \in \localstatepool{S}) \\

%\mcstatekwd & := & \mcstateconkwd :~ \ztype \rightarrow \set{i \mapsto \localstatekwd~\vert~ i \in \coreset } \rightarrow \mclog \rightarrow \hstatekwd\\
\mcstatekwd & := & \mcstateconkwd :~ \ztype \rightarrow \localstatepool{\coreset} \rightarrow \mclog \rightarrow \hstatekwd\\
\end{array}
$

\noindent\fbox{oracle step: $(oracle:\mcoracle{\coreset}{\mclog}{\mcevent}) :  \mcstatekwd \rightarrow \cctracekwd \rightarrow \mcstatekwd \rightarrow \mcprop$}

\begin{mathpar}
\inferrule[progress]
{ \getpstsome{\localstatekwd}{curid}{lsp}{(\localstate{ps}{\bfalse})}\\
\hardwarelocalstep{curid'}{(\localview{ps}{l})}{(\localview{ps'}{l'})}\\
\setpst{\localstatekwd}{curid'}{\localstate{ps'}{\btrue}}{lsp}{lsp'}}
{\oraclestep{oracle}{(\mcstate{curid}{lsp}{l})}{\ccemptytrace}{(\mcstate{curid}{lsp'}{(\listapp{l'}{l})})}}

\inferrule[yield]
{  l_0 = \listcons{\yieldbackev{curid}}{l} \\
l' = \listcons{\yieldev{curid'}}{l_0} \\
\getpstsome{\localstatekwd}{curid'}{lsp}{(\localstate{ps}{\btrue})}\\
\setpst{\localstatekwd}{curid'}{\localstate{ps}{\bfalse}}{lsp}{lsp'}\\
\oget{\coreset}{\mclog}{\mcevent}{l_0}{oracle}{{\yieldbackev{curid}}}}
{\oraclestep{oracle}{(\mcstate{curid}{lsp}{l})}{\ccemptytrace}{(\mcstate{curid}{lsp'}{l'})}}
\end{mathpar}


\subsection{Introduce Partial Machine}
\label{subsec:mc-partial}

\noindent\fbox{environmental state: }

$
\begin{array}{lll}
\localstatekwd &:=& \localstateconkwd : \privatestate \rightarrow \booltype \rightarrow \localstatekwd\\
%\envstatekwd[A : \set{\ztype}]& := & \mcstateconkwd :~ \ztype \rightarrow \set{i \mapsto \localstatekwd~\vert~ i \in A} \rightarrow \mclog \rightarrow \localstatekwd\\
\envstatekwd_{[A : \set{\ztype}]}& := & \mcstateconkwd :~ \ztype \rightarrow \localstatepool{A} \rightarrow \mclog \rightarrow \localstatekwd\\

\end{array}
$

\noindent\fbox{back id function:}

\begin{tabular}{l}
$
\backidkwd (curid : \ztype) (e : \mcevent):= \left\{\begin{array}{lr}
curid' & \text{for } e = \yieldbackev{curid'} \\
\schedid &   \text{for } e =\yieldev{\_} \\
curid & \text{Otherwise}
        \end{array} \right.
 $\\
\end{tabular}


\noindent\fbox{env step: ${}_{[A : \set{\ztype}]} (oracle:\mcoracle{A}{\mclog}{\mcevent}) :  \mcstatekwd_{[A]} \rightarrow \cctracekwd \rightarrow \mcstatekwd_{[A]} \rightarrow \mcprop$}

\begin{mathpar}
\inferrule[progress]
{ \getpstsome{\localstatekwd}{curid}{lsp}{(\localstate{ps}{\bfalse})}\\
\hardwarelocalstep{curid'}{(\localview{ps}{l})}{(\localview{ps'}{l'})}\\
\setpst{\localstatekwd}{curid'}{\localstate{ps'}{\btrue}}{lsp}{lsp'}}
{\envstep{A}{oracle}{(\envstate{A}{curid}{lsp}{l})}{\ccemptytrace}{(\envstate{A}{curid}{lsp'}{(\listapp{l'}{l})})}}

\inferrule[yield]
{  l_0 = \listcons{\yieldbackev{curid}}{l} \\
l' = \listcons{\yieldev{curid'}}{l_0} \\
\getpstsome{\localstatekwd}{curid'}{lsp}{(\localstate{ps}{\btrue})}\\
\setpst{\localstatekwd}{curid'}{\localstate{ps}{\bfalse}}{lsp}{lsp'}\\
\oget{A}{\mclog}{\mcevent}{l_0}{oracle}{\yieldbackev{curid}}}
{\envstep{A}{oracle}{(\envstate{A}{curid}{lsp}{l})}{\ccemptytrace}{(\envstate{A}{curid}{lsp'}{l'})}}

\inferrule[skip]
{  l' = \listcons{e}{l} \\
\getpstnone{\localstatekwd}{curid'}{lsp}\\
\oget{A}{\mclog}{\mcevent}{l}{oracle}{e} \\
\backid{curid}{e}{curid'}}
{\envstep{A}{oracle}{(\envstate{A}{curid}{lsp}{l})}{\ccemptytrace}{(\envstate{A}{curid'}{lsp}{l'})}}
\end{mathpar}

\subsection{Introduce Single Core Machine}
\label{subsec:mc-single}

\noindent\fbox{single state:}

$
\begin{array}{lll}
\singlestatekwd & := & \singlestateconkwd :~ \ztype \rightarrow \localstate \rightarrow \mclog \rightarrow \singlestatekwd\\

\end{array}
$

\noindent\fbox{single step: ${}_{[curid : \ztype]} (oracle : \mcoracle{\set{curid}}{\mclog}{\mcevent}) :  \singlestatekwd \rightarrow \cctracekwd \rightarrow \singlestatekwd \rightarrow \mcprop$}

\begin{mathpar}

\inferrule[progress]
{ ls = \localstate{ps}{\bfalse} \\
ls' = \localstate{ps'}{\btrue} \\
\hardwarelocalstep{curid}{(\localview{ps}{l})}{(\localview{ps'}{l'})}}
{\singlestep{curid}{oracle}{(\singlestate{curid}{ls}{l})}{\ccemptytrace}{(\singlestate{curid}{ls'}{(\listapp{l'}{l})})}}

\inferrule[yield]
{  l_0 = \listcons{\yieldbackev{curid}}{l} \\
l' = \listcons{\yieldev{curid'}}{l_0} \\
ls = \localstate{ps}{\btrue} \\
ls' = \localstate{ps'}{\btrue} \\
\oget{\set{curid}}{\mclog}{\mcevent}{l_0}{oracle}{\yieldbackev{curid'}}}
{\singlestep{curid}{oracle}{(\singlestate{curid}{ls}{l})}{\ccemptytrace}{(\singlestate{curid'}{ls'}{l'})}}

\inferrule[skip]
{  l' = \listcons{e}{l} \\
curid' \neq curid\\
\oget{\set{curid}}{\mclog}{\mcevent}{l}{oracle}{e} \\
\backid{curid'}{e}{curid_0}}
{\singlestep{curid}{oracle}{(\singlestate{curid'}{ls}{l})}{\ccemptytrace}{(\singlestate{curid_0}{ls}{l'})}}

\end{mathpar}

\noindent\fbox{fairness:}

$
\begin{array}{llll}
\timebound & : & \nattype &\mbox{(time bound for wating scheduling)} \\
\end{array}
$

\noindent\fbox{Yield Back Function:}

\begin{tabular}{l}
$ \yieldbackfunckwd_{(n : \nattype)} (curid : \ztype) (l \ res : \mclog) (oracle : \mcoracle{\set{curid}}{\mclog}{\mcevent}) : \mclog := $\\
$ \left\{\begin{array}{lr}
       \optionnone & \text{for } n  = O \\
       
       \optionsome\ res' & \text{for } n = S\ n' \wedge \oget{\set{curid}}{\mclog}{\mcevent}{\listapp{res}{l}}{oracle}{e}\\ 
                  & \wedge res' := \listcons{e}{res} \wedge \yieldbackfunc{n'}{curid}{l}{res'}{oracle}{r} \\
                  & \wedge e = \yieldbackev{curid'} \wedge curid = curid' \\
        r & \text{for }n = S\ n' \wedge \oget{\set{curid}}{\mclog}{\mcevent}{\listapp{res}{l}}{oracle}{e}\\ 
                  & \wedge res' := \listcons{e}{res} \wedge \yieldbackfunc{n'}{curid}{l}{res'}{oracle}{r} \\
                  & (\wedge e \neq \yieldbackev{curid'} \vee (\wedge e = \yieldbackev{curid'} \wedge curid \neq curid')) \\
          \optionnone  & \text{for }n = S\ n' \wedge \ogetnoeq{\set{curid}}{\mclog}{\mcevent}{\listapp{res}{l}}{oracle} = r \wedge r \neq \optionsome \ e \\
        \end{array}\right .
$\\
\end{tabular}

\noindent\fbox{single big step: ${}_{[curid : \ztype]} (oracle : \mcoracle{\set{curid}}{\mclog}{\mcevent}) :  \singlestatekwd \rightarrow \cctracekwd \rightarrow \singlestatekwd \rightarrow \mcprop$}

\begin{mathpar}

\inferrule[progress]
{ ls = \localstate{ps}{\bfalse} \\
ls' = \localstate{ps'}{\btrue} \\\\
\hardwarelocalstep{curid}{(\localview{ps}{l})}{(\localview{ps'}{l'})}}
{\singlebigstep{curid}{oracle}{(\singlestate{curid}{ls}{l})}{\ccemptytrace}{(\singlestate{curid}{ls'}{(\listapp{l'}{l})})}}



\inferrule[yield]
{ l_0 = \listcons{\yieldbackev{curid}}{l} \\
 ls = \localstate{ps}{\btrue} \\
ls' = \localstate{ps}{\bfalse} \\
\yieldbackfunc{\timebound}{curid}{l_0}{\nulllist}{oracle}{l'}}
{\singlebigstep{curid}{oracle}{(\singlestate{curid}{ls}{l})}{\ccemptytrace}{(\singlestate{curid'}{ls'}{l'})}}

\end{mathpar}

\noindent\fbox{rstate:}
$
\begin{array}{lll}
\rstatekwd & := & \rstateconkwd : \privatestate \rightarrow \mclog \rightarrow \rstatekwd\\
\end{array}
$

\fbox{single big2 step:
$
{}_{[curid : \ztype]} (oracle : \mcoracle{\set{curid}}{\mclog}{\mcevent}) :  \rstatekwd \rightarrow \cctracekwd \rightarrow \rstatekwd \rightarrow \mcprop
$}

\begin{mathpar}

\inferrule[progress]
{  l_0 = \listcons{\yieldbackev{curid}}{l} \\
\yieldbackfunc{\timebound}{curid}{l_0}{\nulllist}{oracle}{l_1} \\
\hardwarelocalstep{curid}{(\localview{ps}{(\listapp{l_1}{l_0})})}{(\localview{ps'}{l'})}}
{\singlebigtwostep{curid}{oracle}{(\rstate{ps}{l})}{\ccemptytrace}{(\rstate{ps'}{(\listapp{l'}{\listapp{l_1}{l_0}})})}}

\end{mathpar}

\noindent\fbox{single local view:}
$
\begin{array}{lll}
\singlelocalviewkwd & := & \singlelocalviewconskwd : \privatestate \rightarrow \singlelocalviewkwd\\
\end{array}
$

\noindent\fbox{single local step: $\ztype \rightarrow \singlelocalviewkwd \rightarrow \singlelocalviewkwd \rightarrow \mcprop$}

\begin{mathpar}[private]
\inferrule
{\programcounter{ps}{\privatecmdkwd}\\
\privatestep{curid}{ps}{ps'}}
{ \singlelocalstep{curid}{(\singlelocalview{ps})}{(\singlelocalview{ps'})}}
\end{mathpar}

\fbox{single log step: $\ztype \rightarrow \localviewkwd \rightarrow \localviewkwd \rightarrow \mcprop$}
\begin{mathpar}
\inferrule[acquire shared]
{\programcounter{ps}{\acqsharedcmd{id}}\\
\calowner{l}{id}{\ofreestate{d}}\\
\setsharedstep{ps}{d}{ps'}}
{ \singlelogstep{curid}{(\localview{ps}{l})}{(\localview{ps'}{(\listcons{\acqev{curid}{id}}{\nulllist})})}}

\inferrule[release shared]
{\programcounter{ps}{\relsharedcmd{id}}\\
\calowner{l}{id}{\ownstate{curid}}\\
\getsharedstep{ps}{d}{ps'}}
{ \singlelogstep{curid}{(\localview{ps}{l})}{(\localview{ps'}{(\listcons{\relev{curid}{id}{d}}{\nulllist})})}}

\inferrule[atomic]
{  \programcounter{ps}{\atomiccmd{id}{\atomiceventident{e}}}\\
\atomicstep{curid}{id}{ps}{l}{ps'}{e}}
{ \singlelogstep{curid}{(\localview{ps}{l})}{(\localview{ps'}{(\listcons{\atomicev{curid}{id}{e}}{\nulllist})})}}
\end{mathpar}


\noindent\fbox{srstate:}

$
\begin{array}{lll}
\srstatekwd & := & \srstateconkwd : \privatestate \rightarrow \mclog \rightarrow \mclog \rightarrow \srstatekwd \\
\end{array}
$

\noindent\fbox{single split step: ${}_{[curid : \ztype]} (oracle : \mcoracle{\set{curid}}{\mclog}{\mcevent}) :  \srstatekwd \rightarrow \cctracekwd \rightarrow \srstatekwd \rightarrow \mcprop$}

\begin{mathpar}
\inferrule
{al_0 = \listcons{\yieldev{curid}}{al}\\
l_0 = \listapp{al_0}{l}\\
\yieldbackfunc{\timebound}{curid}{l_0}{\nulllist}{oracle}{l_1}\\
\singlelocalstep{curid}{(\singlelocalview{ps})}{(\singlelocalview{ps'})}}
{\singlesplitstep{curid}{oracle}{(\srstate{ps}{l}{al})}{\ccemptytrace}{(\srstate{ps'}{l}{(\listapp{l_1}{al_0})})}}

\inferrule
{l_0 = \listapp{\listcons{\yieldev{curid}}{al}}{l}\\
\yieldbackfunc{\timebound}{curid}{l_0}{\nulllist}{oracle}{l_1}\\
\singlelogstep{curid}{(\localview{ps}{(\listapp{l_1}{l_0})})}{(\localview{ps'}{l'})}}
{\singlesplitstep{curid}{oracle}{(\srstate{ps}{l}{al})}{\ccemptytrace}{(\srstate{ps'}{(\listapp{l'}{\listapp{l_1}{l_0}})}{\nulllist})}}
\end{mathpar}



\noindent\fbox{single reorder step: ${}_{[curid : \ztype]} (oracle : \mcoracle{\set{curid}}{\mclog}{\mclog}) :  \srstatekwd \rightarrow \cctracekwd \rightarrow \srstatekwd \rightarrow \mcprop$}

\begin{mathpar}
\inferrule
{\singlelocalstep{curid}{(\singlelocalview{ps})}{(\singlelocalview{ps'})}}
{\singlereorderstep{curid}{oracle}{(\rstate{ps}{l})}{\ccemptytrace}{(\rstate{ps'}{l})}}

\inferrule
{\oget{\set{curid}}{\mclog}{\mclog}{l}{oracle}{l_0}\\
\singlelogstep{curid}{(\localview{ps}{(\listapp{l_0}{l})})}{(\localview{ps'}{l'})}}
{\singlereorderstep{curid}{oracle}{(\rstate{ps}{l})}{\ccemptytrace}{(\rstate{ps'}{(\listapp{l'}{\listapp{l_0}{l}})})}}

\end{mathpar}




\noindent\fbox{separate log and sp state:}

$
\begin{array}{lll}
\separateeventkwd & := &\separateacqevkwd\langle from : \ztype\rangle~\vert~\separaterelevkwd\langle from : \ztype, d: \sharedpiece\rangle\\
   & & \vert~\separateatomicevkwd\langle from:\ztype, e:\atomicevent\rangle\\
\separatelogtypekwd & := & \listconstructorkwd\ \separateeventkwd\\
\separatelogkwd & := &   \ztype \rightarrow \separatelogtypekwd \\
\separateoraclelogtypekwd & := & \ztype \times \separatelogtypekwd \times \primitiveid \\
\spstatekwd & := & \spstateconkwd : \privatestate \rightarrow \separatelogkwd \rightarrow spstatekwd \\

\end{array}
$

\noindent\fbox{separate log step: $ \ztype \rightarrow \ztype \rightarrow \localviewkwd \rightarrow \privatestate \rightarrow \separateeventkwd \rightarrow \mcprop$}

\begin{mathpar}
\inferrule[acquire]
{\programcounter{ps}{\acqsharedcmd{id}}\\
\calowner{l}{id}{\ofreestate{d}}\\
\setsharedstep{ps}{d}{ps'}}
{\separatelogstep{curid}{id}{(\localview{ps}{l})}{ps'}{({\separateacqev{curid}})}}

\inferrule[release]
{\programcounter{ps}{\relsharedcmd{id}}\\
\calowner{l}{id}{\ownstate{curid}}\\
\setsharedstep{ps}{d}{ps'}}
{\separatelogstep{curid}{id}{(\localview{ps}{l})}{ps'}{({\separaterelev{curid}{d}})}}

\inferrule[atomic]
{  \programcounter{ps}{\atomiccmd{id}{\atomiceventident{e}}}\\
\atomicstep{curid}{id}{ps}{l}{ps'}{e}}
{\separatelogstep{curid}{id}{(\localview{ps}{l})}{ps'}{({\separateatomicev{curid}{e}})}}
\end{mathpar}


\noindent\fbox{auxiliary functions for the differences of the logs:}

\begin{tabular}{l}
$
   \separateeventtwoeventkwd(ev : \separateeventkwd) (id : \ztype) 
      :=  \left\{\begin{array}{lr}
        \acqev{from}{id} & \text{for } \separateacqev{from}\\
        \relev{from}{id}{d} & \text{for } \separaterelev{from}{d}\\
        \atomicev{from}{id}{e} & \text{for } \separateatomicev{from}{e} \\
        \end{array} \right.
$\\
$
   \separatelogtwologkwd(log : \separatelogtypekwd) (id : \ztype) 
      :=  \left\{\begin{array}{lr}
        \nulllist & \text{for } log = \nulllist \\
        \listcons{e_{sp}}{l_{sp}} & \text{for }  log = \listcons{e}{l} \\
        & \wedge \separateeventtwoevent{e}{id}{e_{sp}}\\
        & \wedge \separatelogtwolog{l}{id}{l_{sp}}\\
        \end{array} \right.
$\\
$
\separateeventtwoidkwd (ev : \separateeventkwd) : \primitiveid :=
 \left\{\begin{array}{lr}
         \acqsharedid& \text{for } \separateacqev{\_}\\
        \relsharedid & \text{for } \separaterelev{\_}{\_}\\
        \atomiceventident{e} & \text{for } \separateatomicev{\_}{e} \\
 \end{array} \right.
$\\
\end{tabular}

$
\begin{array}{lll}
  \getseplogkwd (id :\ztype)(spp : \separatelogkwd)  &:= & slog \hfill{\text{for } (id, slog) \in spp}\\
   \setseplogkwd (id :\ztype) (slog : \separatelogtypekwd) (spp: \separatelogkwd) &:= &(spp - \set{(id, \_)})  \cup \set{(id, slog)}\\
\end{array}
$

\noindent\fbox{single separate step:
${}_{[curid : \ztype]} (oracle : \mcoracle{\set{curid}}{\separateoraclelogtypekwd}{\separatelogtypekwd}) :  \spstatekwd \rightarrow \cctracekwd \rightarrow \spstatekwd \rightarrow \mcprop$
}

\begin{mathpar}
\inferrule[local]
{\singlelocalstep{curid}{(\singlelocalview{ps})}{(\singlelocalview{ps'})}}
{\singleseparatestep{curid}{oracle}{(\spstate{ps}{gl})}{\ccemptytrace}{(\spstate{ps'}{gl})}}

\inferrule[log]
{
\getseparatelog{id}{gl}{l}\\
\oget{\set{curid}}{\separateoraclelogtypekwd}{\separatelogtypekwd}{(id, l, \separateeventtwoidnoeq{e}}{oracle}{l_0} \\
\separatelogtwolog{(\listapp{l_0}{l}}{id}{sl}\\
\separatelogstep{curid}{id}{(\localview{ps}{sl})}{ps'}{e}
\setseplogkwd{id}{(\listapp{\listcons{e}{l_0}}{l})}{gl}{gl'}
}
{\singleseparatestep{curid}{oracle}{(\spstate{ps}{gl})}{\ccemptytrace}{(\spstate{ps'}{gl'})}}

\end{mathpar}

%


\clearpage


\subsection{Instantiating the abstract definition}
\label{subsec:instantiate-mc}

Previous sections deal with multiple intermediate languages and the refinement theorems among those languages. 
They, however, are different from the practical hardware semantics such as x86 assembly languages or ARM. 
To avoid the detailed subtleties of defining intermediate languages and the refinement proofs, 
our definitions are relying on multiple abstract variables, syntax and semantics defined in Fig.~\ref{fig:hardware-abstract-definition}.

Thus, providing the per-CPU machine model that can be connected with the \compcertx, 
providing those abstract definitions are required. 


\compcertx\ is a collection of layers, which give the interface for programs on them.


%first starts from mapping the 
%definitions in CompCert with the abstact definitions in our intermediate languages. 
%One benefit of our framework is only one single instantiation is required for 
%those local state transition semantics, and we can reuse them in all intermediate languages. 
%
%Based on the introduced languages, we are a ble to connect them together and then 
%prove the receptiveness and single event property of CompCert semantics, which makes us to be able to use their simulation proof template.
%
%One place, among all refinements, cannot use the forward-backward simulation technique in CompCert. 
%Since the lowest level semantics (hardware semantics) does not contain any	



\clearpage


\subsection{Linking Definition}
\label{subsec:optimization-mc}

With the definitions that we have written in ~\ref{subsubsec:instantiate-mc}, 
Instantiating the intermediate machine models are straightforward.
We used CompCert's small step semantics to encapsulate them. 

%
\clearpage

\subsection{Connecting Intermediate Languages}
\label{subsec:connect-intermediate-language}



\begin{figure}
\fbox{valid log and valid oracle:}


%
%  (* eback is always generated by scheduler *) 
%  Definition event_source (e: event) :=
%    match e with
%      | EYIELD i => i
%      | EBACK _ => sched_id
%      | EACQ i _ => i
%      | EREL i _ _ => i
%      | EATOMIC i _ _ => i
%    end.
%

$
\eventsourcefunckwd (ev : \mcevent) : \ztype := 
 \left\{\begin{array}{lr}
\schedid & \text{for } ev = \yieldbackev{\_} \\
i  & \text{for } ev =\yieldev{i}  \vee ev= \acqev{i}{\_} \\
   & \vee ev = \relev{i}{\_}{\_} \vee ev =  \atomicev{i}{\_}{\_}\\
\end{array} \right.
$

%
%  Definition event_des (e: event) :=
%    match e with
%      | EYIELD _ => sched_id
%      | EBACK i => i
%      | EACQ i _ => i
%      | EREL i _ _ => i
%      | EATOMIC i _ _ => i
%    end.
%

$
\eventdesfunckwd (ev : \mcevent) : \ztype := 
 \left\{\begin{array}{lr}
\schedid & \text{for } ev =\yieldev{\_} \\
i & \text{for } ev= \yieldbackev{i} \vee ev = \acqev{i}{\_}\\
   & ev=  \relev{i}{\_}{\_} \vee ev = \atomicev{i}{\_}{\_}\\
\end{array} \right.
$

%
%  Definition get_curid_from_log (start_core : Z) (l : Log) : Z :=
%    match l with
%    | nil => start_core
%    | e::l' => event_des e
%    end.
%

$
\getcuridfromlogkwd (start\_core : \ztype) (l : \mclog): \ztype :=
 \left\{\begin{array}{lr}
start\_core & \text{for }  l = \nulllist \\
\eventdesfuncnoeq{ev} & \text{for } l = \listcons{ev}{\_}\\
\end{array} \right.
$


%   
%    Definition valid_log_check (start_core : Z) (l : Log) :=
%      match l with
%      | nil => True
%      | e::l' =>
%        match e with
%        | EBACK j => lastEvTy l' = Some YIELDTY /\ core_set j = true
%        | EYIELD from => lastEvTy l' <> Some YIELDTY /\ from = get_curid_from_log start_core l' /\ core_set from = true
%        | _ => match l' with
%              | EBACK j'::_ => j' = event_source e /\  core_set j' = true
%              | _ => False
%              end
%        end
%      end.

\begin{mathpar}
\inferrule[valid log check - nil]
{\ }
{\validlogcheck{start\_core}{\nulllist}}

\inferrule[valid check - yieldback]
{ l = \listcons{e}{l'} \\
e = \yieldbackev{j} \\
l' = \listcons{\yieldev{\_}}{\_}\\
j \in \coreset}
{\validlogcheck{start\_core}{l}}

\inferrule[valid check - yield]
{l = \listcons{e}{l'} \\
e = \yieldev{from} \\
l' \neq \listcons{\yieldev{\_}}{\_}\\
\getcuridfromlog{start\_core}{l'}{from} \\
from \in \coreset}
{\validlogcheck{start\_core}{l}}

\inferrule[valid check - other]
{l = \listcons{e}{l'} \\
e \neq \yieldev{\_} \\
e \neq \yieldbackev{\_} \\
l' = \listcons{\yieldbackev{j'}}{\_}\\
\eventsourcefunc{e}{j'} \\
j' \in \coreset}
{\validlogcheck{start\_core}{l}}
\end{mathpar}

%
%    Inductive valid_log : Z -> Log -> Prop :=
%    | valid_log_nil :
%        forall tid, core_set tid = true -> valid_log_check tid nil -> valid_log tid nil
%    | valid_log_cons:
%        forall tid e l, core_set tid = true -> valid_log tid l -> valid_log_check tid (e::l) -> valid_log tid (e::l).
%

\begin{mathpar}
\inferrule[valid log - nil]
{tid \in \coreset\\
\validlogcheck{tid}{\nulllist}}
{\validlog{tid}{\nulllist}}

\inferrule[valid log - cons]
{tid \in \coreset\\
\validlog{tid}{l}\\
\validlogcheck{tid}{\listcons{e}{l}}}
{\validlog{tid}{\listcons{ev}{l}}}
\end{mathpar}


%    (* valid oracle conditions only for hw oracle and single oracle *)
%    (* similar to other valid_oracle conditions, 
%       it is defined as rely-guarantee style *)
%    Definition valid_oracle (start_core : Z) (A: ZSet) (o: Oracle) :=
%      forall l e,
%        valid_log start_core l -> 
%        oget A l o = Some e ->
%        (* thanks to this condition, hw_oracle only generate EBACK event, so 
%           we can use this valid oracle condition both for hw oracle and single oracle *)
%        A (event_source e) = false /\
%        valid_log start_core (e::l).

$
\begin{array}{l}
\validoraclekwd (start\_core : \ztype) (A : \set{\ztype}) (o : \mcoracle{A}{\mclog}{\mcevent}) := \ \ \ \ \ \ \ \ \ \ \ \ \ \ \ \ \ \ \ \ \ \ \ \  \ \ \ \ \ \ \ \ \ \ \ \ \ \ \ \ \ \  \\
\hfill \forall\ l \ e .\  \validlog{start\_core}{l} \rightarrow \oget{A}{\mclog}{\mcevent}{l}{o}{e} \rightarrow \\
\hfill \eventsourcefuncnoeq{e} \notin A \wedge \validlog{start\_core}{(\listcons{e}{l})}\\
\end{array}
$

\caption{Valid Log and Valid Oracle}
\label{fig:valid-log-valid-oracle}
\end{figure}




%%%%%%%%%%%%%%
\clearpage

\subsubsection{Oracle Refines Hardware}
\label{subsubsec:oracle-refines-hardware}

\begin{figure}
\fbox{variables for match relation}

$
\begin{array}{lllr}
cpuid & : & \ztype & \mbox{(starting CPU of the system)}\\
hw\_o & : & \mcoracle{\coreset}{\mclog}{\mcevent} &  \mbox{(hardware scheduler oracle)}\\
\end{array}
$

\fbox{local match state: $\ztype \rightarrow \ztype \rightarrow \privatestate \rightarrow \localstatekwd \rightarrow \mcprop$}

\begin{mathpar}
\inferrule[match local neq]
{(curid = i \rightarrow b = \btrue) \\
(curid \neq i \rightarrow b = \bfalse) }
{\matchlocal{curid}{i}{ps}{(\localstate{ps}{b})}}
\end{mathpar}

\fbox{match state : $\mcstatekwd \rightarrow \hstatekwd \rightarrow \mcprop$}
\begin{mathpar}
\inferrule[match state]
{(\forall \ i \ ps \ ls \ . \ i \in \coreset \rightarrow \getpstsome{\privatestate}{i}{psp}{ps} \rightarrow\\\\
\getpstsome{\localstatekwd}{i}{lsp}{ls} \rightarrow  \matchlocal{curid}{i}{ps}{ls})\\\\
 l \neq \listcons{\yieldev{\_}}{\_}\\
 curid \in \coreset \\
 \validlog{cpuid}{l} \\
 \getcuridfromlog{cpuid}{l}{curid}}
 {\matchstatehstate{(\mcstate{curid}{lsp}{l})}{(\hstate{curid}{psp}{l})}}
\end{mathpar}

\fbox{hypothesis}

\begin{enumerate}
\item valid$\_$cpuid: $cpuid \in \coreset$
\item valid$\_$oracle$\_$def: $\validoraclenoeq{cpuid}{\coreset}{hw\_o}$
\item related$\_$hw$\_$step$\_$hw$\_$oracle : \\
$
\begin{array}{l}
\matchstatehstate{(\mcstate{curid}{lsp}{l})}{(\hstate{curid}{psp}{l})} \rightarrow \\
\ \ \ \exists\ ev .\ \oget{\coreset}{\mclog}{\mcevent}{\listcons{\yieldev{curid}}{l}}{hw\_o}{ev} \wedge \\
\ \ \ \ \ \ \forall \ curid'' . \ ev = \yieldbackev{curid''} \rightarrow\\
\ \ \ \ \ \ \ \ \  \forall \ curid' \ ps \ st . \\
\ \ \ \ \ \ \ \ \ \ \ \ \hardwarelocalstep{curid'}{(\localview{ps}{(\listcons{\yieldbackev{curid'}}{\listcons{\yieldev{curid}}{l}})})}{st} \rightarrow\\
\ \ \ \ \ \ \ \ \ \ \ \ \ \ \ curid' = curid'' \\
\end{array}
$
\end{enumerate}

\fbox{top lemmas}

\begin{enumerate}
\item one$\_$step$\_$hw$\_$refines$\_$oracle : \\
$
\begin{array}{l}
\forall \ s_l \ s_l' \ t \ s_h . \ \hardwarestep{curid}{s_l}{t}{s_l'} \rightarrow  \matchstatehstate{s_h}{s_l} \rightarrow \\
\ \ \ \ \ \ \ \ \ \ \ \ \ \ \ \ \exists\ s_h' . \  \ccplusstep{\oraclestep{hw\_o}{s_h}{t}{s_h'}} \wedge  \matchstatehstate{s_h'}{s_l'}\\
\end{array}
$
\item match$\_$state$\_$implies$\_$one$\_$step: \\
$
\begin{array}{l}
\forall \ curid \ lsp \ l \ curid' \ hsp \ l' . \ \matchstatehstate{(\mcstate{curid}{lsp}{l})}{(\hstate{curid'}{hsp}{l'})} \rightarrow \\
\ \ \ \ \ \ \ \ \ \ \ \ \ \ \ \  \exists \ curid'' \ lsp' \ l'  . \ \\
\ \ \ \ \ \ \ \ \ \ \ \ \ \ \ \  \ \ \ \ccstarstep{\oraclestep{hw\_o}{(\mcstate{curid}{lsp}{l})}{\ccemptytrace}{(\mcstate{curid''}{lsp}{l'})}} \wedge\\
\ \ \ \ \ \ \ \ \ \ \ \ \ \ \ \  \ \ \ l' = \listcons{\yieldbackev{curid''}}{\listcons{\yieldev{curid}}{l}} \wedge\\
\ \ \ \ \ \ \ \ \ \ \ \ \ \ \ \  \ \ \ (\forall \ ps  . \ \getpstsome{\localstatekwd}{curid}{lsp}{(\localstate{ps}{\btrue})} \rightarrow \\
\ \ \ \ \ \ \ \ \ \ \ \ \ \ \ \  \ \ \ \ \ \ \ \ \ \  \ \ \setpst{\localstatekwd}{curid}{(\localstate{ps}{\bfalse})}{lsp}{lsp'} \\
\end{array}
$
\end{enumerate}

\caption{Oracle Refines Hardware}
\label{fig:oracle-refines-hardware}
\end{figure}

\clearpage


\subsubsection{Environmental Refines Oracle}
\label{subsubsec:env-refines-oracle}

\begin{figure}
\fbox{variables for match relation}

$
\begin{array}{lllr}
cpuid & : & \ztype & \mbox{(starting CPU of the system)}\\
hw\_o & : & \mcoracle{\coreset}{\mclog}{\mcevent} &  \mbox{(hardware scheduler oracle)}\\
\end{array}
$

\fbox{match state : $\envstatekwd_{[\coreset]} \rightarrow \mcstatekwd \rightarrow \mcprop$}
\begin{mathpar}
\inferrule[match state]
{curid \in \coreset \\ 
\validlog{cpuid}{l} \\
l \neq \listcons{\yieldev{\_}}{\_}\\
\getcuridfromlog{cpuid}{l}{curid}\\
(\forall \ ps \ b \ tid  . \ tid \neq curid \rightarrow \getpstsome{\localstatekwd}{tid}{lsp}{(\localstate{ps}{b})} \rightarrow b = \bfalse)\\
(\forall \ ps \ b  . \ \getpstsome{\localstatekwd}{curid}{lsp}{(\localstate{ps}{b})} \rightarrow b = \bfalse \rightarrow \\\\
\exists \ e \ l' . \ l = \listcons{e}{l'} \wedge e = \yieldbackev{curid})} 
{\matchestatestate{(\envstate{\coreset}{curid}{lsp}{l})}{(\mcstate{curid}{lsp}{l})}}
\end{mathpar}


\fbox{hypothesis}

\begin{enumerate}
\item valid$\_$oracle$\_$def: $\validoraclenoeq{cpuid}{\coreset}{hw\_o }$
\end{enumerate}


\fbox{top lemmas}

\begin{enumerate}
\item one$\_$step$\_$env$\_$refines$\_$oracle : \\
$
\begin{array}{l}
\forall \ s_h \ s_h' \ t \ s_l . \ \envstep{\coreset}{hw\_o }{s_h}{t}{s_h'} \rightarrow  \matchestatestate{s_h}{s_l} \rightarrow \\
\ \ \ \ \ \ \ \ \ \ \ \ \ \ \ \ \exists\ s_l' . \  \ccplusstep{\oraclestep{hw\_o }{s_l}{t}{s_l'}} \wedge  \matchstatehstate{s_h'}{s_l'}\\
\end{array}
$
\end{enumerate}

\caption{Environmental Refines Oracle}
\label{fig:env-refines-oracle}
\end{figure}

\clearpage

\subsubsection{Environmental Refines Environmental}
\label{subsubsec:env-refines-emv}

\begin{figure}

\fbox{auxiliary function - decreasing index:}

\begin{tabular}{l}
$
   \mcmeasurekwd (l : \mclog)  : \nattype :=
     \left\{\begin{array}{lr}
        O & \text{for }  l = \listcons{\yieldev{\_}}{\_} \\
        S \ O & \text{otherwise}\\
        \end{array} \right.$ \\
$ \mcstatemeasurekwd (A : \set{\ztype}) (est : \envstatekwd_{[A]}) : \nattype := \mcmeasurenoeq{l} \ \ \ \ \ \ \ \ \ \ (\text{for } est = \envstate{A}{\_}{\_}{l}) $\\
$ \localstepevkwd (ev : \mcevent : \mathrm{option} \mclog := 
     \left\{\begin{array}{lr}
        \mathrm{Some} \ (\listcons{ev}{\nulllist}) & \text{for }  ev = \acqev{\_}{\_} \\
                                        & \vee  ev= \relev{\_}{\_}{\_} \\
                                        & \vee ev= \atomicev{\_}{\_}{\_} \\
        \mathrm{Some} \ \nulllist & \text{for }  ev = \yieldev{\_} \\
	   \mathrm{None} & \text{otherwise}\\
        \end{array} \right. $ \\
\end{tabular}        
%  Definition measure (l : Log) : nat := 
%    match lastEvTy l with 
%    | Some YIELDTY => O
%    | _ => S O 
%    end.
%  
%  Definition state_measure (A : ZSet) (est : estate (A := A)) :=
%    match est with 
%    | EState _ _ l => measure l 
%    end.
%
%  Definition local_step_ev (ev : event) : option Log := 
%    match ev with 
%    | EACQ _ _ => Some (ev::nil)
%    | EREL _ _ _ => Some (ev::nil)
%    | EATOMIC _ _ _ => Some (ev::nil)
%    | EYIELD _ => Some nil
%    | _ => None
%    end.


\caption{Auxiliary Functions for Refinements between two Environment Machines}
\label{fig:aux-functions-for-env-refines-env}
\end{figure}



\begin{figure}
\fbox{variables for match relation:}

$
\begin{array}{lllr}
cpuid & : & \ztype & \mbox{(starting and the focused CPU of the system)}\\
hw\_o & : & \mcoracle{\coreset}{\mclog}{\mcevent} &  \mbox{(hardware scheduler oracle)}\\
si\_o & : & \mcoracle{\set{cpuid}}{\mclog}{\mcevent} &  \mbox{(oracle for a single cpu)}\\
\end{array}
$

\fbox{match state : $\envstatekwd_{[\set{cpuid}]} \rightarrow  \envstatekwd_{[\coreset]}  \rightarrow \mcprop$}

\begin{mathpar}
\inferrule[match state S]
{\mcmeasure{l}{S\ O} \\
curid \in \coreset \\
\validlog{cpuid}{l}\\
\getcuridfromlog{cpuid}{l}{curid}\\
\getpstnoeq{\localstatekwd}{cpuid}{lsp'} = \getpstnoeq{\localstatekwd}{cpuid}{lsp}\\
(curid = cpuid \rightarrow \forall \ ps \ b . \ \getpstsome{\localstatekwd}{cpuid}{lsp}{(\localstate{ps}{b})} \rightarrow\\\\
b = \bfalse \rightarrow \exists \ e \ l' . \ l = \listcons{e}{l'} \wedge e = \yieldbackev{curid})\\\\
(\forall \ tid  . \ tid \neq curid \rightarrow \forall \ ps \ b . \ \getpstsome{\localstatekwd}{tid}{lsp}{(\localstate{ps}{b})} \rightarrow b = \bfalse)\\\\
(curid \neq cpuid \rightarrow l = \listcons{\yieldbackev{\_}}{\_} \rightarrow\\\\
\forall \ ps \ b . \ \getpstsome{\localstatekwd}{curid}{lsp}{(\localstate{ps}{b})} \rightarrow b = \bfalse)\\\\
(curid \neq cpuid \rightarrow l \neq \listcons{\yieldbackev{\_}}{\_} \rightarrow\\\\
\forall \ ps \ b . \ \getpstsome{\localstatekwd}{curid}{lsp}{(\localstate{ps}{b})} \rightarrow b = \btrue)}
{\matchestateestate{\set{cpuid}}{(\envstate{\set{cpuid}}{curid}{lsp'}{l})}{(\envstate{\coreset}{curid}{lsp}{l})}}

\inferrule[match state O]
{\mcmeasure{l'}{O} \\
\getcuridfromlognoeq{cpuid}{l} \in \coreset \\
\validlog{cpuid}{l'}\\
l' = \listcons{\yieldev{curid}}{l} \\ 
curid' = \schedid\\
\getcuridfromlog{cpuid}{l}{curid} \\
\getpstnoeq{\localstatekwd}{cpuid}{lsp'} = \getpstnoeq{\localstatekwd}{cpuid}{lsp}\\
\getpstnone{\localstatekwd}{(\getcuridfromlognoeq{cpuid}{l})}{lsp'}\\
\oget{\set{cpuid}}{\mclog}{\mcevent}{l}{si\_o}{\yieldev{curid}}\\
(\forall \ tid  . \ tid \neq curid \rightarrow \forall \ ps \ b . \ \getpstsome{\localstatekwd}{tid}{lsp}{(\localstate{ps}{b})} \rightarrow b = \bfalse)\\\\
(curid \neq cpuid  \rightarrow \forall \ ps \ b  . \ \getpstsome{\localstatekwd}{curid}{lsp}{(\localstate{ps}{b})} \rightarrow\\\\
b = \bfalse \rightarrow  l = \listcons{\yieldbackev{\_}}{\_})}
{\matchestateestate{\set{cpuid}}{(\envstate{\set{cpuid}}{curid'}{lsp'}{l'})}{(\envstate{\coreset}{curid}{lsp}{l})}}

\end{mathpar}


\fbox{hypothesis}

\begin{enumerate}
\item valid$\_$cpuid: $cpuid \in \coreset$
\item valid$\_$oracle$\_$cond: $\validoraclenoeq{cpuid}{\set{cpuid}}{si\_o}$
\item related$\_$single$\_$oracle$\_$hw$\_$oracle$\_$def : \\
$
\begin{array}{l}
\forall \ l \ ev  . \ \validlog{cpuid}{l} \rightarrow l = \listcons{\yieldev{\_}}{\_} \rightarrow\\
\ \ \ \ \oget{\set{cpuid}}{\mclog}{\mcevent}{l}{si\_o}{ev} \rightarrow \oget{\coreset}{\mclog}{\mcevent}{l}{hw\_o}{ev} \\
\end{array}
$
\item relate$\_$single$\_$oracle$\_$concrete$\_$step$\_$def: \\ 
$
\begin{array}{l}
\forall \ curid' \ curid \ lsp' \ lsp \ l' \ l \ l\_res \ ps \ ev  . \\
\ \ \ \matchestateestate{\set{cpuid}}{(\envstate{\set{cpuid}}{curid'}{lsp'}{l'})}{(\envstate{\coreset}{curid}{lsp}{l})} \rightarrow\\
\ \ \ \getcuridfromlognoeq{cpuid}{l} \neq cpuid \rightarrow \\
\ \ \ \getpstsome{\localstatekwd}{\getcuridfromlognoeq{cpuid}{l}}{lsp}{(\localstate{ps}{false})} \rightarrow \\
\ \ \ \oget{\set{cpuid}}{\mclog}{\mcevent}{l}{si\_o}{ev} \rightarrow \\
\ \ \ \localstepev{ev}{l\_res} \rightarrow l = \listcons{\yieldbackev{\_}}{\_} \rightarrow \\
\ \ \ \exists\ (ps' : \privatestate) . \ \hardwarelocalstep{(\getcuridfromlognoeq{cpuid}{l})}{(\localview{ps}{l})}{(\localview{ps'}{l\_res})} \\
\end{array}
$
\end{enumerate}


\fbox{top lemmas}

\begin{enumerate}
\item one$\_$step$\_$env$\_$refines$\_$oracle : \\
$
\begin{array}{l}
\forall \ s_h \ s_h' \ t \ s_l. \ \envstep{\set{cpuid}}{si\_o}{s_h}{t}{s_h'} \rightarrow  \matchestateestate{\set{cpuid}}{s_h}{s_l} \rightarrow \\
\ \ \ \ \ \ \ \ \ \ \ \ \ \ \ \ (\exists \ s_l' .\ (\ccplusstep{\envstep{\coreset}{hw\_o}{s_l}{t}{s_l'}} \wedge\\
\ \ \ \ \ \ \ \ \ \ \ \ \ \ \ \ \ \ \ \ \ \ \ \ \  \matchestateestate{\set{cpuid}}{s_h'}{s_l'}) \vee \\ 
\ \ \ \ \ \ \ \ \ \ \ \ \ \ \ \ (\mcstatemeasurenoeq{\set{cpuid}}{s_h'} < \mcstatemeasurenoeq{\set{cpuid}}{s_h} \wedge t = \ccemptytrace \wedge \\ 
\ \ \ \ \ \ \ \ \ \ \ \ \ \ \ \ \matchestateestate{\set{cpuid}}{s_h'}{s_l})\\
\end{array}
$
\end{enumerate}


\caption{Environmental Refines Environmental}
\label{fig:env-refines-env}
\end{figure}


\clearpage


\subsubsection{Single Refines Environmental}
\label{subsubsec:single-refines-emv}

\begin{figure}
\fbox{variables for match relation}

$
\begin{array}{lllr}
cpuid & : & \ztype &  \mbox{(starting and the focused CPU of the system)}\\
\end{array}
$

\fbox{match state : $\singlestatekwd \rightarrow \envstatekwd_{[cpuid]} \rightarrow \mcprop$}
\begin{mathpar}
\inferrule[match state]
{\getpstsome{\localstatekwd}{cpuid}{lsp}{ls}}{\matchsinglestatestate{\set{cpuid}}{(\singlestate{curid}{ls}{l})}{(\envstate{\set{cpuid}}{curid}{lsp}{l})}}
\end{mathpar}

\fbox{top lemmas}
\begin{enumerate}
\item one$\_$step$\_$single$\_$refines$\_$env : \\
$
\begin{array}{l}
\forall \ si\_o \ s_h \ s_h' \ t \ s_l . \ \singlestep{cpuid}{si\_o}{s_h}{t}{s_h'} \rightarrow  \matchsinglestatestate{\set{cpuid}}{s_h}{s_l} \rightarrow \\
\ \ \ \ \ \ \ \ \ \ \ \ \ \ \ \ \exists\ s_l' . \  \ccplusstep{\envstep{\set{cpuid}}{si\_o}{s_l}{t}{s_l'}} \wedge  \matchsinglestatestate{\set{cpuid}}{s_h'}{s_l'}\\
\end{array}
$
\end{enumerate}

\caption{Single Refines Environment}
\label{fig:single-refines-env}
\end{figure}

\clearpage

\subsubsection{Big Refines Single}
\label{subsubsec:big-refines-single}

\begin{figure}
\fbox{variables for match relation}

$
\begin{array}{lllr}
cpuid & : & \ztype & \mbox{(starting CPU of the system)}\\
si\_o & : & \mcoracle{\set{cpuid}}{\mclog}{\mcevent} &  \mbox{(oracle for a single cpu)}\\
\end{array}
$

\fbox{match state : $\envstatekwd_{[\coreset]} \rightarrow \mcstatekwd \rightarrow \mcprop$}
\begin{mathpar}
\inferrule[match state]
{ s_h = s_l \\
s_h = \singlestate{tid}{(\localstate{ps}{b})}{l} \\
\validlog{cpuid}{l} \\
l \neq \listcons{\yieldev{\_}}{\_}\\
\getcuridfromlog{cpuid}{l}{cpuid}\\
(b = \bfalse \rightarrow \exists \ e \ l' . \ l = \listcons{e}{l'} \wedge e = \yieldbackev{cpuid})}
{\matchbsstatesstate{s_h}{s_l}}
\end{mathpar}


\fbox{hypothesis}

\begin{enumerate}
\item valid$\_$cpuid: $cpuid \in \coreset$
\item valid$\_$oracle$\_$def: $\validoraclenoeq{cpuid}{\set{cpuid}}{si\_o }$
\end{enumerate}


\fbox{top lemmas}

\begin{enumerate}
\item start$\_$step$\_$get$\_$env$\_$log: \\
$\begin{array}{l}
\forall \ n \ l \ res \ l' \ ps \ curid . \ cpuid \neq curid \rightarrow \yieldbackfunc{n}{cpuid}{l}{res}{si\_o}{l'} \rightarrow\\
\ccstarstep{\singlestep{cpuid}{si\_o}{(\singlestate{curid}{ps}{(\listapp{res}{l})})}{\ccemptytrace}{(\singlestate{cpuid}{ps}{(\listapp{l'}{l})})}}\\ 
\end{array}
$
\item one$\_$step$\_$big$\_$refines$\_$single : \\
$
\begin{array}{l}
\forall \ s_h \ s_h' \ t \ s_l . \ \singlebigstep{cpuid}{si\_o}{s_h}{t}{s_h'} \rightarrow  \matchbsstatesstate{s_h}{s_l} \rightarrow \\
\ \ \ \ \ \ \ \ \ \ \ \ \ \ \ \ \exists\ s\_l' . \  \ccplusstep{\singlestep{cpuid}{si\_o}{s_l}{t}{s_l'}} \wedge  \matchbsstatesstate{s_h'}{s_l'}\\
\end{array}
$
\end{enumerate}

\caption{Big Refines Single}
\label{fig:big-refines-single}
\end{figure}

\clearpage

\subsubsection{Big2 Refines Big}
\label{subsubsec:bigtwo-refines-big}

\begin{figure}
\fbox{match state : $\rstatekwd \rightarrow \singlestatekwd \rightarrow \mcprop$}

\begin{mathpar}
\inferrule[match state]
{\ }
{\matchrstatesinglestate{cpuid}{(\rstate{ps}{l})}{(\singlestate{cpuid}{(\localstate{ps}{true})}{l}}}
\end{mathpar}


\fbox{hypothesis}

\begin{enumerate}
\item valid$\_$oracle$\_$def: $\validoraclenoeq{cpuid}{\coreset}{hw\_o }$
\end{enumerate}


\fbox{top lemmas}

\begin{enumerate}
\item one$\_$step$\_$big2$\_$refines$\_$big : \\
$
\begin{array}{l}
\forall \ s_h \ s_h' \ t \ s_l . \ \singlebigtwostep{cpuid}{si\_o}{s_h}{t}{s_h'} \rightarrow  \matchrstatesinglestate{cpuid}{s_h}{s_l} \rightarrow \\
\ \ \ \ \ \ \ \ \ \ \ \ \ \ \ \ \exists\ s\_l' . \  \ccplusstep{\singlebigstep{cpuid}{si\_o}{s_l}{t}{s_l'}} \wedge  \matchrstatesinglestate{cpuid}{s_h'}{s_l'}\\
\end{array}
$

\end{enumerate}

\caption{Big2 Refines Big}
\label{fig:bigtwo-refines-big}
\end{figure}

\clearpage


\subsubsection{Split Refines Big2}
\label{subsubsec:split-refines-bigtwo}

\begin{figure}

\fbox{match state : $\srstatekwd \rightarrow \rstatekwd \rightarrow \mcprop$}
\begin{mathpar}
\inferrule[match state]
{l' = \listapp{al}{l}}
{\matchsrstaterstate{(\srstate{ps}{l}{al})}{(\rstate{ps}{l'})}}
\end{mathpar}


\fbox{top lemmas}

\begin{enumerate}
\item single$\_$local$\_$imply$\_$env:\\
$
\begin{array}{l}
\forall \ cpuid \ l \ ps \ ps' .\ \singlelocalstep{cpuid}{(\singlelocalview{ps})}{(\singlelocalview{ps'})} \rightarrow \\
\ \ \ \ \ \ \ \ \ \ \ \ \ \ \ \ \hardwarelocalstep{cpuid}{(\localview{ps}{l})}{(\localview{ps'}{\nulllist})}\\
\end{array}
$
\item single$\_$log$\_$imply$\_$env:\\
$
\begin{array}{l}
\forall \ cpuid \ l \ l' \ ps \ ps' .\ \singlelogstep{cpuid}{(\localview{ps}{l})}{(\localview{ps'}{l'})} \rightarrow \\
\ \ \ \ \ \ \ \ \ \ \ \ \ \ \ \ \hardwarelocalstep{cpuid}{(\localview{ps}{l})}{(\localview{ps'}{l'})}\\
\end{array}
$
\item one$\_$step$\_$split$\_$refines$\_$big2 : \\
$
\begin{array}{l}
\forall \ cpuid \ si\_o \ s_h \ s_h' \ t \ s_l . \ \singlesplitstep{cpuid}{si\_o}{s_h}{t}{s_h'} \rightarrow  \matchsrstaterstate{s_h}{s_l} \rightarrow \\
\ \ \ \ \ \ \ \ \ \ \ \ \ \ \ \ \exists\ s_l' . \  \ccplusstep{\singlebigtwostep{cpuid}{si\_o}{s_l}{t}{s_l'}} \wedge  \matchsrstaterstate{s_h'}{s_l'}\\
\end{array}
$
\end{enumerate}

\caption{Split Refines Big2}
\label{fig:split-refines-bigtwo}
\end{figure}


\clearpage

\subsubsection{Reorder Refines Split}
\label{subsubsec:reorder-refines-split}

\begin{figure}
\fbox{auxiliary function}
%  Definition valid_cache_log_members (tid : Z) (l : Log) :=
%    forall ev, In ev l -> event_source ev <> tid \/ GetEvTy ev = YIELDTY.
\begin{mathpar}
\inferrule[nil]
{ \ }
{\validcachelogmembers{cpuid}{\nulllist}}

\inferrule[nonil]
{ev \in l\\
(\eventsourcefuncnoeq{ev} \neq cpuid \vee ev = \yieldev{\_})}
{\validcachelogmembers{cpuid}{l}}
\end{mathpar}

\fbox{variables for match relation}

$
\begin{array}{lllr}
cpuid & : & \ztype & \mbox{(starting CPU of the system)}\\
si\_o & : & \mcoracle{\set{cpuid}}{\mclog}{\mcevent} &  \mbox{(oracle for a single cpu)}\\
re\_o & : & \mcoracle{\set{cpuid}}{\mclog}{\mclog} &  \mbox{(oracle for a single cpu - reordered)}\\
\end{array}
$

\fbox{match state : $\rstatekwd \rightarrow \srstatekwd \rightarrow \mcprop$}
\begin{mathpar}
\inferrule[match state]
{ ps_h = ps_l \\
l_h = l_l \\
\validlog{cpuid}{(\listapp{al}{l_l})}\\
\getcuridfromlog{cpuid}{\listapp{al}{l_l}}{cpuid}\\
l_h \neq \listcons{\yieldev{\_}}{\_} \\
(al = \nulllist \vee (\exists \ ev \ al' . \ al = \listcons{ev}{al'} \wedge ev = \yieldbackev{cpuid}))\\
\validcachelogmembers{cpuid}{al}}
{\matchrstatesrstate{(\rstate{ps_h}{l_h})}{(\srstate{ps_l}{l_l}{al})}}
\end{mathpar}


\fbox{hypothesis}

\begin{enumerate}
\item valid$\_$oracle$\_$def: $\validoraclenoeq{cpuid}{\set{cpuid}}{si\_o }$
\item YieldBack$\_$consistence$\_$with$\_$reorder$\_$o :\\
$
\begin{array}{l}
\forall \ ps_h \ l_h \ ps_l \ l_l \ al.\ \matchrstatesrstate{(\rstate{ps_h}{l_h})}{(\srstate{ps_l}{l_l}{al})} \rightarrow \\
\ \ \ \ \ \ \ \ \ \ \validlog{cpuid}{(\listcons{\yieldev{cpuid}}{\listapp{al}{l_l}})} \rightarrow \\
\ \ \ \ \ \ \ \ \ \ \exists\ l_1. \ \yieldbackfunc{\timebound}{cpuid}{\listcons{\yieldev{cpuid}}{\listapp{al}{l_l}}}{\nulllist}{si\_o}{l_1} \wedge \\ 
\ \ \ \ \ \ \ \ \ \ \ \ \ \ \ \ \ \forall \ l_1' .\  \oget{\set{cpuid}}{\mclog}{\mclog}{l_h}{re\_o}{l_1'} \rightarrow \\ 
\ \ \ \ \ \ \ \ \ \ \ \ \ \ \ \ \  \ \ \ \ \ \ l_1' = \listapp{l_1}{\listcons{\yieldev{cpuid}}{al}} \\
\end{array}
$
\end{enumerate}


\fbox{top lemmas}

\begin{enumerate}
\item one$\_$step$\_$reorder$\_$refines$\_$split : \\
$
\begin{array}{l}
\forall \ s_h \ s_h' \ t \ s_l . \ \singlereorderstep{cpuid}{re\_o}{s_h}{t}{s_h'} \rightarrow  \matchrstatesrstate{s_h}{s_l} \rightarrow \\
\ \ \ \ \ \ \ \ \ \ \ \ \ \ \ \ \exists\ s_l' .\ \singlesplitstep{cpuid}{si\_o}{s_l}{t}{s_l'} \wedge  \matchrstatesrstate{s_h'}{s_l'}\\
\end{array}
$
\end{enumerate}

\caption{Reorder Refines Split}
\label{fig:reorder-refines-split}
\end{figure}


\clearpage

\subsubsection{Reorder Refines Reorder}
\label{subsubsec:reorder-refines-reorder}



\begin{figure}

%  Fixpoint reduce_log (l: Log) :=
%    match l with
%      | nil => nil
%      | e :: l' =>
%        match log_event_source e with
%          | None => (reduce_log l')
%          | _ => e :: (reduce_log l')
%        end
%    end.

\fbox{auxiliary function}

$
\mcreducelogfunckwd (l : \mclog) : \mclog := 
 \left\{\begin{array}{lr}
\nulllist & \text{for } l = \nulllist \\
\listcons{ev}{l_r'} & \text{for } l = \listcons{ev}{l'} \wedge \mcreducelogfunc{l'}{l_r'}\\
&\wedge(ev \neq \yieldev{\_} \wedge ev \neq \yieldbackev{\_})\\
%& (ev = \acqev{\_} \vee ev = \relev{\_}{\_}{\_} \vee ev = \atomicev{\_}{\_}{\_})\\
l_r' & \text{for } l = \listcons{ev}{l'} \wedge \mcreducelogfunc{l'}{l_r'}\\
&\wedge (ev = \yieldev{\_} \vee ev = \yieldbackev{\_})\\
\end{array} \right.
$


\fbox{variables for match relation}

$
\begin{array}{lllr}
cpuid & : & \ztype & \mbox{(starting CPU of the system)}\\
hi\_re\_o & : & \mcoracle{\set{cpuid}}{\mclog}{\mclog} &  \mbox{(oracle for a single cpu - simplified)}\\
lo\_re\_o & : & \mcoracle{\set{cpuid}}{\mclog}{\mclog} &  \mbox{(oracle for a single cpu - reordered)}\\
\end{array}
$

%\fbox{match state : $\rstatekwd \rightarrow \rstatekwd \rightarrow \mcprop$}
\begin{mathpar}
\inferrule[match state]
{\mcreducelogfunc{l}{l'}}
{\matchrrstaterrstae{(\rstate{ps}{l'})}{(\rstate{ps}{l})}}
\end{mathpar}


\fbox{hypothesis}

\begin{enumerate}
\item relate$\_$reorder$\_$oracle$\_$def: \\
$
\begin{array}{l}
\forall \ ps \ l.\ \matchrrstaterrstae{(\rstate{ps}{(\mcreducelogfuncnoeq{l})}}{(\rstate{ps}{l})} \rightarrow \\
\ \ \ \ \ \ \ \ \ \ \ \ \forall \ l_h. \ \oget{\set{cpuid}}{\mclog}{\mclog}{(\mcreducelogfuncnoeq{l})}{hi\_re\_o}{l\_h} \rightarrow \\
\ \ \ \ \ \ \ \ \ \ \ \ \ \ \ \ \ \ \exists \ l. \ \oget{\set{cpuid}}{\mclog}{\mclog}{l}{lo\_re\_o}{l\_l} \wedge \mcreducelogfunc{l\_l}{l\_h} \\
\end{array}
$
\end{enumerate}


\fbox{top lemmas}

\begin{enumerate}
\item one$\_$step$\_$reorder$\_$refines$\_$reorder : \\
$
\begin{array}{l}
\forall \ s_h \ s_h' \ t \ s_l . \ \singlereorderstep{cpuid}{hi\_re\_o}{s_h}{t}{s\_h'} \rightarrow \matchrrstaterrstae{s_h}{s_l} \rightarrow \\
\ \ \ \ \ \ \ \ \ \ \ \ \ \ \ \ \exists\ s_l' . \  \singlereorderstep{cpuid}{lo\_re\_o}{s_l}{t}{s\_l'} \wedge  \matchrrstaterrstae{s_h'}{s_l'}\\
\end{array}
$
\end{enumerate}

\caption{Reorder Refines Reorder}
\label{fig:reorder-refines-reorder}
\end{figure}


\clearpage

\subsubsection{Separate Refines Reorder}
\label{subsubsec:separate-refines-reorder}

\begin{figure}

\fbox{auxiliary function}
%
%  Fixpoint remove_cache_event (l: Log) (tid: Z) :=
%    match l with
%      | nil => nil
%      | e :: l' =>
%        if zeq tid (event_source e) then l
%        else remove_cache_event l' tid
%    end
%

$
\removecacheeventkwd (l : \mclog) (cpuid : \ztype): \mclog := 
 \left\{\begin{array}{lr}
\nulllist & \text{for } l = \nulllist \\
\listcons{ev}{l_r'} & \text{for } l = \listcons{ev}{l'}  \wedge \eventsourcefunc{e}{cpuid}\\
    &  \wedge \removecacheevent{l'}{cpuid}{l_r'}\\
l_r' & \text{for } l = \listcons{ev}{l'}  \wedge \eventsourcefuncnoeq{e}\neq {cpuid}\\
    &  \wedge \removecacheevent{l'}{cpuid}{l_r'}\\
\end{array} \right.
$


\begin{mathpar}
\inferrule[valid log']
{\mcreducelogfunc{l}{l}}
{\validlogprim{l}}
\end{mathpar}
%
%  Definition valid_log' (l : Log) : Prop :=
%    l = reduce_log l.
%
%  Definition valid_oracle' (tid: Z) (o : Oracle (OracleProp := re_op)) :=
%    forall l l', 
%      valid_log' l ->
%      oget (s_set tid) l o = Some l' ->
%      (forall e, In e l' -> event_source e <> tid) /\
%      valid_log' l'.
%  
%

$
\begin{array}{l}
\validoracleprimkwd (cpuid: \ztype) (o : \mcoracle{\set{cpuid}}{\mclog}{\mclog}) := \\
\ \ \ \ \forall \ l \ l' . \ \validlogprim{l} \rightarrow \oget{\set{cpuid}}{\mclog}{\mclog}{l}{o}{l'} \rightarrow\\
\ \ \ \ \ \ \ \ (\forall \ e . e \in l' \rightarrow \eventsourcefuncnoeq{e} \neq cpuid) \wedge \validlogprim{l'}\\
\end{array}
$

\fbox{variables for match relation}

$
\begin{array}{lllr}
cpuid & : & \ztype & \mbox{(starting CPU of the system)}\\
sep\_o & : & \mcoracle{\set{cpuid}}{\separateoraclelogtypekwd}{\separatelogtypekwd} &  \mbox{(separate log type oracle)}\\
re\_o & : & \mcoracle{\set{cpuid}}{\mclog}{\mclog} &  \mbox{(reordered log oracle)}\\
\end{array}
$

\fbox{match logs : $\ztype \rightarrow \separatelogkwd \rightarrow \mclog \rightarrow \mcprop$}
\begin{mathpar}
\inferrule[match log]
{\getseparatelog{id_{prim}}{l_{sep}}{l_{h}}\\
\loggetatom{l_{glob}}{id_{prim}}{l_l}\\
\separatelogtwolog{l_h}{id_{prim}}{\removecacheeventnoeq{l_l}{curid}}}
{\mcmatchlogs{curid}{l_{sep}}{l_{glob}}}
\end{mathpar}

\fbox{match state : $\ztype \rightarrow \spstatekwd \rightarrow \rstatekwd \rightarrow \mcprop$}
\begin{mathpar}
\inferrule[match state]
{\validlogprim{l_{glob}}\\
\mcmatchlogs{curid}{l_{sep}}{l_{glob}}}
{\matchspstaterstate{curid}{(\spstate{ps}{l_{sep}})}{(\rstate{ps}{l_{glob}})}}
\end{mathpar}

\fbox{hypothesis}

\begin{enumerate}
\item valid$\_$oracle$'\_$def: $\validoracleprim{cpuid}{re\_o }$
\item relate$\_$separate$\_$oracle$\_$reorder$\_$oracle$\_$def:\\
$
\begin{array}{l}
\forall \ ps \ l_{sep} \ l_{glob} \ id  \ l_{id} \ l_{id}' \ e .\ \matchspstaterstate{curid}{(\spstate{ps}{l_{sep}})}{(\rstate{ps}{l_{glob}})} \rightarrow \\
\ \ \ \ \ \ \ \ \ \ \ \ \ \ \ \ \getseparatelog{id}{l_{sep}}{l_{id}} \rightarrow \\
\ \ \ \ \ \ \ \ \ \ \ \ \ \ \ \  \oget{\set{cpuid}}{\separateoraclelogtypekwd}{\separatelogtypekwd}{(id, l_{id}, e)}{sep\_o}{l_{id}'} \\
\ \ \ \ \ \ \ \ \ \ \ \ \ \ \ \   \exists \ l_{glob}' .\ \oget{\set{cpuid}}{\mclog}{\mclog}{l_{glob}}{re\_o}{l_{glob}'} \wedge \\
\ \ \ \ \ \ \ \ \ \ \ \ \ \ \ \  \ \ \ \ \ \ \loggetatom{\listapp{l_{glob}'}{l_{glob}}}{id}{\separatelogtwolognoeq{\listapp{l_{id}'}{l_{id}}}{id}} \\
\end{array}
$
\end{enumerate}


\fbox{top lemmas}

\begin{enumerate}
\item one$\_$step$\_$separate$\_$refines$\_$reorder : \\
$
\begin{array}{l}
\forall \ s_h \ s_h' \ t \ s_l . \ \singleseparatestep{cpuid}{sep\_o}{s_h}{t}{s_h'} \rightarrow  \matchspstaterstate{cpuid}{s_h}{s_l} \rightarrow \\
\ \ \ \ \ \ \ \ \ \ \ \ \ \ \ \ \exists\ s_l' . \  \singlereorderstep{cpuid}{re\_o}{s_l}{t}{s_l'} \wedge  \matchspstaterstate{cpuid}{s_h'}{s_l'}\\
\end{array}
$
\end{enumerate}

\caption{Separate Refines Reorder}
\label{fig:separate-refines-reorder}
\end{figure}

%
\section{Building Certified Multithreaded Layers}
\label{sec:multithreaded-layers}
Multithreaded programs have to deal with interleavings triggered by not only  the hardware scheduler
but also the  explicit invocation
of thread scheduling primitives. 
Therefore, introducing multithreaded layer interface requires additional works to handle thread scheduling primitives. 
First, our framework has to abstract away other threads' behavior 
as an environmental context associated with the thread.
This work is similar with the corresponding parts of our multiprocessor 
layer interface in a sense that abstracting away other components' behavior as environmental contexts.
Secondly, we want to achieve the goal with using the existing framework as much as possible, especially 
the compiler, CompCertX, that we have used in the previous work~\cite{dscal15}.
Even though our framework can handle those issues, showing that a 
large concurrent program can be verified and linked in our framework is an another work. 
Our CCAL Toolkit resolve all the issues.
In this section, we introduce  the certified layers
dealing with scheduling primitives, a new concept of thread-local layer interfaces equipped with
compositional rules, and a thread-safe version of CompCertX.

\subsection{Multithreaded Layer Interface Overview}\label{subsec:machinemodeloverview}

The purpose of multithreaded layer interface is expose the framework that makes users to be able to
program, compile, verify multithreaded programs on top of it with keeping the 
consistency with CPU-local layers that we have built.
In this sense, 
one requirement that the interface has to achieve is an ability to 
link multiple machine models (\textit{i.e.} 
we call them $LAsm_{H}$) in thread-local layers 
with one single machine (\textit{i.e.} we call it $LAsm_{L}$)
model in CPU-local layer while 
preserving the common framework in both sides (multiprocessor layer and multithreaded layer interfaces) 
as much as possible. 
We especially want to re-use the compilation phases from C programs to Assembly programs (CompCertX~\cite{dscal15}) 
in both side due to the size and complexity of those parts. 
However, providing the common framework in both sides 
cannot be done easily due to multiple challenges that we have described. 
To bridge the gap between the  machine model in CPU-local layers and
that in thread-local layers,
we introduce two intermediate machine models, $EAsm$ and $TAsm$. 

\begin{figure}[t]
\includegraphics[scale=.40]{figs/ccal/thread-linking}
\caption{Thread Linking}
\label{fig:thread-linking}
\end{figure}

Figure~\ref{fig:thread-linking} briefly shows how those intermediate machine models help us to 
resolve multiple challenges.
The multiprocessor layer interface (c.f\ Fig.~\ref{fig:thread-linking} (1)) is still a CPU-local layer,
and the layer contains only one register set and one private abstract data in its state which are associated
with the CPU.
The layer definitely captures the execution of the whole thread set of CPU $c$ 
and does not support thread-local reasoning.
The first step of building multithreaded layer interface is 
dividing CPU-local private data (a private register set and a private abstract data) into multiple thread-local
private datum (c.f\ Fig.~\ref{fig:thread-linking} (2)). 
Since the layer contains multiple private datum, we also add the flag for currently-running thread ID $curid$ in the state. 
At this stage, 
context switching becomes simple because we do not need to store and load the corresponding contexts from/to 
the abstract data structure to/from the registers. 
Instead of that, the machine only needs to change the current thread id ($curid$) to mark 
the thread id that currently has the running control. 

Now each thread can use its own private data for its evaluation, but that is not sufficient at all. 
In fact, scheduling switches in this layer (c.f\ Fig.~\ref{fig:thread-linking} (2)) 
has a similar meaning with the ones in the lower layer,
which implies that changing the context from one to an another one for the evaluation.
Ideally, we would like to reason about each thread execution 
independently, and later formally combine the reasoning to obtain a global
property for the full set of threads on the same CPU.
So, we need a machine model that gives semantics to
a partially-composed set of threads to support this.

Therefore,  a new layer (c.f\ Fig.~\ref{fig:thread-linking} (3)) is introduced such that other 
threads' operations can be modeled as input strategies to the layer interface 
(as we did in the multiprocessor layer interface). 
Here, we introduced a new kind of environment context, $\oracle^{t}$.
This $\oracle^t$ will only be queried during the execution of scheduling primitives (thread yield/sleep/wakeup),
since our machine model does not allow
preemption. 
Similar to the environmental context query in the multiprocessor layer interface, 
The execution has two kinds of
behaviors  depending on whether the \emph{target
thread} is active or not.
Considering an execution in Fig.~\ref{fig:thread-linking} (3) with an active thread set
$T = \{0\}$, whenever an execution switches (by $\yield$ or $\sleep$) 
to a thread outside of $T_a$ (i.e., the yellow $\yield$),
it takes environmental steps (i.e., notated as arrows), repeatedly appending the 
events returned by the environment context $\oracle$ and the thread
context $\oracle^t$ to the log until a $\yield$
event indicates that control switches back to an active thread.

This layer is already a thread local because it only captures the behavior of one thread.
However, the strategy query in this layer follows small-step style, 
and this is insufficient to build thread-local layer interface because 
we do not want to query multiple times for a single yield or sleep calls.
Therefore, we introduce another layer (we call the machine model in this layer $TAsm$)
to merge those multiple strategy queries into a single 
big-query (c.f\ Fig.~\ref{fig:thread-linking} (4)). 
Finally, the last thing to do to build a multithreaded layer interface
is to connect the machine state of the current intermediate machine ($TAsm$) 
to the similar form of 
our multiprocessor
 layer interface, which has the form of $(\regs, m, a, l)$.
Therefore, we introduce the last layer (Fig.~\ref{fig:thread-linking} (5)) that will become a 
base in building multithreaded layers.

During the remaining parts of this section, we will first explain  all those machine models 
that are introduced in Fig.~\ref{fig:thread-linking}.
We also show key ideas to prove the compositional theorems to link these machine models. 
After that, we explain how we build and refine the concrete layer interface over them. 
As for the second part, we also show the refinement relation between concrete layer definitions.
As the last part of this section, we talk about one another subtlety issue about memory to build multithreaded layer interface. 


\subsection{Certified Layers for Scheduling Primitives}\label{subsec:pbthreadlayer}
Based on the shared thread queues provided by the multicore
toolkit (c.f\ Sec.~\ref{sec:shared-queue}), we introduce a new  layer interface
$\Lbthread[c]$ that supports multithreading.
At this layer interface, the transitions between threads are done using
scheduling primitives,
implemented in a mix of
C and assembly.

Figure~\ref{fig:exp:sched} shows the implementation
of scheduling primitives using CCAL.
\begin{figure}
\lstinputlisting [language = C, multicols=2] {source_code/ccal/scheduling.c}
\caption{Pseudocode of selected scheduling primitives.}
\label{fig:exp:sched}
\end{figure}


In our multithreaded setting, each CPU $c$ has a private ready queue $\commc{rdq}$ 
and a shared pending queue $\commc{pendq}$ (containing the threads woken up by other CPUs). 
A thread yield sends the first pending thread from
 $\commc{pendq}$ to $\commc{rdq}$ and then
switches to the next ready thread. 
There are also many shared sleeping queues
$\commc{slpq}$. 
When a sleeping thread is woken up,
it will be directly appended
to the ready queue if the thread belongs to the currently-running CPU. 
Otherwise, it
will be appended to the pending queue of the CPU it belongs to.

Thread switching is implemented by the context switch function $\commc{cswitch}$, which
saves the current thread's kernel context (i.e., $\comm{ra},
\comm{ebp}, \comm{ebx}, \comm{esi}, \comm{edi}, \comm{esp}$),
and loads the context of the target thread.
This  $\commc{cswitch}$ (invoked by $\commc{yield}$ and $\commc{sleep}$) can only be implemented at the assembly level,
as it does not satisfy the C calling convention.
A scheduling primitive like $\commc{yield}$  first queries  $\oracle$ to update the log,
appends its own  event, and then invokes $\commc{cswitch}$ to transfer the control.%
\[
\includegraphics[width=.5\textwidth]{figs/ccal/thread1}%
\]%
\noindent{}This layer interface introduces three new events
$c.\yield$, $c.\sleep(i, lk)$
(sleep on queue $i$ while holding the lock $lk$), and $c.\wakeup(i)$ (wakeup the queue $i$).
These events record the thread switches, which can be used to track the currently-running thread
by a replay function $\replay_{\comm{sched}}$.
These three events are converted from the events
of multicore toolkit, i.e.\,
$c.\deq(\comm{pendq}(c))$,
$c.\enq(\comm{slpq}(i))$
and $c.\enq(\comm{pendq}(c))$, respectively.
These events encapsulate
the modifications to the 
shared thread control blocks ($\comm{tcbp}$), 
thread queues ($\comm{tdqp}$), 
and the running thread \allid{} ($\comm{tid}$).
Thus, the corresponding abstract states
are hidden at $\Lbthread$ but
can always be reconstructed
by the replay function $\replay_{\comm{sched}}$ given the current log.

%%% FULL EASM %%%%%%%%
\subsection{Intermediate Concurrent Machine Model}\label{subsec:fulleasm}

The CPU-local interface  $\Lbthread[c]$
captures the execution of the whole
thread set of CPU $c$ and does not support thread-local verification.
Ideally, we would like to formally reason about each thread separately
 and later compose the proofs together to obtain a global
property.
Thus, we introduce a new layer interface and a new intermediate concurrent 
 that is compositional 
and only focuses on a subset of threads of CPU $c$.
To support this, we need a machine model ($EAsm$) that gives semantics to
a partially-composed set of threads.

Let $T_c$ denote the whole thread set running over CPU $c$.
Based upon  $\PLayer{L}{c}{}$,  we construct a 
\emph{multithreaded} layer interface $\TLayer{L}{c}{T_c}$ without introducing 
additional rely/guarantee conditions.
The purpose of this step is dividing the  CPU-local state as multiple thread-local states
(from Fig.~\ref{fig:thread-linking} (1) to  Fig.~\ref{fig:thread-linking} (2)) as follows:
\[(\textit{State})\ \ \ s_{EAsm}  :=  (tid, f_{\regs}, m, f_{a},l)\]
, when $f_{\regs_{T}}$ is a partial amp from thread id to private state (registers), $f_{a}$ is a partial
map from thread id to divided abstract states based on the abstract state ($a$) in CPU local layer $\Lbthread[c]$.
This change implies that all threads will have their designated registers and local states.
One another purpose of introducing this layer is replacing scheduling primitives in $\Lbthread[c]$ layer 
with explicit transition rules in its machine model to
replace assembly style context switch ($\commc{cswitch}$) into a no-op 
like operation. 
To do that, we have added software scheduler related rules ($\comm{yield}$ and $\comm{sleep}$) 
in this intermediate machine model ($EAsm$). 
For example, the execution of $\commc{yield}$ in~Sec.\ref{subsec:pbthreadlayer} will be replaced by
the transition rule:
\begin{center}
\begin{tikzpicture}[->,>={stealth[black]}, auto,  node distance=3cm,draw]
\begin{scope}[every node/.style={font=\sffamily\small}]
    \node (A) at (0,0) {};
    \node (B) [node_w] at (0.13,0) {};
    \node (C) [node_db] at (6.13,0) {};
\end{scope}

\begin{scope}[every node/.style={font=\sffamily\footnotesize},
every edge/.style={draw, thick}]
    \path [->] (A) edge (B);
    \path [->] (B) edge node[above]{(exec$\_$yield$_{EAsm}$)} node[below]{$(tid, f_{\regs}, m, f_{a},l) \rightarrow (tid', f_{\regs}, m, f_{a},l')$} (C);
\end{scope}
\end{tikzpicture}
\end{center} 
when $l'$ is ``$l \cdot tid.\commc{yield} (tid \switch tid')$''.
This implies that the $\commc{yield}$ evaluation will only switch the current running thread and update the current global log 
by adding the corresponding event to the log like we the hardware scheduler do in our concurrent multicore machine model.
Thanks to having these explicit evaluation rules for scheduling primitives in Sec.~\ref{subsec:pbthreadlayer} as well as
hread-local registers and private data structures associated with each thread, 
context switching will become much simpler in this level than the assembly style context switching 
in the previous level that we have described in Sec.~\ref{subsec:pbthreadlayer}.


subsection{Intermediate Concurrent Machine Model and Thread Linking}\label{sec:multi-threaded-partial}
The next step in defining thread-local layer interface
is replacing other threads' evaluation using the strategy as we have already seen in Fig.~\ref{fig:thread-linking}. 
In this step, our machine does not guarantee that the scheduled thread id is always a member 
running or available threads because the machine is not a total machine on CPU $c$. 
If the thread is not in both cases, we categorize it as a thread with an $\mathrm{Environment}$ state.
Formally, let's  assume the set $T_a$ which satisfies $T_a \subseteq T_c$.
Based upon  $\PLayer{L}{c}{}$,  we construct a 
 \emph{multithreaded} layer interface $\TLayer{L}{c}{T_a} := (\PLayer{L}{c}{}.\Layer,
 L[c].\Rely\cup \Rely_{T_a}, L[c].\Guard_{|T_a})$,
which is 
parameterized over a focused thread set $T_a \subseteq T_c$.
Besides $T_q$, strategies of other threads running on $c$ form a thread context $\oracle^{t}$.
Rely conditions of this multithreaded layer interface extend $L[c].\Rely$ with a \emph{valid set} of $\oracle^{t}$ (denoted as ``$\Rely_{T_a}$'') and
guarantee conditions replace $L[c].\Guard(c)$ with the invariants held by threads in $T_a$ (denoted as ``$L[c].\Guard_{|T_a}$''). Since our machine model does not allow
preemption, $\oracle^{t}$ will only be queried during the execution of scheduling primitives, which have two kinds
of behaviors  depending on whether the \emph{target
thread} is focused or not.%
\[
\includegraphics[width=.5\textwidth]{figs/ccal/thread2}
\]%
\noindent{}Consider the above execution with 
$T_a = \{0,1\}$. Whenever an execution switches (by $\commc{yield}$ or $\commc{sleep}$) 
to a thread outside of $T_a$ (i.e., the yellow $\commc{yield}$ above),
it takes environmental steps (i.e., inside the red box), repeatedly appending the 
events returned by the environment context $\oracle$ and the thread
context $\oracle^t$ to the log until a $c.\yield$
event indicates that the control has switched back to a focused thread.
Whenever an execution  switches to a focused one (i.e., the blue $\commc{yield}$ above), 
it will  perform the context switch without asking $\oracle/\oracle^t$
and its behavior  is identical to the one of $\Lbthread[c]$.

\para{Composing Multithreaded Layers.}
Multithreaded layer interfaces with disjoint focused thread sets
can also be composed in parallel (using an extended \textsc{Pcomp} rule) if the guarantee condition implies
the rely condition for every thread.
If the active thread sets $T_{A1}$ and $T_{A2}$ of $\TLayer{L}{c}{T_{A1}}$ and $\TLayer{L}{c}{T_{A2}}$  machines
are disjoint, they can be composed together 
to form a machine with the union active thread set $T_{A1} \bigcup T_{A2}$.
The resulting focused thread set is the union of the composed ones,
and some environmental steps
are ``replaced by'' the local steps of the other thread set.
For example, if we compose $T_a$ in the above example
with thread 2, the previously yellow $\commc{yield}$ of thread 0 will then switch to
a focused thread.
\[
\includegraphics[width=.5\textwidth]{figs/ccal/thread3}
\]%
Here, the event list $l_1$ generated by $\oracle$ and $\oracle^t$
has been divided into two parts: ``$l_{1a}\cons c.\yield$'' (generated by thread 2)
and $l_{1b}$ (consisting of events from threads 
outside \{0,1,2\}).

Thanks to this parallel composition rule, given each multithreaded layer $\TLayer{L}{c}{t}$
with a single active thread $t\in T_c$,
we can repeatedly compose them together
and build a layer for the entire thread set on $c$.

\para{Multithreaded Linking.}
When the whole $T_c$  is focused,
all scheduling primitives fall into the second case and never switch to unfocused ones. Thus, 
its scheduling behaviors are equal to the ones of $\Lbthread[c]$. 
To link multiple thread-local machines as one multithreaded concurrent machine,  
we perform the above iteration of building a single-threaded concurrent machine model in a reversed way.
Let us first focus on two single-threaded concurrent machine models, 
$\EAsmM{[c, \{0\}]}$ and $\EAsmM{[c, \{1\}]}$.
Then, partial maps for thread private data ($f_a$) of $\EAsmM{[c, \{0\}]}$ 
and of $\EAsmM{[c, \{1\}]}$ will be defined as\newline
\noindent
\begin{minipage}[t]{.5\textwidth}
\begin{small}
\[
f_{a}(i):=
\begin{cases}
 \comm{Some \ a} & \comm{when} \ i = 0\\
\comm{None} & \text{otherwise}
\end{cases}
\]
\end{small}
\end{minipage}
\begin{minipage}[t]{.5\textwidth}
\begin{small}
\[
f_{a}(i):=
\begin{cases}
 \comm{Some \ a} & \comm{when} \ i = 1\\
\comm{None} & \text{otherwise}
\end{cases}
\]
\end{small}
\end{minipage}
respectively.
Then, merging those two partial maps will be 
\begin{small}
\[
f_{a}(i):=
\begin{cases}
 \comm{Some \ a} & \comm{when} \ i = 0 \vee i = 1\\
\comm{None} & \text{otherwise}
\end{cases}
\]
\end{small}
, which simply merge two maps together. 
For the partial map of the private register set ($f_\regs$), merging
is using the corresponding register set with the current running thread $tid$.
In terms of environmental context, both machines have the same environmental context, which is 
$\oracle^{T}$ even though they have different strategies from each other ($\oracle^{T}[c, 0])$ and $\oracle^{T}[c, 1]$).
With those environmental contexts, 
building environmental context for the thread set $\{0, 1\}$ is straightforward. 
We need to either  exclude the event generated by thread 1 from $\oracle^{T}[c, 0]$ or 
exclude the event generated by thread 0 from $\oracle^{T}[c, 1]$.
Then, having those two single-threaded local machines,
we can merge their private states, registers, and a environmental context, 
and thus build a two-threaded concurrent machine model, $EAsm_{[c, \{0, 1\}]}$.
If $T_c = \{0, 1\}$, then we do not need further merging, 
and the machine directly turns into a total concurrent machine 
model $EAsm_{[c, T_c]}$. 
If not, we pick one thread $i$ (\textit{i.e.} $i \in (T_c - \{0, 1\})$), 
and merge the thread again with the exactly 
same process that we have done for thread $0$ and thread $1$.
Finally,
we can prove
the following theorem:
\begin{theorem}[Multithreaded Linking]
\label{thread_composition}
$$
\PLayer{\Lbthread}{c}{\oracle}\le_\id \TLayer{\Lhthread}{c}{T_c}
$$%
\end{theorem}%
\noindent This theorem guarantees that,
once the multithreaded machine based on $\Lhthread[c][T_c]$ captures
the whole thread set,
 the properties of  threads running on top
can be propagated down to the layer with concrete
scheduling implementations.

\subsection{Intermediate Thread-Local Machine Model}\label{subsec:tasm}
If the previous machine contains only one single thread,
the machine is already a single threaded machine model, which gives us a full isolation with other threads on the same CPU.
However, $EAsm$ itself is quite different with $LAsm_L$ in terms of its state definition and evaluation rules,
so it is hard for us to show the refinement relation between this $EAsm$ and $LAsm_L$ directly. 
Especially, yield back and environmental steps in $EAsm$ does not match well with our $LAsm_L$ rules. 
If we keep those operational style strategy query evaluation, it is hard for us to define a single step behavior of 
scheduling primitives in our thread-local layer interface.
Due to the differences between $EAsm$ and $LAsm_L$, 
we have introduced one more intermediate machine model, $TAsm$,
which is parameterized by the current CPU ID ($c$) and and the thread id ($t$). 
And using these variables, we define this machine's state as
\begin{small}
\[
st = (t, \ \regs, m, a, l) 
\]
\end{small}
, which has a shared log and only one private data for the thread.
In this machine, all other threads' steps should be replaced by environmental steps
as our single-threaded concurrent machine model, $EAsm$, does.
In terms of environmental context query, however, 
this new intermediate machine model ($TAsm$) has a different style (See Fig.~\ref{fig:thread-linking} (3) and Fig.~\ref{fig:thread-linking} (4)).
The previous model ($EAsm$ with a single thread, Fig.~\ref{fig:thread-linking} (3)) 
queries its environmental context via operational style as we have marked as multiple arrows in the figure. 
On the other hand, $TAsm$ queries its environmental context as a big-step style, and 
will directly return to the current thread id with the updated log in a single evaluation. 
For example, yield evaluation rule in this intermediate machine model will have the following single step transition:
\begin{center}
\begin{tikzpicture}[->,>={stealth[black]}, auto,  node distance=3cm,draw]
\begin{scope}[every node/.style={font=\sffamily\small}]
    \node (A) at (0,0) {};
    \node (B) [node_w] at (0.13,0) {};
    \node (C) [node_db] at (6.13,0) {};
\end{scope}

\begin{scope}[every node/.style={font=\sffamily\footnotesize},
every edge/.style={draw, thick}]
    \path [->] (A) edge (B);
    \path [->] (B) edge node[above]{(exec$\_$yield$_{TAsm}$)} node[below]{$(tid, {\regs}, m, {a},l) \rightarrow (tid', {\regs}, m, {a},l')$} (C);
\end{scope}
\end{tikzpicture}
\end{center}
, where the thread id is always remained as same (\textit{i.e.} $tid = tid'$),
and $l' = l \cdot l''$ when $l'' = (?(\oracle, \oracle^{T}), !tid.\commc{yield}(tid \switch tid''))$
 and $tid'' \in T_c$.
This difference is quite small, but works as an another crucial part to build  multithreaded layer interface 
that uses $LAsm$ like machine model and the approach that we have discussed in the previous parts of this section. 

\subsection{Thread-Local Machine Model}\label{subsec:hasm}
By applying the whole processes,
we finally can define the machine model $LAsm_H$ 
which has lots of common aspects with the machine model of CPU-local layer interface ($LAsm_L$)
enough to use the existing compiler \compcertx~\cite{dscal15} as it is.
For example, the yield call will perform the transition as follows:
\begin{center}
\begin{tikzpicture}[->,>={stealth[black]}, auto,  node distance=3cm,draw]
\begin{scope}[every node/.style={font=\sffamily\small}]
    \node (A) at (0,0) {};
    \node (B) [node_w] at (0.13,0) {};
    \node (C) [node_db] at (6.13,0) {};
\end{scope}

\begin{scope}[every node/.style={font=\sffamily\footnotesize},
every edge/.style={draw, thick}]
    \path [->] (A) edge (B);
    \path [->] (B) edge node[above]{(exec$\_$yield$_{LAsm_H}$)} node[below]{$({\regs}, m, {a},l) \rightarrow ({\regs}, m, {a},l')$} (C);
\end{scope}
\end{tikzpicture}
\end{center}
, where  $l' = l \cdot l''$ when $l'' = (?(\oracle, \oracle^{T}), !tid.\commc{yield}(tid \switch tid'))$,
$tid$ is the current thread id, and $tid' \in T_c$.
Since this machine will always have the same current thread id as its player, the machine state 
does not need to keep the running thread id information any more. 
This implies that we are able to 
utilize the whole power of \compcertx\ and build thread-local layers both written in 
C and in Assembly with the guarantee that those layers preserve the same behavior in our CPU-local machine.
In addition to that, scheduling primitives ($\commc{yield}$ and $\commc{sleep}$) only updates its global log, so these primitives calls
follow C calling convention. 
In this sense we can use those primitives when we program and verify C functions in thread local layers ($\Lhthread[c]$), which was 
not possible in our CPU-local layers ($\Lbthread[c]$).
 
This machine model, however, has one big difference with $LAsm_L$ (that is why we have
distinguished $LAsm_L$ with $LAsm_H$ in their notations).
This machine need to allow a dynamic initial state for each thread, and this dynamically assigned 
initial state should be satisfied by our system invariant.
The reason of enabling dynamic initial states is that machine sates of threads on one CPU are 
closely related to  others' behaviors.
For example,
all threads have to wait until its parents spawn them at some point of the whole system's view.
While waiting the spawning, the state that corresponds to the thread in the system will be changed  
(before this arbitrary thread starts its evaluation), 
so the machine model ($LAsm_H$) of the thread needs to capture it properly while 
keeping the consistency between the state of the whole system and the state of the thread.
That is not only for the shared state (notated as our global log) 
but also for the thread private data structures, such as 
page allocation table and container information, due to the initialization of those data structures. 
In detail, the main thread of the system should initialize all threads' private datum (\textit{i.e.}
page allocation table, container information, etc)
and all other threads should use the properly initialized their private data structures 
even though they do not have a power to initialize them.
Also, the initial state of each thread should have the proper value to some of its private data before it starts its evaluation. 
For instance, the quota in the thread's container information should have the proper value that its 
parent have allowed while the parent spawn the thread.
To resolve the problem with keeping its generality,
our thread-local layer interface provides one abstract definition.

The abstract function to calculate the initial state (for each thread's private data) has a type of
\begin{small}
\[
fun_{st_{init}} : \mathbb{Z} \rightarrow l \rightarrow (\regs, a, l)
\]
\end{small}
which gets the current thread id and a initial log (mapped with the current thread id) as its arguments 
and returns a \textit{initial private register set}, \textit{initial private abstract data} and a initial log for the 
current thread id.
Using this function, 
this machine model can assign proper initial states for all threads even though 
the machine is fully isolated from other threads' machines.

This definition also gives consistency between our multithreaded concurrent machine model 
and this thread-local machine model.
When looking at the initial state definition for $EAsm$ in Sec.~\ref{subsec:fulleasm},
finding the similarity between both definitions is straightforward.
For the guarantee about preserving our system invariant in the initial state, we only need to prove that 
the calculated initial state satisfies our invariant.


\subsection{Thread-Local Layer Interface} \label{subsec:phthreadlayer}
If a multithreaded interface $\TLayer{L}{c}{t}$  focus only on
 a single thread $t\in T_c$,
 $\commc{yield}$ and $\commc{sleep}$ primitives  always
switch to an unfocused thread and then repeatedly query $\oracle$ and $\oracle^t$
until yielding back to $t$ as we have discussed in Sec.~\ref{subsec:hasm}.
\[
\includegraphics[width=.5\textwidth]{figs/ccal/thread4}
\]%
We can prove that this yielding back procedure in our system always terminates. This proof relies on the fact that the software scheduler is \emph{fair} and every running
thread gives up the CPU within a finite number of steps.
In this sense, and using the machine model that we explain in Sec.~\ref{subsec:hasm}, 
We call $\TLayer{L}{c}{t}$ a ``thread-local'' layer interface
because scheduling primitives
always  end up switching back to the same thread;
they do not modify the kernel context 
(i.e.\, $\comm{ra},
\comm{ebp}, \comm{ebx}, \comm{esi}, \comm{edi}, \comm{esp}$) and effectively act as a ``no-op'',
except that the shared log gets updated.
Thus, these scheduling primitives indeed satisfy C calling conventions  as you have seen in Sec.~\ref{subsec:hasm}.



\subsection{Queuing Lock}\label{subsec:qlockimplementation}

\begin{figure}[t]
\lstinputlisting [language = C, multicols=2] {source_code/ccal/queue_lock.c}
\caption{Pseudocode of queuing lock.}
\label{fig:exp:qlock}
\end{figure}
Based upon thread-local layer interfaces,
we build additional synchronization toolkits, such as
a queuing lock (c.f\ Fig.~\ref{fig:exp:qlock}).
With  queuing locks, waiting threads are put to sleep to avoid busy spinning.
Reasoning about this locking algorithm is particularly
challenging since its C implementation utilizes both
spinlocks and low-level scheduler primitives (i.e.\, $\commc{sleep}$ and $\commc{wakeup}$).
This verification task can be decomposed into a bunch of layers above $\TLayer{\Lhthread}{c}{t}$ using CCAL.

The correctness property of a queuing lock consists of
two parts: mutual exclusion and starvation freedom.
The lock implementation (Fig.~\ref{fig:exp:qlock}) is mutually exclusive
because the busy value of the lock ($\commc{ql\_busy}$)
is always equal to the lock holder's thread \allid{}.
This busy value is set either
by the lock requester when the lock is free (line 6 of 
Fig.~\ref{fig:exp:qlock})
or by the previous lock holder when releasing the lock
(line 12).
With the atomic interface of the spinlock, the starvation-freedom proof of queuing lock
is mainly about the termination of the sleep primitive call
(line 4). By showing that all the lock holders
will eventually release the lock,
we prove that all the sleeping threads will be 
added to the pending queue or ready queue within a finite number
of steps. Thus, $\commc{sleep}$ will terminate
thanks to the \emph{fair} software scheduler.
Note that all these properties proved at the C level can be propagated down to the assembly level using the thread-safe CompCertX.
   
%\begin{figure*}[th]
\begin{small}
\begin{mathpar}
\inferrule{
  \disjointunion{m_1}{m_2}{m}
}{
  \nextblock{m} = \max(\nextblock{m_1}, \nextblock{m_2})
}(\textsc{Nb})
\and
\inferrule{
  \disjointunion{m_1}{m_2}{m}
}{
  \disjointunion{m_2}{m_1}{m}
}(\textsc{Comm})
\and
 \inferrule{
  \disjointunion{m_1}{m_2}{m}  \\
  \load{m_2}{\ell} = \some{v}
}{
  \load{m}{\ell} = \some{v}
}(\textsc{Ld})
\and
\inferrule{
  \disjointunion{m_1}{m_2}{m}
}{
  \disjointunion{m_1}{\store{m_2}{\ell}{v}}{\store{m}{\ell}{v}}
}(\textsc{St})
\and
\inferrule{
  \disjointunion{m_1}{m_2}{m} \\
  \nextblock{m_1} \leq \nextblock{m_2} 
}{
  \disjointunion{m_1}{\alloc{m_2}{l}{h}}{\alloc{m}{l}{h}}
}(\textsc{Alloc})
\and
\inferrule{
  \disjointunion{m_1}{m_2}{m} \\
  \nextblock{m_1} \leq \nextblock{m_2}
}{
  \disjointunion{m_1}{\liftnextblock{m_2}{n}}{\liftnextblock{m}{n}}
}(\textsc{Lift-R})
\and
\inferrule{
  \disjointunion{m_1}{m_2}{m} \\
  \nextblock{m_1} \leq \nextblock{m_2}
}{
  \disjointunion{\liftnextblock{m_1}{n}}{m_2}{\liftnextblock{m}{n - (\nextblock{m} - \nextblock{m_1})}}
}(\textsc{Lift-L})
\end{mathpar}
\end{small}
\caption{Algebraic memory model} \label{fig:algmem}
\end{figure*}

\subsection{Thread-Safe Compilation and Linking}
\label{sec:comp}

In this section, we show how to adapt Gu et al.'s CompCertX verified
separate compiler \cite[\S 6]{dscal15} to handle programs that call scheduling primitives.
Section~\ref{subsec:phthreadlayer} shows how thread-local layer interfaces  allow us to
give \emph{C style} specifications to scheduling primitives
($\commc{yield}$ and $\commc{sleep}$) which are partly implemented in assembly.
Thus, code of each thread can be verified at the C level over $\Lhthread[c][t]$ and 
individual threads can then be composed
into programs on $\Lbthread[c]$ by Thm.~\ref{thread_composition}.
However, it is still challenging to show that the compiled programs at the assembly level
are also compatible with this parallel composition
because
of a small snag which we glossed over until now: \emph{stack frames}.
In the CompCert memory model~\cite{leroy08},
whenever
a function is called, a fresh memory block has to be allocated in the
 memory for its stack frame. This means that, on top of the thread-local layer $\Lhthread[c][t]$, a function called within a thread will allocate its stack frame
into the thread-private memory state, and conversely, a thread is
never aware of any newer memory blocks allocated by other
threads. In comparison, on top of the CPU-local layer $\Lbthread[c]$, all stack frames have to be allocated
in the CPU-local memory (i.e.\, \emph{thread-shared} memory) regardless of which thread they belong to;
thus, in the thread composition proof, we need to account
for all such stack frames.

Our solution is based on a special \emph{memory
extension} \cite[\S 5.2]{leroy08} that only removes the access permissions of some memory 
blocks(See Fig.~\ref{fig:mem}). 
To enable the  thread composition, we extended the semantics of $\commc{yield}$ and $\commc{sleep}$ on the thread-local layer $\Lhthread[c][t]$. Besides generating a $c.\yield/c.sleep$ event, such a scheduling primitive also allocates  \emph{empty} memory blocks 
as ``placeholders" for other threads' new stack frames during this $\commc{yield}/\commc{sleep}$.
These empty blocks are the ones without any access permissions.
We write ``$\nextblock{m}$'' to denote the total number
of blocks in $m$, and write ``$\liftnextblock{m}{n}$'' as the memory   extended
from $m$ by allocating $n$ new
empty blocks. 

With the extended semantics for scheduling primitives, we can prove
that a ternary relation ``$\disjointunion{m_1}{m_2}{m}$'' holds between the
private memory states $m_1, m_2$ of two disjoint thread sets and the
thread-shared memory state $m$ after the parallel composition. This relation among memory states is called the ``algebraic memory model'', which is defined by the axioms shown in Fig.~\ref{fig:algmem}.
 
Rule \textsc{Nb} states that the block number of the composed memory $m$  is equal to  ``$\max(\nextblock{m_1}, \nextblock{m_2})$.''
Rule \textsc{Comm} says that the parallel memory composition is
commutative. Rule \textsc{Ld} and \textsc{St} state that the behaviors of memory load and store (over $m_1$ or $m_2$) are preserved by the composed memory $m$. It is because that every non-shared memory block of $m_1$ either does not exist in $m_2$ or corresponds to an empty block in $m_2$, and vice versa.

%These properties guarantee that all the threads' private memory states are compatible.
%scheduling primitives like $\commc{yield}/\commc{sleep}$ are well behaved. 

All the remaining rules  in Fig.~\ref{fig:algmem} share the condition
``$\nextblock{m_1} \leq \nextblock{m_2}$.'' This condition indicates that thread 2 is ``more-recently scheduled/running'' because only \emph{running}
thread  can allocate memory blocks. 
Thus, memory allocations on $m_2$ can be preserved by the composed memory $m$ (see Rule \textsc{Alloc}).
In addition, if thread 2 is still the next scheduled thread and there are $n$ new stack frames allocated by threads other than $\set{1,2}$, we can then simply allocate $n$
 empty blocks in $m_2$, which will be preserved by $m$ (see Rule \textsc{Lift-R}). If thread 1 is the next thread to run, after allocating $n$ new empty blocks to $m_1$, the composed memory $m$ only need to allocate the blocks that have not been captured by $m_2$ (see Rule \textsc{Lift-L}).

Based on the parallel composition for two memory states, we can use Rule \textsc{Lift-R}  and
\textsc{Lift-L} to generalize to $N$ threads by saying that
$m$ is a composition of the private memory states ``$m_1, \dots, m_N$''
of $N$ threads (on a single processor) if, and only if, there exists a
memory state $m'$ such that $m'$ is a composition of ``$m_1, \dots,
m_{N-1}$'' and $\disjointunion{m_N}{m'}{m}$ holds.
For example, 
let $m_1, m_2$ two partial memory states containing the thread-private
memories of two disjoint thread sets. Then, we say that global memory
state $m$ is \emph{a disjoint union} of $m_1$ and $m_2$ (written $\disjointunion{m_1}{m_2}{m}$)
if, and only if, all the following conditions hold:
$\mathsf{nb}(m) = \max(\mathsf{nb}(m_1), \mathsf{nb}(m_2))$; $\forall i$,
memory extension from $\mathsf{liftnb}(m_i, \mathsf{nb}(m) - \mathsf{nb}(m_i))$ to $m$;
and no valid common memory locations in both $m_1$ and $m_2$;
then we satisfy the algebraic memory model axioms of Fig.~\ref{fig:algmem}.

   
%\subsection{Layer Refinement for Thread Linking}\label{subsec:concrete-impl}

In the previous parts of this section,
we show the whole process to 
build a thread-local layer interface
that tackles 
multiple challenges in building multithreaded layer interface.
However, introducing this interface does not mean that all about building thread-local layers and 
linking multiple thread-local layers are straightforward. 

The machines that we have introduced are rely on a global log (a list of events), an abstract data,
and the function that calculates the initial state of each thread in Sec.~\ref{subsec:hasm},  
which are abstract in the interface level. 
In this sense, those things should be concretely instantiated when building layers. 
Therefore, users of our thread local layer interface need a responsibility to 
define instances of those abstract definitions
which are valid with our interface. 
They also need to show that the validity by proving the related lemmas and theorems 
in each step of our processes to build the thread local layer interface.
Obviously, some data structures, such as a memory page permission table
or an IPC (inter process communication) channel, 
cannot be trivially divided into into multiple thread-wise data structures.
In this sense, dividing the whole data structure of concurrent system requires a deep knowledge about how the system works,
and requires an amount of effort. 
Our Coq implementation contains all those instances and proofs, and 
we will automate and optimize this phase in the near future.     
% END CONTENTS



\chapter{Case Study: Concurrent CertiKOS}
\label{chapter:concurrent-certikos}

%\section{Introduction}
\label{sec:intro}

%%% Outline
%% structure 
%% 1. concurrent verification is done in several works 
%% 2. how about showing the non-deterministic full machine model refines  ... 
%% 3. For example CCAL provide a useful tool for building concurrent abstraction layer 
%% 3-1. building layers is feasible 
%% 3-2. However proving the refinement between concurrent machine model and the per-instance machine model 
%%
%% 3-3. Based on the CCAL, we show how we build the linking for them 
%% 3-3-1. Multicore Linking 
%% 3-3-1-1. t provides the universal abstract semantics for multicore non-deterministic machine (with sequential consistency)
%% 3-3-1-2. it provides detailed refinement between those abstract functions 
%% 3-3-1-3. it provides the concrete instance of those proofs by connecting them with the lowest layer of CompCertX layer 
%% 3-3-2. Multithreaded Linking 
%% 3-3-3-1. It provides the CompCert Assembly machine models for CompCertX to build per-thread machine models 
%% 3-3-3-2. it provides the refinement between those machine models (parameterized by any kinds of Layers with the guarantee about the certain properties) 
%%                 - that allows us to allocate the proper dynamic initial state for each thread / invariant preserving in the initial state / using the same compiler with 
%%                    CompCertX                    
%% 3-3-3-3. it provides the actual proofs using the example in the certified layers (the language and the proofs are parameterized by the concrete layer definition)
%%                  - shows the identity of the private state change while  sleep and yield 
%%                  - mutual exclusion of user memory regions 
%%                  - mutual exclusion of other private states  






%
%Dependencies due to shared data
%•
%Subtle effects of synchronizations
%•
%Often manually parallelized
%–
%Difficult to debug
%•
%too many 
%interleavings
%of threads
%•
%hard to reproduce bugs
%
%
%
%

%%% concurrent program verification is necessary 
The prevalence of shared-memory multicore machine 
brings the eminent changes in the  software. 
With the machine, achieving higher performance on a single computer than before 
becomes possible, 
but it requires us to facilitate 
concurrency, running multiple threads on multiple cores.
Concurrency, however, 
brings the whole new challenges in terms of software correctness. 
They are well known 
to be difficult to get right and to debug because 
of their intrinsic characteristic, numerous number (usually unbounded) of interleavings among multiple components of the system. 
Testing is also not a promising way to provide the high-assurance of those programs. 
Due to a plethora of possible interleavings, 
reproducing a bug is unfeasible unless testers knows the 
precise interleaving order of them. 
In this sense, 
Building reliable concurrent programs 
needs verification of them, which formally shows that those programs correct reflects the 
desirable behavior (\textit{i.e.,} are stated in their specifications) 
without missing any single interleaving cases. 

%%% Composition is required
The concurrent program verification requires compositional reasoning in its essence,
since it provides an isolation of each instance of concurrent program
(on a single core or a single thread) separately  
 in its verification
without directly considering complex interleaving 
with other components in the system. 
This feature is crucial in some sorts 
of concurrent programs such as 
operating systems, libraries, or application interfaces
because the
proof of them 
are usually need to be parameterized by 
other programs running on them. 
In those cases, composition and proof isolation 
give  an enough power 
to state and prove the correctness property 
of those programs upon any arbitrary context programs run with the targeted programs. 

%%%% several previous works and machine checkable proof  

In this sense, 
multiple previous works handle compositional reasoning about concurrent programs.
There are two traditional different approaches,
rely-guarantee~\jieung{cite rely guarantee} and separation logic~\jieung{CSL cite separation logic  - need to refer View for citation},
and many other approaches that stem from either or both of them
\jieung{SAGL (2007) / Bornat-at (2005) RGSep (2007) Gotsman-al (2007) RSL (2013) Deny Guarantee (2009) LRG (2009) RGSim (2012) Liang-Feng (2013) 
Lili (2016) / Iris (2015) Iris 2.0 (2016) FCSL (2014) (SCSL (2013) FTCSL (2015) CoLoSL (2015) CAP (2010)   View paper / CCAL paper / CSpec (MIT)
- Please refer the specification of POSIX File Systems slide}.
In addition, some of them are not only focusing on the functional correctness but also 
shows liveness~\jieung{LiLi}. 
Some, CSpec and CCAL, also provides a verified layered structure to build modular verification, an another important 
feature to build a large scaled program verification in a modular ways.


%%%% several previous works and machine checkable proof  
Bsed on them, few works \jieung{verifying concurrent software using movrs in CSPEC / preemtive kernel verification (Xinyu Feng - CAV), CertiKOS, MCSLock CCAL} 
organizes machine checked proofs 
about concurrent execution. 
Among them, both CSPEC and CertiKOS facilitates layered structures 
for scalable and modular verification and formally connect top level operations into bottom-layer operations.

%%%% CCAL - what is missing 
They, however, overlook the difficulty in one another piece of machine checked concurrent program verification, 
provide the evidence of concurrent linking.
The concurrent linking shows 
the precise evidence of the composition that the underlying logics provide. 
In this sense, 
it requires the definition of 
concurrent machine model that can run multiple instances of concurrent program together (\textit{e.g.,} multicore and multithreaded machine) 
as well as 
the linking proofs between the program runs on top of concurrent machine and the composition of multiple single instances together. 
It also requires the proof that 
shows the single instance of the concurrent program correctly reflects
the program run on the multicore machine model. 

They are necessary to show the full correctness of the program, 
but providing concurrent machine model is bothersome, especially when the model is close to that of bare machines, 
and the proof between it wiith the machine that runs the single instance is also a subtle work.
To handle those challenges,
CCAL slightly mentioned these issues,
but it only carries out
a key idea of
linking without exposing underlying multiple obstacles.  
In this sense, 
providing the information about which steps are necessary for concurrent linking and what kind of things that 
the users have to fill out is desired.
In this sense, the idea in the paper is far from 
the enough idea to achieve how 
concurrent linking can be worked in such 
a large scaled concurrent program. 

\jieung{need to add sentence about CompCertX}


%%%% The contribution of this paper

Therefore, our paper aim to deliver all necessary 
and important ides for concurrent linking,
which includes modeling the generic concurrent machine model, 
necessary information to prove refinements between them, 
and how to connect those concurrent linking with the 
proof layers of concurrent programs in a generic way. 
It is definitely not able to be achieved in a single shot.
We introduce multiple intermediate languages and 
context that users has a responsibility to 
connect the generic concurrent linking proof with 
their one verified programs.
We, in this paper, handle all of them in detail. 
In short, he key contribution of this paper is as follows: 

\begin{itemize}
\item We formally define non-deterministic multicore semantics and multiple intermediate languages that are independent from specific machines (such as x86 or ARM). 
\item We provide the refinement proofs between them that can be used for \compcertkwd-style backward simulation. 
\item We connect those intermediate languages and proofs with the CPU local CCAL layer, that uses \compcertkwd-like sequential x86 assembly model with 
environment context.
\item We provide multithreaded machine model with minimal assumptions about a certain CPU local CCA layer, which implies that the machine model does not stick to the specific layer definition.
\item We provide intermediate languages to introduce per thread machines and refinement proofs among them. 
\item We connect those intermediate languages and refinement proofs with the specific layer definition in CertiKOS, which fully link the layer on per-thread machine with the layer on per-CPU machine.
\end{itemize}

The structure of remaining paper is as follows:
Section~\ref{sec:overview} shows a brief high level idea of CCAL as well as how our linking works. Section~\ref{sec:multicore} shows the details of multicore linking,
and Sect.~\ref{sec:multithreaded} shows the implementation of our intermediate machine models for our multithreaded environment.
Section~\ref{sec:multithreaded-linking-impl} shows how are framework 
can be fitted into the actual concurrent kernel implementations.
Evaluations about our implementation can be found in Sect.~\ref{sec:evaluation} 
and the related work and conclusion is in Sect.~\ref{sec:related}.


%
%
%\begin{figure}
%\caption{Requirements in Concurrent Program Verification}
%\label{fig:concurrent-verification-challenge}
%\end{figure}
%
%However, even with the importance of concurrent program verification and 
%a large body of recent work on shared-memory concurrency verification ~\jieung{cite},
%there are few certified programming tools for a large scale software due to the requirement of multiple challenges described in Fig.~\ref{fig:concurrent-verification-challenge}.
%
%\jieung{ need to site ESOP papers too}
%
%They first have to 
%provide a way to build the software in multiple layers
%that enable us to build a large scale program as a modular way. 
%For example, 
%operating systems can be divided into multiple parts, 
%memory management, process management, and so on.
%
%They also have to provide \jieung{need different word} a methodology to 
%represent the behavior of other components in the concurrent environment. 
%For the program running on multicore environment, 
%the single instance of the program, which is a program runs on top of 
%a single CPU, has to correctly capture the 
%environmental behavior (the behavior of programs on other CPUs). 
%
%In addition to that, 
%providing the end-to-end theorem also requires us 
%to link the multiple proof instances to 
%form a single proof that is based on
%the concurrent environment itself which does not have 
%any environmental contexts at all. 
%In the example of the operating system on multicore environment,
%the end-to-end theorem 
%has to prove that 
%the program running on the single CPU is correctly refined by 
%the whole thread programs running on the multicore machine. 
%
%Previous works, CertiKOS~\jieung{need cite} and Certified Concurrent Abstraction Layer~\jieung{need cite}, 
%tackles all the above examples.  
%CCAL is a tool to build a certified concurrent layers, which provides 
%a way to build concurrent abstraction layers, 
%
%
%
%However, the paper does not handle how the linking process works with the concrete machine models. 
%It briefly mentions the high level idea of linking and the memory extension for linking framework. 
%
%Therefore, this paper aims the gap between the high level perspective of CCAL and the 
%low level details of concurrent proof linking. 
%This low level details contains two parts. 
%First, it requires us to define and build multiple intermediate languages to connect
%the x86 multicoro machine model with the LAsm, which is the machine model for one single CPU. 
%In addition to that, 
%the framework also needs to show the refinement 
%between layers on those intermediate machine models to formally link
%all those proofs together. 
%CCAL also briefly provide the idea of how they implement the practical machine models that can be used with CompCertX.
%However, only providing few details does not provide 
%the  useful information to show how it works with the actual running large scale software.
%Thus, our paper tackles the issues that CCAL overlooked in the paper 
%by providing the formal rules and proofs.
%The key contribution of this paper is as follows: 
%
%\begin{itemize}
%\item We provide the detailed intermediate language semantics for multicore machine model based on CCAL, 
%and instantiate all those intermediate language semantics and refinement proofs 
%to link them with CompCertX with environmental context 
%\item We provide the intermediate machine models to build single threaded machine model from a single CPU machine model. 
%Based on the machine models, we provide the linking theorem in between 
%two abstraction layers, which contains different semantics for software schedulers. 
%\end{itemize}
%
%The structure of remaining paper is as follows:
%Section~\ref{sec:overview} shows a brief high level idea of CCAL as well as how our linking works. Section~\ref{sec:multicore} shows the details of multicore linking,
%and Sect.~\ref{sec:multithreaded} shows the implementation of our intermediate machine models for our multithreaded environment.
%Section~\ref{sec:multithreaded-linking-impl} shows how are framework 
%can be fitted into the actual concurrent kernel implementations.
%Evaluations about our implementation can be found in Sect.~\ref{sec:evaluation} 
%and the related work and conclusion is in Sect.~\ref{sec:related}.
%
%
%
%\ignore{
%Despite the importance of concurrent layers and a large body of recent work on 
%shared-memory concurrency verification, 
%
%
%there are no certified programming tools that can specify, compose, and compile concurrent layers to form a whole system [6]. Formal reasoning across multiple concurrent layers is challenging because different layers often exhibit different interleaving semantics and have a different set of observable events. For example, the spinlock module in Fig. 1 assumes a multicore model with an overlapped execution of instruction streams from different CPUs. This model differs significantly from the multithreading model for building high-level synchro- nization libraries: each thread will block instead of spinning if a queuing lock or a CV event is not available; and it must count on other threads to wake it up to ensure liveness.
%
%
%
%
%many of these abstraction layers also become concurrent in nature. Their interfaces not only hide the concrete data representations and algorithmic de- tails, but also create an illusion of atomicity for all of their methods: each method call is viewed as if it completes in a single step, even though its implementation contains com- plex interleavings with operations done by other threads. Herlihy et al. [19, 20] advocated using layers of these atomic objects to construct large-scale concurrent software systems.
%
%
%The importance of software systems' accuracy is growing rapidly these days. 
%In addition to that, 
%the concurrent environment, including multicore and device drivers, are ubiquitous in modern periods. 
%Therefore, 
%the verification methodology for concurrent programs is critical now. 
%
%In this sense, several previous works propose
%proof logics and tools for that purpose \jieung{need cite}.
%
%However, few of them are working on the linking multiple instances of 
%verified concurrent programs with concrete machine models that can be run 
%on the bare machines. 
%
%One tool, Certified Concurrent Abstraction Layers, 
%provides the tool that can be used for building a practical concurrent programs 
%such as a small operating system or distributed system. 
%It also provides the tool to link the 
%}

% ADD CONTENTS
%\section{Building Certified Multicore Layers}
\label{sec:prog}

In this section, we start to show how to apply our techniques to verify shared objects in the CCAL toolkit.
We begin by considering the ones
shared among CPUs: spinlocks and shared queue objects protected by spinlocks.
All layers are built upon the CPU-local layer interface
$\PLayer{\PBoot}{c}{\oracle}$.

\subsection{Spinlocks}
\begin{figure}[t]
\lstinputlisting [language = C, multicols=2] {source_code/ccal/ticket_lock.c}
\caption{Pseudocode of ticket lock using $\push/\pull$.}
\label{fig:exp:real_ticket_lock}
\end{figure}

Spinlocks (e.g., the ticket lock algorithm described in Sec.~\ref{sec:informal}) 
are one of the most basic synchronization
methods for multicore machines; they are used as building
blocks for shared objects and more sophisticated synchronizations.

A spinlock enforces mutual exclusion by restricting CPU access to
a memory location $b$. Therefore, lock operations can be viewed
as ``safe'' versions of $\cpush/\cpull$ primitives.
For example, when the lock acquire  for $b$ succeeds,
the corresponding shared memory is guaranteed
to be ``free'', meaning that it is safe to 
pull the contents to the local copy at this point (line 4 in Fig.~\ref{fig:exp:real_ticket_lock}).
Therefore, as can be seen in Fig.~\ref{fig:exp:real_ticket_lock},
the $\acq/\rel$ functions invoke the $\push/\pull$ primitives.
We now show how to build layers for the spinlock
in Fig.~\ref{fig:exp:real_ticket_lock}, which uses a ticket lock algorithm. Note that query points are denoted as ``$\intp$'' in pseudocode.

\para{Bottom Interface $\PBoot[c]$.}
We begin with the CPU-local interface $\PBoot[c]$ extended with shared primitives
$\commc{FAI\_t}$, $\commc{get\_n}$, and $\commc{inc\_n}$.
These primitives directly manipulate the lock state $\commc{t}$
(next ticket) and $\commc{n}$ (now serving ticket)
via x86 atomic instructions. 

\begin{figure}[t]
\lstinputlisting [language = Caml] {source_code/ccal/lock.v}
\caption{Pseudocode of ticekt lock specifications in Coq.}
\label{fig:exp:tlock}
%\end{wrapfigure}%
\end{figure}
Each of them generates a corresponding event in the 
log. As an example, the specification of $\commc{FAI\_t}$ is shown in Fig.~\ref{fig:exp:tlock},
where the replay function $\replay_{\comm{ticket}}$ calculates  the lock state.
 
\para{Fun-Lift to $L_\comm{lock\_low}[c]$.}
We have shown how to establish the strategy simulation
for this low-level interface $L_\comm{lock\_low}[c]$ (i.e.\, $L_1'[c]$, see Sec.~\ref{sec:informal}).
Note that $\ssem{\commc{acq}}{L_\comm{lock\_low}[c]}$ contains extra silent moves (e.g.\, assigning $\comm{myt}$, line 2 in Fig.~\ref{fig:exp:real_ticket_lock}) compared with $\strat{\comm{acq}}'[c]$.
The simulation relation 
$R_\comm{lock}$ not only states the equality between logs but also maps the lock
state in the memory to the ones calculated by $\replay_{\comm{ticket}}$.
Here we must also handle potential \emph{integer overflows} for $\commc{t}$ and $\commc{n}$.
We can prove that, as long as the total number of CPUs (i.e.\, $\#CPU$) in the machine is less than $2^{32}$ (determined by $\comm{uint}$), the mutual exclusion property will not be violated  even with overflows. 
Based on the CPU-local layer, we first verify that the C implementations of 
the ticket lock satisfy the $\acq$ and $\rel$ specifications,
which are defined in terms of logs with low-level events
related to $\comm{t}$ and $\comm{n}$.
Thus, we can build $\ltyp{\PLayer{\PBoot}{c}{}}{R_\comm{lock}}{(\modulef{\acq}
\oplus \modulef{\rel})}{\PLayer{L_\comm{lock\_low}}{c}{}}$
using the \textsc{Fun} rule,
where the simulation relation $R_\comm{lock}$ maps the lock
fields in the memory to the ones calculated by $\replay_{\comm{ticket}}$.

\para{Log-Lift to $L_\comm{lock}[c]$.}
We then lift the $\acq$ and $\rel$ primitives to an atomic interface, meaning that each
invocation produces exactly one event in the log (see Sec.~\ref{sec:informal}).
These atomic lock interfaces (or strategies) are similar to $\cpull/\cpush$ specifications,
except that the former ones are \emph{safe} (i.e.\, will not get stuck).
This safety property can be proved using rely conditions $L_\comm{lock}[c].\Rely$ saying that,
for any CPU $c'\neq c$, its $c'.\acq$ event must be followed by a sequence of its own events (generated in the critical state)
ending with $c'.\rel$. The distance between $c'.\acq$ and $c'.\rel$ in the log is less than some number $n$.

By enforcing the fairness of the scheduler in rely conditions, saying that any CPU can be scheduled
within $m$ steps, we can show the liveness property (i.e., starvation-freedom): the while-loop in $\commc{acq}$ terminates
in ``$n \times m \times \#CPU$'' steps.


\begin{figure}
\lstinputlisting [language = C, multicols=2] {source_code/ccal/dequeue.c}
\caption{Pseudocode of dequeue implementation.}
\label{fig:exp:dequeue}
\end{figure}


\subsection{Shared Queue Object}
\label{sec:shared-queue}
Shared queues are widely used in concurrent programs, e.g.\, 
as buffers for producers/consumers,
as the list of threads in a scheduler, etc.
The specification of a shared queue object should provide a high level abstract
interface by hiding all the low level details,
to ease the verification of programs using the object.
However, the large gap between this high level abstract specification
and the low level efficient implementation makes the verification of the shared queue object
itself extremely challenging.
Furthermore,
In previous work~\cite{lili16},
due to the lack of layering support,
the verification of any shared object
required inlining the lock implementation
and duplicating the lock-related proofs.
In the following, we illustrate how to utilize concurrent
abstraction layers to verify a shared queue module
(c.f.\ Fig.~\ref{fig:exp:dequeue}
using fine-grained locks  introduced by $L_\comm{lock}[c]$.

\para{Fun-Lift to $L_\comm{q}[c]$.}
The shared queues are implemented as doubly linked lists, and are protected
by spinlocks. For example, the dequeue ($\commc{deQ}$) operation
first acquires the spinlock associated with queue $i$(via the auxiliary function $\qloc$),
then performs the actual dequeue operation in the critical state,
and finally releases the lock.
Instead of directly verifying $\commc{deQ}$ in one shot, 
we first introduce an intermediate function
$\commc{deQ\_t}$, which contains
code that performs the dequeue operation
over a local copy, under the assumption that
the corresponding lock is held.
Since no environmental queries are needed in the critical state, 
building concurrent layers for $\commc{deQ\_t}$ is similar to building a sequential layer~\cite{dscal15}:
we first introduce the abstract states $a.\comm{tcbp}$ and $a.\comm{tdqp}$, which stand for the thread control block (i.e.\, $\commc{tcb}$) array
and the thread queue array.
The  abstract $\comm{tdqp}$ is a partial map from the \emph{queue index}
to an \emph{abstract queue}, which is represented as
a list of $\comm{tcb}$ indices.
Then we can show that $\commc{deQ\_t}$ meets its specification
$\spec_{\deq\comm{\_t}}$:%
\lstinputlisting [language = Caml] {source_code/ccal/deq_t.v}%

\para{Fun- and Log-Lift to $L_\comm{q\_high}[c]$.}
Finally, we have to show that the $\commc{deQ}$ function that wraps $\commc{deQ\_t}$ with lock primitives indeed meets an atomic interface.
With a simulation relation $R_\comm{lock}$ that merges two queue-related lock events
(i.e.\, $c.\acq$ and $c.\rel$) into a single event $c.\deq$
at the higher layer, we can prove the following strategy simulation:%
\begin{center}
\begin{tikzpicture}[->,>={stealth[black]}, auto,  node distance=3cm,draw]
\begin{scope}[every node/.style={font=\sffamily\small}]
    \node (F1) at (-0.8,0) {$\ssem{\commc{deQ}}{L_{\comm{q}}[c]}:$};
    \node (F2) at (-0.8,-1) {$\strat{\comm{deQ}}[c]:$};
    \node (G) at (-0.8,-0.45) {$\le_{R_{\comm{lock}}}$};
    \node (A) at (-0.1,0) {};
    \node (B) [node_w] at (0.4,0) {};
    \node (C) [node_b] at (2.6,0) {};
    \node (D) [node_b] at (5.3,0) {};
    \node (E) [node_d] at (6.7,0) {};
    \node (A2) at (-0.1,-1) {};
    \node (B2) [node_w] at (0.4,-1) {};
    \node (E2) [node_d] at (3,-1) {};
\end{scope}
\begin{scope}[every node/.style={font=\sffamily\small},
every edge/.style={draw, thick}]
    \path [->] (A) edge (B);
    \path [->] (B) edge node[above] {$?\oracle, !c.\comm{acq}(i), \return q$} (C);
       \path [->] (C) edge node[above] {$\spec_{\comm{deQ\_t}}(q) = (q', r)$} (D);
       \path [->] (D) edge node[above] {$!c.\comm{rel}(i,q')$} (E);
        \path [->] (A2) edge (B2);
    \path [->] (B2) edge node[above] {$?\oracle, !c.\comm{deQ}(i), \return r$} (E2);
\end{scope}
\end{tikzpicture}
\end{center}

    
%\section{Evaluation and Experience}
\label{sec:kernel}

\begin{table}[t]
\caption{Lines of proofs in Coq for the toolkit.}
\vspace{-5pt}
\begin{center}
\begin{small}
\renewcommand{\arraystretch}{1} 
\begin{tabular}{|c|c||c|c|}
\hline
Component & LOC & Component & LOC \\
\hline
\hline
Auxiliary library & 6,200 & Multilayer linking & 17,000 \\
\hline
C verifier & 2,200 & Multithread linking & 10,000 \\
\hline
Asm verifier & 800 & Multicore linking & 7,000 \\
\hline
Simulation library & 1,800 & Thread-safe CompCertX & 7,500\\
\hline
\end{tabular}
\end{small}
\end{center}
%\vspace{-9pt}
%\hrulefill
\label{table:toolkit}
\vspace{-5pt}
\end{table}

%
\begin{table}[t]
\caption{Statistics for implemented components.}
\vspace{-5pt}
\begin{center}
\begin{small}
\renewcommand{\arraystretch}{1.1}
\setlength{\tabcolsep}{0.3em}
\begin{tabular}{|c|c|c|c|c|c|c|}
\hline
 & \makecell{ C\&Asm \\Source} & Spec. & \makecell{Invariant \\ Proof} & \makecell{C \& Asm \\Proof} & \makecell{Simulation \\ Proof} \\
%\cline{3-4}
% & Source & Specification & Invariant Proof & Proof & Proof  \\
\hline
Ticket lock & 74 & 615 & 1,080 & 1,173 & 2,296* \\
\hline
MCS lock & 287 & 1,569 & 2,299  &  1,899 & 3,049** \\
\hline
Local queue & 377 & 554 & 748 & 2,821& 3,647 \\
\hline
Shared queue &  20 & 107 & 190 & 171& 419\\
\hline
Scheduler & 62 & 153 & 166 & 1,724 & 2,042 \\
%\hline
%Multithreaded linking & N/A & 2350 & N/A & N/A & 31,500 \\ 
\hline
Queuing lock & 112 & 255 & 992 & 328 & 464 \\
\hline
\end{tabular}
\end{small}
%\newline
\end{center}

\begin{flushright}\begin{scriptsize}
* Starvation proof for Ticket lock included: 980\\
** Starvation proof for MCS lock included: 2,455
\end{scriptsize}
\end{flushright}
\label{table:evaluation}

\end{table}


We have implemented the CCAL toolkit (see
Fig. \ref{fig:toolchain}) in the Coq proof assistant. Table
\ref{table:toolkit} presents the number of lines (in Coq) for each component in Fig.~\ref{fig:toolchain}. The auxiliary library contains the common
tactics and lemmas for 64 bit integers, lists, maps, integer
arithmetic, {etc}.



\para{Case Studies.}

To evaluate the framework itself, we have implemented, specified, and verified various
concurrent programs in the framework. Table \ref{table:evaluation} presents some of the
statistics with respect to the implemented components in terms of the number of lines of C \& assembly
source code, the number of layers used to specify and verify the module, the size of the
specification, and the number of lines
(in Coq) used to perform the invariant proof, code proof, and refinement proof, respectively.

Lock implementations are  big parts in our examples. 
The first reason or this large LoC is the subtlety of busy waiting in the lock implementation 
(line 14 in Fig.~\ref{fig:exp:ticket_lock_example}). This requires us a large number of proofs to show 
 this loop
terminates within a bound (i.e.\, starvation-freedom). 
The other reason is mainly related to the strategy simulation 
that we have explained in Sec.~\ref{sec:informal}. 
Since our main goal is facilitating our toolkit for a large concurrent software system 
with scalability,  we have to make sure the the spinlock module provides
a general and simple interface.
This requires us to lift the low-level strategies (or specifications)
of two lock implementation to the same atomic interface (see Sec.~\ref{sec:informal}), which make proofs big.
Despite of this large LoC, however, the client of spinlocks can be verified over a simple interface
thanks to our toolkit.

Their source code contains not only
the code of the associated functions,
but also the data structures and their initialization.
In addition to the top level interface, the specification contains all the 
specifications used in the intermediate layers.
For both the ticket and MCS locks,
the refinement proof column
includes the proof of starvation freedom (about 3,500 lines) in addition to the correctness proof.
The gap between the underlying C implementation and the high level specification of the locks
also contributes to the large proof size for these components.
For example, the intermediate specification of the ticket lock uses an unbounded
integer for the ticket field,
while the implementation uses a binary integer
which wraps back to zero.
Similarly, the queue is represented as a logical list in the specification,
while it is implemented as a doubly linked list.


Our development is compositional. Both ticket  and MCS locks share the same
high-level atomic specifications (or strategies) shown in Sec.~\ref{sec:informal}.
Thus the lock implementations can be freely interchanged without affecting any proof
in the higher-level modules using locks. When implementing the shared queue library, we
also reuse the implementation and proof of the local (or sequential) queue library:
to implement the atomic queue object, we simply wrap the
local queue operations with lock acquire and release statements.
As shown in Table \ref{table:evaluation}, using verified lock modules to build
atomic objects such as shared queues is relatively simple and does not require
many lines of code.

Following the same philosophy, 
Gu et al. \cite{certikos-osdi16} has further extended our work with paging-based
dynamically allocated virtual
memory, device drivers with in-kernel interrupts, a synchronous inter-process
communication (IPC) protocol using the queuing lock, a shared-memory IPC protocol with
a shared page, and Intel hardware virtualization support;
our CCAL toolkit was used to produce the world's
first fully certified concurrent OS kernel
with fine-grained locking. 


\para{Performance Evaluation.} 
The verification should not by any means hinder the performance.
One benefit of the layered approach is that
concrete and highly optimized (and thus complex) implementations can be abstracted
into much simpler logical specifications that are easier to reason about.
In our toolkit, the verified source code (both C and assembly) are encoded as
abstract syntax trees in Coq, while CompCertX is written directly in
Coq. We then use Coq's extraction mechanism to produce a program which
compiles the C source code and outputs a single piece of assembly code
for the system. The resulting code is efficient.
We have measured the performance
of the ticket lock on an Intel 4-Core i7-2600S (2.8GHz) processor with 16GB memory.
Initially, the ticket lock implementation incurred a latency of 87 CPU cycles in the
single core case.
After a short investigation, we found that we forgot to remove some function calls
to ``logical primitives'' used for manipulating ghost abstract states. After we removed
these extra null calls, the latency dropped down to only 35 CPU cycles.
Gu et al. \cite{certikos-osdi16} also presented performance
evaluations of their OS kernel built using CCAL.

\para{Limitations.}

Our concurrent  machine models assume strong sequential consistency (SC)
for atomic primitives. 
Previous work~\cite{SewellSONM10} demonstrated that race-free programs on a
TSO model do indeed behave as if executing on a
sequentially consistent machine. Since safe programs on our push/pull model are race-free, 
we believe extending our work from SC to TSO is promising. In our future work,
we will formalize and integrate this proof in Coq.
%We leave the support for relaxed memory models as future work.
Since our model requires all non-atomic memory
operations to be synchronized through locking, and the x86 TSO model
guarantees proper order for the atomic operations used to implement
locks, we are confident that our proof is valid for that model as
well.  However, a formal proof of this is left as future work.
Furthermore, the current event-based contextual refinement proofs still require
quite a bit of manual proof. We are working on developing more
automation tactics to further cut down the proof effort. In addition to this
general toolkit that can support a broad range of concurrent programs,
we also plan to provide more aggressive automation for commonly-used
concurrent programming patterns, either through additional tactic
libraries or using specific program logics targeting such patterns.

Our layered methodology enables us the modular development of the proofs, by proving
each module at the right abstraction level with minimum dependencies, avoiding unnecessary
tedious dependency proofs. But still, as the code base gets larger, the slowdown
in the compilation time starts to affect efficiency of the development more and more.
For example, in rare cases, if we change the file that contains global definitions used by
most of other files, the compilation take very long to get to the file we are working on.
We are actively considering porting our framework to the newest version of Coq which support
asynchronous edition and compilation.
    
% END CONTENTS



%\iflongabs
Operating System (OS) kernels form the backbone of all system
software. They can have a significant impact on the resilience
and security of today's computers. Recent efforts
have demonstrated the feasibility of building formal proofs of
functional correctness for simple general-purpose kernels, but they
have ignored the important issues of concurrency, which include not
just user- and I/O concurrency on a single core, but also
multicore parallelism with fine-grained locking.  Many 
researchers believe that building a certified concurrent kernel is
very challenging, and even if it is possible, its cost would far
exceed that of verifying a single-core sequential kernel.

In this paper, we 
\else
Complete formal verification of a non-trivial concurrent OS kernel is
widely considered a grand challenge.  We 
\fi
present a novel compositional approach for building certified
concurrent OS kernels. Concurrency allows interleaved execution of
kernel/user modules across different layers of abstraction. Each such
layer can have a different set of observable events. We insist on
formally specifying these layers and their observable events, and then
verifying each kernel module at its proper abstraction level. To
support certified linking with other CPUs or threads, we prove a
strong contextual refinement property for every kernel function, which
states that the implementation of each such function will behave like
its specification under any kernel/user context with any valid
interleaving. We have successfully developed a practical concurrent OS
kernel and verified its (contextual) functional correctness in Coq.
Our certified kernel is written in 6500 lines of C and x86 assembly
and runs on stock x86 multicore machines. To our knowledge, this is
the first proof of functional correctness of a complete,
general-purpose concurrent OS kernel with fine-grained locking.



%%\sectskip
\section{Introduction}
\label{sec:intro}
%\asectskip

Operating System (OS) kernels and hypervisors form the backbone of
\ignore{every}safety-critical software systems in the world.  Hence it is
highly desirable to formally verify the correctness of these
\ifanonymized programs.  \else programs~\cite{shao10}.  \fi Recent
efforts~\cite{klein2009sel4,hawblitzel10,klein14,ironclad14,dscal15,fscq15,cogent16,chen16}
have shown that it is feasible to formally prove the functional
correctness\ignore{property} of simple general-purpose kernels, file systems,
and device drivers. However, none of these systems have addressed the
important issues of concurrency~\cite{kaashoek15,ospp11}, 
including not only user and I/O concurrency on a single CPU, but also
multicore parallelism with fine-grained locking. This severely limits
the applicability and power of today's formally verified system
software.

What makes the verification of concurrent OS kernels so challenging?
First, concurrent kernels allow interleaved execution of kernel/user
modules across different abstraction layers; they contain many
interdependent components that are difficult to untangle.  Several
researchers~\cite{vontessin13,peters15} believe that the combination
of concurrency and the kernels' functional complexity makes formal
verification of functional correctness intractable, and even if it is
possible, its cost would far exceed that of verifying a single-core
sequential kernel.

Second, concurrent kernels need to support all three types of
concurrency (user, I/O, or multicore) and make them work coherently
with each other. User and I/O concurrency rely on thread
yield/sleep/wakeup primitives or interrupts to switch control and
support synchronization; these constructs are difficult to reason
about since they transfer control from one thread to another.
Multicore concurrency with fine-grained locking requires sophisticated
spinlock implementations such as MCS locks~\cite{mcs91}, which are also
hard to verify.

Third, concurrent kernels should also guarantee that each of their
system calls eventually returns, but this depends on the progress of
the concurrent primitives used in the kernels. Proving
starvation-freedom~\cite{Herlihy08book} for concurrent objects only
became possible recently~\cite{lili16}.  Standard Mesa-style condition
variables~\cite{lampson80} do not guarantee starvation-freedom; this
can be fixed by using a FIFO queue of condition variables, but the
solution is not trivial and even the popular, most up-to-date OS
textbook~\cite[Fig.~5.14]{ospp11} has gotten it
wrong~\cite{anderson16}.

Fourth, given the high cost of building concurrent kernels, it is
important that they can be quickly adapted to support new hardware
platforms and applications~\cite{dune12,unikernel13,engler95}.  One
advantage of a certified kernel is the formal specification for all of
its components. In theory, this allows us to add certified kernel
plug-ins as long as they do not violate any existing invariants.  In
practice, however, if we are unable to encapsulate interference, even
a small edit could incur huge verification overhead.

In this paper, we present a novel compositional approach that tackles
all these challenges. We believe that, to control the complexity of
concurrent kernels and to provide strong support for extensibility, we
must first have a {\em compositional} specification that can untangle
{\em all} the kernel interdependencies and encapsulate interference
among different kernel objects. Because the very purpose of an OS
kernel is to build layers of abstraction over bare machines, we insist
on meticulously uncovering and specifying these layers, and then
verifying each kernel module at its {\em proper} abstraction level.

The functional correctness of an OS kernel is often stated as a {\em
  refinement}.  This is shown by building {\em forward
  simulation}~\cite{Lynch95} from the C/assembly implementation of a
kernel ($K$) to its abstract functional specification ($S$). Of
course, the ultimate goal of having a certified kernel is to reason
about programs running on top of (or along with) the kernel. It is
thus important to ensure that given any kernel extension or user
program $P$, the combined code $K\join{}P$ also refines
$S\join{}P$. If this fails to hold, the kernel is simply still
incorrect since $P$ can observe some difference between $K$ and $S$.
\citet{dscal15} advocated proving such a {\em contextual refinement}
property, but they only considered the {\em sequential} contexts (\ie,
$P$ is sequential).

For concurrent kernels, proving the {\em contextual refinement}
property becomes essential. In the sequential setting, the only way
that the context code $P$ can interfere with the kernel $K$ is when
$K$ fails to encapsulate its private state; that is, $P$ can
modify some internal state of $K$ without $K$'s permission.
In the concurrent setting, the {\em environment} context ($\oracle$)
of a running kernel $K$ could be other kernel threads or a copy of $K$
running on another CPU. With shared-memory concurrency, 
interference between $\oracle$ and $K$ is both necessary and 
common; the sequential {\em atomic} specification $S$ is now replaced
by the notion of linearizability~\cite{herlihy90} plus a progress
property such as starvation-freedom~\cite{Herlihy08book}.

In fact, linearizability proofs often require event reordering that
preserves the happens-before relation, so $K\,\join{}\,\oracle$ may not
even refine $S\,\join{}\,\oracle$.  Contextual refinement in the
concurrent setting requires that for any $\oracle$, we can find a {\em
  semantically related}\ $\oracle'$ such that $K\,\join{}\,\oracle$ refines
$S\,\join{}\,\oracle'$.  Several
researchers~\cite{filipovic10,liang13,lili16} have shown that
contextual refinement is precisely equivalent to the linearizability
and progress requirements for implementing compositional concurrent
objects~\cite{Herlihy08book,herlihy90}.

Our paper makes the following contributions:
\begin{itemize}[leftmargin=*] %,itemsep=0pt] 
\item We present {\bf{}\CTOS}---a new extensible architecture for
  building certified concurrent OS kernels.  \CTOS\ uses 
  contextual refinement over the ``concurrent'' {\em environment contexts}
  ($\oracle$) as the {\em unifying} formalism for composing 
  different concurrent kernel/user objects at different
  abstraction levels.  Each $\oracle$ defines a specific instance on how
  other threads/CPUs/devices respond toward the events generated by
  the current running threads.  Each abstraction layer, 
  parameterized over a specific $\oracle$, is an
  assembly-level machine extended with a particular set of
  abstract objects (\ie, abstract states plus atomic primitives).
  \CTOS\ successfully decomposes an
  otherwise prohibitive verification task into many simple and easily
  automatable ones.
%%%%%%%
\item We show how the use of an environment context at each
  layer allows us to apply standard techniques for
  verifying sequential programs to verify concurrent programs.
  Indeed, most of our kernel programs are written in a variant of C
  (called ClightX)~\cite{dscal15}, verified at the source level, and
  compiled and linked together using
  CompCertX~\cite{dscal15,ccal16}~--- a {\em thread-safe} version of the CompCert
  compiler~\cite{compcert,leroy09}. As far as we
  know, \CTOS\ is the first architecture that can truly build
  certified concurrent kernels and transfer global properties proved
  for programs (at the kernel specification level) down to the
  concrete assembly machine level.
%%%%%%%
\item We show how to impose temporal invariants over these environment
  contexts so we can verify the progress of various
  concurrent primitives. For example, to verify the starvation-freedom
  of ticket locks or MCS locks, we must assume that the multicore
  hardware (or the OS scheduler) always generates a {\em fair}
  interleaving, and those threads/CPUs which requested locks before
  the current running thread will eventually acquire and then release
  the lock. In a separate paper~\cite{ccal16}, we present the formal
  theory of environment contexts and show how these assumptions
  can be discharged when we compose different threads/CPUs to form
  a complete system.
%%%%%%%
\item Using \CTOS, we have successfully developed a fully certified
  concurrent OS kernel (called \mCTOS) in the Coq proof
  assistant~\cite{coq}. Our kernel supports both fine-grained locking
  and thread yield/sleep/wakeup primitives, and can run on stock x86
  multicore machines. It can also double as a hypervisor and boot
  multiple instances of Linux in guest VMs running on different CPUs.
  Our certified hypervisor kernel consists of 6500 lines of C and x86
  assembly. The entire proof effort for supporting
  concurrency took less than 2 person years. To our knowledge, this
  is the first proof of functional correctness of a complete,
  general-purpose concurrent OS kernel with fine-grained locking.
\end{itemize}

The rest of this paper is organized as follows.
Section~\ref{sec:overview} gives an overview of our new
\CTOS\ architecture. Section~\ref{sec:machine} shows how we use
environment contexts to turn concurrent layers into sequential ones.
Section~\ref{sec:base} presents the design and development of the
\mCTOS\ kernel and how we verify various concurrent kernel
objects. Section~\ref{sec:imp} presents an evaluation of \CTOS.
Sections~\ref{sec:related}-\ref{sec:concl} discuss related work and
then conclude.
% The formal theory of a general compositional concurrent
% model and a detailed description of the language and compiler toolchains
% are presented in a separate paper~\cite{ccal16}.















%%%%%%%%%%%%%%%%%%%%%%%%%%%%%%%%%%%%%%%%%%%%%%%%%%%%%%%%%%%%%%%%%%%%%%%%%%%%
{
%\setlength{\floatsep}{-10pt}
\begin{figure}\centering
%\ifanonymized
\includegraphics[scale=.32]{figs/mainthmA}
%\else
%\includegraphics[scale=.3]{figs/mainthm}
%\fi
\caption{Certified OS kernels: what to prove?}
%\rule[0in]{\columnwidth}{.15mm}
\label{fig:mainthm}
\end{figure}
}
%%%%%%%%%%%%%%%%%%%%%%%%%%%%%%%%%%%%%%%%%%%%%%%%%%%%%%%%%%%%%%%%%%%%%%%%%%%

%\sectskip

\section{Overview of Our Approach}
\label{sec:overview}
%\asectskip

\ignore{
What is the high-level picture like?
\begin{itemize}
\item What to prove and why? What is the main theorem?
  (We can state such a theorem in Coq! need a figure)
\item Introducing abstract state and abstract primitives
       (this is necessary to specify a kernel! explain challenges)
\item Challenge: how to cut deep dependency? use layered refinement;
      viewing kernel verification as building a "certified compiler",
      that is, how to compile system calls w. spec into actual assembly code!
\item cutting deep dependency needs layered decomposition
         (Invariant decomposition: each layer has its own invariant)
\item Challenge: C too low level? 
       Mixing Clight/AsmX with abstract state/primitives, use CompCertX
\item Challenge: C too high level? 
       Define abstraction layer at the AsmX level?
\item Proving contextual refinement between two layers
       (Connecting all layers together to establish the new theorem)
\end{itemize}
}

The ultimate goal of research on building certified OS kernels is
not just to verify the functional correctness of a particular kernel,
but rather to find the best OS design and development methodologies that can be used to build provably reliable, secure, and efficient
computer systems in a cost-effective way. We enumerate a few important
dimensions of concerns and evaluation metrics which we have used so
far to guide our work toward this goal:
%%%%%%%%%
\begin{itemize}[leftmargin=*,itemsep=0pt] 
\item {\bf Support for new kernel design.}  Traditional OS kernels use
  the hardware-enforced ``red line'' to define a single system call
  API. A certified OS kernel opens up the design space significantly as
  it can support multiple certified kernel APIs at different
  abstraction levels. It is important to support kernel
  extensions~\cite{bershad95,engler95,unikernel13} and novel ring-0
  or guest-domain processes~\cite{hunt07,dune12} so we can experiment
  and find the best trade-offs.
%%%%%%%%%  
\item {\bf Kernel performance.} Verification should not
  impose significant overhead on kernel performance. Of course,
  different kernel designs may imply different performance
  priorities.  An L4-like microkernel~\cite{liedtke95} focuses on
  fast inter-process communication (IPC), while a
  Singularity-like kernel~\cite{hunt07} emphasizes efficient
  support for type-safe ring-0 processes.
%%%%%%%%%
\item {\bf Verification of global properties.}
  A certified kernel is much less interesting if it cannot be
  used to prove global properties of the complete system built on top
  of the kernel.  Such global
  properties include not only safety, liveness, and security properties
  of user-level processes and virtual machines, but also resource usage
  and availability properties (\eg, to counter denial-of-service attacks).
%%%%%%%%%  
\item {\bf Quality of kernel specification.}  A good kernel
  specification should capture precisely the {\em contextually observable}
  behaviors of the implementation~\cite{dscal15}. It must
  support transferring global properties proved at a high abstraction
  level down to any lower abstraction level~\cite{costanzo16}.
%%%%%%%%%
\item {\bf Cost of development and maintenance.} Compositionality
  is the key to minimize such cost. If the machine model is stable,
  verification of each kernel module should only need
  to be done once (to show that it {\em implements} its deep
  functional specification~\cite{dscal15}). Global properties (e.g., 
  information flow security) should
  be derived from the kernel deep specification alone~\cite{costanzo16}. 
%%%%%%%%%  
\item {\bf Quality of formal proofs.} We use the term {\em certified
  kernels} rather than {\em verified kernels} to emphasize the
  importance of third-party machine-checkable proof
  certificates~\cite{shao10}. Hand-written paper proofs are
  error-prone~\cite{findler12}. Program verification without explicit
  machine-checkable proof objects has been subject to
  significant controversy~\cite{demillo77}.
\end{itemize}
%%%%%%%%%

\vspace*{-10pt}
\paragraph{Overview of \CTOS}
Our new \CTOS\ architecture aims to address all these concerns and
also tackle the challenges described in Section~\ref{sec:intro}.  The
\CTOS\ architecture leverages the new certified programming
methodologies developed by \citet{dscal15,ccal16} and applies them to
support building certified concurrent OS kernels.

A {\em certified abstraction layer} consists of a language construct
$\layer{L_1}{M}{L_2}$ and a mechanized proof object showing that the
layer implementation $M$, built on top of the interface $L_1$ (the
{\em underlay}), is a {\em contextual refinement} of the desirable
interface $L_2$ above (the {\em overlay}). A {\em deep} specification
($L_2$) of a module ($M$) captures everything {\em contextually
  observable} about running the module over its underlay ($L_1$). Once
we have certified $M$ with a deep specification $L_2$, there is no
need to ever look at $M$ again, and any property about $M$ can be
proved using $L_2$ alone.

%%%%%%%%%%%%%%%%%%%%%%%%%%%%%%%%%%%%%%%%%%%%%
\begin{figure}\centering
\includegraphics[scale=0.7]{figs/refine_layer}
\caption{Contextual refinement between concurrent layers}
\label{fig:spec:refine_layer}
\vspace*{-10pt}
\end{figure}
%%%%%%%%%%%%%%%%%%%%%%%%%%%%%%%%%%%%%%%%%%%%%


%%%%%%%%%%%%%%%%%%%%%%%%%%%%%%%%%%%%%%%%%%%%%
{
%\setlength{\floatsep}{-10pt}
%\setlength{\abovecaptionskip}{-2pt}
%\setlength{\belowcaptionskip}{-10pt}
\begin{figure*}\centering
\includegraphics[scale=.75]{figs/sysarch}
\caption{System architecture for the \mCTOS\ kernel}
%\rule[0in]{\columnwidth}{.15mm}
\label{fig:sysarch}
\vspace*{-10pt}
\end{figure*}
}
%%%%%%%%%%%%%%%%%%%%%%%%%%%%%%%%%%%%%%%%%%%%%

In Figure~\ref{fig:mainthm}, we use x86mc to denote an assembly-level
multicore machine.  Suppose we load such a machine with the
\mCTOS\ kernel $K$ (in assembly) and user-level assembly code $P$, and
we use {$\sem{\rm{}x86mc}{\cdot}$} to denote the whole-machine
semantics for x86mc, then proving any global property of such a
complete system amounts to reasoning about the semantic object
{$\sem{\rm{}x86mc}{K\join{}P}$}, i.e., the set of observable behaviors
from running $K\join{}P$ on x86mc.

Reasoning at such a low level is difficult, so we formalize a new
\mCTOS\ machine that extends the x86mc machine with the (deep) high-level
specification of all system calls implemented by $K$.
We use $\sem{\rm\mCTOS}{\cdot}$ to denote its
whole-machine semantics.  The contextual refinement property about the
\mCTOS\ kernel can be stated as:
%%%%
 \[
\forall{}P,\;\sem{\rm{}x86mc}{K\join{}P}\Refrel{}\sem{\rm\mCTOS}{P}
\]%
%%%%
\noindent
Hence any global property proved about
$\sem{\rm\mCTOS}{P}$ can be transferred to
$\sem{\rm{}x86mc}{K\join{}P}$.

To support concurrency, for each layer interface $L$,
we parameterize it with an {\em active} thread set $A$ 
and then carefully define its set of valid {\em environment contexts},
denoted as $\ectxt{L,A}$. Each environment context $\oracle$ captures
a specific
instance---from a particular run---of the list of events that other
threads or CPUs (i.e., those not in $A$) return when responding
to the events generated by those in $A$.  We can then define a
new {\em thread-modular} machine $\mach{L(A)}(P,\oracle)$ that will
operate like the usual assembly machine when $P$ switches control to
those threads in $A$, but will only obtain the list of events from the
environment context $\oracle$ when $P$ switches control to those outside
$A$. The semantics for a concurrent layer machine $L$ is then:
%%%%
\[
\sem{L(A)}{P} = \{ ~ \mach{L(A)}(P,\oracle) ~ \mid ~ \oracle \in \ectxt{L,A} ~ \}
\]%
%%%%
\noindent{}To support parallel layer composition, 
we carefully design $\ectxt{L,A}$ so that the following
property holds:
%%%%
\[
\sem{L(A\cup{}B)}{P} ~ = ~ \sem{L(A)}{P} ~ \cap ~ \sem{L(B)}{P} ~~ {\it if} ~ A\cap{}B=\emptyset
\]%
%%%%
\noindent{}The formal details for $\ectxt{L,A}$ and 
$\sem{L(A)}{\cdot}$ are presented in a separate paper~\cite{ccal16}.
Note that if $A$ is a singleton, for each $\oracle$,
$\mach{L(A)}$  behaves like a sequential machine.

With our new compositional layer semantics, we can take a multicore
machine like x86mc and zoom into a specific active CPU $i$ by creating
a {\em logical} ``single-core'' machine layer for CPU $i$, and then
apply techniques from \citet{dscal15} to build a collection of
certified ``sequential'' (per-CPU) layers (see
Figure~\ref{fig:spec:refine_layer}).
%%%%
When we want to introduce kernel- or
user-level threads, we can further zoom into a particular thread
(\eg, $i0$) and create a corresponding logical machine
layer.
%%%%%
We can impose
specific invariants over the environment contexts (i.e., the ``rely''
conditions) and use them to ensure that per-CPU or per-thread
reasoning can be soundly composed (when their ``rely'' conditions are
compatible with each other).
%%%%%
After we have added all the kernel components and
implemented all the system calls, we can combine these per-thread
machines into a single concurrent machine.

Under \CTOS, building a new certified concurrent kernel (or
experimenting with a new design) is just a matter of composing a
collection of certified concurrent layers, developed in a variant of C
(called ClightX) or assembly.  \citet{dscal15} have
developed a certified compiler (CompCertX) that can compile certified
ClightX layers into certified assembly layers. Since all
concurrent primitives in \CTOS\ are treated as CompCert-style
external calls or built-ins,
they cannot be reordered or optimized away by the compiler. Memory
accesses over these external calls cannot be reordered either.
Therefore, each concurrent ClightX module (running over a particular
per-thread or per-CPU layer) is compiled by CompCertX as if it were a
sequential program performing many external-call events. The
correctness of CompCertX guarantees that the generated x86 assembly
behaves the same as the source ClightX module.  CompCertX can
therefore serve as a {\em thread-safe} version of CompCert.

\CTOS\ can thus enjoy the full programming power of both an ANSI C
variant and an assembly language to certify any efficient routines
required by low-level kernel programming.  The layer mechanism allows
us to certify most kernel components at higher abstraction levels,
even though they all eventually get mapped (or compiled) down to an
assembly machine.

\vspace*{-10pt}
\paragraph{Overview of the \mCTOS\ kernel}
Figure~\ref{fig:sysarch} shows the system architecture of \mCTOS.  The
\mCTOS\ system was initially developed in the context of a large
DARPA-funded research project.  It is a concurrent OS kernel that can
also double as a hypervisor.  It runs on an Unmanned Ground Vehicle
(UGV) with a multicore Intel Core i7 machine.  On top of
\mCTOS, we run three Ubuntu Linux systems as guests (one each
on the first three cores). Each virtual machine runs several RADL (The
Robot Architecture Definition Language~\cite{radl15}) nodes that have
fixed hardware capabilities such as access to GPS, radar, \etc\  
The kernel also contains a few simple device drivers (\eg, interrupt
controllers, serial and keyboard devices). More complex devices are
either supported at the user level, or passed through (via IOMMU) to
various guest Linux VMs. By running different RADL nodes in different
VMs, \mCTOS\ provides strong isolation support so that even if
attackers take control of one VM, they still cannot break into other
VMs to compromise the overall mission of the UGV.

Within \mCTOS, we have various shared objects such as spinlock modules
(Ticket, MCS), sleep queues (SleepQ) for implementing queueing locks
and condition variables, pending queues (PendQ) for waking up a
thread on another CPU, container-based physical and virtual memory
management modules (Container, PMM, VMM), a Lib Mem module for
implementing shared-memory IPC, synchronization modules (FIFOBBQ,
CV), and an IPC module. Within each core (the purple box), we have
the per-CPU scheduler, the kernel-thread management module, the process
management module, and the virtualization module (VM Monitor). Each
kernel thread has its own thread-control block (TCB), context, and stack.

\vspace*{-10pt}
\paragraph{What have we proved?}
Using \CTOS, we have successfully built a fully certified version of
the \mCTOS\ kernel and proved its contextual refinement property with
respect to a high-level deep specification for \mCTOS.  This important
functional correctness property implies that all system calls and
traps will strictly follow the high-level specification and always run
{\em safely} and {\em terminate} eventually; and there will be no data
race, no code injection attacks, no buffer overflows, no null pointer
access, no integer overflow, \etc

More importantly, because for any program $P$, we have 
$\sem{\rm{}x86mc}{K\join{}P}$ refines
$\sem{\rm\mCTOS}{P}$, we can also derive the
important {\em behavior equivalence} property for $P$, that is,
whatever behavior a user can deduce about $P$ based on the high-level
specification for the \mCTOS\ kernel $K$, the actual linked system
$K\join{}P$ running on the concrete x86mc machine would indeed behave
exactly the same.  All global properties proved at the system-call
specification level can be transferred down to the lowest assembly
machine.

\vspace*{-10pt}
\paragraph{Assumptions and limitations}
The \mCTOS\ kernel is obviously not as comprehensive as real-world
kernels such as Linux.  The main goal of this paper is to show that it
is feasible to build certified concurrent kernels with fine-grained
locking.  We did not try to incorporate all the latest advances for
multicore kernels into \mCTOS.

Our assembly machine assumes strong sequential consistency for all
atomic instructions. We believe our proof should remain valid for
the x86 TSO model because (1) all our concurrent layers guarantee that
non-atomic memory accesses are properly synchronized; and (2) the TSO
order guarantees that all atomic synchronization operations are
properly ordered. Nevertheless,
more formalization work is needed to turn our proofs over
sequential-consistent machines into those over the TSO
machines~\cite{vontessin13}.

Since our machine does not model TLB, any code for addressing TLB
shootdown cannot be verified.

The \mCTOS\ kernel currently lacks a certified storage system.
%and a certified network stack.
%The Linux community~\cite{lkmap} sorted these
%into different stacks of abstraction layers (based on their underlying
%hardware devices).\vilhelm{it's not clear to me what point this sentence makes.}
We plan to incorporate recent advances in building
certified file systems~\cite{fscq15,cogent16} into \mCTOS\ in the near future.

Our assembly machine only covers a small part of the full x86
instruction set, so our contextual correctness results only apply to
programs in this subset. Additional instructions 
can be easily added if they have simple or no interaction with our
kernel.  \citet[Sec. 6]{costanzo16} shows how the fidelity of the
CompCert-style x86 machine model would impact the formal correctness
or security claims, and how such gap can be closed.

The CompCertX assembler for converting assembly into machine code is
unverified. We assume correctness of the Coq proof checker
and its code extraction mechanism.

The \mCTOS\ kernel also relies on a bootloader, a {\it PreInit}
module (which initializes the CPUs and the devices), and an ELF
loader. Their verification is left for future work. 

\ignore{
The bottom-most x86 layer of our certified kernels is called 
\code{PreInit}, which initializes the drivers, \eg, serial, disk,
console, {\it etc}. Device drivers are not verified because our
current machine semantics lacks device models for expressing the
corresponding semantics.

Like most existing verified kernel efforts, we assume that interrupts
are only enabled in user or guest mode.  The challenges in handling
interrupts and preemption are similar to those for
concurrency~\cite{feng09:jar,feng08:vstte}. We believe that similar
approaches can be readily supported in our \CTOS\ framework.
\vilhelm{But then in section 4.3 ``Device drivers'', we say that
we have ported Chen et al's work to allow interrupts inside the kernel; 
which is it?}
}

%%%%%%%%%%%%%%%%%%%%%%%%%%%%%%%%%%%%%%%%%%%%%%%%%%%%%%%%%%%%%%%%%%%%%%%%%%%%%%



%\input{oldcontents/certikos/spec}
%\section{Certifying the {\mCTOS} Kernel}
\label{sec:base}

%%%%%%%%%%%%%%%%%%%%%%%%%%%%%%%%%%%%%%%%%%%%%%%%%%%
\begin{figure*}[t]
\begin{tabular}{c|c}
\begin{tabular}{cc}
(a) &\hspace{-10pt}
\begin{tabular}{c}
\includegraphics[scale=.33]{figs/mem_model_1} 
\end{tabular}
\end{tabular}
& 
\begin{tabular}{cc}
(b) &
\begin{tabular}{c}
\includegraphics[scale=.35]{figs/mem_model_2} 
\end{tabular}
\end{tabular}
\vspace*{-10pt} 
\end{tabular}
\caption{(a) Hardware MMU using two-level page map; (b) Virtual address space $i$ set up by page map $i$}
\label{fig:spec:memmodel}
\vspace*{-5pt}
\end{figure*}
%%%%%%%%%%%%%%%%%%%%%%%%%%%%%%%%%%%%%%%%%%%%%%%%%%%

\ignore{
\begin{figure}
\begin{minipage}{.22\textwidth}
\lstinputlisting [language = C] {source_code/lock_producer.c}
\end{minipage}\hfill
\begin{minipage}{.22\textwidth}
\lstinputlisting [language = C] {source_code/lock_consumer.c}
\end{minipage}
\ifTR{}{\vspace*{-5pt}}
\caption{Lock-based producer-consumer implementation}
\label{fig:exp:lock}
\end{figure}
}


\begin{comment}
\begin{figure*}
\begin{center}
\begin{scriptsize}
\begin{tabular}{ |l|l||l|p{4.5cm}| }
  \hline
  \multicolumn{2}{|c||}{\textbf{Memory Management}} 
  & \multicolumn{2}{|c|}{\textbf{Thread and Process Management}} \\
  \hline
  \hline    
  \multicolumn{2}{|l||}{\textbf{abstract state}} 
  & \multicolumn{2}{|l|}{\textbf{abstract state}}\\
  \hline
  \verb"AT" & physical page allocation table
  & \verb"kctxp" & kernel context (\verb"kctx") pool\\
  \hline 
  \verb"PFInfo" & save the address and \verb"PC" that page fault occurs
  & \verb"Ltdqp" & low abstract thread queue pool\\
  \hline
  \verb"ptp" & page table (\verb"pt") pool 
  
  & \verb"Htdqp"& high abstract thread queue (\verb"Htdq") pool\\ 
  \hline
   \verb"ipt"& whether \verb"pt"'s invariant  should hold or not
  
  & \verb"uctxp" & user context pool\\
  \hline
\verb"PT" & index of the current \verb"pt"
  & \verb"chanp" & channel pool\\
  \hline
  \verb"pbit" & bit map for free \verb"pt" indexes
  & \verb"Htcbp"& high abstract TCB pool \\
  \hline
  \multicolumn{2}{|l||}{\textbf{primitive}} 
  & \multicolumn{2}{|l|}{\textbf{primitive}}\\
  \hline	
  \verb"setcr3" & set the starting address of the \verb"pt"
  & \verb"kctx_new" & allocate the first free \verb"pt" and \verb"kctx"\\
  \hline
  \verb"meminit" & initialize the allocation table
  & \verb"Henqueue" & append a thread to the \verb"Htdq"\\
  \hline
  \verb"palloc" & allocate a page 
  & \verb"thread_kill" & kill and free a thread\\
  \hline
  \verb"pt_insrt" & insert a page map into a given \verb"pt" 
  & \verb"thread_sleep" & sleep, schedule to the 1st ready thread\\
  \hline
  \verb"pt_resv" & allocate a page for a given linear addr 
  & \verb"kctx_switch" & switch \verb"kctx" between threads\\
  \hline 
  \verb"PTInit" & init kernel's \verb"pt" and enable paging 
  & \multirow{2}{*}{\texttt{resv\_chan}} & 
  receive msg from the channel, wake\\
  \cline{1-2}
  \verb"pt_new" & allocate the first free \verb"pt" & & up the first sleeping thread
    \\	  
  \hline
  \hline
  \multicolumn{2}{|c||}{\textbf{Virtualization}}
  &\multicolumn{2}{|c|}{\textbf{Trap Handler}} \\
  \hline
  \hline    
  \multicolumn{2}{|l||}{\textbf{abstract state}} 
  & \multicolumn{2}{|l|}{\textbf{primitive}} \\
  \hline
  \verb"npt" & nested page table for guest
  & \verb"trap_arg" & get arguments of system calls\\
  \hline
  \verb"hctx"& host context
  & \verb"hpagefault" & page fault handler\\ 
  \hline
  \verb"vmcb" & virtual machine (\verb"VM") control control block
  & \verb"sys_yield" & system calls for yielding\\
  \hline
  \verb"xvmst" & registers not saved in \verb"vmcb" 
  & \verb"sys_wait_chan" & system calls to sleep on a channel\\
  \hline
  \multicolumn{2}{|l||}{\textbf{primitive}} 
  & \verb"sys_run_vm" & system calls to run \verb"VM"\\
  \hline	
  \verb"npt_insrt" & insert into the nested page table
  & \verb"sys_proc_create" & system calls to create a process\\
  \hline
  \verb"switch2guest" &  switch to guest mode 
  & \verb"sys_getexitinfo" 
  & get the information about \verb"VM" exit\\
  \hline
  \verb"set_vmcb" & set value in virtual machine control block
  & \verb"sys_injectevent" & inject interrupt and exception to \verb"VM"\\
  \hline 
  \verb"run_vm" & save host context, restore \verb"vmcb", start \verb"VM" 
  & \verb"kernel_init" & initialization function of the kernel\\  
  \hline  

\end{tabular}
\end{scriptsize}
\caption{Key abstract states and primitives for \mCTOSbase{} and \mCTOShyper{}}
\label{table:layers}
\end{center}
%\vspace*{-14pt}
\end{figure*}
\end{comment}

\begin{comment}
\begin{figure}
\includegraphics[scale=0.38]{figs/memory_management_layer}	
\caption{Layers of PreInit and memory management}
\label{fig:base:mm:layers}
\vspace*{-14pt}
\end{figure}

\begin{figure}
\includegraphics[scale=0.37]{figs/process_management_layer}	
\caption{Layers of process management}
\label{fig:base:pm:layers}
\vspace*{-14pt}
\end{figure}

\begin{figure}
\includegraphics[scale=0.37]{figs/trap_management_layer}	
\caption{Layers of trap management}
\label{fig:base:tm:layers}
\vspace*{-14pt}
\end{figure}
\end{comment}

\ignore{
\begin{figure}
\includegraphics[scale=0.34]{figs/mctos_layer}	
\caption{Layers of \mCTOSbase{}}
\label{fig:base:ctos:layers}
\vspace*{-14pt}
\end{figure}
}

Contextual refinement provides an elegant formalism for decomposing
the verification of a complex kernel into a large number of
small tractable tasks: we define a series of logical abstraction
layers, which serve as increasingly higher-level specifications
for an increasing portion of the kernel code. We design
these abstraction layers in a way such that complex
interdependent kernel components are untangled and converted
into a well-organized kernel-object stack with clean
specification (\cf ~Fig.~\ref{fig:spec:refine_layer}). 

In the \mCTOSbase{} kernel, the pre-initialization module is the
bottom layer that connects to the \emph{CPU-local machine model}
$\mach{loc}$, instantiated with a particular \emph{active CPU} (\cf
Sec.~\ref{ssec:spec:seq}).  The trap handler contains the top layer
that provides system call interfaces and serves as a specification of
the whole kernel, instantiated with a particular active thread running
on that active CPU.  Our main theorem states that any global
properties proved at the topmost abstraction layer can be transferred
down to the lowest hardware machine.  In this section, we explain
selected components in more details.

%\paragraph{Pre-initialization module}
%\label{sec:base:preinit}

Each CPU-local pre-initialization machine defines the x86 hardware
behaviors including page table walk upon memory load (when paging is
turned on), saving and restoring the trap frame in the case of
interrupts and exceptions (\eg, page fault), and the data exchange
between devices and memory. The hardware memory management unit (MMU)
is modeled in a way that mirrors the paging hardware (\cf
Fig.~\ref{fig:spec:memmodel}a). When paging is enabled, memory
accesses made by both the kernel and the user programs are translated
using the page map pointed to by \code{CR3}.  When a page fault
occurs, the fault information is stored in \code{CR2}, the CPU mode is
switched from user mode to kernel mode, and the page fault handler is
triggered.


\ignore{
Some privileged
memory regions (\eg, allocation table) and 
instructions (\eg, modifying control registers)
are only available in kernel mode.}
\ignore{
\code{CR0} selects the memory protection mode,
\code{CR2} stores the Page Fault Linear Address (PFLA)
as well as the address of the instruction that caused the page fault, and
\code{CR3} stores the starting point of the page map.
}
\ignore{
The switch function models the change of the \code{ikern} flag
and the remaining tasks involved with trap handling,
such as saving and restoring user and kernel contexts,
and dispatch over the trap type,
are verified at the assembly level.
}


\ignore{
{\color{red}Jan: Do we need the following paragraph?}
The initialization primitive at this bottom-most layer is the bootloader,
which initializes \code{MM} and necessary drivers
(tsc, disk, console, timer, keyboard, serial, {\it etc.}),
loads the kernel into the memory,
and sets the initialization flag to be \code{true}.
}

\vspace*{-10pt}
\paragraph{The spinlock module}\label{sec:base:lock} 
provides fine-grained
lock objects as the base of synchronization mechanisms. 

\textbf{Ticket Lock} depends on an \emph{atomic ticket object}, which
consists of two fields: \code{ticket} and \code{now}.
Figure~\ref{fig:exp:ticket_lock} shows one implementation of a ticket
lock. Here, \texttt{L} is declared as an array of ticket locks; each
shared data object can be protected with one lock in the array,
identified using a specific lock index (\texttt{i}).  The atomic
increment to the ticket is achieved through the atomic
\texttt{fetch-and-increment} (FAI) operation (implemented using the
\texttt{xaddl} instruction with the \texttt{lock} prefix in x86).  As
described in Section~\ref{ssec:spec:seq}, the \emph{{\intptext}s} at
this abstraction level have been shuffled and merged so that there is
exactly one {\intptext} before each atomic operation.  Thus, the lock
implementations generate a list of events; \ignore{The atomic
  operations generate ticket-object events;} for example, when CPU $t$
acquires the lock $i$ (stored in \code{L[i]}), it continuously
generates the event ``$\event{t.get\_now\ i}$" (line~10) until the
latest \code{now} is increased to the ticket value returned by the
event ``$\event{t.inc\_ticket\ i}$" (line~9), and then followed by the
event ``$\event{t.pull\ i}$" (line~11):
\[
\includegraphics[scale=.6]{figs/ticket_log}
\]
The event list is as below:
$$[\intp,\event{t.inc\_ticket\ i},\intp,\event{t.get\_now\ i},\cdots,\intp,\event{t.get\_now\ i}]$$
 
Verifying the linearizability and starvation-freedom of the ticket
lock object is equivalent to proving that under a {\em fair} hardware
scheduler $\hardoracle$, the ticket lock implementation is a {\em
  termination-sensitive} contextual refinement of its atomic
specification~\cite{liang13,lili16}.  There are two main proof
obligations: (1) the lock guarantees \emph{mutual exclusion}, and (2)
the \code{acq\_lock} operation eventually succeeds.

%%%%%%%%%%%%%%%%%%%%%%%
\begin{figure}
\lstinputlisting [language = C, multicols=2] {source_code/ticket_lock.c}
\caption{Pseudocode of the ticket lock implementation}
\label{fig:exp:ticket_lock}
%\vspace*{-10pt}
\end{figure}
%%%%%%%%%%%%%%%%%%%%%%%
 
\emph{Mutual exclusion} is straightforward for a ticket lock.  At any
time, only the thread whose ticket is equal to the current serving
ticket (\ie, \code{now}) can hold the lock.  Furthermore, each
thread's ticket is unique as the $\texttt{fetch-and-increment}$
operation is atomic (line~9).  Thanks to this \emph{mutual exclusion}
property, it is safe to \emph{pull} the shared memory associated with
the lock $i$ to the local copy at line~11.  Before releasing the lock,
the local copy is \emph{pushed} back to the shared memory at line~14.

To prove that \code{acq\_lock} eventually succeeds, from the fairness
of $\hardoracle$, we assume that between any two consecutive events
from the same thread, there are at most $m$ events generated by other
threads (for some $m$). We also impose the following invariants on the
environment:
\begin{invariant}[Invariants for ticket lock]
\label{inv:lock}
An environment context that holds the lock $i$
(1) never acquires lock $i$ again before releasing it;
and (2) always releases lock $i$ within $k$ steps
(for some $k$).
\end{invariant}
\begin{lemma}[Starvation-freedom of ticket lock]
\label{lemma:lock}
Acquiring ticket-lock in the {\mCTOS} kernel eventually succeeds.
\proof 
The full proofs are mechanized
in Coq; here we highlight the main ideas.
Let $n$ be the maximum number of the total threads.  Then (1) there
are at most $n$ threads waiting before the current one; (2) the thread
holding the lock releases the lock within $k$ steps, which generates
at most $k$ events; and (3) the environment context generates at most $m$
events between each step of the lock holder.  Hence there are at most
$n\times m\times k$ events generated by the \emph{context} of the
threads waiting before the current one.  Since the current thread
belongs to this ``context" and each read to the \code{now} field
generates one \code{get\_now} event, there are at most $n\times
m\times k$ loop iterations at line~10 in
Fig.~\ref{fig:exp:ticket_lock}.  Thus, acquiring lock always succeeds.
\qed
\end{lemma}
After we abstract the lock implementation into an atomic
specification, each acquire-lock call in the higher layers
only generates a single event ``$\event{t.acq\_lock\ i}$.'' We can
compose such per-CPU specification with those of its environment CPUs as long
as they all follow Invariant~\ref{inv:lock}.

\ignore{
\begin{figure}
\lstinputlisting [language = C, multicols=2] {source_code/mcs_lock.c}
\caption{MCS Lock Implementation}
\label{fig:exp:mcs_lock}
\end{figure}
}

\textbf{MCS Lock} is known to have better scalability than ticket lock
over machines with a larger number of CPUs.  In {\mCTOS}, we have also
implemented a version of MCS locks~\cite{mcs91}.
The starvation-freedom proof is similar to that of the ticket lock.
The difference is that the MCS lock-release operation waits in a loop
until the next waiting thread (if it exists) has added itself to a
linked list, so we need similar proofs for both acquire and release.

\ignore{
\paragraph{Device drivers}
\label{sec:base:device}
\citet{chen16} developed a new framework for building certified
interruptible kernels and device drivers.
%%%%%%%%%%%%%%%
\ignore{Their core idea is to treat the driver for each device as if
  it were running on a ``logical'' CPU dedicated to that device; their
  new framework systematically enforces the isolation among
  different ``devices'' and the rest of the kernel.}
%%%%%%%%%%%%%%%
Their work has been successfully ported into our setting 
thanks to the fact that their event-based model is consistent
with our interleaving machine model.
}

\ignore{\subsection{Memory management}
\label{sec:base:memm}

The memory management of {\mCTOS} consists of  the
\emph{physical memory management} (4 layers), 
\emph{virtual memory management} (7 layers), and
\emph{shared memory management} (3 layers).
}

\ignore{
\begin{figure}
\includegraphics[scale=0.35]{figs/dynamic}	
\caption{The state transition of page object}
\label{fig:base:dynamic}
\vspace*{-14pt}
\end{figure}
}

\vspace*{-10pt}
\paragraph{Physical memory management}
\label{sec:base:memm}
introduces the page allocation table $\texttt{AT}$ (with
$\texttt{nps}$ denoting the maximum physical page number).  Since
$\texttt{AT}$ is shared among different CPUs, we associate it with a
lock $\texttt{lock\_AT}$.  The page allocator is then refined into an
atomic object where the implementation for each of its methods (e.g.,
\texttt{palloc} in Fig.~\ref{fig:exp:palloc}) is proved to satisfy an
atomic interface, with the proof that lock utilization for
$\texttt{lock\_AT}$ satisfies Inv.~\ref{inv:lock}.  Once the atomic
allocator is introduced, lock acquire and release for
$\texttt{lock\_AT}$ are \emph{not allowed to be invoked} at higher
layers.  Thus, in this layered approach, it is not possible that a
thread holding a lock defined at a lower layer tries to acquire
another lock introduced at a higher layer, \ie, the order that a
thread acquires different locks is guided by the layer order that the
locks are introduced.  This implicit order of lock acquisitions
prevents \emph{deadlocks} in {\mCTOS}.

Another function of the physical memory management is to dynamically
track and bound the memory usage of each thread. A \emph{container}
object is used to record information for each thread (array \code{cn}
in Fig.~\ref{fig:exp:palloc}); one piece of information tracked is the
thread's \emph{quota}. Inspired by the notions of containers and
quotas in HiStar~\cite{zeldovich06}, a thread in {\mCTOS} is spawned
with some quota specifying the maximum number of pages that the thread
will ever be allowed to allocate. As can be seen in
Fig.~\ref{fig:exp:palloc}, \code{palloc} returns an error code if the
requesting thread has no remaining quota (lines~2 and~3), and the
quota is decremented when a page is successfully allocated (line~13).
Quota enforcement allows the kernel to prevent a denial-of-service
attack, where one thread repeatedly allocates pages and uses up all
available memory (thus denying other threads from allocating
pages). From a security standpoint~\cite{costanzo16}, it also
prevents the undesirable information channel between different threads
that occurs due to such an attack.

\vspace*{-10pt}
\paragraph{Virtual memory management}
provides consecutive virtual address spaces on top of physical memory
management (see Fig.~\ref{fig:spec:memmodel}b),
\ignore{Because much of the code assumes that the memory
management sets up the virtual address space properly, initialization
has been a sticking point.}  We prove that the primitives 
manipulating page maps are correct, and 
the \emph{initialization procedure} sets up the two-level page
maps properly in terms of the hardware address translation.

\begin{invariant}
\label{inv:virtual}
(1) paging is enabled only after all the page maps are initialized;
(2) pages that store kernel-specific data must have the kernel-only
permission in all page maps; (3) the kernel page map is an identity map;
and (4) non-shared parts of user processes' memory are isolated.
\end{invariant}

By Inv.~\ref{inv:virtual}, we show that it is safe to run both the
kernel and user programs in the virtual address space when paging is
enabled.  In this way, memory accesses at higher layers operate on the
basis of the high-level, abstract descriptions of address spaces
rather than concrete page directories and page tables stored in the
memory itself.

%%%%%%%%%%%%%%%%%%%%%%%%
\begin{figure}[t]
\lstinputlisting [language = C, multicols=2] {source_code/palloc.c}
\caption{Pseudocode of \texttt{palloc}}
\label{fig:exp:palloc}
\vspace*{-5pt}
\end{figure}
%%%%%%%%%%%%%%%%%%%%%%%%

\vspace*{-10pt}
\paragraph{Shared memory management} provides a protocol to share physical
pages among different user processes. 
A physical page can be mapped into multiple processes' page maps.  
For each page, we maintain a \emph{logical owner set}.  
For example, a
user process $k_1$ can share its private physical page $i$ to another
 process $k_2$ and the logical owner
set of page $i$ is changed from $\set{k_1}$ to $\set{k_1,k_2}$.
A shared page can only be
freed when its owner set is a \emph{singleton}.
\ignore{\subsection{Process management}

Process management  introduces the
\emph{abstract queue library} (4 layers),
\emph{thread management} (6 layers),
\emph{condition variable} (3 layers),
and \emph{IPC} module (2 layers).}

\ignore{\begin{figure}
\lstinputlisting [language = C, multicols=1] {source_code/enqueue.v}
\vspace{-5pt}
\caption{Specifications of local queue operations}
\label{fig:exp:queue}
\vspace{-10pt}
\end{figure}}

\vspace*{-10pt}
\paragraph{The shared queue library}
\label{sec:base:procm}
abstracts the queues implemented as \emph{doubly-linked lists} into
\emph{abstract queue states} (\ie, Coq lists).  The local {\it
  enqueue} and {\it dequeue} operations are specified over the
abstract lists.  As usual, we associate each shared queue with a lock.
The atomic interfaces for shared queue operations are represented by
queue events $``\event{t.enQ\ i\ e}"$ and $``\event{t.deQ\ i}"$, which
can be replayed to construct the shared queue.  
For instance, starting
from an empty initial queue, if the current log of the $i$-th
shared queue is
$[\intp,\event{t_0.enQ\ i\ 2},\intp,\event{t_0.deQ\ i}]$, and the
event lists generated by the \emph{environment context} at two
{\intptext}s are $[\event{t_1.enQ\ i\ 3}]$ and
$[\event{t_1.enQ\ i\ 5}]$, respectively, then the complete log
for the queue $i$ is:
\[[\event{t_1.enQ\  i\ 3},\event{t_0.enQ\  i\ 2},\event{t_1.enQ\  i\ 5},\event{t_0.deQ\ i}]\]%
By replaying the log, the shared queue state becomes $[2,5]$,
and the last atomic dequeue operation returns 3.
\ignore{
\[
\includegraphics[scale=.67]{figs/queue_log}
\]}

\vspace*{-10pt}
\paragraph{Thread management}
introduces the thread control block and manages the resources of
dynamically spawned threads (\eg, quotas) and their meta-data (\eg,
children, thread state).  For each thread, one page (4KB) is
allocated for its \emph{kernel stack}.  We use an external
tool~\cite{veristack} to show that the stack usage of our compiled
kernel is  less than 4KB, so stack overflows cannot occur inside
the kernel.

One interesting aspect of the thread module is the context switch
function.  This assembly function saves the register set of the
current thread and restores the register set from the kernel context
of another thread on the same CPU.  Since the instruction pointer
register (\code{EIP}) and stack pointer register (\code{ESP}) are
saved and restored in this procedure, this kernel context switch
function is verified at the assembly level, and linked with other code
that is verified at the C~level and then compiled by CompCertX.

The thread scheduling is done by three primitives: \code{yield},
\code{sleep}, and \code{wakeup}. They are implemented using the shared queue library (\cf
Fig.~\ref{fig:exp:fig:scheduler}).  Each CPU has a \emph{private ready
  queue} ReadyQ and a \emph{shared pending queue} PendQ.
The context CPUs can insert threads to the current CPU's pending
queue.  The {\mCTOS} kernel also provides a set of  shared \emph{sleeping
  queues} SleepQs.  As shown
in Fig.~\ref{fig:exp:fig:scheduler},  \code{yield}  moves
a thread from the pending queue to the ready queue and then switches to
the next ready thread.  The \code{sleep} primitive simply adds the
running thread to a sleeping queue and runs the next ready thread.
The \code{wakeup} primitive contains two cases.  If the thread to be
woken up belongs to the current CPU, then the primitive adds the
thread to its ready queue.  Otherwise, \code{wakeup} adds the thread
to the pending queue of the CPU it belongs to.  Except for the ready
queue, all the other thread queue operations are protected by
\emph{fine-grained} locks.

\begin{figure} \centering
\includegraphics[scale=.72]{figs/scheduler} 
\caption{Scheduling routines \texttt{yield}, \texttt{sleep},
and \texttt{wakeup}}
\label{fig:exp:fig:scheduler}
\vspace*{-10pt}
\end{figure}

\ignore{
\begin{figure}
\lstinputlisting [language = C, multicols=2] {source_code/thread_management.c}
\caption{Implementation of the Scheduler Module}
\label{fig:exp:scheduler}
\end{figure}
}

\vspace*{-10pt}
\paragraph{Thread-local machine models}
can be built based on the thread management layers.  The first
step is to extend the environment context with a \emph{software
  scheduler} (\ie, abstracting the concrete scheduling procedure),
resulting in a new environment context $\oracle_{ss}$.  The scheduling
primitives generate the $\event{yield}$ and $\event{sleep}$ events and
$\oracle_{ss}$ responds with the next thread ID to execute.  One
invariant we impose on $\oracle_{ss}$ is that a sleeping
thread can be rescheduled only after a $\event{wakeup}$ event is
generated.  The second step is to introduce the \emph{active thread
  set} to represent the \emph{active} threads on the \emph{active CPU}, and
extend the $\oracle_{ss}$ with the \emph{context threads}, \ie, the
rest of the threads running on the active CPU. The composition
structure is similar to the one of Lemma~\ref{lemma:compose}.  In this
way, higher layers can be built upon a thread-local machine model with
a single active thread on the active CPU (\cf
Fig.~\ref{fig:spec:refine_layer}).

%%%%%%%%%
\begin{figure}[t]
\lstinputlisting [language = C, multicols=2] {source_code/fifoq3.c}
\caption{Pseudocode of the remove method for FIFOBBQ}
\label{fig:exp:fifo}
\vspace*{-5pt}
\end{figure}
%%%%%%%%%

\vspace*{-5pt}
\paragraph{Starvation-free condition variable}
A \emph{condition variable} (CV) is a synchronization object that
enables a thread to wait for a change to be made to a
shared state (protected by a lock).  Standard Mesa-style
CVs~\cite{lampson80} do not guarantee starvation-freedom: a thread
waiting on a CV may not be signaled within a bounded number of
execution steps. We have implemented a starvation-free version of CV
using condition queues as shown by \citet[Fig.~5.14]{ospp11}. However,
we have found a bug in the FIFOBBQ implementation shown in that
textbook: in some cases, their system can get stuck by allowing all
the signaling and waiting threads to be asleep simultaneously, or the
system can arrive at a dead end where the threads on the remove queue
(rmvQ) can no longer be woken up.  We fixed this issue by postponing
the removal of the CV of a waiting thread from the queue, until the
waiting thread finishes its work (\cf Fig.~\ref{fig:exp:fifo}); the
remover is now responsible for removing itself from the rmvQ (line~24)
and waking up the next element in the rmvQ (line~27). Here, \code{peekQ}
reads the head item of a queue; and \code{my\_cv} returns the CV
assigned to the current running thread. 

\ignore{
\paragraph{IPC}
We have implemented and verified a single-copy
inter-process communication (IPC) protocol using condition variables
and the FIFO Blocking Bounded Queue.
Additionally, we have verified a
zero-copy IPC for user programs that is built on top of the
shared memory infrastructure.
}

%%%%%%%%%%%%%%%%%%%%%%%%%%%%%%%%
\ignore{
\begin{figure}
\lstinputlisting [language = C, multicols=2] {source_code/ipc.c}
\caption{Implementation of Single-Copy IPC}
\label{fig:exp:ipc}
\end{figure}
}
%%%%%%%%%%%%%%%%%%%%%%%%%%%%%%%%

%%%%%%%%%%%%%%%%%%%%%%%%%%%%%%%%
\ignore{
\begin{figure}
\begin{center}
\includegraphics[scale=0.33]{figs/pagefault2}	
\caption{Call graph of page fault handler}
\label{fig:base:pagefault}
\end{center}
\vspace*{-14pt}
\end{figure}
}
%%%%%%%%%%%%%%%%%%%%%%%%%%%%%%%%

\ignore{
\paragraph{Trap module}\label{sec:base:trapm} 
specifies the behaviors of exception handlers and
\mCTOSbase\ system calls.
In \mCTOSbase{}, exception handlers are registered in a table of first-class code pointers.
When an exception occurs, the kernel consults this table
and invokes the corresponding exception handler.

For example, a page fault at the user level traps into the kernel, saves the
current trap frame (done by both the hardware and software), and then jumps
to the page fault handler.
The page fault handler reserves a page for \texttt{PFLA} (if necessary)
and returns to the user level by restoring the saved user context.
The verification of the page fault handler depends on objects
introduced at various abstraction levels. 
% (see Fig.~\ref{fig:base:pagefault}).
Therefore, the behavior of the page fault handler is interpreted by
the concrete first-class code pointer until all the dependent layer
objects are introduced.  Then the handler code is verified and
the behavior is interpreted using its abstract atomic specification.

To further simplify the reasoning about user code, we have implemented and
verified the user level system call libraries directly in the user space.
Since our machine semantics models hardware behaviors
like paging and ring switch, the specifications of user system call
libraries closely corresponds to the real execution model in the actual
hardware. With this \emph{atomic} system call semantics in the user level,
the user code can be proved much more easily.

The top layer of \mCTOSbase{} offers a set of system calls for user programs, 
such as IPC calls and calls to invoke the scheduler.
The specifications of system calls are defined and verified at the user level
by wrapping the system call handler's specification
with the ring switch specification.
We can reason about user-level programs directly with these atomic system calls' specifications.
}



\ignore{
\newman{Already in Section 2}
\subsection{Other properties}
Except for the above features, we also prove the following properties of \mCTOSbase{}:
\begin{itemize}
\item Since the contextual refinement is termination sensitive, we prove the total
correctness of our kernel, meaning that our kernel will not get stuck
and all system calls for user program will terminate.
\item There is no integer overflow inside the kernel.
\item There is not stack overflow inside the kernel. (Statically checked by the analysis tool,
refer to Quentin's work)
\item All the pointers stored in the kernel objects are valid.
\end{itemize}
}

%\input{oldcontents/certikos/imp}
%\section{Related Work}
\label{sec:related}

Dijkstra~\cite{dijkstra68a,Dijkstra72} proposed to ``realize'' a
complex program by decomposing it into a hierarchy of linearly ordered
abstract machines.  Based on this idea, the PSOS team at
SRI~\cite{psos80} developed the Hierarchical Development Methodology
(HDM) and applied it to design and specify an OS using 20
hierarchically organized modules. HDM was later also used for the KSOS
system~\cite{ksos84}.
\citet{dscal15} developed new languages and tools for building
certified abstraction layers with {\em deep} specifications, and
showed how to apply the layered methodology to construct fully
certified (sequential) OS kernels in Coq.

\citet{costanzo16} showed how to prove sophisticated global properties
(e.g., information-flow security) over a deep specification of a
certified OS kernel and then transfer these properties from the
specification level to its correct assembly-level implementation.
\citet{chen16} extended the layer methodology to build certified
kernels and device drivers running on multiple {\em logical}
CPUs. They treat the driver stack for each device as if it were
running on a logical CPU dedicated to that device. Logical CPUs do not
share any memory, and are all eventually mapped onto a single physical
CPU.
%%%%
None of these systems, however, can support shared-memory concurrency
with fine-grained locking.

\ignore{Our new \CTOS\ framework adds several
significant novelties (e.g., new models and refinement proofs for
concurrent layer machines, new layer design with environment context),
but it still connects back to the previous
work~\cite{dscal15,chen16,costanzo16} really nicely.  A concurrent
layer with a specific environment context can be composed freely just
as sequential layers~\cite{dscal15}.  The invariants over the
environment contexts (i.e., the ``rely'' conditions) are used to
guarantee that per-CPU or per-thread reasoning can be soundly composed
(when their ``rely'' conditions are compatible with each other).
}

The seL4 team~\cite{klein2009sel4,klein14} was the first to verify the
functional correctness and security properties of a high-performance
L4-family microkernel. The seL4 microkernel, however, does not support
multicore concurrency with fine-grained locking.  \citet{peters15}
and \citet{vontessin13} argued that for an seL4-like microkernel,
concurrent data accesses across multiple CPUs can be reduced to a
minimum, so a single {\em big kernel lock (BKL)} might be good enough
for achieving good performance on multicore machines.
\citet{vontessin13} further showed how to convert the single-core seL4
proofs into proofs for a BKL-based clustered multikernel.

\ignore{
One high-level difference between seL4 and \CTOS\ is that the seL4
team~\cite{klein14} focused on verifying a particular microkernel. The
designers of the L4-family kernels~\cite{liedtke95,heiser13} advocated
the {\em minimality principle}: a concept is tolerated inside the
microkernel only if moving it outside the kernel would prevent the
implementation of the system's required functionality.  This is a
reasonable principle but its interpretation of the ``kernel-user''
boundary (as the hardware-enforced ``red-line'') is quite narrow.  Our
new \CTOS\ architecture advocates replacing the traditional ``red
line'' with a large number of certified abstraction layers enforced by
formal specification and proofs; hardware mechanism (such as address
protection) is just one (quick) way of ensuring that a specific
process will not violate the invariants required by a particular
kernel abstraction layer.
}

The Verisoft team~\cite{verisoft07,leinenbach09,alkassar10} applied
the VCC framework~\cite{vcc09} to formally verify Hyper-V, which is a
widely deployed multiprocessor hypervisor by Microsoft consisting of
100 kLOC of concurrent C code and 5 kLOC of assembly. However, only
20\% of the code is verified~\cite{vcc09}; it is also only verified
for function contracts and type invariants, not the full functional
correctness property.  There is a large body of other
work~\cite{bevier89,hawblitzel10,ironclad14,fscq15,ironfleet15,verdi15,cogent16,uberspark16}
showing how to build verified OS kernels, hypervisors, file systems, device
drivers, and distributed systems, but they do not address the issues
on concurrency.

\citet{xu16} developed a new verification framework by combining
rely-guarantee-based simulation~\cite{RGSim} with Feng~{et~al.}'s
program logic for reasoning about interrupts~\cite{feng08:aim}.
They have successfully verified key modules in the $\mu$C/OS-II
kernel~\cite{ucosii}. Their work supports preemption but only on a
single-core machine. They have not verified any assembly code nor
connected their verified C-like source programs to any certified
compiler so there is no end-to-end theorem about the entire
kernel. They have not proved any progress properties so even their
verified kernel modules or interrupt handlers could still diverge.


\ignore{
\citet{lili16} presented the first program logic (Lili) that can apply
contextual refinement techniques to prove both linearizability and a
progress property for various concurrent objects (including ticket
locks and MCS locks). Their assertion language does not allow
assertions on event traces, so temporal invariants must be described
using special predicates (called {\em definite actions}).  Our new
\CTOS\ framework, on the other hand, directly reasons about the
environment contexts so temporal properties can be expressed uniformly
as other invariants. The Lili language also only supports the
high-level parallel composition construct, so it is unclear how their
logic can be used to verify the yield/sleep/wakeup primitives in \mCTOS.
}


\ignore{

Bevier~\cite{bevier89} developed a full correctness proof
for a highly idealized kernel in an automated theorem prover. The
Verisoft team~\cite{verisoft07} has done a large body of work aiming
to verify OS kernels and
hypervisors~\cite{leinenbach09,alkassar10}. The Verve
project~\cite{hawblitzel10} managed to prove the type safety of an
entire kernel by combining the partial correctness proof of a nucleus
and the type-safety guarantee from a certifying C\# compiler (for the
rest of the kernel); by using powerful automated proving tools (\eg,
Boogie and Z3), Verve managed to certify the nucleus in 9
person-months.


Hawblitzel~{\em et al}~\cite{ironclad14} has recently developed a set
of new tools based on the Dafny verifier~\cite{dafny10} and Z3 SMT
solver~\cite{moura08}, and applied them to build their Ironclad system
which includes a verified kernel (based on Verve~\cite{hawblitzel10}),
verified drivers, verified system and crypto libraries, and several
applications.  This is another impressive effort that advances the
frontier of system software verification. Ironclad, however, only
proves the partial correctness property (at the assembly level), which
is weaker than the total correctness properties proved by seL4 and
\CTOS. All properties proved by Ironclad are not ``contextual'' so it
is unclear how properties proved on Ironclad apps would still hold
when new extensions are added into their system. Ironclad also differs
from seL4 and \CTOS\ in that its proofs are all done by an SMT solver
which does not produce any machine-checkable proof objects.


\citet{filipovic10} showed that proving linearizability for
concurrent objects is precisely equivalent to proving
termination-insensitive contextual refinement for a simple
object-based concurrent language. \citet{liang13,li} extended this
result


\paragraph*{Comparison with seL4}
As mentioned in Section~\ref{sec:intro}, the seL4 team only proved the
{\em refinement} property but not the {\em contextual refinement}
property, so the global properties (\eg,
security~\cite{murray13,sewell11}) proved at the abstract
specification level cannot be transferred to the C-implementation
level.\david{they do transfer security to the C level; need to reword this} 
The root cause of this problem is their rather simplistic
C-level state machine which they used to verify their 7500 lines of C
code. This machine is too high level to model
several key OS features (e.g, kernel initialization,
context switches, address translation, and page-fault
handling). Indeed, these features happen to coincide with the
unverified C and assembly code in their kernel.

Sewell {\em et al.}~\cite{sewell13} used translation validation to
build a refinement proof between the semantics of the verified C
source code and the corresponding binary (compiled by GCC).  This
proof is not as high quality as the rest of the seL4 effort because
it was not done in a proof assistant (thus it has no machine-checkable
proof) and the translation validator itself still has not been
verified.

Even with this work by Sewell {\em et al.}~\cite{sewell13}, the
previously unverified C code (1200 lines) and assembly code (600
lines) in seL4 still remain unverified. These are actually quite {\em
  major} assumptions for a verified kernel because they include the
correctness of context switches, kernel initialization, address
translation, and linking between verified C and assembly; all of which
were considered as major challenge problems by many researchers
working in this
field~\cite{verisoft06,ni07,feng08:vstte,BedrockPLDI11,vaynberg12}.

Using \CTOS, we have successfully tackled all of these challenges:
context switches, kernel initialization, address translation, and page
fault handling are all certified. All kernel components (in C and
assembly) are correctly linked together to form a complete system in
an assembly machine and all our proofs are machine-checkable in Coq.

Much of the implementation complexity of the seL4 kernel lies on its
support of capability-based access control. Capabilities are important
in seL4 as they are used to prevent unwanted interference between
different kernel components. However, they significantly increase the
complexity of the seL4 kernel.  In contrast, the \CTOS-family kernels
we have built so far rely on the CompCert memory
model~\cite{leroy12} to enforce isolation and prove contextual
refinement.



Vaynberg {\em et al.}~\cite{vaynberg12} also advocated a layered approach
and used it to verify a small virtual memory manager. Their layers
are not linearly ordered; instead, their seven abstract machines
form a DAG with potential upcalls (i.e., calls from a lower layer to
upper ones). As a result, their initialization function (an upcall)
was much harder to verify. Their refinement proofs between layers are
insensitive to termination, from which they can only prove partial
correctness but not the strong contextual refinement property which we
prove in our current work.

}


%\section{Conclusion}
\label{sec:concl}

We have presented a novel extensible architecture for building
certified concurrent OS kernels that have not only an efficient
assembly implementation but also machine-checkable contextual
correctness proofs.  OS kernels developed using our layered
methodology also come with a clean, rigorous, and layered
specification of all kernel components.  We show that building
certified concurrent kernels is not only feasible but also
quite practical. 
%can also be done quite economically.
\ignore{
Traditional OS kernels use a hardware-enforced ``red line'' to isolate
the behaviors of user programs and to protect the integrity of the
kernel code.
}
Our layered approach to certified concurrent kernels replaces the
hardware-enforced ``red line'' with a large number of abstraction
layers enforced via formal specification and proofs. We believe this
will open up a whole new dimension of research efforts toward building
truly reliable, secure, and extensible system software.  

\ignore{
While we initially pursued layered decomposition to reduce the cost of
verification, we later found that such decomposition is also critical
for managing the complexity of specification.  Decomposing a kernel
into many smaller, well-specified components also makes it possible to
aggressively apply proof automation.  Indeed, the majority of our
development effort was spent on layer definitions; once all the layers
are in place, the verification of source-level code and the layer
refinement proofs were done quickly by using shared Coq tactic
libraries.
}

\ignore{
}

%\vspace*{-5pt}
\section*{Acknowledgments}
We would like to acknowledge the contribution of many former and
current team members on various CertiKOS-related projects at Yale,
especially Jérémie Koenig, Tahina Ramananandro, Shu-Chun Weng, Liang
Gu, Mengqi Liu, Quentin Carbonneaux, Jan Hoffmann, Hernán Vanzetto,
Bryan Ford, Haozhong Zhang, Yu Guo, and Joshua Lockerman. We also want
to thank our shepherd Gernot Heiser and anonymous referees for helpful
feedbacks that improved this paper significantly.  This research is
based on work supported in part by NSF grants 1065451, 1521523, and
1319671 and DARPA grants FA8750-12-2-0293, FA8750-16-2-0274, and
FA8750-15-C-0082.  Hao Chen's work is also supported in part by China
Scholarship Council. The U.S. Government is authorized to reproduce
and distribute reprints for Governmental purposes notwithstanding any
copyright notation thereon. The views and conclusions contained herein
are those of the authors and should not be interpreted as necessarily
representing the official policies or endorsements, either expressed
or implied, of DARPA or the U.S. Government.




\chapter{Case Study: Multi-registred Paxos}
\label{chapter:wormspace}

%\section{Introduction}
\label{sec:intro}

%%% Outline
%% structure 
%% 1. concurrent verification is done in several works 
%% 2. how about showing the non-deterministic full machine model refines  ... 
%% 3. For example CCAL provide a useful tool for building concurrent abstraction layer 
%% 3-1. building layers is feasible 
%% 3-2. However proving the refinement between concurrent machine model and the per-instance machine model 
%%
%% 3-3. Based on the CCAL, we show how we build the linking for them 
%% 3-3-1. Multicore Linking 
%% 3-3-1-1. t provides the universal abstract semantics for multicore non-deterministic machine (with sequential consistency)
%% 3-3-1-2. it provides detailed refinement between those abstract functions 
%% 3-3-1-3. it provides the concrete instance of those proofs by connecting them with the lowest layer of CompCertX layer 
%% 3-3-2. Multithreaded Linking 
%% 3-3-3-1. It provides the CompCert Assembly machine models for CompCertX to build per-thread machine models 
%% 3-3-3-2. it provides the refinement between those machine models (parameterized by any kinds of Layers with the guarantee about the certain properties) 
%%                 - that allows us to allocate the proper dynamic initial state for each thread / invariant preserving in the initial state / using the same compiler with 
%%                    CompCertX                    
%% 3-3-3-3. it provides the actual proofs using the example in the certified layers (the language and the proofs are parameterized by the concrete layer definition)
%%                  - shows the identity of the private state change while  sleep and yield 
%%                  - mutual exclusion of user memory regions 
%%                  - mutual exclusion of other private states  






%
%Dependencies due to shared data
%•
%Subtle effects of synchronizations
%•
%Often manually parallelized
%–
%Difficult to debug
%•
%too many 
%interleavings
%of threads
%•
%hard to reproduce bugs
%
%
%
%

%%% concurrent program verification is necessary 
The prevalence of shared-memory multicore machine 
brings the eminent changes in the  software. 
With the machine, achieving higher performance on a single computer than before 
becomes possible, 
but it requires us to facilitate 
concurrency, running multiple threads on multiple cores.
Concurrency, however, 
brings the whole new challenges in terms of software correctness. 
They are well known 
to be difficult to get right and to debug because 
of their intrinsic characteristic, numerous number (usually unbounded) of interleavings among multiple components of the system. 
Testing is also not a promising way to provide the high-assurance of those programs. 
Due to a plethora of possible interleavings, 
reproducing a bug is unfeasible unless testers knows the 
precise interleaving order of them. 
In this sense, 
Building reliable concurrent programs 
needs verification of them, which formally shows that those programs correct reflects the 
desirable behavior (\textit{i.e.,} are stated in their specifications) 
without missing any single interleaving cases. 

%%% Composition is required
The concurrent program verification requires compositional reasoning in its essence,
since it provides an isolation of each instance of concurrent program
(on a single core or a single thread) separately  
 in its verification
without directly considering complex interleaving 
with other components in the system. 
This feature is crucial in some sorts 
of concurrent programs such as 
operating systems, libraries, or application interfaces
because the
proof of them 
are usually need to be parameterized by 
other programs running on them. 
In those cases, composition and proof isolation 
give  an enough power 
to state and prove the correctness property 
of those programs upon any arbitrary context programs run with the targeted programs. 

%%%% several previous works and machine checkable proof  

In this sense, 
multiple previous works handle compositional reasoning about concurrent programs.
There are two traditional different approaches,
rely-guarantee~\jieung{cite rely guarantee} and separation logic~\jieung{CSL cite separation logic  - need to refer View for citation},
and many other approaches that stem from either or both of them
\jieung{SAGL (2007) / Bornat-at (2005) RGSep (2007) Gotsman-al (2007) RSL (2013) Deny Guarantee (2009) LRG (2009) RGSim (2012) Liang-Feng (2013) 
Lili (2016) / Iris (2015) Iris 2.0 (2016) FCSL (2014) (SCSL (2013) FTCSL (2015) CoLoSL (2015) CAP (2010)   View paper / CCAL paper / CSpec (MIT)
- Please refer the specification of POSIX File Systems slide}.
In addition, some of them are not only focusing on the functional correctness but also 
shows liveness~\jieung{LiLi}. 
Some, CSpec and CCAL, also provides a verified layered structure to build modular verification, an another important 
feature to build a large scaled program verification in a modular ways.


%%%% several previous works and machine checkable proof  
Bsed on them, few works \jieung{verifying concurrent software using movrs in CSPEC / preemtive kernel verification (Xinyu Feng - CAV), CertiKOS, MCSLock CCAL} 
organizes machine checked proofs 
about concurrent execution. 
Among them, both CSPEC and CertiKOS facilitates layered structures 
for scalable and modular verification and formally connect top level operations into bottom-layer operations.

%%%% CCAL - what is missing 
They, however, overlook the difficulty in one another piece of machine checked concurrent program verification, 
provide the evidence of concurrent linking.
The concurrent linking shows 
the precise evidence of the composition that the underlying logics provide. 
In this sense, 
it requires the definition of 
concurrent machine model that can run multiple instances of concurrent program together (\textit{e.g.,} multicore and multithreaded machine) 
as well as 
the linking proofs between the program runs on top of concurrent machine and the composition of multiple single instances together. 
It also requires the proof that 
shows the single instance of the concurrent program correctly reflects
the program run on the multicore machine model. 

They are necessary to show the full correctness of the program, 
but providing concurrent machine model is bothersome, especially when the model is close to that of bare machines, 
and the proof between it wiith the machine that runs the single instance is also a subtle work.
To handle those challenges,
CCAL slightly mentioned these issues,
but it only carries out
a key idea of
linking without exposing underlying multiple obstacles.  
In this sense, 
providing the information about which steps are necessary for concurrent linking and what kind of things that 
the users have to fill out is desired.
In this sense, the idea in the paper is far from 
the enough idea to achieve how 
concurrent linking can be worked in such 
a large scaled concurrent program. 

\jieung{need to add sentence about CompCertX}


%%%% The contribution of this paper

Therefore, our paper aim to deliver all necessary 
and important ides for concurrent linking,
which includes modeling the generic concurrent machine model, 
necessary information to prove refinements between them, 
and how to connect those concurrent linking with the 
proof layers of concurrent programs in a generic way. 
It is definitely not able to be achieved in a single shot.
We introduce multiple intermediate languages and 
context that users has a responsibility to 
connect the generic concurrent linking proof with 
their one verified programs.
We, in this paper, handle all of them in detail. 
In short, he key contribution of this paper is as follows: 

\begin{itemize}
\item We formally define non-deterministic multicore semantics and multiple intermediate languages that are independent from specific machines (such as x86 or ARM). 
\item We provide the refinement proofs between them that can be used for \compcertkwd-style backward simulation. 
\item We connect those intermediate languages and proofs with the CPU local CCAL layer, that uses \compcertkwd-like sequential x86 assembly model with 
environment context.
\item We provide multithreaded machine model with minimal assumptions about a certain CPU local CCA layer, which implies that the machine model does not stick to the specific layer definition.
\item We provide intermediate languages to introduce per thread machines and refinement proofs among them. 
\item We connect those intermediate languages and refinement proofs with the specific layer definition in CertiKOS, which fully link the layer on per-thread machine with the layer on per-CPU machine.
\end{itemize}

The structure of remaining paper is as follows:
Section~\ref{sec:overview} shows a brief high level idea of CCAL as well as how our linking works. Section~\ref{sec:multicore} shows the details of multicore linking,
and Sect.~\ref{sec:multithreaded} shows the implementation of our intermediate machine models for our multithreaded environment.
Section~\ref{sec:multithreaded-linking-impl} shows how are framework 
can be fitted into the actual concurrent kernel implementations.
Evaluations about our implementation can be found in Sect.~\ref{sec:evaluation} 
and the related work and conclusion is in Sect.~\ref{sec:related}.


%
%
%\begin{figure}
%\caption{Requirements in Concurrent Program Verification}
%\label{fig:concurrent-verification-challenge}
%\end{figure}
%
%However, even with the importance of concurrent program verification and 
%a large body of recent work on shared-memory concurrency verification ~\jieung{cite},
%there are few certified programming tools for a large scale software due to the requirement of multiple challenges described in Fig.~\ref{fig:concurrent-verification-challenge}.
%
%\jieung{ need to site ESOP papers too}
%
%They first have to 
%provide a way to build the software in multiple layers
%that enable us to build a large scale program as a modular way. 
%For example, 
%operating systems can be divided into multiple parts, 
%memory management, process management, and so on.
%
%They also have to provide \jieung{need different word} a methodology to 
%represent the behavior of other components in the concurrent environment. 
%For the program running on multicore environment, 
%the single instance of the program, which is a program runs on top of 
%a single CPU, has to correctly capture the 
%environmental behavior (the behavior of programs on other CPUs). 
%
%In addition to that, 
%providing the end-to-end theorem also requires us 
%to link the multiple proof instances to 
%form a single proof that is based on
%the concurrent environment itself which does not have 
%any environmental contexts at all. 
%In the example of the operating system on multicore environment,
%the end-to-end theorem 
%has to prove that 
%the program running on the single CPU is correctly refined by 
%the whole thread programs running on the multicore machine. 
%
%Previous works, CertiKOS~\jieung{need cite} and Certified Concurrent Abstraction Layer~\jieung{need cite}, 
%tackles all the above examples.  
%CCAL is a tool to build a certified concurrent layers, which provides 
%a way to build concurrent abstraction layers, 
%
%
%
%However, the paper does not handle how the linking process works with the concrete machine models. 
%It briefly mentions the high level idea of linking and the memory extension for linking framework. 
%
%Therefore, this paper aims the gap between the high level perspective of CCAL and the 
%low level details of concurrent proof linking. 
%This low level details contains two parts. 
%First, it requires us to define and build multiple intermediate languages to connect
%the x86 multicoro machine model with the LAsm, which is the machine model for one single CPU. 
%In addition to that, 
%the framework also needs to show the refinement 
%between layers on those intermediate machine models to formally link
%all those proofs together. 
%CCAL also briefly provide the idea of how they implement the practical machine models that can be used with CompCertX.
%However, only providing few details does not provide 
%the  useful information to show how it works with the actual running large scale software.
%Thus, our paper tackles the issues that CCAL overlooked in the paper 
%by providing the formal rules and proofs.
%The key contribution of this paper is as follows: 
%
%\begin{itemize}
%\item We provide the detailed intermediate language semantics for multicore machine model based on CCAL, 
%and instantiate all those intermediate language semantics and refinement proofs 
%to link them with CompCertX with environmental context 
%\item We provide the intermediate machine models to build single threaded machine model from a single CPU machine model. 
%Based on the machine models, we provide the linking theorem in between 
%two abstraction layers, which contains different semantics for software schedulers. 
%\end{itemize}
%
%The structure of remaining paper is as follows:
%Section~\ref{sec:overview} shows a brief high level idea of CCAL as well as how our linking works. Section~\ref{sec:multicore} shows the details of multicore linking,
%and Sect.~\ref{sec:multithreaded} shows the implementation of our intermediate machine models for our multithreaded environment.
%Section~\ref{sec:multithreaded-linking-impl} shows how are framework 
%can be fitted into the actual concurrent kernel implementations.
%Evaluations about our implementation can be found in Sect.~\ref{sec:evaluation} 
%and the related work and conclusion is in Sect.~\ref{sec:related}.
%
%
%
%\ignore{
%Despite the importance of concurrent layers and a large body of recent work on 
%shared-memory concurrency verification, 
%
%
%there are no certified programming tools that can specify, compose, and compile concurrent layers to form a whole system [6]. Formal reasoning across multiple concurrent layers is challenging because different layers often exhibit different interleaving semantics and have a different set of observable events. For example, the spinlock module in Fig. 1 assumes a multicore model with an overlapped execution of instruction streams from different CPUs. This model differs significantly from the multithreading model for building high-level synchro- nization libraries: each thread will block instead of spinning if a queuing lock or a CV event is not available; and it must count on other threads to wake it up to ensure liveness.
%
%
%
%
%many of these abstraction layers also become concurrent in nature. Their interfaces not only hide the concrete data representations and algorithmic de- tails, but also create an illusion of atomicity for all of their methods: each method call is viewed as if it completes in a single step, even though its implementation contains com- plex interleavings with operations done by other threads. Herlihy et al. [19, 20] advocated using layers of these atomic objects to construct large-scale concurrent software systems.
%
%
%The importance of software systems' accuracy is growing rapidly these days. 
%In addition to that, 
%the concurrent environment, including multicore and device drivers, are ubiquitous in modern periods. 
%Therefore, 
%the verification methodology for concurrent programs is critical now. 
%
%In this sense, several previous works propose
%proof logics and tools for that purpose \jieung{need cite}.
%
%However, few of them are working on the linking multiple instances of 
%verified concurrent programs with concrete machine models that can be run 
%on the bare machines. 
%
%One tool, Certified Concurrent Abstraction Layers, 
%provides the tool that can be used for building a practical concurrent programs 
%such as a small operating system or distributed system. 
%It also provides the tool to link the 
%}


%Distributed systems are notoriously complex due to the many possible interleavings of their coarsely-connected instances as well as the possibility of errors in both  those instances and the network environment. For these reasons, verification of distributed systems is desirable to remove the possibility of bugs and guarantee their safety and correctness. However, much current verification work still requires a great deal of effort and sometimes has limitations.

We present a verification approach that uses \textit{write-witness-passing}, which is simple but novel in distributed system verification. It is a scalable, reusable, and extensible approach that can be directly linked with the low-level implementations of distributed protocols through contextual refinement. Write-witness-passing can capture the common behaviors of many distributed protocols, and provides both a simple way of understanding the protocols as well as an easy methodology for verifying them.

To demonstrate how write-witnesses work, we verify the functional correctness and safety of Paxos, one of the most famous consensus protocols. We implement the key routines of Paxos in C, and use Coq to verify both the functional correctness of the implementation as well as the safety properties of the protocol within less than 4 person-months. We also describe how we can apply our approach to other distributed protocols to illustrate its generality.

%\section{Introduction}
\label{sec:intro}


%Distributed systems are based on the consensus protocols to guarantee consistency among the nodes.
%However, consensus protocols are notorious due to their complexity originated by their complex interleaving.
%Due to that, proving basic safety proofs for consensus protocols is known as a challenging problem. 
%Recent works, \needcite\jieung{verdi, DISEL, ironfleet, PLDI18, oopsla 2017, CPP for verdi, ESOP} propose
%the works to provide robust verification on consensus protocols. 
%However, some of them fail to connect the safety proof of consensus protocols with the low-level implementations,
%others are not able to allow re-configuration of the system, which is usually necessary in the real deploy of the system, 
%hard to reuse safety proofs for the system that are built based on the protocols 

% \jieung{need to compare, others are quite simple to compare... but we need to write a 
%simple comparison with inronfleet}

%Therefore, we propose a way to understand multiple consensus protocols easy and thus makes the safety verification of them simple.

%To demonstrate the approach, we implement Paxos \needcite\jieung{paxos} with a small TCB code for network communications, 

\topic{Distributed systems are difficult to verify.}
Distributed systems form the underlying base of many applications these days.
Unfortunately, verifying these systems is difficult because of the inherent concurrency and the possibility of failure in both the nodes and the network.
For distributed nodes to collaborate and overcome these failures seamlessly,
distributed systems must employ sophisticated protocols.
\topic{Existing non-machine checkable proofs are less useful in practice.}
Although many of these distributed protocols have pencil-and-paper proofs of their correctness,
their subtle and complex nature makes them difficult to implement faithfully in actual code.
Even though the industry rigorously applies various testing strategies 
for software development, there are continuous reports about distributed software bugs that can shut down entire data centers~\cite{awsdown, gmaildown}.

\topic{Interactive theorem provers created opportunities to verify distributed systems.}
Machine-checkable verification tools open up new opportunities to provide distributed systems
with end-to-end correctness guarantees by
verifying low-level implementations of distributed code and linking them with safety proofs of the abstract protocols.
Specifically, theorem provers allow line-by-line verification of the code
with support for partial proof-automation.
Previous works have used
both automated tools such as Z3~\cite{moura08} and
Dafny~\cite{dafny} as well as
interactive tools, such as Coq~\cite{coq}.
\ignore{
The most common tools are the ones based on SMT solvers, such as Z3\cite{moura08} and 
Dafny\cite{dafny}, which are well automated but works for only first-order logic and decidable problems, 
and Coq\cite{coq}, which is less automated but works on high-order logic.}

\topic{But distributed system verification is still difficult.}
Regardless of which tool one uses,
verifying distributed system code with an interactive theorem prover requires much more work
than with a hand-written proof.
The verification must cover every low-level corner case that is mostly related to the underlying network error 
or optimization.
In addition, the collective view
of all distributed nodes and the network -- which we call the \globalstate{} -- 
should be created and made available to the proof because key safety properties of distributed protocol 
always related to not with a single node but with all nodes in the system. 
Thus, even given a high-level proof of a theorem for a distributed protocol, 
there is still a significant proof burden to bridge the gap between the code and the model.

\topic{Reasons why others have failed: verdi, ironfleet, sergey et al., disel, etc.}
For this reason, proofs of distributed protocols done using interactive theorem provers often simplify the \globalstate{}
and rely on additional tools or assumptions to fill in the missing pieces.
For example, in Ironfleet's~\cite{ironfleet} model, all distributed nodes are connected via an asynchronous network,
and they show that the code refines the \globalstate{}.
However, in part due to the expressiveness limitations of Dafny, Ironfleet's verification relies on pencil-and-paper proofs to show
that a realistic, arbitrarily-interleaved network refines their more restrictive \globalstate{}.
Verdi~\cite{verdi} similarly models a global state but assumes an ``idealized'' network and carries out refinement proofs to show that the code refines the \globalstate{}. 
It then relaxes its assumptions by automatically applying valid transformations to the code, such as adding sequence numbers to tolerate packet duplication.
However, starting with an assumption of an ideal network is not suitable for verifying systems such as Paxos~\cite{paxos} that assume a faulty network.
% which the same authors showed the safety proof later in a separate paper~\cite{cppraft}. % TODO: fit this reference in another way
Several other papers suggest methodologies to verify distributed systems using interactive theorem provers,
but they focus on specific topics such as automation~\cite{modular} or isolating protocols~\cite{disel} and do not fully address this problem.

\topic{We propose a new and easy-to-understand distributed system proving techniques: new global model + witness-passing.}
To fill in this gap, we propose a \globalstate{} of a distributed system where the proof can be written entirely in Coq,
and a novel write-witness-passing scheme that can promote the understanding of distributed system protocols and simplify the verification process.
Our \globalstate{} includes an asynchronous network, and the states in the \globalstate{} are constructed by composing the operations and local state of each node.
Write-witness-passing adds a logical data structure to the messages sent between nodes that remembers what each node has seen so far.
The contents of this data structure could come from the sending node's own state or from observations of other nodes' state.
As this information accumulates it can show how each node reached its current state, with evidence demonstrating the validity of each transition.
Using these witnesses, the behavior of a node can easily be verified with respect to the \globalstate{} through primarily local reasoning.

\topic{Rationale behind our approach.}
To understand why the write-witness-passing scheme is useful, one should first understand how distributed systems are designed.
Distributed systems have evolved to hide complex protocols using simple abstractions and to send as little information as possible between nodes to save network bandwidth.
Messages sent over the network may be lost depending on network assumptions and information received over the network is often discarded right after it is used to update any relevant local states.
The code for each node often blindly executes an operation without the global view of the entire distributed system and assumes that all other nodes are working correctly.
Therefore, the context that can be extracted from the code for a local node is typically not enough to reason about the validity of the node's state against the entire distributed system.

\topic{Our design: 1) Network model: we don't make any extra assumptions}
Our \globalstate{}'s network is asynchronous and allows that packets can be dropped, delayed, reordered, interleaved, and duplicated, but never corrupted,
which most other works on distributed system verification also assume.
Such a realistic network model is necessary because verification based on a weaker model will be invalid in an actual deployment.
Depending on the need, our network model can sometimes be refined to a more restrictive model, but, because of the simulation relation,
properties proved using this model are also guaranteed to hold for the realistic one.
Existing work sometimes assumes that nodes operate atomically between send and receive (for clients) or receive and send (for servers)~\cite{verdi},
but our model does not have any such additional assumptions.

\topic{Our design: 2) Our global model.}
State in our \globalstate{} is a collection of all local states that are affected by the network.
To prove correctness of the whole system, we must show that the interactions among the distributed nodes are correct.
Our model of the global state is not very different from other work, but the write-witness-passing scheme takes advantage of the state in an unique way.

\topic{Our design: 3) Why witness-passing: how it works easier to reason and prove.}
While typical proofs of distributed systems involve showing that a local node's behavior refines the \globalstate{}~\cite{verdi, ironfleet},
write-witness-passing works in the opposite direction;
we start from a \globalstate{} and bring necessary global state information into the local state.
The imported state is what the other nodes have witnessed at the time of sending messages.
This information provides the context that is necessary for a local node to reason about its correctness within the entire distributed system,
but was not required to simply execute the distributed protocol.
The imported global state constitutes a witness-tree that keeps track of the path and evidence for how a node's current state was reached.
This structure can be used for checking invariants and carrying out inductive proofs of the protocol.
The witness-tree has partial information about the entire system state, but only the parts that are relevant to the node currently holding the tree.
Thus, the verification can take place within the local context of a node without having to worry about other complex states in the rest of the distributed system.
Because the information in the witness-tree was taken from the global state, an invariant that the data in the tree corresponds to something in the \globalstate{} naturally holds.
In addition to easier verification, a proof based on write-witness-passing provides insight into why the protocol works,
because the verification takes place in a local context that more closely mirrors the implementation.

We use Paxos as an example to demonstrate how our \globalstate{} and write-witness-passing scheme can facilitate the verification of distributed systems.
Paxos is a good example to explore the power of write-witness-passing because it requires communication with at least a majority of acceptors,
and the weak network assumptions require reasoning about failure cases.
Therefore, being able to handle a weak network model and having a clear sight on the global view of the system are necessary to verify the system.
The Paxos consensus protocol is also notoriously difficult to understand just by observing the information that is passed around.
Write-witness-passing can gather necessary global state into a tree to provide a clearer insight into how the protocol works.

\begin{figure}
\includegraphics[scale=.70]{figs/overall_structure}
\caption{Overall Structure of Distributed System Verification with Write-Witness-Passing. 
All important components are explained in the later sections; (1) in Sect.~\ref{subsec:network-primitives} and 
Sect.~\ref{subsec:low-level-network-syntax-and-semantics}; (2) in Sect.~\ref{subsec:network-primitives} and Sect.~\ref{subsec:functional-correctness};
(3) in Sect.~\ref{subsec:distributed-transition-semantics} and Sect.~\ref{subsec:witness-passing-semantics-in-paxos}
(4) in Sect.~\ref{subsec:witness-write}, Sect.~\ref{subsec:distributed-transition-semantics-with-witness-passing}, and Sect.~\ref{subsec:witness-passing-semantics-in-paxos};
(5) in Sect.~\ref{subsec:paxos-safety} and Sect.~\ref{subsec:extensibility-of-verified-paxos}; and 
(6) in Sect.~\ref{subsec:extensibility-of-verified-paxos}}
\label{fig:overall-structure}
\end{figure}

To realize the write-witness-passing scheme, we use certified concurrent abstraction layers (CCAL)~\cite{concurrency} as the base verification framework
and build necessary components such as the \globalstate{}, which includes the network model and node states, and the state transition framework within the \globalstate{}.
The contextual refinement scheme that is proposed by the CCAL approach adds more benefits to our distributed system verification framework,
such as enabling vertical and horizontal composition of verified protocols.
Figure~\ref{fig:overall-structure} shows the overall structure of 
distributed system verification with write-witness-passing, and the important components are numbered in the figure.
In a later section, we describe those components one by one, and
we show how our framework and write-witness-passing can be used to prove Paxos leader election and reconfiguration, as well as other distributed protocols.


This paper makes the following contributions:
\begin{itemize}
	\item We propose a general distributed system verification approach with the capability to link Coq-verified specifications with executable C code without relying on any external verification tools.
	\item We propose a novel write-witness-passing scheme that facilitates local reasoning about distributed systems and provides insight into how distributed system protocols work.
	\item We present a complete safety proof of Paxos using our verification framework and sketch how our framework can be used to verify other distributed systems.
\end{itemize}


The remaining parts of this paper are organized as follows.
Section~\ref{sec:overview} is an overview of our verification approach using Paxos.
Section~\ref{sec:witness-passing-semantics} describes in detail the formal definitions of our verification approach and of write-witnesses.
Sections~\ref{sec:paxos-verification}-\ref{sec:evaluation} provide examples of
our verification approach applied to Paxos and variants of Paxos, and evaluate
our proof methodology. 
Section~\ref{sec:related} investigates related work and gives our conclusions.

\topic{Benefit of CCAL: 1) network model is flexible, 2) vertical and horizontal composition.}

\topic{Contributions: 
1) provide simple way to represent consensus protocols as well as for safety proofs of those protocols, 
2) provide the way to link those representations with low-level implementations (scalable, reusable way),
3) verify Paxos using the approach with small human efforts.}

%\jiyong{We need to know what are the typical ways that others model the global state to verify distributed systems.}
%\jieung{ 
%\begin{enumerate}
%\item Ironfleet: takes too much human effort. Some parts are treated as assumptions (network reduction is not able to be verified)
%\item ESOP18, Verdi: made atomic handler and collect those atomic handlers to define global transition systems - connecting local behaviours with global transition systems. for the global properties, they need to reason about the state transitions for all the packets due to their representation.  Human efforts is high
%\item DISEL: horizontal composition. They argue that they can reuse their verification when it combined with other protocols. But, they cannot support vertical composition, which seems that they only can verify multiple distributed systems together if they are clearly divided. 
%\item OOSPLA17, PLDI18 (automations): OOSPLA17 cannot generate the the runnable code. they cannot be linked with executable code, which is desired - they works with Paxos variants well, but have questions about other distributed systems. For PLDI18, they argue that they verified Raft and MultiPaxos, but for the MultiPaxos, they are unclear that what they have proved. they are not able to verify network reduction like Ironfleet, and they are not able to support concurrency yet (but.. I need to recompare PLDI18 paper again. PLDI18 paper also assume the all synthesis as a TCB
%\end{enumerate}
%}


%%%%%%%%%%%%%%%%%%%%%%%%%%%%%%%%%%%%%%%%%%%%%%%%%%%%%%%%%%%%%%%%%%%%%%
% some brain storming
%%%%%%%%%%%%%%%%%%%%%%%%%%%%%%%%%%%%%%%%%%%%%%%%%%%%%%%%%%%%%%%%%%%%%%
%Distributed protocol is important but complex:\\
%Several works verify distributed protocols either using automated tools or using interactive theorem provers\\
%people knows that interactive theorem provers are subtle\\
%But, it has benefits:\\
%Not even with its expressiveness, we can connect protocol layer verification with the low-level implementation \\
%Also, re-usability can be achieved using contextual refinement - this one cannot be achieved by automated system (contextual refinement)
%
%And, Paxos, immutability, is actually unbounded. 
%In this sense, it is a little bit hard for automated approaches to prove that one in general (need to check PLDI18 and OOPSLA 17 papers)
%
%We want to claim that the complexity is not because of using interactive theorem prover, 
%
%We propose  a way that dramatically simplify the proof of distributed system 
%
%1. witness passing \\
%2. global transition system \\
%3. simple enough to understand the distributed system in a few minute \\
%4.compositionality \\
%5. and other benefits (link the proof with low-level implementation )
%

 


% ADD CONTENTS
%\section{Overview}
\label{sec:overview}

Distributed systems often require a collection of nodes to reach consensus on some value.
Thus, to show the effectiveness of our approach,
we use the Paxos algorithm~\cite{paxos} as a simple but useful example.
After first briefly explaining Paxos, we illustrate how we can prove the functional correctness of the low-level implementation.
Finally, we show how write-witness-passing captures the essential information of Paxos as well as
reducing the complexity of the verification.

\subsection{Example: Paxos} 
\label{subsec:paxos} 

\begin{figure}
\begin{minipage}{\linewidth}
\noindent
\begin{multicols}{2}
\lstinputlisting[numbers = left, language=TeX]{source_code/multipaxos/paxos_spec.c}
\end{multicols}
\end{minipage}
\caption{Paxos: Informal Description}
\label{fig:paxos-pseudocode}
\end{figure}

The Paxos algorithm is one of the most popular asynchronous consensus algorithms, and was even almost treated as a synonym of
consensus for decades.
Although an informal description of Paxos can be expressed in a single page,
implementations of the algorithm often exceed thousands of lines of code due to its underlying complexity.
Moreover, this hidden complexity is made evident by the difficulty that many people have in trying to understand the algorithm.
Several works~\cite{raft, rvrpaxos} have noted Paxos' lack of clarity despite multiple attempts to present
it in a more understandable way ~\cite{paxosmadesimple, Lampson1996, Lampson2001, dpaxos}.
Part of the difficulty stems from the fact that all nodes in the distributed system perform their transitions in a local manner in a fault-prone environment.
Their states and behaviors need to be consistent, but
the only way to learn another node's state is through the network, which can fail in multiple ways.
The result is a system that guarantees global properties through local behaviors, but the relation between the two is not made clear by the algorithm.

\begin{figure}
\begin{center}
\includegraphics[scale=.34]{figs/multipaxos/paxos_example_nowitness}
\end{center}
\caption{Paxos: Execution Example}
\label{fig:paxos-example}
\end{figure}

Paxos can be treated as a concurrent state machine consisting of a cluster of two types of nodes: proposers and acceptors.
Proposers operate with arbitrary speed and their role is to propose a value to write in the system.
Although there can be multiple proposers in the system, they are isolated from one another;
thus they do not coordinate to reach consensus.
Acceptors, on the other hand, are responsible for deciding which values suggested by the proposers to write.
The system is said to reach consensus if a majority of the acceptors (here defined to simply be more than half), have chosen the same value.
In this sense, the acceptors are cooperating to reach consensus, but they do not actually communicate with one another.
Instead, each acceptor works only using its local state, and certain invariants of the algorithm guarantee that the states of all acceptors remains consistent.

Figure~\ref{fig:paxos-pseudocode} informally illustrates the key steps of Paxos.
Both proposers and acceptors participate in two phases, which are notated as Phase 1 (Prepare) and Phase 2 (Write) in the figure.
In addition to modifying local state, transitions in both phases can also involve network communications between nodes.
When a proposer wants to write a value, it first asks acceptors to prepare (Phase 1a) with a unique round number ($crndp$).
This number is totally ordered among the whole system, but proposers cannot know which numbers have been proposed by other proposers directly.
When an acceptor receives the message, it will first compare the round number with the highest round number it has seen up to that point.
If the new number is higher, it will respond with its stored value and the round in which it was stored (Phase 1b).
If no such value and round exist then it returns a special null value and the round number that is smaller than any that can be proposed.
If a proposer receives Phase 1b responses from a majority of acceptors it can send a Phase 2a message with a value to write.
If any of the Phase 1b messages contained a non-null value then the proposer must try to write the value associated with the highest round number.
Otherwise, if all of the values are null, the proposer is free to choose any value.
Upon receiving the Phase 2a message, acceptors will again check the round number to ensure that it is greater or equal to the maximum it has seen.
Otherwise they do not update their state and they ignore the request.

%%%%%%%
% fault tolerant 
%%%%%% 

This high-level description is relatively short and seemingly simple.
The complexity, however, arises when all nodes run together concurrently in an asynchronous environment with the possibility of multiple types of failures.
In an asynchronous environment, nodes are loosely connected with others.
There are no bounds on timing, so each clock on each node can run arbitrarily fast or slow.
Additionally, network communication may take a potentially unbounded amount of time,
and nodes themselves may be unpredictably slow in responding to messages.
The possibility of failures then compounds the complexity.
Communication may involve message duplication, loss, and reordering, which means that a simple send-receive pair can have many possible interleavings.
Figure~\ref{fig:paxos-example} (a) shows some of the possible interleavings of messages sent between multiple acceptors and a proposer.
In the example, the proposer (P1) communicates with three acceptors (A1, A2, and A3).
It first proposes the round number 5, and gets an acknowledgement from A1 and A2.
A3 also responds, but its message is duplicated, and one of them is lost.
The other one is delayed and does not arrive until after P1 has already heard from a majority of acceptors, so it is ignored.
Then, in Phase 2, P1 tries to write the value $v$, which will succeed assuming its messages reaches a majority of the acceptors and no
other proposer has proposed a larger round number in the meantime.
Although Paxos does guarantee that it will still work under these conditions, even in this simple example,
the number of cases to consider is large and proving it even informally is not straightforward.

%%%%%%%
% mininum requirement for distributed system verificatiaon
%%%%%% 
In general, implementation and verification of a distributed system must handle the following:
\begin{enumerate}
\item \textbf{Network model}:
Most distributed systems and consensus protocols are based on certain assumptions about the network,
such as packet duplication, loss, and reordering. In order to verify such a system, one must make a model of the network that matches
those assumptions.

\item \textbf{Functional correctness}: 
To reason about correctness at all, there must be a specification of how the system should behave.
A key part of the verification of a system is showing that the implementation correctly simulates this specification.
Distributed systems, including Paxos, can often have a large gap between their implementation and specification due to optimizations and careful handling of failure cases.
This mismatch between code and specification makes proofs of functional correctness especially important.

\item \textbf{Safety of the protocol}: 
Proving that the implementation refines the specification is necessary, but is often insufficient in distributed system verification.
Since distributed systems consist of multiple nodes, which may have different functionalities,
functional correctness of one node does not imply the correctness or safety of the whole system.
Instead, verification of such global properties requires modeling the entire system and
proving that the behaviors of individual nodes ensures that the system runs correctly.
\end{enumerate}

%%%%%%%%%%%%%%%%%%%%%%%%%%%%%%%%%%%%%%%%
% Network Model 
%%%%%%%%%%%%%%%%%%%%%%%%%%%%%%%%%%%%%%%%

%\newcommand\sdiff{\triangle}
%\newcommand\ssame{\odot}

\newcommand{\envcontext}{\varepsilon}
\newcommand{\layerdef}{\mathcal{L}}
\newcommand{\relyrule}{\mathcal{R}}
\newcommand{\guaranteerule}{\mathcal{G}}
\newcommand{\primid}{id}
\newcommand{\primspec}[1]{\sigma_{#1}}
\newcommand{\igchar}{\_}
\newcommand{\sendpkt}[3]{{#1}.\mathrm{SEND}{[#2]}.{#3}}
\newcommand{\recvpkt}[3]{{#1}.\mathrm{RECV}{[#2]}.{#3}}

\subsection{Network Primitives}
\label{subsec:network-primitives}

In distributed systems, the network acts as a shared communication channel among the participating nodes.
Nodes in the system typically run independently and concurrently, and can only guarantee the order of their own evaluations.
To represent this, several previous approaches~\cite{verdi, disel}
define the network as a collection of events corresponding to send and receive transitions.
They then build local transitions on top of these network primitives that can update local state as well as the network,
and finally they compose multiple nodes and their transitions together to represent the global state machine.
Instead of doing that, our approach lifts the composition down to the bottom by using methodologies
inspired by the concurrent linking framework~\cite{concurrency} and game semantics~\cite{gsinvite},
which treats each node as a participant in a game with communication via the network.

A key idea of the approach is to keep a snapshot of the global shared state from the view of a single node,
and use an environment context (\textit{i.e.} $\envcontext_i$ when focusing on node $i$) to model the
other nodes' behaviors.
The environment context is a collection of all of the possible behaviors of the other nodes and
can be defined as a set of functions from a local log of network events to another log representing the next steps taken by the environment.
At certain synchronization points the node under consideration will query the context to learn what steps
the environment has taken and decide what to do next based on the result.
By parametrizing proofs over all valid environment contexts we can guarantee that every node will behave correctly
even in a non-deterministic environment.
As a concrete example, consider the following possible linearized trace of Fig.~\ref{fig:paxos-example}.
\begin{center}
\begin{tabular}{c}
$[\sendpkt{P1}{5, non}{A1}] \ssame [\sendpkt{P1}{5, non}{A2}]  \ssame [\recvpkt{A1}{5, non}{P1}]$\\
$\ssame [\sendpkt{P1}{5, non}{A3}]  \ssame [\recvpkt{A2}{5, non}{P1}]    \ssame [\sendpkt{A2}{5, 0, \igchar}{P1}]$\\
$ \ssame \cdots \ssame [\sendpkt{P1}{5, v}{A1}] \ssame 
[\recvpkt{A1}{5, v}{P1}] \ssame \ssame [\recvpkt{A1}{5, v}{P2}]$\\
$ \ssame \cdots \ssame [\recvpkt{P1}{5, 3, v'}{A3}] $\\ 
\end{tabular}
\end{center}
where [$\sendpkt{i}{msg}{j}$] means that $i$ sends the message ($msg$) to $j$,
[$\recvpkt{i}{msg}{j}$] means that $i$ receives the message ($msg$) from $j$,
and $\ssame$ is the concatenation operator.
% \wolf{TODO: not sure what this next sentence means}
% With the single linearized instance among all possible executions,
% constructing other nodes' behavior when the current status is given is possible.
After each $[\sendpkt{P1}{\any, \any}{A}]$ event P1 queries its environment context ($\envcontext_{P1}$) to learn what the acceptors have done.
In this case, $\envcontext_{P1}$ is defined as
\begin{center}
\begin{tabular}{c}
$\envcontext_{P1} := \set{\cdots, ([\sendpkt{P1}{5, non}{A1}] \ssame [\sendpkt{P1}{5, non}{A2}]   \rightarrow [\recvpkt{A1}{5, non}{P1}]), \cdots}$\\
\end{tabular}
\end{center}
This means that just before P1 sends its message to A3, it first
queries $\envcontext_{P1}$ and passes it its local view of the network,
$[\sendpkt{P1}{5, non}{A1}] \ssame [\sendpkt{P1}{5, non}{A2}]$. 
Then the environment context returns $[\recvpkt{A1}{5, non}{P1}]$ and P1 appends this event to its log before continuing
with sending $[\sendpkt{P1}{5, non}{A3}]$.
The behavior of receive is defined similarly.
Continuing in this manner, P1 builds up a trace of events that can be replayed to reconstruct its
state at any point.

\ignore{\wolf{Is this paragraph saying that we can define the environment context so the last event is always one that affects the current node's state?}}
Similarly,
it is possible to build the environment context for node $i$ ($\envcontext_i$)
from arbitrary send and receive patterns.
The network primitives will
add multiple events to the network log, which will always end with an event generated by the node $i$.
Send and receive primitives then consist of two
adjacent transitions.
The first one only updates node $i$'s view of the other nodes based on what the environment context returns
in response to the current network state.
The other is a local transition that adds a single packet associated with node $i$.

The benefits of following this approach are as follows:
\begin{itemize}
 \item We hide the non-determinism of the network as soon as possible by introducing send and receive primitives that use the environment context.
   Because the environment context is a collection of deterministic functions, it acts as an oracle that knows in advance what the network will do.
   But, by parametrizing over all valid environments, the proofs are still not tied to any particular pattern of events.
 \item The global network history snapshot is recorded in the network log. This implies that the local view from each node
contains all the relevant information on the network, including packets that are not associated with the current node.
\end{itemize}
We present the formal rules for network primitives in Section~\ref{subsec:low-level-network-syntax-and-semantics}.

\subsection{Link with Low-level Code Verification}
\label{subsec:link-with-low-level-code-verification}

Implementations of distributed systems are sometimes quite different in structure from their formal specifications.
One cause of this is because they typically must handle many corner cases, such as network timeouts or ignoring invalid messages.
They also often add optimizations that are not present in the specification.
For these reasons, proving that the implementation refines the specification is desirable.
To make the proofs manageable, we follow a strategy employed by many systems that consist of multiple abstraction layers
(e.g., OS kernels, hypervisors, device drivers, network protocols).
Each layer defines a state machine that implements a particular set of functionalities.
The layer then provides an abstract interface that hides the implementation details.
This way, client programs can be built on top of a state machine without concerning themselves with the actual implementation.
Previous work~\cite{concurrency} has proposed a way of building abstraction layers with the
capability to handle concurrent programs, and we extend this approach to work for distributed systems as well.

\ignore{
A concurrent certified abstraction layer (CCAL) is a predicate $L'[i] \vdash_R  M : L[i]$
plus a mechanized proof object
that indicates that the layer implementation $M$, built on top of the interface $L'[i]$, rigorously implements
the layer $L[i]$ with the two layers related by $R$.}

\ignore{
\wolf{it seems like we are defining concurrent certified abstraction layers twice} }
The definition of a concurrent certified abstraction layer over the node identifier $i$
is a tuple with four elements, notated as $L[i] := (\layerdef, \envcontext_i, \relyrule, \guaranteerule)$.
The first component $\layerdef$ contains a partial map from
identifiers to transition specifications
($\layerdef := \set{\primid \mapsto \primspec{\primid}}$ where $ \primspec{\primid}$ is
the specification of the primitive $\primid$).
The state of the layer can be interpreted as a pair
of the private abstract state for the node $i$ ($lst_i$), and
a log of events ($l$) that represents the network ($st_i := (lst_i, l)$).
The local state consists of multiple machine-dependent concrete definitions such as register and memory values,
as well as abstract objects that correspond to
certain regions of memory through some relation.
For specifications of primitives that only touch local state, the transition is straightforward.
Specifications of primitives that contain network transition primitives, on the other hand,
must use the environment context for the node $i$ ($\envcontext_i$)
to model the behavior of other nodes (discussed in Sect.~\ref{subsec:network-primitives}).
For example, the specification of a function that broadcasts a message from a proposer to a set of acceptors must query the environment context
between each send to learn how the environment has changed.
The other two components of the layer definition, $\relyrule$ and $\guaranteerule$,
provide invariants about the network and the distributed system
following the approach of previous work on rely/guarantee systems~\cite{RGSim, LRG}.
The invariants in $\relyrule$ and $\guaranteerule$ are complementary to each other.
Each layer must contain evidence that all of the local transitions satisfy the conditions in $\guaranteerule$.
Conversely, we can restrict the behavior of other nodes by using assuming the conditions in $\relyrule$ hold.
One example of a rely/guarantee rule in Paxos concerns the relation between the round number and value in a Phase 1b message.
It is true that the value is $\bot$ if and only if the round number is $\bot$.
To prove that this invariant always holds it is enough to show that
for layer $L'[i]$, every transition of node $i$ will satisfy it.
Having satisfied the guarantee, we can rely on the fact that the invariant will hold for all other nodes in the system.

\begin{figure}
\begin{center}
\includegraphics[scale=0.47]{figs/multipaxos/network_reduction.pdf}
\end{center}
\caption{Network Reduction}
\label{fig:network-reduction}
\end{figure}

CCAL also provides a way to build a layer on top of another using a program module $M$, which consists of code written in C or assembly.
The predicate $L'[i] \vdash_R  M : L[i]$ indicates that the layer implementation $M$, built on top of the interface $L'[i]$, rigorously implements
the layer $L[i]$ with the two layers related by $R$.
CCAL can compile these C modules using the CompCertX certified compiler~\cite{deepspec, concurrency},
which is a modified version of CompCert~\cite{compcert}.
This, combined with the {\em contextual} correctness property,
lets us define contextual refinement over abstraction layers with the ability to compile layers.
A certified layer converts any {\em safe} client program $P$ running on top of $L'[i]$ into one that has the
same behavior but runs on top of $L[i]$ by compiling the abstract
primitives in $L[i]$ into their implementation in $M$.
If we use ``$\sem{L[i]}{\cdot}$'' to denote the behavior of the layer machine based on
$L[i]$, the correctness property of ``$\ltyp{L'[i]}{R}{M}{L[i]}$'' is written
formally as ``$\forall{}P.\sem{L'[i]}{P\oplus{}M} \refines_R \sem{L[i]}{P}$''
where $\oplus$ denotes a linking operator over programs $P$ and $M$ and 
the relation ($\refines_R$) is formally defined as a forward
simulation~\cite{Lynch95,leroy09,Milner71,Park81} with the (simulation) relation $R$.

\ignore{CCAL compiles these modules with the CompCertX certified compiler~\cite{deepspec, concurrency},
which is a modified version of CompCert~\cite{deepspec, compcert}.
The {\em implements} relation ($\refines_R$) is formally defined as a forward
simulation~\cite{Lynch95,leroy09,Milner71,Park81} with the (simulation) relation $R$.
This guarantees that if $L'[i] \vdash_R  M : L[i]$ holds,
the behaviors allowed by layer $L[i]$ simulate those allowed by $L'[i]$.
\wolf{or is it the other way around?}}

\ignore{Certified layers enforce a {\em contextual} correctness property as well.
A certified layer converts any {\em safe} client program $P$ running on top of $L'[i]$ into one that has the
same behavior but runs on top of $L[i]$ by compiling the abstract
primitives in $L[i]$ into their implementation in $M$.
If we use ``$\sem{L[i]}{\cdot}$'' to denote the behavior of the layer machine based on
$L[i]$, the correctness property of ``$\ltyp{L'[i]}{R}{M}{L[i]}$'' is written
formally as ``$\forall{}P.\sem{L'[i]}{P\oplus{}M} \refines_R \sem{L[i]}{P}$''
where $\oplus$ denotes a linking operator over programs $P$ and $M$.}

The implements relation also applies to the environment context and the network.
Formally,
$$\forall \varphi_l \in \envcontext'_i, \exists \varphi_h,  \varphi_h \in \envcontext_i \wedge R_{\envcontext', \envcontext}(\varphi_l , \varphi_h) \ \ \ \ \ \ (\mbox{where} \ L'[i] = (\_,  \envcontext'_i, \_, \_) \ \mbox{and} \
L[i] = (\_,  \envcontext_i, \_, \_))$$
This allows us to simplify our view of the possible network behaviors by showing that certain reductions refine the interleaved pattern.
Figure~\ref{fig:network-reduction} shows two common types of network reduction.
The events in boxes are generated by the current node, and the gaps represent places where the environment context will fill in other nodes' events.
Example (a) shows the reductions that can be done for broadcasting sends.
Since the same message is sent to every acceptor, the environment's behavior between the sends is not relevant to P1.
Therefore, it is possible to define a network reduction rule that reorders those send messages and collects them together (Fig.~\ref{fig:network-reduction} (a) (2)).
Then, because the sends always appear together and their order is not important, we can further refine them into a single broadcasting send message (Fig.~\ref{fig:network-reduction} (a) (3)).
Another example of possible network reductions is for a busy-waiting receive pattern.
In example (b), P1 is waiting for a response from A1.
Since the first message comes from A3 instead, P1 will ignore it, and so it can safely be dropped from the log.
Similarly, the third messages arrives after P1 has already received the message it was waiting for so it can also be removed.



\subsection{Write-Witness-Passing}
\label{subsec:witness-write}
\newcommand{\witness}{\varpi}

Concurrent certified abstraction layers provide a powerful framework for showing functional correctness.
However, that alone is not sufficient in distributed system verification.
To fully verify Paxos, for example, one must prove that certain safety properties hold, such as immutability and durability.
These proofs are often considered to be the main obstacles in distributed system verification.
For example, although Verdi~\cite{verdi} provides a framework
that can automatically convert a program assuming a perfect network
into one that can handle a fault-prone environment,
a follow-up work~\cite{cppraft} claims that the safety proofs are still challenging.
Ironfleet~\cite{ironfleet} proves the safety Multi-Paxos,
but they leave some parts of network reduction to a pencil-and-paper proof
(\textit{e.g.} the reduction between Fig.~\ref{fig:network-reduction} (a) (1) and (a) (2)).
Other works~\cite{EPRdistributed, modular} provide automated approaches to reduce the amount of human effort needed,
but it is unclear how well they can handle some distributed protocol services such as leader election in Raft.

\begin{figure}
\begin{center}
\includegraphics[scale=.34]{figs/multipaxos/paxos_example_witness}
\end{center}
\caption{Paxos: Execution Example with Witness}
\label{fig:paxos-example-with-witness}
\end{figure}

In order to simplify these types of proofs and create a methodology that will work for many systems,
we focus on two factors that are common across most distributed systems.
\begin{enumerate}
\item Most distributed protocols involve some totally ordered, unique round identifier (\textit{i.e.} the round number in Paxos, the term number in Raft).
These values monotonically increase as a node executes, and when a value is written, the corresponding round identifier is stored with it.
\item To write a value, a proposer (client, leader, etc) needs to first get acknowledgment
from a quorum of nodes that a certain value is safe to write.
The definition of what constitutes a quorum varies between distributed protocols.
\end{enumerate}

Based on these observations, we propose write-witness-passing as a generic distributed system proof technique.
The high level idea is to make the second point more explicit by gathering the acknowledgements from the quorum
and attaching them to the messages that follow.
This is purely a logical change; the implementation will not actually send any additional information, but only the specification
will be enriched.
We claim that this captures the essential aspects of many distributed systems, and makes it possible to reason about
safety properties without having to consider every failure case.

A formal definition of a write-witness is given in Sect.~\ref{subsec:distributed-transition-semantics-with-witness-passing},
but intuitively, a write-witness is an extension of the pair ($rnd, val$) with additional logical information.
Figure~\ref{fig:paxos-example-with-witness} shows a brief example of using witnesses.
When acceptors store a value, they now store a witness along with it that demonstrates the validity of the value.
For example, acceptor $A1$ contains the witness $\witness_2$ for value $v$ and $A3$ contains witness $\witness_3$ for value $v'$.
Because $A2$ has not yet stored a value, it has the bottom witness $\witness_0$, which is formally defined later.

In order for $P1$ to write a value in round 5, it must create a new witness $\witness_5$ by collecting the responses it received from $A1$ and $A2$.
A witness consists of several components including
the round number, the value to be written, the set of acceptors that participated in the quorum, the set of all acceptors,
and the messages sent by the quorum.
If the value to be written came from one of the acceptors in the quorum and was not chosen by the proposer, then the witness
also contains the previous witness corresponding to the round in which that value was originally written.
All of these can be seen in the definition of $\witness_5$ in Figure~\ref{fig:paxos-example-with-witness}.
This information is nearly all that is needed to prove the safety of Paxos
(we argue in Section~\ref{sec:witness-passing-semantics-with-paxos-variants} that it can be adapted to other systems as well),
and building these witnesses is straightforward.

The following is an informal description of Paxos enriched with witnesses, but it is quite similar
to the previous description in Figure~\ref{fig:paxos-pseudocode}.
\begin{enumerate}
\item Phase 1 (Prepare).
\begin{itemize}
\item Proposer: This step is the same as before except that a witness is generated as acceptor responses are received.
\item Acceptor: If an acceptor returns a previously stored value, it also sends the stored witness associated with it.
\end{itemize}
\item Phase 2 (Write).
\begin{itemize}
\item Proposer: The message now also contains the witness generated in Phase 1.
\item Acceptor: When writing a new value, the acceptor also writes the associated witness.
\end{itemize}
\end{enumerate}
It is clear that adding witnesses to a system is practically free because they are not actually used anywhere in the protocol.
They are merely logically passed around and only used in reasoning about the algorithm.

Witnesses also free one from having to consider all of the ways the environment can fail.
For example, an important invariant in Paxos is that if an acceptor has a stored value, then a majority of acceptors
claimed it was safe to write that value in an earlier round.
To prove this without witnesses requires stepping back through the network history to demonstrate the existence of these message.
This is a tedious process that involves accounting for all the possible network timeouts and dropped packets that could occur.
With write-witness-passing on the other hand, the acceptor already has the witness with exactly these messages.
By simply gathering relevant parts of the global state and storing them locally, write-witnesses are
powerful enough to dramatically reduce the effort required in distributed system verification.

%\section{Witness Passing}
\label{sec:witness-passing-semantics}

This section illustrates the formal definition of write-witness-passing.
This includes
the network model discussed in Sect.~\ref{subsec:network-primitives}, 
the write witness in Sect.~\ref{subsec:witness-write}, and 
how to connect them together via contextual refinement and CCAL mentioned in Sect.~\ref{subsec:link-with-low-level-code-verification}.
We first illustrate the formal definition of our network model,  and then define how it works to 
build the global state transition machines for distributed systems by
building abstraction layers.
After that, we define the formal rules for witnesses and show how the global state transition machine refines
the machine with write witness. 

\newcommand{\msg}{\mathrm{msg}}
\newcommand{\argus}{\mathrm{args}}
\newcommand{\packet}{\mathrm{pkt}}
\newcommand{\canrecv}{\mathrm{valid_{recv}}}
\newcommand{\getsrc}{\mathrm{getSrc}}
\newcommand{\getowner}{\mathrm{getOwner}}
\newcommand{\getdes}{\mathrm{getDes}}
\newcommand{\checkcidone}{{\mathrm{checkCid}_1}}
\newcommand{\checkcidtwo}{{\mathrm{checkCid}_2}}
\newcommand{\lstate}[1]{\mathrm{lst}_{#1}}
\newcommand{\stfori}[1]{\mathrm{st}_{#1}}
\newcommand{\fsendpkt}[4]{[{#1}.\mathrm{SEND}\langle{#2}\rangle{[#3]}.{#4}]}
\newcommand{\frecvpkt}[4]{[{#1}.\mathrm{RECV}\langle{#2}\rangle{[#3]}.{#4}]}
\newcommand{\ftimeoutpkt}[2]{[{#1}.\mathrm{TOUT}\langle{#2}\rangle]}
\newcommand{\fghostpkt}[3]{[{#1}.\mathrm{GHOST}\langle{#2}\rangle[{#3}]]}
\newcommand{\networklog}{\mathrm{l}_{net}}
\newcommand{\layerfori}[1]{L{[#1]}}
\renewcommand{\replay}{\mathbb{R}}
\newcommand{\accsset}{\mathrm{ASET}}
\newcommand{\propsset}{\mathrm{PSET}}
\newcommand{\propkwd}{\mathrm{PROP}}
\newcommand{\dsset}{\mathrm{DSET}}
\newcommand{\getcid}{{\mathrm{getCid}}}

\begin{figure}
\raggedright
\fbox{Variables:}

$
\begin{array}{llll}
\msg: Type & \mbox{(message)} \\
\argus: \mathrm{list}\ \mathbb{Z} & \mbox{(arguments for ghost packets)} \\
cid,\ owner,\ des, src: \mathbb{Z} & \mbox{(channel ID, owner, destination, and source in packets)}\\ 
\lstate{i}: Type & \mbox{(local state of the node}\ i\mbox{)} \\
\end{array}$
\vspace{0.5em}

\raggedright
\fbox{Packet, network, and state definitions:}

$
\begin{array}{lll}
\packet & :=& \fsendpkt{owner}{cid}{\msg}{des}~\vert~\frecvpkt{owner}{cid}{\msg}{src}~\vert~\ftimeoutpkt{owner}{cid}\\
&&\vert~\fghostpkt{owner}{cid}{\argus}\\
\networklog & := & \mbox{list}\ \packet  \ \ \ \ \ \ \ \ \ \ \ \ \ \mbox{(network log with the local view from node} \  i \mbox{)}\\
\stfori{i} & := & (\lstate{i}, \networklog)  \ \ \ \ \ \ \ \ \ \mbox{(state of  node} \ i\mbox{)}\\
\end{array}
$
\vspace{0.5em}

\raggedright
\fbox{Auxiliary functions:}

$
\begin{array}{ll}
\getsrc : \msg \rightarrow \mathbb{Z} & \mbox{(return source node ID of message}\ \msg\mbox{)}\\
\getowner :  \packet \rightarrow \mathbb{Z} & \mbox{(return  owner node ID of  packet}\ \packet\mbox{)}\\
\getdes :  \packet \rightarrow \mathbb{Z} & \mbox{(return destination node ID of the packet}\ \packet\mbox{)}\\
\checkcidone : \mathbb{Z}  \rightarrow \mathbb{Z}  \rightarrow \mathbb{Z}  \rightarrow \propkwd & 
\mbox{(validity of channel ID for send and receive packets)}\\
\checkcidtwo : \mathbb{Z} \rightarrow \mathbb{Z}  \rightarrow \propkwd & 
\mbox{(validity of channel ID for timeout and ghost packets)}\\
\ssame:   \networklog \rightarrow \networklog \rightarrow \networklog & \mbox{(append two network logs)} \\
\cons : \networklog \rightarrow \packet \rightarrow \networklog & \mbox{(add one packet into one network log)} \\
\end{array}
$
\vspace{0.5em}

\raggedright
\fbox{Specifications for send and receive primitives:}
\begin{mathpar}
\inferrule
{\layerfori{owner} = (\layerdef, \envcontext_{owner}, \igchar, \igchar) \\
\stfori{owner} = (\lstate{owner}, \networklog) \\
(\networklog, \networklog ') \in  \envcontext_{owner} \\
\packet = \fsendpkt{owner}{cid}{\msg}{des}\\
src = \getsrc(\msg)\\  
\checkcidone(cid, owner, des) \\
\stfori{owner}' = (\lstate{owner},\ \networklog \ssame\ \networklog '\ \cons\ \packet)}
{\layerfori{owner} \vdash \stfori{owner} \xrightarrow{\sigma_{send(cid, des, \msg)}}  \stfori{owner}'}

\inferrule
{\packet =  \frecvpkt{owner}{cid}{\msg}{src}\\
\checkcidone(cid, owner, src) \\
src = \getsrc(\msg)  \\\\
\exists cid', \checkcidone(cid', src, des) \wedge \fsendpkt{src}{cid'}{\msg}{owner} \in \networklog}
{\canrecv(src, \packet, \networklog)}

\inferrule
{\packet = \ftimeoutpkt{owner}{cid} \\ 
\checkcidtwo(cid, owner)}
{\canrecv(src, \packet, \networklog)}

\inferrule
{\layerfori{owner} = (\layerdef, \envcontext_{owner}, \igchar, \igchar) \\
\stfori{owner} = (\lstate{owner}, \networklog) \\
(\networklog, \networklog ') \in  \envcontext_{owner} \\
\canrecv(from, \packet, \networklog \ssame \networklog ') \\
\stfori{owner}' = (\lstate{owner},\ \networklog \ssame\ \networklog '\ \cons\ \packet)}
{\layerfori{owner]} \vdash \stfori{owner} \xrightarrow{\sigma_{recv(from)}} \stfori{owner}'}
\end{mathpar}
\caption{Network Model and Related Definitions}
\label{fig:net-syntax-semantics}
\end{figure}

\subsection{Low Level Network Syntax and Semantics}
\label{subsec:low-level-network-syntax-and-semantics}

Our network is defined as a log of packets, and the key definitions are
found in Figure~\ref{fig:net-syntax-semantics}.
There are four kinds of packets; 1) send; 2) receive; 3) timeout; and 4) ghost.
The first three correspond to send and receive primitives, and those 
packets consist of multiple elements including a channel identifier.
The model contains multiple logical channels and communication in one channel can be distinguished from communication in others.
Channel identifiers provide a clear distinction between packets in different channels even though our network history contains 
packets in multiple channels.
The channel identifier also makes defining projection functions straightforward.
There are often two nodes per channel,
but multiple nodes can be associated with one channel as well.
Additionally, one node or one set of nodes can participate in multiple channels, which is a crucial property
of the network model to build extensible distributed systems.
For example, one channel can be designated for communications from acceptors to proposers in Paxos
and another channel can be used for communications from proposers to acceptors.
It is also possible for the model to assign another channel between proposers (clients)
to build distributed applications based on the underlying Paxos implementation.
By allowing that, multiple kinds of distributed transactions could be built based on the same consensus protocols
but their network communications would remain logically separated.

The existence of multiple channels also relates to the network reductions in Fig.~\ref{fig:network-reduction}.
A reduction can be represented as introducing a new channel with the simplified semantics and disallowing the old channel.
For example, the specification of the broadcasting send function (\textit{i.e.} Fig.~\ref{fig:network-reduction} (a) (2)) can be first defined as
sending the same message to all acceptors in the set ($\accsset$) as
\begin{mathpar}
\inferrule
{\stfori{owner}' = 
\mbox{\textbf{reduce}}\ (\lambda\ des. \ \layerfori{owner}_{send_{low}}(send)\ (cid_{low}, des, \msg))\ 
\ \accsset\  \  \stfori{owner}}
{\layerfori{owner}_{send_{low}} \vdash \stfori{owner} \xrightarrow{\sigma_{bsend(cid_{low}, \msg)}}  \stfori{owner}'}
\end{mathpar}
assuming that  $\layerfori{owner}_{send_{low}} $ contains the primitive $bsend$.
Then, we simplify the function by assigning the new channel $cid_{high}$ without changing 
the actual specification of the send primitive so other nodes can continue to use it.
The new specification of the broadcasting send primitive (\textit{i.e.} Fig.~\ref{fig:network-reduction} (a) (3))
can be defined as a single send primitive call with the newly assigned channel
\begin{mathpar}
\inferrule
{\stfori{owner}' = \layerfori{owner}_{send_{high}}(send)\ (cid_{high}, \igchar, \msg))\  \stfori{owner}}
{\layerfori{owner}_{send_{high}} \vdash \stfori{owner} \xrightarrow{\sigma_{bsend(cid_{high}, \msg)}}  \stfori{owner}'}
\end{mathpar} 
with the condition $cid_{low} \neq cid_{high}$.
The rely/guarantee conditions must also be modified, and the two layers must be related by a proper relation $R$ that also relates their environment contexts.
Intuitively, the relation $R$ matches multiple send packets  in channel $cid_{low}$, with $n$ when $n$ is the number of elements in $\accsset$,
into a single send event $cid_{high}$. 

Figure~\ref{fig:net-syntax-semantics} also contains send and receive specifications.
They first query the environment context based on the current network status, represented as $(\networklog, \networklog ') \in  \envcontext_{owner} $.
Then, they
update the current network status using the result from the environment context query, notated as $ \networklog \ssame\ \networklog '$.
After that, both add the next event, either send, receive, or timeout packet, at the end of the network history. 
The specification of the receive function requires an additional validity condition, $\canrecv$,
to check whether the currently received packet has a matching send in the network history.
This rules out the possibility of packets being generated from nothing.
Besides validity checks for received messages and channel identifiers,
we do not have any additional auxiliary conditions in the network primitives.
That implies that the receive primitive can receive any packets that have been sent previously, and some send packets may never have a corresponding receive packets.
The receive primitive also can generate a timeout packet, which implies that no packets were received at that time.

The final type of packet, a ghost packet, is to represent the transitions that only trigger changes in the local state.
As having observed in the specifications of send and receive,
all nodes contain their local view of network history 
via the environment context that properly captures interleaving among other nodes.
This helps us to directly define the global state transition machines without considering complex composition with other nodes in the system. 
However, only having send, receive, and timeout packets is sometimes not sufficient
to define all possible transitions.
Ghost packets help us to define them, and Sect.~\ref{subsec:distributed-transition-semantics} shows how we use them.

\subsection{Distributed Transition Semantics}
\label{subsec:distributed-transition-semantics}

Another step required in distributed system verification is defining the global state and transitions to represent the status and the behavior of 
all the participants in the system.
Based on the network primitives including send and receive functions and the local
instructions that manipulate the local state,
implementing and verifying the functional correctness of the transitions on each node
(e.g. proposers and acceptors) are feasible by building multiple certified abstraction layers. 
Distributed systems, however, are a cluster of nodes.
However, exciting safety properties such as immutability and linearizability for distributed systems are usually stated about the status and the behavior of all nodes in the system. 
Therefore, proving the functional correctness of the low-level implementation only for a local node is not sufficient.
The verification has to consider the global behavior of the entire distributed system because.
thus specifications for transitions also need to represent the alternative behaviors that make use of a protocol,
which are on the whole local states of all nodes in them. 
Our approach provides a simple, but sufficient solution for that purpose. 


Our network history contains not only the packets triggered by the node that calls network primitives but also 
the packets generated by other nodes in the system.
This feature is achieved by the network history update using environmental context,
and thus it implies that the history contains all the communication information in the network as the linearized version by the view of each node.
By using all the network packets in the system, 
the network replay function provides a way to figure out how local states of all nodes can be changed during if all local transitions can be mapped with packets in the network history.
We define the
replay function for the distributed system $DS$ as
$$\replay_{DS}: \networklog \rightarrow \mbox{option}\ \set{\lstate{DS_i}~\vert~\forall i, i \in \dsset} $$
, where $\dsset$ is the set of nodes that participate in the distributed system and
$\lstate{DS_i}$ is the local state for node $i$ that only contains the information
regarding $DS$. 
 
However, only using send, receive, and timeout packets to define the proper replay function is 
impossible or restricted in its applicability.
Even though most local transitions in the distributed systems are associated with the network primitive, 
there are some local transitions can be purely associated with the local state itself. 
To model those transitions in the network replay function properly, 
we use ghost packets defined in~\ref{fig:net-syntax-semantics}.
These packets do not participate in the network communication but work as stamps that tell the 
replay function to change the local state that is associated with that node id $i$.
Allowing ghost packets for node id $i$ in the network history also implies that adding ghost packets in the network for other nodes is also possible by using environment context and rely/guarantee conditions.
In this sense, it is possible for us to model all local transitions with the network replay functions even though some are not associated with network primitives. 

\begin{figure}
\begin{center}
\begin{mathpar}
\inferrule
{L_{impl}[owner] = (\layerdef_{impl}, $\_$, $\_$, $\_$) \\
L_{abs}[owner] = (\layerdef_{abs}, $\_$, $\_$, $\_$) \\\\
\stfori{impl_{owner}} = (\lstate{impl_{owner}},\ \networklog) \\
\stfori{abs_{owner}} = (\lstate{abs_{owner}},\ \networklog) \\\\
\layerdef_{impl}(fid) = \sigma_{fid}\\
fid \in \mathrm{dom}(\layerdef_{abs})\\
\sigma_{fid}(args, \lstate{impl_{owner}}) = \lstate{owner}'\\\\
cid = \getcid(fid)\\
\checkcidtwo(cid, owner)\\
\packet = \fghostpkt{owner}{cid}{args}  \\
\replay_{DS}(\networklog)  = \mbox{Some}\ \set{\cdots ,\lstate{DS_{owner}},  \cdots }\\
\replay_{DS}(\networklog\ \cons\ \packet)  = \mbox{Some}\ \set{\cdots ,\ \lstate{DS_{owner}}',\  \cdots } \\ 
\stfori{impl_{owner}}' = (\lstate{impl_{owner}}',\ \networklog \ \cons\  \packet) \\
\stfori{abs_{owner}}' = (\lstate{abs_{owner}},\ \networklog \ \cons\  \packet)\\
R (\lstate{impl_{owner}},\ (\lstate{abs_{owner}}, \lstate{DS_{owner}}))\\
R (\lstate{impl_{owner}}',\ (\lstate{abs_{owner}}, \lstate{DS_{owner}}'))}
{L_{impl}[owner](fid) \refines_R  L_{abs}[owner](fid)}
\end{mathpar}
\end{center}
\caption{Relation for the Refinement Proof of Function $fid$ in Two Layers, $L_{impl}$ and  $L_{abs}$}
\label{fig:relation-for-the-refinement-proof}
\end{figure}

Providing the refinement theorem in between those two representations is also necessary to link the layer with global state transition representation and the layer only contains the information about the local functional correctness.
For example, when the layer $L_{abs}$ is the layer that contains the network replay function to represent the global state of the distributed system and the $L_{impl}$ is the layer that only contains the specifications focusing on the local functional correctness, 
the relation for refinement proof of $fid$ is defined in Fig.~\ref{fig:relation-for-the-refinement-proof}, 
which implies that the evaluation of the network replay function  for $fid$ in the $L_{abs}$  layer 
will be matched with evaluation of the local transition of the function $fid$ in the $L_{impl}$ layer 
with the refinement relation $R$.

Similar to the example, providing the simulation proofs for all possible transitions between 
those two layers are possible with using proper ghost packets and the network replay function.
This process finally gives us the refinement theorem 
$L_{impl}[owner]  \refines_R L_{abs}[owner]$, and also the contextual refinement theorem 
with the client program $p$, which will be stated as $\forall{}P.\sem{L_{impl}[owner]}{P} \refines_R \sem{L_{abs}[owner]}{P}$.

\subsection{Distributed Transition Semantics with Write-Witness-Passing}
\label{subsec:distributed-transition-semantics-with-witness-passing}
\newcommand{\ballotnum}{\mathrm{bn}}
\newcommand{\votedballotnum}{\mathrm{vbn}}
\newcommand{\isquorums}{\mathrm{is}\_\mathrm{qrm}}
\newcommand{\dsvalue}{\mathrm{val}}
\newcommand{\nodeid}{\mathrm{nid}}
\newcommand{\partialset}{\mathrm{accs}}
\newcommand{\acceptors}{\mathrm{ASET}}
\newcommand{\funwitness}{\mathcal{F}_{\witness}}
\newcommand{\getbn}{\mathrm{getBN}}
\newcommand{\getvbn}{\mathrm{getVBN}_{msg}}
\newcommand{\getvbnp}{\mathrm{getVBN}_{\packet}}
\newcommand{\getdsvalue}{\mathrm{getVal}_{\packet}}
\newcommand{\getwitness}{\mathrm{get}\witness_{\packet_{\witness}}}
\newcommand{\projpkt}{\mathrm{proj}_{\packet}}
\newcommand{\projnet}{\mathrm{proj}_{\mathrm{net}}}

This section illustrates a novel but straightforward way, \textit{write-witness-passing}, that can aid protocol-related property proofs (\textit{e.g.} safety) of distributed systems alongside with the functional correctness of them. 
The write-witness-passing does not only make the proofs simple but 
but also helps to provide a generic interface that many distributed systems have.


The main idea of write-witness-passing is facilitating the methodology that we use \textit{(2) functional correctness} in
\textit{(3) safety verification} as much as possible by
introducing ghost state in the local state, by linking the ghost state with the low-level
implementation using contextual refinement, and by capturing the universal aspects of the write operation in the distributed system.
Write operation of several distributed systems usually requires a step ahead before write, prepare. 
The prepare step is for 
asking each acceptor to be prepared with the proposed write
and more importantly, asking each acceptor to give the value that has been previously written if it exists.
Distributed systems typically do not guarantee successes of all write attempts due to possible contentions from other write attempts for 
other nodes in the system
and error that may occur during the operation. 
However, most of them require specific evidence before each write attempt regardless of the result of the write operation. 
Therefore, we generalize the form of the evidence and define the generic form, \textit{write-witness} ($\witness$).

\begin{figure}
\raggedright
\fbox{Variables:}

$
\begin{array}{llll}
\ballotnum: Type & \mbox{(ballot value -- totally ordered)} \\
\dsvalue: Type & \mbox{(value that the distributed system stores)} \\
\nodeid : Type & \mbox{(node ID --  unique value for each node)} & \\
\partialset : \set{\nodeid} & \mbox{(set of node IDs)}\\
\accsset : \set{\nodeid} & \mbox{(set of node IDs for all acceptors at the moment)}\\
\witness_{\mathrm{el}} : (\ballotnum \times \dsvalue \times accs  \times \accsset  \times \set{\packet}) & \mbox{(each  element in the witness)}\\
\witness  : \mbox{list}\ \witness_{\mathrm{el}}  & \mbox{(witness)}\\
\end{array}$
\vspace{0.7em}

\fbox{Inductive definition of write-witness-passing:}
\vspace{0.2em}

$
\begin{array}{ll}
\witness_{\bot} := \mbox{nil} & \mbox{(base case of the witness definition)}\\
\witness_{\ballotnum} := \witness_{\ballotnum'}\ \cons\ (\ballotnum, \dsvalue, accs, \accsset, \set{\packet}) &  \mbox{(inductive case of the witness definition)}\\
\end{array}
$
\vspace{0.7em}


\raggedright
\fbox{Packets with write-witness-passing:}
\vspace{0.2em}

$
\begin{array}{llll}
\getvbn : \msg \rightarrow \ballotnum  & \mbox{(voted ballot value of message}\ \msg\mbox{)} \\
\msg_\witness : (\msg, \mbox{option}\ \witness) & \mbox{(message with its witness -- the witness of}\ \getvbn(\msg)\mbox{)}\\
\end{array}
$
\vspace{0.7em}

$
\begin{array}{lll}
\packet_\witness & :=& \fsendpkt{owner}{cid}{\msg_{\witness}}{des}~\vert~\frecvpkt{owner}{cid}{\msg_\witness}{src}~\vert~\ftimeoutpkt{owner}{cid}\\
&&\vert~\fghostpkt{owner}{cid}{\argus}\\
{\networklog}_{\witness} & := & \mbox{list}\ \packet_{\witness}  \ \ \ \ \ \ \ \ \ \ \ \ \ \mbox{(network log with the local view from node)} \  i \\
\end{array}
$
\vspace{0.7em}

\fbox{Function for write-witness-passing:}
\vspace{0.2em}

$
\begin{array}{llll}
\isquorums : \set{\mathbb{Z}} \rightarrow  \set{\mathbb{Z}} \rightarrow \propkwd & 
 \mbox{(function that checks the quorum)} \\
\le_{\ballotnum} : \ballotnum \rightarrow \ballotnum \rightarrow \propkwd & \mbox{(binary relation for ballot values)}\\
\funwitness : \le_{\ballotnum}  \rightarrow  \isquorums  \rightarrow \set{\nodeid}  \rightarrow \ballotnum \rightarrow \set{\packet_\witness} \rightarrow \dsvalue \rightarrow \witness 
& \mbox{(generate a write witness)} \\
\end{array}
$
\vspace{0.7em}

\fbox{Generic properties with write-witness-passing:}
\vspace{0.2em}

$
\begin{array}{llll}
\getvbnp : \packet \rightarrow \ballotnum  & \mbox{(voted ballot value of  packet}\ \packet\mbox{)} \\
\getdsvalue : \packet \rightarrow \dsvalue  & \mbox{(stored value of  packet}\ \packet\mbox{)} \\
\getwitness : \packet_\witness \rightarrow \mathrm{option} \ \witness  & \mbox{(witness of  packet}\ \packet_{\witness} \mbox{)} \\
\projpkt : \packet_\witness \rightarrow \packet & \mbox{(remove witness from}\ \packet_{\witness}\mbox{)} \\
\projnet : {\networklog}_{\witness} \rightarrow \networklog & \mbox{(remove witnesses from all packes in}\  {\networklog}_{\witness}\mbox{)} \\
NoDup: \mathrm{list}\ Type \rightarrow \propkwd & \mbox{(check the uniqueness of all elements in the list)} \\
\end{array}
$
\vspace{0.7em}

$$
\begin{array}{c}
 \funwitness(\le_{\ballotnum},~\isquorums,~\accsset, ~\ballotnum, ~pkts) = \witness_{\ballotnum'} \ \cons \ (\ballotnum, ~\dsvalue', ~accs, ~\accsset, ~pkts') \rightarrow\\
\isquorums(accs, \accsset)~\wedge~NoDup(accs)~\wedge~accs \subseteq  \accsset ~\wedge~ pkt' \subseteq \projnet(pkt)~\wedge\\
(\forall a.~a \in accs \rightarrow \exists p.\ p \in pkts'~\wedge~p = \frecvpkt{\igchar}{\igchar}{\igchar}{a})~\wedge\\

(\forall p.~p \in  pkt' \rightarrow \\
(\exists \ src \ \msg. \ p = \frecvpkt{\igchar}{\igchar}{\msg}{src}~\wedge~
src \in accs~\wedge~\getvbn(\msg) <_{\ballotnum} \ballotnum)) \\ 

(\exists p\ p_{\witness}.~p \in  pkt'~\wedge~p_{\witness} \in pkt~\wedge\\
\projpkt(p_{\witness}) = p~\wedge~~\getwitness(p_{\witness}) = \mathrm{Some}\ \witness_{\ballotnum'}~\wedge~
\getvbnp(p) = \ballotnum'~\wedge\\
(\forall p'.\ p' \in pkt' \rightarrow \getvbnp(p') \le_{\ballotnum} \getvbnp(p))~\wedge\\
( \getvbnp(p) = \bot \rightarrow \getdsvalue(p) = \bot~\wedge~ \dsvalue' = \dsvalue)~\wedge \\
( \getvbnp(p) \neq \bot \rightarrow \dsvalue' = \getdsvalue(p)) \\
\end{array}
$$
\caption{Write-Witness-Passing and Related Definitions}
\label{fig:witness-witness-formal}
\end{figure}


Figure~\ref{fig:witness-witness-formal} shows  key definitions related the write-witness-passing.
The write-witness can be viewed as a record that keeps the essential and generic information for each write attempt in the distributed system,
and it is a list of evidence for the all write attempts that are associated with the current one. 


Each element of the witness consists of five components that can be instantiated according to the distributed system that we want to verify.
The first one is a ballot number $\ballotnum$, a unique identifier for the write-witness.
Most distributed systems use their own ballot values ($\ballotnum$) to distinguish each write attempt from others. 
Those values do not need to be defined as simple singleton values such as integer numbers, but they have to satisfy three conditions. 
First, all values need to be members of a totally ordered set to provide the order 
between all possible two different ballot values.
Also, distributed systems need to guarantee the uniqueness of each value for each write attempt from others of other write attempts
with having a particular constraint, and regardless of contentions or network error.
Two examples are round numbers in Paxos and the pair of term and index numbers in Raft.
Those facts imply that a particular ballot value has only one and unique witness if there was a write attempt associated with the ballot value.
One another condition related to $\ballotnum$ is that no write attempts are allowed with $\bot$, 
when $\bot$ implies the bottom value with the condition that for all possible ballot numbers are greater than or equal to $\bot$, 
 $\forall \ballotnum, \ \bot <_{\ballotnum} \ballotnum \vee \ballotnum = \bot$.
The second field is a stored value ($\dsvalue$) for the distributed system, and each stored value is mapped with a certain $\ballotnum$.
There is no restriction on the stored value, but if the current $\ballotnum$ is $\bot$ in the local state of a node $i$, 
 then the corresponding stored value in the node $i$'s local state also needs to be $\bot$.
Using the constraints of both $\ballotnum$ and $\dsvalue$, we can treat $(\bot, \bot)$ as an initial state for the slot. 

Other two fields of the write-witness, two sets of node IDs, are to 
represent the node IDs that participate in the quorum for the write attempt and
the node IDs that represent the all candidates that could participate in the write attempt.
The last field, $\set{\packet}$, is a set of concrete evidence for the write attempt. 
When a proposer collects the evidence for the write attempt, 
write-witness also collect additional information, which is a set that contains all the received packets from the acceptors that participate in the quorum ($accs$ in the definition).
During the proof, we can treat them as a projection from the whole network history to the packets that are related to witness of the associated write attempt. 

The write-witness, list of each evidence discussed in the above, contains all the information from the initial write to the latest one associated with
the ballot value to keep track of the write history.
The list does not need to have the witness for the all possible ballot values less than or equal to the current ballot value. 
The list of identifiers, projection from the list of witness elements, is monotonically increasing but is not continuous.
However, the information in the list is sufficient for us to deduct where the stored value with the ballot value comes from and the history of the stored value. 
In Paxos, for example, 
each write operation usually picks up one value among promise messages from acceptors 
if the previous write has been recorded on one of the acceptors that participate in the quorum. 
This implies that the witness can keep track of the previous write that has been occurred on the acceptor. 
Combined with the consensus mechanism of Paxos,
this also implies that all stored values ($\dsvalue$) in all elements of the write-witness in Paxos
need to be same regardless of the consensus of the entire system. 
If the consensus has been raised with the ballot number $\ballotnum$, 
all write attempts with $\ballotnum'$ ($\ballotnum <_{\ballotnum} \ballotnum'$) after the consensus contains 
$\witness_{\ballotnum}$ as  the prefix of their witnesses.

%\topic{how we pass those write witness}

Providing write-witness-passing also requires us to integrate the witness generation process with the previous distributed transition system, combining local states with the write-witness as well as extending the network model with the write-witness, which are straightforward with contextual refinement.
Regarding local state modification, adding write witness implies adding one more \textit{logical} field in each local state of the participant. 
For example, $\lstate{DS_i}$, the local state of node $i$ in the global transition system, is a tuple of three components, 
$(\ballotnum, \ballotnum, \dsvalue)$ for Paxos.
Then $\lstate{DS\witness_i}$, the local state of node $i$ in the global transition system with the write-witness, 
is a tuple with the form $(\ballotnum, \ballotnum, \dsvalue, \witness_{\ballotnum})$.
The network model also needs a small change to make it possible to use the write-witness in the communication. 
Each message of packets can contain its own write-witness for the value stored in the message
if it is required, and the definition for the modified packet is in Fig.~\ref{fig:witness-witness-formal}.
As in the figure, adding the write-witness in the packet is optional.
For instance, the prepare packet in Paxos described in the Fig.~\ref{fig:paxos-pseudocode} does not need to include the write-witness in it.
With those modifications, we provide the form for the network replay function with the write-witness as
$$\replay_{DS\witness}: {\networklog}_{\witness} \rightarrow \mbox{option}\ \set{\lstate{DS\witness_i}~\vert~\forall i, i \in \dsset} $$
, which is similar to the network replay function in Sect.~\ref{subsec:distributed-transition-semantics}, but can handle the write-witness in it. 

%\topic{how we collect them}

Collecting those write-witness is mostly compatible with the prepare phase in distributed protocols. 
The type of the function, $\funwitness$, that generates the write-witness is also defined in Fig.~\ref{fig:witness-witness-formal}.
The first argument, $ \le_{\ballotnum}$ is the binary comparison relation for the set of ballot values,
and the second one is the definition of the quorum. 
Two definitions depend on distributed protocols, and  different services in the same distributed protocol may have 
different definitions for them, too.
For example, the binary relation in  Paxos is a binary comparison relation on natural numbers, 
and $\isquorums$ is the function that checks whether the witness contains the majority, which is more than half, of all acceptors. 
On the other hand, the binary relation on the election phase of Raft is a relation on two elements, 
term and index numbers. 
The third argument is a set of node IDs for all acceptors in the system for the write attempt associated with the ballot number $\ballotnum$.
Several distributed protocols, such as Vertical Paxos and Raft, allow reconfiguration. 
Allowing reconfiguration implies that a set of node IDs for all acceptors associated with a single write attempt may vary with the set related to other attempts even they are write attempts inside the same distributed protocols.
The next two arguments of the function correspond to the values that we need when we collect the evidence of the quorum in the prepare phase.
The ballot value ($\ballotnum$) is the value that we use for the write attempt and thus also in the prepare phase.
The $\packet_\witness$ are the packets that the function received from all acceptors while the prepare phase.
For example, they are the same packets with promise packets from acceptors in Paxos and vote packets in the election phase of Raft. 
The last argument ($\dsvalue$) is the value that the $\ballotnum$ try to write on the system but is not guaranteed.
Writing  another value different from $\dsvalue$  happens frequently in multiple distributed systems, especially consensus protocols.
Most consensus protocols sometimes need to rewrite the values that they have gotten from acceptors instead of 
their proposed value.
Picking up the value for the write attempt associated with $\ballotnum$ depends on the distributed protocol.

%\topic{what do they guarantee?}

The witness generated by $\funwitness$ guarantees multiple safety properties with the argument of the function and those properties are described in Fig.~\ref{fig:witness-witness-formal}
with
$$\funwitness(\le_{\ballotnum},~\isquorums,~\accsset,~\ballotnum,~pkts) = \witness_{\ballotnum'} \ \cons \ (\ballotnum,~\dsvalue',~accs,~\accsset,~pkts')$$ The first line of the property shows that
$accs$ is a collection of acceptor node IDs that participate in the quorum.
It also provides the property that there are a set of packets ($pkt'$), pieces of evidence for the write that are extracted from a subset of $pkt$ when $pkt$ is a set of packets that the proposer received from all acceptors related to the write. 
The second line implies every element that participates in the quorum sent the proper message to provide the evidence of the write.
The next two lines are for properties that all shreds of evidence need to hold. 
All packets in $pkt'$ contain the proper evidence, which includes the fact that ballot values of those packets are always less than $\ballotnum$, the identifier of the current write-attempt.
The last five lines are for inferring the source of the values and finding out the previous write-witness for the current write.
Some properties may vary depending on the protocol, and adding or removing those properties in the write-witness is striaghtforward with our approach. We discuss how it helps the verification of distributed systems in Sect.~\ref{sec:paxos-verification} using the running example.

%\lstdefinelanguage{mycoq}
{morekeywords={let, in, Function, replay_log, WOR_write, if, then, else},
sensitive=false,
morecomment=[s]{(*}{*)},
morestring=[b]",
}

\lstset{
  basicstyle={\footnotesize\linespread{0.5}\normalfont}, %\ttfamily %\small, %\ttfamily
  showstringspaces=false,
  columns=flexible,
  breakatwhitespace=false, 
  breaklines=true, 
  commentstyle=\color[HTML]{444444},%\textit,
  keywordstyle=\color{black}\textbf,
  mathescape,
}


\section{Formal Verification}\label{sec:formal_verification}



	WormSpace acts as a foundation for verifying distributed systems. We verify WormSpace once and reuse its proof for verifying systems built on top while hiding the complexity of distributed protocol verification. To do so, we extend the Certified Concurrent Abstraction Layer (CCAL) approach~\cite{deepspec, concurrency} introduced in Section~\ref{sec:background}, modeling an asynchronous network of distributed nodes in order to verify WormSpace. We apply CCAL beyond a single system verification for the first time and link the proof of WormSpace, WormSpace applications and a verified OS. 


\subsection{Layer Structure for Verification}



WormSpace consists of two separate stacks of verification layers, the client library (17 layers) and the wormserver (2 layers), over a common set of base functionalities (5 layers). While the number of layers may seem excessive, it matches a conventional software stack designed for modularity: each layer is a C component implementing some interface. A simplified layer diagram is shown in Figure~\ref{fig:layerdiagram}.

\begin{figure}
\centering
\includegraphics{figs/multipaxos/layer_diagram.pdf}
\caption{Layer diagram: client and server stacks are combined as a distributed system in the GHOSTLAYER and the distributed nature is invisible from the WOR layer.}
\label{fig:layerdiagram}
\vspace{-0.1in}
\end{figure}


Both stacks share a common set of base layers: the bottom layer provides an interface to the trusted computing base (TCB), including network communication functions and a small number of system calls. Above this bottom layer, we introduce a data layer which implements various data structures over the trusted primitives. Above the data layer, the client and server stacks diverge. The server stack includes Paxos acceptor layers and the wormserver code above it. The client stack includes layers for Paxos proposer logic and a wormclient layer that issues individual Paxos proposals.


	The GHOSTLAYER horizontally composes the two stacks and proves properties across multiple wormservers and clients. The GHOSTLAYER 
	includes a global state transition system
	that can reason all concurrent client and server interactions based on a network model. Safety properties of Paxos are proved in this layer.
The contextual refinement proof between the GHOSTLAYER and the composition of wormserver and wormclient provides a powerful guarantee for the layers built on top of the GHOSTLAYER. Any layer that the GHOSTLAYER contextually refines is guaranteed to be correct with respect to both client and server layers. It is guaranteed that any concurrent behaviors of distributed nodes using the client and server layers are correct. Verified distributed protocol properties hold in higher layers while complex proofs are encapsulated in the GHOSTLAYER.

Verification above the GHOSTLAYER is as easy as verifying a sequential program.
For example, the top-level specification for a write in WormSpace is simply translating the global address to a segment address and offset and passing the captureID (cid) to call the lower-level write which is already proved safe under concurrent distributed accesses:

\begin{lstlisting}[language=mycoq, basicstyle=\small]%, basicstyle=\linespread{0.5}]

  Function WormSpace_write (addr: Z) (val: Payload)
   (cid: Z) (adt: EnvVars) : option (EnvVars * Z) :=
    let segment:= addr / WOS_SIZE in 
    let offset := addr mod WOS_SIZE in
    WOR_write segment offset val cid adt.
\end{lstlisting}
\noindent We verify the WOR abstraction, the WOS abstraction, and the WormSpace API. The client stack can be extended to applications such as WormPaxos, WormLog, and WormTX. 

\subsection{Network Model}\label{subsec:network_modeling}

To model a real-world network and to prove distributed properties about the system, we employ techniques from concurrency verification~\cite{concurrency}. Our network model includes two basic primitives, \textit{send\_msg} and \textit{recv\_msg}, which manipulate the modeled network state. The model includes a logically linearized sequence of network operations, which we call the global network log. Each distributed node can extract its local interaction with the network from the log, and the log is used to reason about the interaction between nodes.


However, we depart from single-node concurrency verification by modeling the network as unreliable (but non-Byzantine). In our model, \textit{send\_msg} simply creates a \textsc{send} event in the log, while \textit{recv\_msg} creates either \textsc{timeout} (this models dropped packet) or \textsc{recv} events in an arbitrary future location (this models packet delays) than the \textsc{send} event in the log. In between a pair of \textsc{send} and \textsc{timeout/recv}, any other nodes can freely record their operations (this models packet reordering). A \textsc{recv} after a \textsc{send} does not necessarily mean that the \textsc{recv} event received the value sent by this \textsc{send}. The actual value can be a duplicate message from a previous send (this models duplicate packets).




Network communication patterns can be complex when a client interacts with multiple wormservers in a one-to-many request pattern.
Abstraction and contextual refinement can help us manage this complexity without reducing the fidelity of verification.
Accordingly, we create a network log layer with simpler semantics, and prove that the original log refines the simplified log.
The simplified log coalesces broadcasts and receptions into singleton events and eliminates duplicates simplifying global property proofs.


\begin{figure} 

\begin{lstlisting}[language=mycoq, basicstyle=\small]%, basicstyle=\linespread{0.5}]

Function WOR_ghost_write (addr: Z) (val: Payload) (cid: Z)
 (adt: EnvVars) : option (EnvVars * Z) :=
  let net_l := adt.net_l (* get net log from Env context *)
  let nid := get_node_id adt in (* get current node id *)
  (* replay the net log; get the local node state; and
     check if the node is in a writable status *)
  if (can_write ((replay_log(net_l))[nid]) addr val cid)
  then
    (* log write intent with a ghost msg to the net log *)
    let net_l$_1$ := (ghost_write nid addr val cid) :: net_l in
    (* broadcast msgs and collect acks: reflect behaviors 
       of other nodes to add send/recv events by this and
       other nodes to the net log *)
    let net_l$_2$ := bcast_n_recv nid addr val cid net_l$_1$ adt in
    (* replay the net log to compute global state; get 
       node's local state; and check the quorum status *)
    let result := is_qrm ((replay_log(net_l$_2$))[nid]) addr in
    (* log the result using a ghost msg to the net log *)
    let net_l$_3$ := (ghost_result nid result) :: net_l$_2$ in
    (* return the updated net log and the result *)
    (adt{net_l := net_1$_3$}, result)
  else None.
\end{lstlisting}

\caption{A simplified log construction function example. It logs local and network events of a node to the network log and calls the log replay function to check state changes.}
\label{fig:spec}
\end{figure}

\subsection{Proving Global Properties}
\label{subsec:safety_verification}

	The global state transition system in the GHOSTLAYER models a distributed system with multiple concurrent Paxos clients and acceptors from the viewpoint of the global network to enable the distributed protocol verification. It includes (network) log construction functions, a (network) log replay function, and a global state. The log construction function models how each client/server operation affects the network; it governs the communication pattern of each node in the network log to define the Paxos protocol. The log replay function constructs the global state, which is a snapshot of the entire distributed system state or a combination of Paxos-related states in all nodes, by interpreting network events in the network log. Log construction and replay functions are derived from wormclient and wormserver specifications and their refinement relations for the derivation are verified. 


Log construction functions interact with the network log and the global state to introduce new network events in the network log. To record local state changes of a node which do not involve network operations, ghost messages are written to the network log. Log construction functions use the log replay function to learn and use state changes incurred by other concurrent nodes and itself (Figure~\ref{fig:spec}).


The global state transition system includes a \textit{network log replay function}, which can reason about all state transitions in the distributed system. The log replay function maps a network event in the network log with the state transition function to reconstruct the global state. Figure~\ref{fig:spec} shows a simplified specification of a write function in the ghost layer that updates the global state and writes and reads the network log. Based on this global view of the system, we verify the properties of WormSpace.

The log replay function by itself can replay all behaviors and state changes of a distributed system step by step from the global network log. Based on this capability we prove the Paxos-based safety/immutability property of WormSpace:
\\
\textit{\textbf{Theorem 1.} Once a value is written to a WOR, the value in the WOR never changes.}


\noindent To prove Theorem 1, we prove the following key lemma: 
\\
\textit{\textbf{Lemma 1.} Given a valid network log $\ell$, if there exists a Paxos round $n$ where a value $v$ is successfully written to a WOR $r$, any following write to $r$ in Paxos rounds $n' > n$ in the log $\ell$
can only attempt to write $v' = v$.}

\noindent\\
	The valid network log is the log that preserves verified invariants such as communication patterns derived from log construction functions.
Lemma 1 is proved by induction on writes in the log using other supporting lemmas: e.g., $n'$ is unique and is monotonically increasing, the Paxos-phase-1a/capture at round $n'$ on $r$ returns the written value $v$, etc.
Based on Theorem 1, the immutability and uniqueness of WOS allocation (including leader/sequencer election for WormPaxos/WormLog) and WOS trim are easily verified. 




\subsection{Top-Level Theorem of WormSpace}

	The top-level theorem that we prove for WormSpace is, \\
\textit{\textbf{Theorem 2.} $\forall t, L_{TCB} (i_{AllWormSpace} \oplus t) \sqsubseteq L_{WormSpace} (t)$}, \\
%where $t$ is the context and $i_{AllWormSpace}$ is the implementation of all WormSpace layers combined. The contextual refinement proof between all adjacent layers are used as lemmas. Theorem 2 guarantees the correctness of the code and verified Paxos properties in the GHOSTLAYER hold in WormSpace. 
where $t$ is the context and $i_{AllWormSpace}$ is the implementation of all WormSpace layers combined. The contextual refinement proof between all adjacent layers are used as lemmas to guarantee the correctness of the entire code. Theorem 2 also guarantees that the verified Paxos properties in the GHOSTLAYER hold for the WormSpace implementation.



\subsection{Reusability and Linking}
\label{subsec:proof_effort}

Because the GHOSTLAYER encapsulates the distributed nature of WormSpace, the verification of WormPaxos, WormLog, and WormTX does not have to reason about complex Paxos proofs. % is as easy as proving non-distributed code.
The verification of any additional distributed protocols above WormSpace reuses the same network model, but requires a new GHOSTLAYER. Protocols at different levels of the stack are independently verified within separate GHOSTLAYERS; invariants of interfaces to the protocol and contextual refinement proofs guarantee non-interference among protocols. 

The top-level theorems that we prove for WormPaxos, WormLog, and WormTX are in the same format:\\
\textit{\textbf{Theorem 3.} $\forall t, L_{WormSpace} (i_{WormApp} \oplus t) \sqsubseteq L_{WormApp} (t)$},\\
where WormApp can be one of WormPaxos, WormLog, and WormTX.
By reusing Theorem 2 and transitively combining it with Theorems 3, applications are guaranteed to be correct with respect to all layers of WormSpace and to encapsulate verified Paxos properties. Similarly, Theorem 2 can be reused to verify any system in Section~\ref{sec:sysbackground} to guarantee WOR semantics, if we use WormSpace as a building block. 


To enable end-to-end verification of WormSpace, WormPaxos, WormLog, and WormTX, we link WormSpace to CertiKOS. The linking requires contextual refinement proof between two interfacing layers.
When linking independently developed and verified software pieces together, it is important to check that the specification exposed by the lower layer matches the expectations of the higher layer. Since WormSpace and its applications were co-designed, such a consistency check was unnecessary, but linking WormSpace to CertiKOS required careful consistency checks. Once we link WormSpace with CertiKOS the correctness of WormSpace and the applications is guaranteed from the bottom-level ($L_{x86asm}$) of the OS without any side-effects~\cite{shimlayer}; this verifies and guarantees,

\noindent\textit{\textbf{Theorem 4.} $\forall t, L_{x86asm} (i_{CertiKOS} \oplus i_{WormSpace} \oplus i_{WormApp} \oplus t) \\
\null$\qquad\qquad\quad\quad$ \sqsubseteq L_{WormApp} (t)$}. 

The extensibility of WormSpace verification to applications and the OS is difficult for other verified systems~\cite{ironfleet, hyperkernel} to achieve. Especially, it is unnatural and difficult to support contextual refinement, which is based on high-order logic, when the verification tool is based on a SMT solver or first-order logic (e.g., Dafny~\cite{dafny} and Z3~\cite{z3}). 



%% paxos verification example

\section{Paxos Verification}
\label{sec:paxos-verification}

As a running example of our approach, 
we implement Paxos using C, 
and then prove functional correctness and safety of our implementation using Coq.
Figure~\ref{fig:paxos-proposer-example} shows the source code for slightly 
optimized version of 
Phase 1 (Prepare) of Paxos  (in  Fig.~\ref{fig:paxos-pseudocode}
with  multiple auxiliary functions in underlying layers. 
Due to the space limit, we focus on those two functions in this section.
The whole implementation can be found in the follow site: \url{https://sites.google.com/site/witnesspassing/}.


\begin{figure}
\begin{minipage}{\linewidth}
\noindent
\begin{multicols}{2}
\lstinputlisting[numbers = left, language = C]{source_code/paxos_verify_example.c}
\end{multicols}
\end{minipage}
\caption{Parts of Paxos Implementation to build  $LPAX_{impl}[i]$}
\label{fig:paxos-proposer-example}
\end{figure}

\newcommand{\paxset}{\mathrm{SET}_{\mathrm{Paxos}}}
\newcommand{\paxpropset}{\mathrm{PSET}_{\mathrm{Paxos}}}
\newcommand{\paxaccset}{\mathrm{ASET}_{\mathrm{Paxos}}}

\subsection{Environment}
\label{subsec:environment}

Verifying Paxos requires assumptions on environment.
We assume there is an arbitrary number of acceptors and proposers in the system.
Each acceptor or proposer use unique integer number as their node identifiers, and two sets for node identifiers of acceptors and proposers
are represented by $\paxaccset$ and $\paxpropset$, respectively. 
For the simplicity,  we have omitted leaner in the system.
We also assume that the network communication can trigger packet duplication, delay, reordering, and loss as our network model assumes. 

We assign five channels for the communication of Paxos verification. 
The first channel is for the communication from proposers to acceptors. 
Proposers can add send packets and acceptors can add receive packets in the channel. 
The second  and the third  are 
for the communications from acceptors to proposers and between proposers, respectively.
The third is for the distributed applications based on our verified Paxos. 
We briefly discuss how we use them in Sect.~\ref{subsec:extensibility-of-verified-paxos}.
The fourth channel is for writing ghost packets.
In our implementation, only proposers can write ghost packets in this channel because acceptors do not have any purely local transitions.
The last one is the reserved one for other services in the future, and it can be divided into multiple subchannels later.

We do not assume the possibility of message corruptions as well as the existence of adversaries. 
All nodes that are participated in Paxos in our implementation are either acceptors or proposers, and they are honest.


\subsection{Functional Correctness}
\label{subsec:functional-correctness}

Verifying functional correctness is based on the  CCAL
as discussed in Sect.~\ref{subsec:link-with-low-level-code-verification}.
The bottom layer based on the node $i$, $LPAX_{btm}[i]$, works as shim of our verification that contains network related primitives 
including ``send'' and ``receive''.
Based on the bottom layer, 
the purpose of functional correctness proof is providing 
the abstraction layer,  $LPAX_{impl}[i]$, which contains the primitives that 
corresponds to the functionality of Paxos,
``prepare'' and ``write'' primitives for proposers and 
accept handlers that can be divided into ``prepare handler'' and ``accept handler'' primitives for acceptors.

The layers for the functional correctness proof are divided into three parts.
The first part is for the abstraction of Paxos related local states. 
Those layers add several setter and getter primitives that access 
the local states such as \texttt{get$\_$paxos$\_$rnd} and  \texttt{set$\_$paxos$\_$rnd} in the Fig.~\ref{fig:paxos-proposer-example}.
The second is for the functional correctness of the implementation of acceptor handlers, and the top-most layer of the second part provides primitives for accepter handlers including \texttt{prepare$\_$handler} in it. 
The last part is for providing the abstraction layer that contains primitives corresponding to the implementation of proposer's functionality, 
and the top-most layer of the third part is $LPAX_{impl}[i]$, which contains all primitives 
related to Paxos in it. 

This process also includes network reductions.
Two kinds of network reductions has been done while 
building layers for proposer's codes,
and they are exactly same with the network reductions in Fig.~\ref{fig:network-reduction}.
The interface for the one of them are shown in the Fig.~\ref{fig:paxos-proposer-example}, 
\texttt{broadcast}. 
The primitive only record one send packet from $i$ and other multiple packets via the result of the environmental context 
in the network history, while the underlying implementation performs multiple 
send operations to broadcast the same messages to all acceptors.

As a result, we provide the theorem 
that shows the implementation of Paxos ($Pax_{impl}$) plus a program $P$ in C or Assembly running on the shim layer $LPAX_{btm}[i]$ faithfully refines  the program $P$ running on $LPAX_{impl}[i]$.
The functional correctness theorem is formally defined in Thm.~\ref{thm:contextual-refinement}.

\begin{theorem}
\label{thm:contextual-refinement}

\noindent$\forall{}P.\sem{LPAX_{btm}[i]}{Pax_{impl} \oplus P} \refines_R \sem{LPAX_{impl}[i]}{P}$
\end{theorem}

\ignore{
The set also can be used when we distinguish the communication channel too. 
By using the set, layers can horizontally combined layers that are disjoint and one of two is 
for acceptors and the other is for proposers. 

The layers for the acceptors only required us to build two layers due to its simplicity. 
For the proposers, however, broadcast and wait$\_$quorum involves multiple network communications in them, and it requires the simplification of network log and environmental context discussed in Sect.~\ref{subsec:link-with-low-level-code-verification}. 
Two simplifications has been used, in the broadcasting  and the busy waiting. 
Both contains the loop in their implementations and it could increase the complexity of the 
global state transition machines if they exist in the high-level specifications. 
To reduce the complexity, we have simplified the log and the environmental context in two places. 

}

\subsection{Write-Witness-Passing Semantics in Paxos}
\label{subsec:witness-passing-semantics-in-paxos}

Functional correctness proof using contextual refinement provides a way to show that our implementation is correct and matched with the specifications. 
It also provides a certified abstraction layer $LPAX_{impl}[i]$ that contains the all primitives related to Paxos in it.

The next step is introducing the distributed transition semantics with write-witness-passing 
via two separate steps discussed in Sect.~\ref{subsec:distributed-transition-semantics} and Sect.~\ref{subsec:distributed-transition-semantics-with-witness-passing}.
To provide them, we define the network replay function $(\replay_{PAX})$ that handles all necessary cases for Paxos. 
For example, 
the following case in the replay function:
\lstinputlisting[numbers = left]{source_code/paxos_acceptor_case.v}
corresponds to the \texttt{prepare$\_$handler} in Fig.~\ref{fig:paxos-proposer-example}, and thus the 
high-level specification of the aceptor handler  primitive in  $LPAX_{impl}[i]$.

We also introduce two ghost packets that help us to build the replay function. 
One of them is related to \texttt{set\_new\_rnd} and  \texttt{qrm\_prepare\_reset}   
in Fig.~\ref{fig:paxos-proposer-example}, which locally increase 
the round number of node $i$ and initialize the local state before 
starting the prepare phase. 
By following this approach, building the layer $LPAX_{abs}[i]$, which enables us to 
calculate all local states for the node in $\paxset$ ($\paxset = \paxpropset \uplus \paxaccset$), is possible.

Adding the write-witness in the definition of $LPAX_{abs}[i]$
is the next step of our write-witness-passing style verification. 
Adding the write-witness is how we instantiate the variables and abstract functions in Fig.~\ref{fig:witness-witness-formal}.Some of them
The concrete definition of some of those variables and functions are as follows:
\begin{itemize}
\item $\termnum$: The ballot number in Paxos is the round number. Round numbers are elements of natural numbers, thus 
they are totally ordered. 
\item $\dstate$: In our implementation, the stored value is unsigned integer numbers. This is for the simplification. Using the value with more
complex types are possible for the $\dstate$.  
\item $\nodeid$: we use  integer numbers as node identifiers. 
\item $\accsset$: $\paxaccset$, which is a set that contains all node identifiers for acceptors in the system. The value can be treated as 
a constant value in our Paxos verification because our implementation does not allow reconfiguration.
\item $\witness_{\mathrm{el}}$ : The instantiate of this witness element is possible by using the above definitions. Among the all fields in 
the witness, we remove $\accsset$ for this Paxos verification because that is a constant value.
\item $\isquorums$: The concrete definition for Paxos is checking the cardinality of the set $accs$ according to the 
the cardinality of $\accsset$. If the number of elements of $\accsset$ is $2N + 1$, the number of elements in $accs$ 
needs to be bigger than  $N + 1$ if the write attempt wants to proceed its operation.
\end{itemize}

We provide all of those definitions in our implementation. 
For example, the followings are for ``$\isquorums$'', ``$\witness_{\mathrm{el}}$'', and ``$\witness$`` written in Coq:
\lstinputlisting[numbers = left]{source_code/paxos_witness_definition.v}

With the instantiated definitions, we also define  the network replay function $(\replay_{PAX\witness})$ that returns the state with the following type,
$$\mbox{option} \ \{\lstate{{PAX\witness}_i}~\vert~ \forall i.\ i \in \paxset\} \ \ \ \ \ \ \ (\mbox{where}\ \lstate{{PAX\witness}_i} : 
(\mathbb{N} \times \mathbb{N} \times \mathbb{Z} \times \witness))$$
,and also possible to build the layer $LPAX_{abs\witness}[i]$ that uses the network replay function, $(\replay_{PAX\witness})$.
Providing the contextual refinement theorem from the bottom layer to the layer is also possible, and the theorem is
stated in Thm.~\ref{thm:contextual-refinement-witness}.

\begin{theorem}[Contextual Refinement]
\label{thm:contextual-refinement-witness}
\noindent$\forall{}P.\sem{LPAX_{btm}[i]}{Pax_{impl} \oplus P} \refines_R \sem{LPAX_{abs\witness}[i]}{P}$
\end{theorem}


\subsection{Paxos Safety}
\label{subsec:paxos-safety}

The main safety property that we verify with our implementation is about the consistency in the consensus which 
is stated in Thm.~\ref{thm:immutability}.

\begin{theorem}[Immutability]
\label{thm:immutability}
Assuming that there are two nodes  $i$ and $j$, which are members of $\paxpropset$,
and the corresponding local states are  $\lstate{PAX\witness_i}$ and 
$\lstate{PAX\witness_j}$, respectively.
If both nodes succeed in the Phase 2 (Write) of Paxos, 
the stored value in both local states are same, which can be stated as\ $\lstate{PAX\witness_i}.\dstate  = \lstate{PAX\witness_j}.\dstate$.
\end{theorem}

Proving the theorem requires two important sub lemmas. 
The first one is related to the uniqueness of the write attempt with a particular ballot number, $\termnum$.

\begin{lemma}[write$\_$once]
\label{lemma:write-once}
Let's assume that there are two write send packets, $\msg_{\witness_1}$ and $\msg_{\witness_2}$
in the network ${\networklog}_{\witness}$. 
If those two messages are associated with the same round number $\termnum$, 
their messages are same as stated as $\msg_{\witness_1} = \msg_{\witness_2}$.
\end{lemma}

Proving the lemma is related to the fundamental property of $\termnum$.
Since all ballot numbers are unique as discussed in Sect.~\ref{subsec:distributed-transition-semantics-with-witness-passing}, 
We can prove that there will be only one write message in the network history that corresponds to the particular ballot number. 
The proof of this lemma does not facilitate the properties of  write-witness-passing. 
The next and the most crucial lemma, however, use properties of  write-witness-passing in the proof.

\begin{lemma}[write$\_$lift]
\label{lemma:write-lift}
Assuming that there is a round number $\termnum_1$, which succeed the write on majority of acceptors with the messages
$\msg_{\witness_1}$ 
Then, for all write attempts associated with the round number $\termnum_2$ always try to write  
the same value on acceptors when $\termnum_1 < \termnum_2$, which is stated as 
$\msg_{\witness_1}.\dstate = \msg_{\witness_2}.\dstate$.
\end{lemma}

Proving the lemma requires the induction on all write attempts between $\termnum_1$ and $\termnum_2$. 
To apply the induction on the number of write attempts, we make an auxiliary function, \texttt{nth$\_$send} as follows:
\lstinputlisting[numbers = left]{source_code/nth_send.v}


Proofs using the number achieved by  \texttt{nth$\_$send} implies that all cases are associated with a particular write attempt. 
In particular, 
for each write attempt in between $\termnum_1$ and $\termnum_2$,
we need to prove that 
there is a proper witness for the write in the network history, 
which satisfies the fact that the witness is  a subset of acceptors that raise the quorum
as well as the fact that the value of $\msg_{\witness_2}$ comes from the acceptor who is a participant of the quorum. 

To prove them, we facilitate generic properties of our write-witness-passing for Paxos,
which is the instance of the properties in Fig.~\ref{fig:witness-witness-formal}.
Those properties can be achieved with the refinement when adding witnesses in our states. Thus we do not need to search all the network history to find out them. 

As have shown in this section, 
our safety proof mostly relies on local aspects of the distributed systems. 
One lemma, Lemma~\ref{lemma:write-once} require us to 
prove global invariant of the distributed protocols, 
but other parts do not need  global reasoning at all because 
each write-witness provide the proper projection from the whole network history to the pieces of evidence regarding 
the attempt. 
Besides, 
proofs do not need any complex case analysis 
for network errors such as packet duplication, reordering, and lost.
Whatever some of those errors occur with the write attempt, 
each write always have the valid write-witness that always satisfy the generic properties. 
In this sense, write-witness-passing makes our proofs tremendously easy.  

\ignore{
\begin{theorem}[Contextual Refinement with Immutability]
\noindent$\forall{}P.\sem{L_{abs}[i]}{Pax_{impl} \oplus P}$
\end{theorem}
}

\subsection{Extensibility of Verified Paxos}
\label{subsec:extensibility-of-verified-paxos}

Distributed system verification has a high cost compared to design and implementation.
Some distributed systems, however, are usually served as bases of other distributed systems. 
For example, Paxos usually works as a tool to provide consensus for other distributed programs.
Thus, providing the extensible proofs for them are desirable. 
Our verification approach provides extensible, scalable, and reusable proofs not only for the functional correctness but also for safety properties. 

The immutability property provides that all state transitions in $\sem{LPAX_{abs\witness}[i]}{P}$ with a program $P$ always preserve consistency related to the Paxos states. 
In the layer, there are no direct transitions that modify the Paxos states except the transitions with the replay function,  $\replay_{PAX\witness}$.
Therefore, the layer built on the of the layer can access the states related to Paxos by only calling the primitives that use $\replay_{PAX\witness}$.
Thus, the above layers on $LPAX_{abs\witness}[i]$ always satisfy consistency property regardless of the program $P$. 


%\section{Extensibility of Write-Witness-Passing Semantics}
\label{sec:witness-passing-semantics-with-paxos-variants}
In this section we sketch how write-witness-passing can be applied to several distributed systems.
We mention how our write-witness-passing help the verification of variants of Paxos as well as other distributed protocols.

\subsection{Multi-Paxos}
\label{subsec:multi-paxos}
Paxos as it was originally presented only reaches consensus for a single value, but real systems
typically need to decide values for a set of slots.
These multiple-decision versions of Paxos are broadly referred to as Multi-Paxos, and although there
are many variants, they share some common features.
The simplest implementation is to run a new instance of Paxos for each slot, but better performance
can be achieved by choosing a node to be the leader.
Choosing a leader requires a consensus algorithm itself, but once one has been elected, if it is
guaranteed to be the only node trying to write to a set of slots, then it can skip the prepare phase after the
first round.

Because the safety of Multi-Paxos depends on very similar invariants to Single-Paxos, witness-passing
also works in much the same way.
The witnesses collected from the prepare phase remain the same, but now they will be reused in multiple
writes.
To demonstrate that this is safe, the leader should be able to prove that it actually won the election.
We can use witnesses again for this purpose, but this time they are collected from the participants in the
leader election instead of the acceptors.
How exactly the leader is given these witnesses depends on the details of the leader election algorithm,
but, by the end of the election, it should have evidence that shows it received enough votes to win.

\subsection{Vertical Paxos}
\label{subsec:vertical-paxos}
Paxos is designed to tolerate some amount of node failures, but each crash brings the system closer
to this threshold.
It is therefore desirable to allow the system to be reconfigured by adding new nodes and replacing
old ones.
In standard Paxos this can only happen between writes to different slots, never between rounds for
the same slot.
Vertical Paxos \cite{vertpaxos} is designed to support dynamic reconfiguration of acceptor and leader nodes,
where dynamic means consecutive rounds for the same slot can have different configurations.
Configurations are decided round-by-round by a separate master process and include the set of
acceptors and the single leader to be used that round.
Unlike standard Paxos, Vertical Paxos allows the quorums of acceptors for the prepare and write
phases to differ with the restriction that all write quorums must have a non-empty intersection
with all other quorums.
The common case is to let every acceptor be a prepare quorum on its own, which implies that the only
possible write quorum is the set of all acceptors.
The main challenge that dynamic reconfiguration introduces is that leaders must learn what value an
acceptor from an earlier round and possibly a different configuration voted for.
This can prevent removing old acceptors from the system because they might still need to be consulted in the
prepare phase of a later round.
To solve this, the master process tracks the highest \emph{complete} round number, which is a round
in which a value was successfully written, or one where any value can be proposed.
With this extra information, the leader is guaranteed to only have to communicate with acceptors from
rounds as most as early as the last complete round, so older acceptors can safely be removed from the system.

Vertical Paxos is a variant of Multi-Paxos, so write-witness-passing can be used in largely the same way
as before for the communication between a leader and the acceptors.
The key difference is in the communication between a leader and the master process.
To guarantee that a write is safe, the max complete round sent to the leader must actually be complete.
If the master, in addition to sending the complete round number, also sends a witness demonstrating its
completeness, then checking that this invariant holds during verification becomes almost trivial.
The master would update this witness every time it updates the max complete round by simply storing the
message indicating that a successful write just took place in that round.

\subsection{Raft}
\label{subsec:raft}
Raft \cite{raft} is an alternative to Paxos with leader election and reconfiguration built in.
Every node stores a log of commands along with the terms in which they were decided.
New commands are proposed by clients to the leader, which attempts to get the other nodes to store
that command in their logs.
Initially all nodes begin as candidates and an election round begins with one of them broadcasting a
request for votes along with its current term number, the length of its log, and the term of the last
element of its log.
Each node votes yes if it has not already voted this term, the candidate's term is larger than its
own, and the candidate's log is at least as up-to-date as its own.
A candidate becomes the leader when it receives a yes vote from a majority of nodes, at which point
it begins sending out heartbeats with its term number to inform the other nodes of the result of the
election.
To replicate a log entry, the leader broadcasts a message with its term, the new values to be
replicated, and the index and term of the log entry just before the new ones.
A follower can reject the request if the leader's term is too small, or if its own log does not match
the leader's at the index before the new values.
Otherwise it updates its log by first deleting any entries that conflict with the new values as
well as everything that follows the conflict, then appending the rest of the new values that are not already in the
log.
Reconfiguration in Raft is handled by first switching to a joint configuration that contains the union of the nodes
from both the old and new configurations.
Configuration changes are communicated as log entries and are replicated as normal except that consensus now requires
agreement from a majority of nodes in the new configuration.
Once the leader sees that the joint configuration has been committed, it can try to commit the new configuration,
and only when that has succeeded can old nodes be removed.

Write-witness-passing can be applied to all three of Raft's main components.
In leader election, each node includes a witness with a copy of its log in its vote so it is easy to check that a
new leader must have in its log all the values committed by a majority of nodes up to that point.
Witnesses also include the logs during replication, which guarantees that a majority of the nodes' logs match the leader's exactly for
the newly committed entries and the terms match for the previous entry.
Maintaining the invariant that only one value can be decided per term implies that the entire
log must then be the same.
Because reconfiguration reuses much of the machinery from log replication,
the write-witness-passing technique can also be reused without much change.
The main modification that must be made is that upon a configuration change,
the witnesses' definition of $ASET$ (see Figure \ref{fig:witness-witness-formal}) must be updated.

\subsection{Other Systems}
\label{subsec:other-systems}
Write-witness-passing is an effective reasoning tool even for distributed systems other than Paxos variants.
\ignore{ % Chain Replication doesn't fit this pattern so maybe it's just confusing to say this
Any system that fits the pattern of broadcasting a request then collecting responses can benefit
from witness-passing.
}
For example, two-phase commit is a protocol where a single coordinator tries to atomically commit a
transaction among multiple participant nodes.
Similar to Paxos, in the first phase the coordinator tries to get confirmation that the transaction
can be committed by asking each node.
If every node responds that the command is valid locally, then the coordinator sends back a
commit message; otherwise it tells the participants to abort and roll back the transaction.
Using write-witness-passing, the responses from the participants would contain a piece of their
state as evidence that the transaction is valid.

Another system that write-witness-passing fits into is chain replication \cite{chainreplication}.
In this protocol, all writes are sent to a single head node.
The head then forwards the value to the next node in the chain, which continues passing it down
until it reaches the tail.
Reads can only be performed at the end of the chain, so a value is ``committed'' once it reaches the
tail.
A witness in this case could contain the sequence of write messages from all of the earlier nodes in
the chain.
This would make it clear that the value stored at the tail is the same as the one from the original write message.


\ignore{
\subsection{Extensibility of distributed protocols}

Most distributed transactions assume correctness of underlying consensus protocols to guarantee the basic safety. 
In this sense, re-usability of the proofs as well as implementations are desirable.
We use \textit{contextual refinement} approach to enable this extensible feature of our verification.

\jieung{This section may be mostly related to contextual refinement...}
\ignore{
\jieung{EPaxos}
\jieung{2PC}
\jieung{CorfuDB}
}
}
%\section{Evaluation}
\label{sec:evaluation}

\para{Proof Effort}

Our verification of Paxos can be divided into two main parts: functional correctness and safety.
All of the proofs of functional correctness took approximately 1.5 person-months, in large part thanks to the power of the CCAL framework.
The safety proofs using write-witness-passing were completed in 2 person-months with a small amount of trial-and-error.

\para{Comparison with non-witness-passing proof}

Our initial attempt to prove the safety of Paxos did not use witnesses and instead went by induction on the round number or the network log.
We quickly found that this strategy created large, intractable proofs due to the difficulty of trying to reconstruct global state
from only partial local information.
Adding in write-witness-passing required only minor changes to our specifications, but made many of the proofs significantly easier.
Because we never completed the proofs without write-witness-passing, we cannot directly compare how long each method took;
however, before using witnesses we struggled to make any significant progress, and afterwards progress was steady.

\para{Hand-written Proof vs. Mechanized Proof}

The overall structure of our Paxos safety proof is similar to some hand-written proofs.
However, these proofs often proceed by induction on the round numbers or network transitions,
which, as stated earlier, we found to be difficult in a mechanized proof.
The nature of machine-checked proofs means that every corner-case and detail must be accounted for,
and where a hand-written proof might gloss over some seemingly trivial fact, the full proof can turn out
to be surprisingly involved.
The reward for this extra work is a more rigorous proof that can be linked to the actual implementation,
thus providing an even stronger assurance of correctness.

% END CONTENTS



\chapter{Providing Generic Verification Toolkit for Leader Based Distributed Systems }
\label{chapter:witness-passing}

%Distributed systems are notoriously complex due to the many possible interleavings of their coarsely-connected instances as well as the possibility of errors in both  those instances and the network environment. For these reasons, verification of distributed systems is desirable to remove the possibility of bugs and guarantee their safety and correctness. However, much current verification work still requires a great deal of effort and sometimes has limitations.

We present a verification approach that uses \textit{write-witness-passing}, which is simple but novel in distributed system verification. It is a scalable, reusable, and extensible approach that can be directly linked with the low-level implementations of distributed protocols through contextual refinement. Write-witness-passing can capture the common behaviors of many distributed protocols, and provides both a simple way of understanding the protocols as well as an easy methodology for verifying them.

To demonstrate how write-witnesses work, we verify the functional correctness and safety of Paxos, one of the most famous consensus protocols. We implement the key routines of Paxos in C, and use Coq to verify both the functional correctness of the implementation as well as the safety properties of the protocol within less than 4 person-months. We also describe how we can apply our approach to other distributed protocols to illustrate its generality.

%\section{Introduction}
\label{sec:intro}

\topic{Distributed systems are difficult to verify.}

Distributed systems are challenging to formally model, reason about, and verify
due to their inherent concurrency and weak failure assumptions. Distributed
nodes run concurrently over an asynchronous network and both the node and the
network link can fail at any moment. To clear these hurdles a distributed
protocol must have sophisticated error handling and typically relies on implicit
global invariants, which complicates formal reasoning about the system.

\topic{There is a need to verify multiple distributed systems for practical
reasons.}

While even verifying a single distributed system is challenging, in practice
distributed applications rely on several distributed systems. An application
might employ different distributed systems for distinct functionalities (e.g.
consensus~\cite{vivaladifference}, distributed transactions~\cite{gray:2006},
and distributed locks~\cite{chubby, zookeeper} as part of a high reliability
distributed database) or it might use systems that achieve the same goal (e.g.
multi-Paxos~\cite{paxosmadesimple, rvrpaxos} vs. Raft~\cite{raft}) in different
parts of the application depending on performance considerations or simple
preference. Therefore, to realize a verified distributed system environment,
methodologies to extend formal reasoning to multiple distributed systems are
necessary.

\topic{Leader based systems are widely used but is not studied well as a whole}

We find that distributed systems that realize strong semantics are typically
designed under a common pattern: systems exploit a leader node
(or a centralized coordinator) explicitly or implicitly to coordinate
distributed state changes. Indeed, for simplicity of management and understanding,
this leader-based scheme is commonly used to implement critical distributed
functions. For example, multi-Paxos and Raft elect a leader to replicate states
across multiple nodes, two-phase commit employs a transaction manager to
coordinate transactions over multiple resource managers, and coordination
services grant a lock to a requester to have mutually exclusive access to a
distributed shared state.

While the leader-based scheme has a huge presence in the design of distributed systems,
little attention has been paid to formally modeling and studying the common properties
of leader-based distributed systems as a whole. Previous work mostly focused
on verifying individual leader-based distributed systems~\cite{ironfleet, cppraft}
or on general approaches to verify arbitrary distributed systems by relaxing network
models, finding better automation schemes, and isolating and modularizing the
proof structure~\cite{verdi, disel, modular}. We argue that formally
modeling and studying the class of leader-based distributed systems can yield greater
insight into and expedite verification of many systems in that category.

\topic{We model a leader based distributed system using a common specification.}
\subtopic{Covers all leader based system.}

Therefore, we propose \sysname{}, a verification framework for
leader-based distributed systems that promotes the understanding of
a variety of leader-based distributed systems and facilitates the verification
of each system. \sysname{} demonstrates how seemingly very different distributed systems,
such as multi-Paxos, Raft, and distributed locks, can be modeled and verified under
a generic specification as long as they follow the leader-based design.
\sysname{} includes a formal model of leader-based distributed systems and
proofs of generic properties of the model.
It also acts as a proof template that can be instantiated with a particular leader-based
system to show that the general properties hold for it.

%\jieung{hope to shorten the previous paragraph, but will rephrase it later}

%In this paper, we focus on generalizing the leader-based distributed system
%(i.e., the distributed system that uses the leader-based protocol) to propose a
%formal model to reason about and verify various leader-based systems under a
%common framework. The common framework includes a general specification of the
%leader based distributed system, a proof of common characteristics of the
%system that are exposed by the specification, and detailed low-level
%specifications and their proof of a particular leader-based system that one
%wants to verify using the framework.

We identify a few common characteristics of leader-based distributed systems:
first, they use a logical clock under different names such as round number,
term, timestamp, or epoch, to order operations and to tag the leader's
leadership period;
and second, they take strongly consistent steps in the sense that a
sequence of successful operations by the leader takes effect and is made visible
system-wide in the same sequential order (a linearizable order).
We propose a generic specification that models all leader-based systems using a
term number $\termnum$, template functions $\tplfunction$, and
a version number $\seqnum$. $\termnum$ represents the term in which a leader's leadership is valid,
$\tplfunction$ describes common behaviors of leader-based distributed systems,
and $\seqnum$ is tagged to the system state to order successful calls to $\tplfunction$.

\subtopic{Intro to the specification.}

\sysname{} models leader-based systems using two template functions:
1) $\tplldrfunction$, which elects a new leader of the system and
2) $\tplopfunction$, which the elected leader issues over the entire system
to mutate the state. These functions work as a high-level specification or a
template to elect a leader and execute operations over the entire system.
$\tplldrfunction$ and $\tplopfunction$ take the term number $\termnum$ as an identifier
and corresponding state update functions $\gldrfunction$ or
$\gopfunction$ that specify how a system modeled under \sysname{}
actually elects a leader or reaches a new state, respectively (Figure~\ref{fig:process-flow}).
$\tplfunction$ is designed to be generic enough to host most
leader-based distributed system and $\tplfunction$ and $\protocolfunction$
constitute the full specification of a leader-based distributed system.

%The state $\mathcal{S}$ is the distributed system which encapsulates a
%distributed state $\dobj$.
%$\mathcal{G}$ should implement strongly consistent semantics
%such that $\mathcal{F}$ that uses $\mathcal{G}$ can mutate the distributed
%state $\dobj$ atomically.

%\wolf{is the distinction between $\mathcal{F}$ and $\mathcal{G}$ important? Also, could we give the function types?}
%
%For example, to model multi-Paxos using \sysname{}, $\ldrfunction$
%elects one node in $\fullset$ as the leader using a
%protocol defined by $\gldrfunction^{mp}$ and $\opfunction$
%mutates $\dobj$ by running Paxos protocols to append the state
%changes to $\dobj$ into the state machine replication log using
%$\gopfunction^{mp}$. One of the main roles of $\gopfunction^{mp}$ is to
%implement consensus protocol on a single Paxos instance which is an entry of the log.
%The correctness of $\gopfunction^{mp}$ can be verified similar
%to how other literatures~\cite{sdpaxos, dpaxos, paxosepr} have verified Paxos consensus,
%but \sysname{} extends this verified single step to guarantee the correctness of
%an infinite sequence of these steps to model how distributed systems host and
%update a distributed object $\dobj$ in a general way.
%\sysname{} allows any leader-based distributed systems to plug in their
%behaviors on $\fullset$ to maintain $\dobj$ in a form of the function
%$\protocolfunction$ as illustrated in Figure~\ref{fig:examples} and detailed in Section~\ref{sec:xxx}.
%\wolf{I don't think this paragraph does much to clarify how one can use \sysname{} or what it gains you.
%It makes it sound like you just verify your system like usual and then plug it in to $\gldrfunction$
%or $\gopfunction$ but it doesn't explain how that helps}
%\wolf{Maybe this paragraph should be cut and replaced with a pointer to a later section (3 maybe)
%for more details}

%The model expressed by $\mathcal{F}$ and $\mathcal{G}$ is similar to
%to the specification of CASPaxos~\needcite{CASPaxos} which applies arbitrary
%functions atomically to a distributed shared object, or that of a state machine
%replication~\needcite{smr}.

\subtopic{Can verify linearizability of state mutation in a common way.}

%\wolf{can we list these assumptions?}
By making a few assumptions  about $\protocolfunction$
such as monotonic increasing of $\seqnum$ in the distributed object,
the immutability of previous committed values, and atomicity of the resulting
state, \sysname{} provides proofs of few common properties of leader-based distributed systems:
linearizability of state updates and the soundness (uniqueness at term $\termnum$)
of the leader.
Therefore \sysname{} provides a reusable proof template in the sense that these properties
are given ``for free'' once $\protocolfunction$ has been shown to satisfy the assumptions.
In addition to specifying the system behavior,
the state update function $\protocolfunction$ is responsible for exposing necessary
states and information, which we call the witness, to the template for the proof.

\topic{Internally we use a witness based verification approach.}
\subtopic{Witness makes easier reasoning.}

A witness is an abstract data structure that simplifies reasoning about and
verifying properties of distributed systems. Although we make use of witnesses
in \sysname{}, they are applicable in more general systems as well. A witness is
a logical component of messages sent between distributed nodes that share what
each node has seen so far. The contents of this data structure are
protocol-specific and could be the sending node's own state or an observed state
of another node. As this information is passed around and accumulates, it can
act as evidence to demonstrate how each node reached its current state and to
justify a node's next decision with respect to the protocol. The behavior of a
node can then easily be verified in relation to the entire distributed system.
% through primarily local reasoning by using the witness.
%and the linearizability of entire
%distributed system state update becomes easily deducible.
%\wolf{we should make the distinction between local and global reasoning clearer}

\subtopic{Witness also generates necessary information to the template.}
\subtopic{We show the witness based approach can be composed.}

Another advantage of witnesses is that they are composable. Witness data
structures that are used to verify different components of the system can be
combined to set the ground for verifying the entire system behavior or the
interaction between system components. For example, $\tplldrfunction$ uses every
node's vote history $\ldrwitness$ as a witness to elect a sound leader and to
justify that leader's use of $\tplopfunction$. Then, the composition of witnesses
$\opwitness$, which is created during the execution of $\tplopfunction$, and
$\ldrwitness$ becomes the source of what function $\protocolfunction$ passes to
the proof template to prove the linearizability of the entire system.

%\wolf{this could also probably be shortened/cut and deferred to a more detailed discussion in section 4}

Proving the correctness of the state update function $\protocolfunction$ is equivalent to
verifying a distributed system while leaving the verification for linearizability
and soundness of the leader. However, modeling and verifying different systems
under the same template allows a deeper understanding of subtle differences among systems.
A comparison of commonalities and differences in state update functions
and proofs yields such insight which we outline in Section~\ref{sec:examples}.

%\jiyong{Also compare with other distributed system verification framework
%papers: e.g. DISEL, other group of Paxos verification paper.}
%
%Few previous works also propose ways to verify distributed systems in a modular way or with compositionality.~\needcite{DISEL, OOPLSA17}.
%However, they do not provide a generic way to prove  \jieung{Is it true with OOPSLA17 and PLDI18?}
%key properties of
%leader based systems, such as
%mutual exclusiveness of leaders at a particular virtual time
%and the linearizabilty of the log that they have manipulated.

\topic{As an example we study a multi-Paxos system in detail}

To illustrate how a leader-based system is verified in \sysname{},
we take multi-Paxos as an example.
%and Raft~\cite{raft} and Lamport's
%distributed lock~\cite{lamportclock} as secondary examples.

\sysname{} uses the certified concurrent abstraction layer (CCAL)~\cite{concurrency}
approach and extends CCAL with a network model, witnesses, and the generic model
and proof template of leader-based distributed systems. Thanks to CCAL, we
can connect our proofs about the high-level specifications down to their C implementations
and we can easily modularize and compose the proofs and witnesses.
We demonstrate that \sysname{} can effectively model different
leader-based systems and facilitate the verification of their common and
distinct properties while also yielding great insight into these systems.

This paper makes the following three contributions:
\begin{itemize}[leftmargin=*]
\item We propose a general framework to model and verify leader-based
	distributed systems where the framework enables automatic verification
	of linearizability and soundness of the leader and easy comparison of
	different leader based distributed systems.
\item We propose a novel witness-passing scheme that facilitates
	reasoning about distributed systems in a modular way
	and provides insight for how distributed system protocols work and
	interact with each other.
\item We detail the safety proof of Paxos and the linearizability and leadership
	proof of multi-Paxos using our framework and compare the proofs with
	those of Raft, two-phase commit and Lamport's distributed lock using \sysname{}.
\end{itemize}

%The rest of the paper is organized as follows.
%Section~\ref{sec:background} provides background information about the
%leader-based distributed system and our verification scheme and
%Section~\ref{sec:spec} describes the specification of the generic model of the
%leader-based distributed systems.
%Section~\ref{sec:witness-passing-semantics} details the formal definitions of
%our verification approach using the witness.
%Sections~\ref{sec:paxos-verification}-\ref{sec:evaluation} provide examples of
%our framework and verification approach applied to multi-Paxos and other systems
%and evaluate our proof methodology.
%Section~\ref{sec:related} investigates the related work and
%Section~\ref{sec:conclusion} presents our conclusion.
%\jiyong{We can remove this paragraph if we need space.}

%=========================================
%
%\topic{Rationale behind our approach.}
%To understand why the write-witness-passing scheme is useful, one should first understand how distributed systems are designed.
%Distributed systems have evolved to hide complex protocols using simple abstractions and to send as little information as possible between nodes to save network bandwidth.
%Messages sent over the network may be lost depending on network assumptions and information received over the network is often discarded right after it is used to update any relevant local states.
%The code for each node often blindly executes an operation without the global view of the entire distributed system and assumes that all other nodes are working correctly.
%Therefore, the context that can be extracted from the code for a local node is typically not enough to reason about the validity of the node's state against the entire distributed system.
%
%
%
%%Compared to ad-hoc systems such as a peer-to-peer system with
%
%
%%Distributed systems, especially state machines form the underlying base of many applications these days.
%%Unfortunately, verifying these systems is difficult because of the inherent concurrency and the possibility of failure in both the nodes and the network.
%%For distributed nodes to collaborate and overcome these failures seamlessly,
%%distributed systems must employ sophisticated protocols.
%%\topic{Existing non-machine checkable proofs are less useful in practice.}
%%Although many of these distributed protocols have pencil-and-paper proofs of their correctness,
%%their subtle and complex nature makes them difficult to implement faithfully in actual code.
%%Even though the industry rigorously applies various testing strategies
%%for software development, there are continuous reports about distributed software bugs that can shut down entire data centers~\cite{awsdown, gmaildown}.
%
%\topic{Interactive theorem provers created opportunities to verify distributed systems.}
%Machine-checkable verification tools open up new opportunities to provide distributed systems
%with end-to-end correctness guarantees by
%verifying low-level implementations of distributed code and linking them with safety proofs of the abstract protocols.
%Specifically, theorem provers allow line-by-line verification of the code
%with support for partial proof-automation.
%Previous works have used
%both automated tools such as Z3~\cite{moura08} and
%Dafny~\cite{dafny} as well as
%interactive tools, such as Coq~\cite{coq}.
%\ignore{
%The most common tools are the ones based on SMT solvers, such as Z3\cite{moura08} and
%Dafny\cite{dafny}, which are well automated but works for only first-order logic and decidable problems,
%and Coq\cite{coq}, which is less automated but works on high-order logic.}
%
%\topic{But distributed system verification is still difficult.}
%Regardless of which tool one uses,
%verifying distributed system code with an interactive theorem prover requires much more work
%than with a hand-written proof.
%The verification must cover every low-level corner case that is mostly related to the underlying network error
%or optimization.
%In addition, the collective view
%of all distributed nodes and the network -- which we call the \globalstate{} --
%should be created and made available to the proof because key safety properties of distributed protocol
%always related to not with a single node but with all nodes in the system.
%Thus, even given a high-level proof of a theorem for a distributed protocol,
%there is still a significant proof burden to bridge the gap between the code and the model.
%
%\topic{Reasons why others have failed: verdi, ironfleet, sergey et al., disel, etc.}
%For this reason, proofs of distributed protocols done using interactive theorem provers often simplify the \globalstate{}
%and rely on additional tools or assumptions to fill in the missing pieces.
%For example, in Ironfleet's~\cite{ironfleet} model, all distributed nodes are connected via an asynchronous network,
%and they show that the code refines the \globalstate{}.
%However, in part due to the expressiveness limitations of Dafny, Ironfleet's verification relies on pencil-and-paper proofs to show
%that a realistic, arbitrarily-interleaved network refines their more restrictive \globalstate{}.
%Verdi~\cite{verdi} similarly models a global state but assumes an ``idealized'' network and carries out refinement proofs to show that the code refines the \globalstate{}.
%It then relaxes its assumptions by automatically applying valid transformations to the code, such as adding sequence numbers to tolerate packet duplication.
%However, starting with an assumption of an ideal network is not suitable for verifying systems such as Paxos~\cite{paxos} that assume a faulty network.
%% which the same authors showed the safety proof later in a separate paper~\cite{cppraft}. % TODO: fit this reference in another way
%Several other papers suggest methodologies to verify distributed systems using interactive theorem provers,
%but they focus on specific topics such as automation~\cite{modular} or isolating protocols~\cite{disel} and do not fully address this problem.
%
%\topic{We propose a new and easy-to-understand distributed system proving techniques: new global model + witness-passing.}
%To fill in this gap, we propose a \globalstate{} of a distributed system where the proof can be written entirely in Coq,
%and a novel write-witness-passing scheme that can promote the understanding of distributed system protocols and simplify the verification process.
%Our \globalstate{} includes an asynchronous network, and the states in the \globalstate{} are constructed by composing the operations and local state of each node.
%Write-witness-passing adds a logical data structure to the messages sent between nodes that remembers what each node has seen so far.
%The contents of this data structure could come from the sending node's own state or from observations of other nodes' state.
%As this information accumulates it can show how each node reached its current state, with evidence demonstrating the validity of each transition.
%Using these witnesses, the behavior of a node can easily be verified with respect to the \globalstate{} through primarily local reasoning.
%
%\topic{Rationale behind our approach.}
%To understand why the write-witness-passing scheme is useful, one should first understand how distributed systems are designed.
%Distributed systems have evolved to hide complex protocols using simple abstractions and to send as little information as possible between nodes to save network bandwidth.
%Messages sent over the network may be lost depending on network assumptions and information received over the network is often discarded right after it is used to update any relevant local states.
%The code for each node often blindly executes an operation without the global view of the entire distributed system and assumes that all other nodes are working correctly.
%Therefore, the context that can be extracted from the code for a local node is typically not enough to reason about the validity of the node's state against the entire distributed system.
%
%\topic{Our design: 1) Network model: we don't make any extra assumptions}
%Our \globalstate{}'s network is asynchronous and allows that packets can be dropped, delayed, reordered, interleaved, and duplicated, but never corrupted,
%which most other works on distributed system verification also assume.
%Such a realistic network model is necessary because verification based on a weaker model will be invalid in an actual deployment.
%Depending on the need, our network model can sometimes be refined to a more restrictive model, but, because of the simulation relation,
%properties proved using this model are also guaranteed to hold for the realistic one.
%Existing work sometimes assumes that nodes operate atomically between send and receive (for clients) or receive and send (for servers)~\cite{verdi},
%but our model does not have any such additional assumptions.
%
%\topic{Our design: 2) Our global model.}
%State in our \globalstate{} is a collection of all local states that are affected by the network.
%To prove correctness of the whole system, we must show that the interactions among the distributed nodes are correct.
%Our model of the global state is not very different from other work, but the write-witness-passing scheme takes advantage of the state in an unique way.
%
%\topic{Our design: 3) Why witness-passing: how it works easier to reason and prove.}
%While typical proofs of distributed systems involve showing that a local node's behavior refines the \globalstate{}~\cite{verdi, ironfleet},
%write-witness-passing works in the opposite direction;
%we start from a \globalstate{} and bring necessary global state information into the local state.
%The imported state is what the other nodes have witnessed at the time of sending messages.
%This information provides the context that is necessary for a local node to reason about its correctness within the entire distributed system,
%but was not required to simply execute the distributed protocol.
%The imported global state constitutes a witness-tree that keeps track of the path and evidence for how a node's current state was reached.
%This structure can be used for checking invariants and carrying out inductive proofs of the protocol.
%The witness-tree has partial information about the entire system state, but only the parts that are relevant to the node currently holding the tree.
%Thus, the verification can take place within the local context of a node without having to worry about other complex states in the rest of the distributed system.
%Because the information in the witness-tree was taken from the global state, an invariant that the data in the tree corresponds to something in the \globalstate{} naturally holds.
%In addition to easier verification, a proof based on write-witness-passing provides insight into why the protocol works,
%because the verification takes place in a local context that more closely mirrors the implementation.
%
%We use Paxos as an example to demonstrate how our \globalstate{} and write-witness-passing scheme can facilitate the verification of distributed systems.
%Paxos is a good example to explore the power of write-witness-passing because it requires communication with at least a majority of acceptors,
%and the weak network assumptions require reasoning about failure cases.
%Therefore, being able to handle a weak network model and having a clear sight on the global view of the system are necessary to verify the system.
%The Paxos consensus protocol is also notoriously difficult to understand just by observing the information that is passed around.
%Write-witness-passing can gather necessary global state into a tree to provide a clearer insight into how the protocol works.
%
%\begin{figure}
%\includegraphics[scale=.70]{figs/overall_structure}
%\caption{Overall Structure of Distributed System Verification with Write-Witness-Passing.
%All important components are explained in the later sections; (1) in Sect.~\ref{subsec:network-primitives} and
%Sect.~\ref{subsec:low-level-network-syntax-and-semantics}; (2) in Sect.~\ref{subsec:network-primitives} and Sect.~\ref{subsec:functional-correctness};
%(3) in Sect.~\ref{subsec:distributed-transition-semantics} and Sect.~\ref{subsec:witness-passing-semantics-in-paxos}
%(4) in Sect.~\ref{subsec:witness-write}, Sect.~\ref{subsec:distributed-transition-semantics-with-witness-passing}, and Sect.~\ref{subsec:witness-passing-semantics-in-paxos};
%(5) in Sect.~\ref{subsec:paxos-safety} and Sect.~\ref{subsec:extensibility-of-verified-paxos}; and
%(6) in Sect.~\ref{subsec:extensibility-of-verified-paxos}}
%\label{fig:overall-structure}
%\end{figure}
%
%To realize the write-witness-passing scheme, we use certified concurrent abstraction layers (CCAL)~\cite{concurrency} as the base verification framework
%and build necessary components such as the \globalstate{}, which includes the network model and node states, and the state transition framework within the \globalstate{}.
%The contextual refinement scheme that is proposed by the CCAL approach adds more benefits to our distributed system verification framework,
%such as enabling vertical and horizontal composition of verified protocols.
%Figure~\ref{fig:overall-structure} shows the overall structure of
%distributed system verification with write-witness-passing, and the important components are numbered in the figure.
%In a later section, we describe those components one by one, and
%we show how our framework and write-witness-passing can be used to prove Paxos leader election and reconfiguration, as well as other distributed protocols.
%
%
%This paper makes the following contributions:
%\begin{itemize}
%	\item We propose a general distributed system verification approach with the capability to link Coq-verified specifications with executable C code without relying on any external verification tools.
%	\item We propose a novel write-witness-passing scheme that facilitates local reasoning about distributed systems and provides insight into how distributed system protocols work.
%	\item We present a complete safety proof of Paxos using our verification framework and sketch how our framework can be used to verify other distributed systems.
%\end{itemize}
%
%
%The remaining parts of this paper are organized as follows.
%Section~\ref{sec:overview} is an overview of our verification approach using Paxos.
%Section~\ref{sec:witness-passing-semantics} describes in detail the formal definitions of our verification approach and of write-witnesses.
%Sections~\ref{sec:paxos-verification}-\ref{sec:evaluation} provide examples of
%our verification approach applied to Paxos and variants of Paxos, and evaluate
%our proof methodology.
%Section~\ref{sec:related} investigates related work and gives our conclusions.
%
%\topic{Benefit of CCAL: 1) network model is flexible, 2) vertical and horizontal composition.}
%
%\topic{Contributions:
%1) provide simple way to represent consensus protocols as well as for safety proofs of those protocols,
%2) provide the way to link those representations with low-level implementations (scalable, reusable way),
%3) verify Paxos using the approach with small human efforts.}

%\jiyong{We need to know what are the typical ways that others model the global state to verify distributed systems.}
%\jieung{
%\begin{enumerate}
%\item Ironfleet: takes too much human effort. Some parts are treated as assumptions (network reduction is not able to be verified)
%\item ESOP18, Verdi: made atomic handler and collect those atomic handlers to define global transition systems - connecting local behaviours with global transition systems. for the global properties, they need to reason about the state transitions for all the packets due to their representation.  Human efforts is high
%\item DISEL: horizontal composition. They argue that they can reuse their verification when it combined with other protocols. But, they cannot support vertical composition, which seems that they only can verify multiple distributed systems together if they are clearly divided.
%\item OOSPLA17, PLDI18 (automations): OOSPLA17 cannot generate the the runnable code. they cannot be linked with executable code, which is desired - they works with Paxos variants well, but have questions about other distributed systems. For PLDI18, they argue that they verified Raft and MultiPaxos, but for the MultiPaxos, they are unclear that what they have proved. they are not able to verify network reduction like Ironfleet, and they are not able to support concurrency yet (but.. I need to recompare PLDI18 paper again. PLDI18 paper also assume the all synthesis as a TCB
%\end{enumerate}
%}


%%%%%%%%%%%%%%%%%%%%%%%%%%%%%%%%%%%%%%%%%%%%%%%%%%%%%%%%%%%%%%%%%%%%%%
% some brain storming
%%%%%%%%%%%%%%%%%%%%%%%%%%%%%%%%%%%%%%%%%%%%%%%%%%%%%%%%%%%%%%%%%%%%%%
%Distributed protocol is important but complex:\\
%Several works verify distributed protocols either using automated tools or using interactive theorem provers\\
%people knows that interactive theorem provers are subtle\\
%But, it has benefits:\\
%Not even with its expressiveness, we can connect protocol layer verification with the low-level implementation \\
%Also, re-usability can be achieved using contextual refinement - this one cannot be achieved by automated system (contextual refinement)
%
%And, Paxos, immutability, is actually unbounded.
%In this sense, it is a little bit hard for automated approaches to prove that one in general (need to check PLDI18 and OOPSLA 17 papers)
%
%We want to claim that the complexity is not because of using interactive theorem prover,
%
%We propose  a way that dramatically simplify the proof of distributed system
%
%1. witness passing \\
%2. global transition system \\
%3. simple enough to understand the distributed system in a few minute \\
%4.compositionality \\
%5. and other benefits (link the proof with low-level implementation )
%



% ADD CONTENTS
%\section{Background}
\label{sec:background}

This section describes the type of distributed system we handle,
the multi-Paxos system that we use as an example,
and high-level constructs that are used to implement \sysname{}.

\subsection{Leader-Based Distributed Systems}

%\begin{figure}
%\includegraphics[scale=0.5]{figs/witnesspassing/leader-election.jpg}
%\caption{Leader Election}
%\label{fig:leader-election}
%\end{figure}
%
%\begin{figure}
%\includegraphics[scale=0.5]{figs/witnesspassing/state-replication.jpg}
%\caption{State Replication}
%\label{fig:state-replication}
%\end{figure}

We define a leader-based distributed system as a system with a set of nodes
$\fullset$ that are connected over the network and host a distributed shared state/object
$\dobj$. At each monotonically increasing term $\termnum$, at most one leader
$L \in \fullset$ can make changes to $\dobj$. There can be many ways to elect
$L$, but the election protocol must assign at most one $L$ per $\termnum$,
or the system should include a conflict resolution scheme to only apply
state changes by at most one $L$ at $\termnum$.

%\lucas{Notational nitpick: If we want to talk about the number of nodes anywhere in
%this paper (i.e. $2n+1 nodes$, it will be inconvenient to have defined $n \in N$ as
%a node instance. Could we perhaps use $h \in H$ and $l \in H$ (for the leader)?}

The constraints of state updates, which are specified in $\protocolfunction$ in
\sysname{},
are that
update operations can fail during execution, but the effect on $\dobj$ should be
atomic. For example, a multi-Paxos system can reach a
state where a Paxos instance did not reach a consensus, but contains an
incomplete state change proposal from $\gstate$ to $\gstate^\prime$.
Then $\protocolfunction$ should make sure that this
incomplete state is either safely abandoned (in favor of retaining state $\gstate$)
or completed (in favor of adopting $\gstate^\prime$ in its entirety)
before executing any subsequent
operations. Any leader-based systems that cannot be modeled in this way cannot
be hosted by \sysname{}.

We further formally model the elements of a leader-based distributed system in
\pref{sec:specs-for-leader-based-system}.
%and our main example system, multi-Paxos, in the next subsection.

\subsection{Multi-Paxos}

We use the fixed-configuration multi-decree Paxos protocol (a simplified variant
of the protocol commonly referred to as multi-Paxos)~\cite{rvrpaxos}
as an example protocol to explain how \sysname{} facilitates modeling and
verifying leader-based distributed systems. Multi-Paxos is a
state machine replication protocol. In multi-Paxos, a system's state changes
are chosen by synod consensus of a Paxos instance and replicated across acceptor nodes.
The Paxos instance consists of a group of acceptor nodes and the Paxos protocol
ensures that a state chosen by the instance is immutable.

Each Paxos instance requires a prepare and an accept phase to choose a new
state. A proposer sends the request and acceptors accept the new state.
A successful prepare requires a quorum (majority) of acceptor acknowledgments
and the state is successfully chosen only if a quorum of acceptors
accepted the state. For an acceptor to accept a state, the proposer's prepare
request should be the last request (ordered by a round number $r$)
that the acceptor has received. Once a state is chosen by the Paxos instance,
proposers trying to propose new state to the same instance are obligated to
propose the already-chosen state; this makes the chosen state immutable.
\ignore{
If there are contending proposers trying to propose a new
state to a Paxos instance, which does not have a chosen state, the proposers
can indefinitely race with each other by overwriting the prepare request.
}

As an optimization for successive proposals, multi-Paxos elects a
leader (dedicated proposer), which can propose new state changes to multiple Paxos
instances for as long as no other proposer attempts to prepare to write. However, the
original Paxos paper~\cite{paxos, paxosmadesimple} mostly focuses on the operation
of a single Paxos instance
and developers are left to choose their own implementation of multi-Paxos.

Our multi-Paxos example employs the scheme that is described as below:
\begin{itemize}[leftmargin=*]
	\item {\textbf{Leader election: }} whenever the leader at $\termnum$ is
		suspected to be dead, a proposer increments $\termnum$ to $\termnum'$ and
		sends a vote request to the acceptors.
		If a proposer receives a majority of votes with a $\termnum'$ tag it
		becomes the leader for round $\termnum'$.
	\item {\textbf{Preparation of new leader: }} the new leader checks and completes any
		partially finished tasks by the previous leader and figures out the
		tail of the log $t$. Then the leader batch prepares the log entries
		(Paxos instances) that come after $t$ with a round number $r = \termnum'$.
	\item {\textbf{Proposing new states: }} the leader sends proposals
		to Paxos instance $t+1$ whenever new state changes need to be chosen.
		The leader makes sure that $t$ is incremented only after the
		request at $t+1$ is successfully chosen (retries if the
		request fails) to prevent holes in the log.
\end{itemize}


%Ever since Paxos was published, many variants of multi-Paxos were
%proposed~\needcite{RvR Paxos, fast, disk paxos, vertical paxos}.
%While Paxos, sometimes called single-decree Paxos, defines how to reach a consensus on a
%single decision, multi-Paxos extends Paxos to a infinite sequence of decisions
%by maintaining an array of Paxos instances and accessing them in a log-order.
%
%A Paxos proposers first sends prepare requests to Paxos acceptors and then sends accept
%requests to have the proposer's proposal to be chosen by the acceptors. The acceptors
%accept the proposer's proposal if the proposer's prepare request is the latest prepare
%request that the acceptor has received. Therefore, if there are multiple proposers
%trying to send the proposal, the proposal may or may not be accepted depending on
%the recency of the prepare request. At the worst case if two proposers overwrites
%each other's prepare request before the other sends the accept request, a proposal
%will never be chosen.
%
%Multi-Paxos typically assigns a leader as an optimization to obviate this contention.
%The leader is the only node proposing the proposal so the decision making can smoothly
%make progress. The leader election scheme is one of the key factors that differentiates
%different multi-Paxos implementations. In our multi-Paxos system, we use a similar
%leader election scheme to Raft: if a leader is suspected to be dead, a new leader
%candidate increments the round number and asks all accpetor nodes for a vote; if the candidate
%receives a majority vote from the acceptors, it becomes the leader. The new leader
%recovers or discards unfinished operation by the previous leader, batch prepares the acceptors,
%and continues to send accept requests for new proposals.
%

\subsection{Concurrent Certified Abstraction Layer}


\sysname{} is built using the concurrent certified abstraction layer
(CCAL)~\cite{concurrency, deepspec} framework,
which divides a verification target into modular layers for simple reasoning.
CCAL is a predicate
$\ltyp{L'[S]}{R}{M}{L[S]}$,
which says that the implementation $M$ (written in C or assembly) run over the underlay layer $L'$
faithfully {\em implements} the desired interface $L$ via a simulation relation
$R$ for a subset $S$ of the participants $D$ ($S \subseteq D$)
in a concurrent program.
In a distributed setting $D$ can be viewed as the set of participating nodes.
\ignore{
The entire system participants $D$ can be interpreted as
whole CPUs in the multicore environment and all nodes in distributed systems.

When focusing on the distributed system and a singleton set for $S$,
CCAL is viewed as a framework to build a concurrent certified abstraction layer
for a single node with the capability to express all possible behaviors of
other nodes in the distributed system.

Therefore, if the predicate with a single node $i$ is modeled as ``$\ltyp{L'[i]}{R}{M}{L[i]}$'',
then $L'[i]$, which is the state machine for the module $M$, consists of a set
of abstract state and primitives that manipulates the abstract state.
$L'[i]$ also contains environmental context for other nodes $(D - i)$ in the
distributed system to capture all valid interleaved behaviors of
every node in the system, which is inspired by game semantics~\needcite{game semantics paper}.
In other words, the program itself contains the local move of the state machine
and the layer models the environmental move using the environmental context encapsulated in the layer.
The safety of environmental behaviors can be guaranteed by a generalized version
of a "rely" (or "assume") condition in rely-guarantee-based reasoning~\needcite{rely-guarantee works}
which are specified in the environmental context.
}

A layer $L[S]$ acts as an object consisting of abstract state, primitives that manipulate that state,
and an environment context, which captures the behavior of the nodes in the set $D - S$.
By \textit{relying} on the environment to have certain behaviors while also \textit{guaranteeing} that
the nodes in $S$ satisfy those behaviors, one can reason about nodes independently and
later link them together to prove properties of the whole system~~\cite{feng07:sagl,vafeiadis:marriage,LRG,fu10:roch,sergey15}.

%The safety of environmental behaviors can be guaranteed by a set of valid environment contexts
%that are specified for each layer interface.  The validity of those moves are based on a generalized
%version of the "rely" (or "assume") condition in rely guarantee-based reasoning \needcite{rely-guarantee works}



\ignore{
The implementation $M$ is a program module written in assembly (or C).
The abstraction layer $L[i]$ is constructed with $M$ and the implementation of
layer $L'[i]$. The construction process requires a proof object that guarantees
that $M$ on $L'[i]$ correctly refines $L[i]$ with the ($\refines_R$) relation
which is formally defined as a forward simulation~\cite{Lynch95,leroy09,Milner71,Park81}
with the (simulation) relation $R$.

%With this implementation $M$ and the layer $L'[i]$, we build an another
%abstraction layer $L[i]$. This process also contains
%providing the proof object about the program $M$ on $L[i]'$ correctly refines
%$L[i]$  with the {\em implements} relation
%($\refines_R$) which is formally defined as a forward
%simulation~\cite{Lynch95,leroy09,Milner71,Park81} with the
%(simulation) relation $R$.

\jieung{I will rephrase the following two paragraph later}
This CCAL enforces a {\em contextual} correctness property: a
correct layer is similar to a ``certified compiler,'' converting any {\em
  safe} client program $P$ running on top of $L[i]$ into one that has the
same behavior but runs on top of $L'[i]$ (i.e., by ``compiling'' abstract
primitives in $L$ into their implementation in $M$).  If we use
``$\sem{L[i]}{\cdot}$'' to denote the behavior of the layer machine based on
$L$, the correctness property of ``$\ltyp{L[i]'}{R}{M}{L[i]}$'' is written
formally as ``$\forall{}P.\sem{L[i]'}{P\oplus{}M} \refines_R \sem{L[i]}{P}$''
where $\oplus$ denotes a linking operator over programs $P$ and $M$.

CCAL also allows us to connect the layer that contains the protocol specific specifications
with the layer that contains a generic form of leader-based distributed systems.
}

\subsection{Global vs Local Reasoning}
\label{subsec:global-local-reasoning}

One of the challenges of distributed system verification is that it requires reasoning about
global invariants, i.e. properties that hold for the entire distributed state rather than just
one node's local state.
There is often not a straightforward relation between the local and global invariants, so
to reason globally one must trace the network history back in time and consider multiple
nodes' states at once.
This is further complicated if one assumes a network that can have failures such as duplicated
or lost messages.
In \sysname{} we attempt to reduce the amount of global reasoning by logically gathering sufficient
information in the local state to prove the desired properties.
We call this gathered information a \textit{witness}, and while we provide a formal treatment
in \pref{sec:witness-passing}, it is helpful to first see a simple example.

\begin{figure*}[t]
\begin{minipage}{\linewidth}
\noindent
\begin{multicols}{3}
  \lstinputlisting[numbers = left, language=C, mathescape=true, escapeinside={(*}{*)},
  morekeywords={such,that,forall,in,null,to}]{source_code/witnesspassing/paxos_spec.c}
\end{multicols}
\end{minipage}
\vspace{-1em}
\caption{Paxos with Witnesses. Witness extensions in \bfseries{bold}.}
\label{fig:paxos-witness}
\vspace{-1em}
\end{figure*}

\pref{fig:paxos-witness} contains pseudocode for single-decree Paxos.
The parts in bold are the necessary additions to augment Paxos with witnesses.
These additions are purely logical in that they do not show up in the actual implementation.
It is clear also that the overall behavior of the protocol does not depend on the witnesses;
they exist purely to aid in reasoning about the protocol.
By having the witnesses and assuming a well-formedness property certain safety invariants are
made more obvious.

For example, one of the primary safety properties of Paxos is immutability, which says that it
is safe for an acceptor to write a value $v$ when either no value is currently written in a
majority of acceptors or $v$ is already written in a majority of acceptors.
Looking at \pref{fig:paxos-witness} we can see that in order for a proposer to write a value system-wide,
it must have the highest round number that a majority of acceptors have seen.
It can also be shown that whenever an acceptor writes a value, it exactly matches the one in the witness.
Then by inspecting the promises stored in the witnesses, we can show that every value in the witnesses
came from a proposer with the highest round number among a majority of acceptors.
Therefore we know that the witnesses contain a history of all written values.
We use this fact to sketch a proof of immutability.

\begin{proof}
Consider a proposer in phase 2a with a witness $wit$ and a value $val$.
Proceeding by induction on the length of $wit$, the base case is trivial because no witnesses
means no value is written so it is safe to write anything.
In the inductive case we have a value $old$ that was the last write in round $rnd$.
If $val$'s round number is $rnd'$ then it must be the case that $rnd' >= rnd$ because
the acceptors would have rejected the messages otherwise.
If $rnd' = rnd$ then $val = old$ because round numbers are unique and only one value can be proposed
per round, so we are done.
If $rnd' > rnd$ then that implies there was a write in a round after $rnd$, which contradicts our
assumption that $rnd$ was the latest write.
Therefore, for any acceptor in phase 2b, the value in the witness is always safe to write and since
it matches the value in the message from the proposer, immutability is preserved.
\end{proof}

\begin{figure}
\begin{center}
%\includegraphics[width=.45\textwidth,page=1,trim=0 120 0 50,clip]{figs/witnesspassing/construct_witness}
	\includegraphics[page=1]{figs/witnesspassing/witnessspmp}
\end{center}
	\vspace{-0.2in}
\caption{Single-Paxos Witnesses: fields are (round number, value) (promises omitted)}
\label{fig:paxos-witness-table}
\vspace{-1em}
\end{figure}

\begin{figure}
\begin{center}
%\includegraphics[width=.45\textwidth,page=2,trim=40 90 0 0,clip]{figs/witnesspassing/construct_witness}
	\includegraphics[page=2]{figs/witnesspassing/witnessspmp}
\end{center}
	\vspace{-0.2in}
\caption{Multi-Paxos Witnesses: fields are (cell-index, round number, value) (promises omitted)}
\label{fig:multipaxos-witness-table}
\vspace{-1em}
\end{figure}

Another strength of witnesses is their compositionality.
\pref{fig:paxos-witness-table} shows the witnesses for two instances of Paxos.
The second write on each Paxos instance keeps track of witnesses for each
instance: writes at round 3 in both instances shows that value $v_{1}$ and
$v_{2}$ originally came from writes at round 1 and round 2 respectively. 
If we want to extend our proofs about single-Paxos to multi-Paxos, we can construct
a function that composes the witnesses in the manner shown in \pref{fig:multipaxos-witness-table}.
The composed witness is almost equivalent to what is actually stored in the sequence of
Paxos instances in multi-Paxos, which shows step-by-step state changes.
Even in a leader-based distributed system that does not maintain an explicit log 
to record state changes, the composed witness can provide the entire history of
state changes that can be used to reason about the system. 


%
\begin{figure*}
\includegraphics{figs/witnesspassing/overviewfig}
\vspace{-1em}
\caption{Leader-based distributed system process flow: the process flow can be
	expressed using template functions and system-specific functions can
	be plugged into the template. Examples show multi-Paxos, 
	Raft~\cite{raft}, two-phase commit with external membership service, 
	and Lamport's distributed lock~\cite{lamportclock}.}
\label{fig:process-flow}
\vspace{-1em}
\end{figure*}

\section{Specifications for Leader Based Systems}
\label{sec:specs-for-leader-based-system}

\sysname{} specifies leader-based distributed systems with a generic
specification and a system-specific specification. 
The generic specification defines the high-level behavior of the leader-based 
distributed system and the system-specific specification, which is encapsulated 
within the generic specification describes the low-level behaviors. 

The generic specification obviates the need for the user to write specifications
of essential safety properties of the leader-based system and instead enables them to
focus on writing the system-specific parts. The
generic and system-specific specifications are designed in a modular fashion and
can be composed to form a full specification of a leader-based distributed system.

Modeling the leader-based system requires two kinds of actors in the system:
requesters and approvers. 
Requesters seek permission to perform actions, which approvers either authorize or deny.
This separation need not be physical as in some protocols, such as Raft, one node can act
both as a requester and approver simultaneously.
The generic specification for both requesters and approvers 
works as a template for writing system-specific
specifications and for common safety property proofs.
Safety properties can be verified over the generic specification once,
and the system-specific specification, which meets the pre-defined template in
the generic specification, is automatically guaranteed to be safe. The safety
properties that \sysname{} guarantees are linearizability of state update and
soundness of leader selection.

%Proofs about those generic functions also can be combined with the
%multiple properties of protocol-specific update functions later.
%For example, with the proper properties of update functions,
%linearizabilty can be proved with the generic specification.
%Then, if the concrete definition of protocol specific functions are valid with
%the pre-defined properties,
%the linearizability proof can be automatically achieved
%for the protocol.



%later compose them with the common specification and the proof.
%
%
%it first provides a way for users to implement the safe distributed systems
%by giving the place that the users has to fill out.
%the generic function also hides the detail of the implementation.
%by doing that clients that uses those two functions can use them
%without considering the detail of their implementation, which has a benefit in terms of
%compositionality, too.


%The generic specification consists of two parts.
%The first is about the specification that leader-based distributed systems
%commonly follow.
%The other is protocol-specific parts that are encapsulated in the
%generic specification and will be concretely implemented later.
%providing the generic specification for leader-based distributed systems
%has a huge advantage in the verification.
%it first provides a way for users to implement the safe distributed systems
%by giving the place that the users has to fill out.
%
%the generic function also hides the detail of the implementation.
%by doing that clients that uses those two functions can use them
%without considering the detail of their implementation, which has a benefit in terms of
%compositionality, too.
%
%one another advantage of having them is
%providing the generic template for the safety proof of those functions.
%Proofs about those generic functions also can be combined with the
%multiple properties of protocol-specific update functions later.
%For example, with the proper properties of update functions,
%linearizabilty can be proved with the generic specification.
%Then, if the concrete definition of protocol specific functions are valid with
%the pre-defined properties,
%the linearizability proof can be automatically achieved
%for the protocol.

%Two generic functions are defined as follows:
The generic specification includes two template functions for both actors that describe the
abstract behavior of a leader-based distributed system:
\begin{itemize}[leftmargin=*]
	\item $\ldrfunction$ (the requester's spec) and  $\ldrhfunction$ (the approver's spec) specify how a leader is elected, and
	\item $\opfunction$  (the requester's spec) and $\ophfunction$ (the approver's spec)  specify how distributed states in the system are
		updated.
\end{itemize}
The operations of a leader based system can be described using these functions
as illustrated in \pref{fig:process-flow}.
We first define the definitions that are necessary to understand the
functions and then explain both functions in detail.

\subsection{State Definition}
\label{subsec:state-definition}

\begin{figure}
\begin{small}
\raggedright

	\vspace{0.05in}
\textbf{Local State}\\
	\vspace{0.05in}
$
\begin{array}{llll}
\nodeid &:& Type & \mbox{(Node ID)}  \\
\termnum &:& Type & \mbox{(Term num)}^* \\
\seqnum &:& Type & \mbox{(Version num)}^*\\
\dsvalue &:& Type & \mbox{(Local object)}\\
\dstate &:=& (\termnum  \times \seqnum \times \dsvalue) & \mbox{(Local state)} \\
\end{array}
$
\\
\raggedleft

* : Total order required

\raggedright
	\vspace{0.05in}
\textbf{Global State}\\
	\vspace{0.05in}
$
\begin{array}{llll}
\fullset &:=& \set{\nodeid} & \mbox{(Full node ID set)} \\
\gstate &:=& \set{ \nodeid \mapsto \dstate ~\vert~ \forall \nodeid \in \fullset}
	& \mbox{(Global state)}\\
\end{array}
$
\\
	\vspace{0.05in}
\textbf{Network} \\
	\vspace{0.05in}
$
\begin{array}{llll}
\packet &: & Type & \mbox{(Network event)}\\
\networklog &:=& \mbox{list}\ \packet & \mbox{(Network log)}\\
\replay &:& \networklog \rightarrow  \gstate& \mbox{(Log replay function)}\\
\end{array}
$
\\
	\vspace{0.05in}
\textbf{Protocol-specific Functions}  \\
	\vspace{0.05in}
$
\begin{array}{llll}
	\updatefunc & := & \dstate \rightarrow (\seqnum \times \dsvalue) &
	\mbox{(Obj update function)} \\
\gldrfunction & := & \nodeid \rightarrow \networklog \rightarrow \networklog  & \mbox{(Protocol functions)} \\
\gopfunction & := & \nodeid \rightarrow \updatefunc \rightarrow \networklog \rightarrow \networklog  & \mbox{(Protocol functions)} \\

\sendfunction & := & \nodeid \rightarrow \dstate \rightarrow \networklog & \mbox{(Send function)}\\
\recvfunction & := & \nodeid \rightarrow \dstate \rightarrow \networklog & \mbox{(Recv function)}\\
\end{array}
$
\end{small}
\vspace{-1em}
\caption{Basic Definitions}
\label{fig:basic-state}
\vspace{-1.5em}
\end{figure}

\ignore{\wolf{in \pref{fig:basic-state} is $\networklog$ a type or a variable? if it's a type then
it should be $:= list \packet$ instead of $: list \packet$, if it's a variable it should be
$\replay : list \packet \rightarrow \gstate$. similarly for many of the definitions in the protocol-specific functions part}}
\ignore{\lucas{I concur. $\dstate$, $\networklog$, and everything in Protocol-specific
Functions should probably use $:=$ instead of $:$ based on how they're used
elsewhere. I've changed it myself, since this doesn't seem controversial.}}
\pref{fig:basic-state} presents the necessary definitions for the
generic specification.
Nodes must have unique identifiers of type $\nodeid$, which can be any type with decidable equality.
A term number of type $\termnum$ acts as a logical temporal indicator in the system.
Term numbers do not need to correspond to any physical clock,
but they are used to resolve ordering conflicts between operations within the system,
so they must be totally ordered.
Nodes also contain a version number of type $\seqnum$, which is associated with the sequence of updates
of the local object $\dsvalue$.
A local object is one node's partial view of the global distributed object.
Since each node's local state may not be exactly in sync globally due to network or
node failures, the version number indicates how up-to-date a node's state is.
\ignore{\wolf{is it true that if two nodes have the same version number then they must have identical state?}}
%Nodes may also contain additional system-specific state inside their distributed object.
\ignore{\wolf{is the local object $\dsvalue$ not the system-specific part?}}

The global state $\gstate$ is the collection of all the local states hosted by the
entire system, which is composed of the set of distributed nodes $\fullset$.
There is a partial map from each node in the distributed system ($\fullset$)
to its local state ($\dstate$), which is a tuple consisting of a term number,
a version number, and a local object.

Nodes in a distributed system communicate via the network,
which we treat as a shared resource.
Inspired by the approach in CCAL~\cite{concurrency},
the global state $\gstate$ can be reconstructed by
applying the log replay function $\replay$ to the
network log $l : \networklog$, which is a list of network events ($\packet$).
The effect of a call to either $\tplldrfunction$ or $\tplopfunction$ by a node is atomic,
but function calls by other nodes can be interleaved %\wolf{is this true of $\ldrhfunction$ and $\ophfunction$ too?}.
The interleaving is expressed by our network model, described in \pref{sec:low-level-implementation} in detail.
%\wolf{I don't understand the last part of this sentence.
%Does it mean the network model has a channel per protocol that is shared among nodes in that protocol,
%or is there one channel that all protocols share?}.
Our framework keeps track of each function call
within the network log and the replay function $\replay$ computes the
global state based on the network log.
\ignore{\wolf{not sure what the previous two sentences mean. is it saying we record calls to
$ldrfunction$ and $\opfunction$ in the network log? should that be deferred to where we
talk about witnesses?}}
Finally, \sysname{}'s generic specification is parameterized by some protocol-specific
functions that we explain later in this section.

\ignore{
The specify the leader election scheme

For example, multi-Paxos requires replaying the entire log where each log
entry is determined by a quorum of node states. The version number in each node
represents the log index that the node has recorded the state so far. A sharded key-value
store involves simply combining partitioned key-value ranges from all nodes and may
not maintain strongly consistent semantics that require comparing version numbers from
different nodes.

In detail, the specification parameterized with the player's id $\nodeid$,
a term number $\rho$, which is a temporal
indicator, state update functions $\mathcal{G}_{ldr}$ and $\mathcal{G}_{op}$,
and the log $\networklog$ of the distributed systems.

Leader based systems need to guarantee that there is at most one leader at all
times. Therefore, the leader is elected whenever $\rho$, our notion of time, is
updated and $\mathcal{G}_{ldr}$ should implement and guarantee the soundness of
leader election. A successful $\mathcal{G}_{ldr}$ call tags the argument of the
$\mathcal{F}_{ldr}$ function as leader $L$.
Mapping leader election scheme of Raft, which has an explicit notion of a leader,
is straightforward to this model. A coordination service that uses a distributed lock
may not have a strong leader concept but we can regard the node that holds the lock
as a short-lived leader and treat the process of acquiring the lock as $\mathcal{G}_{ldr}$
and assign unique $\rho$ to for each acquire and release lock phase. }

%$\mathcal{G}$ defines the system specific protocols that applies to $N$.

%The common features that our model is intrested in for $\mathcal{S}$
%are the leader node $L \in N$ which is elected for each $\rho$
%and a distributed object $O$ (e.g. an entire key-value store state hosted within
%multi-Paxos) maintained within $N$. Because $O$ is an abstract state maintained
%within a specific system over distributed nodes in $N$, extracting $O$ out of
%$N$ would be system specific (e.g. for multi-Paxos, the entire log should be
%replayed).
%\jiyong{We should associate version number to $O$.}

%They consist of a set of multiple nodes, so each node
%contains their identifier ($\nodeid$).

%Leader based systems need to guarantee that there is at most one leader at all
%times. Therefore, the leader is elected whenever $\rho$, our notion of time, is
%updated and $\mathcal{G}_{ldr}$ should implement and guarantee the soundness of
%leader election. A successful $\mathcal{G}_{ldr}$ call tags the argument of the
%$\mathcal{F}_{ldr}$ function as leader $L$.
%Mapping leader election scheme of Raft, which has an explicit notion of a leader,
%is straightforward to this model. A coordination service that uses a distributed lock
%may not have a strong leader concept but we can regard the node that holds the lock
%as a short-lived leader and treat the process of acquiring the lock as $\mathcal{G}_{ldr}$
%and assign unique $\rho$ to for each acquire and release lock phase.

%In this sense,
%systems contain $\termnum$,
%which works as a virtual time clock of the system, in it.
%Each node can contain their own term number in their state,
%but the number is used in the entire system for global synchronization.
%For example, the term number in Raft is a protocol-specific representation of
%this term number in our generic specification.

%On the other hand,
%distributed object can be viewed as a set of local copies that does not need to be strongly
%synchronized with other copies in other nodes.
%Therefore, distributed objects need an indicator
%to keep track of the version history of their update and contents.
%In this sense,
%each node contains a version number in its local state.
%For example, Raft and multipaxos use an index number as their version number.
%Due to the assumptions about possibilities of failures,
%the state definition contains two version numbers in it,
%which indicates the possibly partially written values and successfully written values.
%In Raft, for instance,
%the first version number is an index number, while
%the second is a committed index number.

%The last element in the state definition is an actual distributed object that
%contains application-specific datum.
%It could be a singleton value, a list of values, or user-defined typed values.
%With those generic form of data structures, we now be able to define specifications for
%two important functions, \textit{prepare} and \textit{commit} functions, in the leader-based system.

\subsection{Leader Election}
\label{subsec:leader-election}

%\lucas{I'd like to suggest we use ``selection'' rather than ``election'' unless
%we are specifically referring to a system in which each node in some nontrivial
%set gets some kind of say in which node becomes leader. Also, are we using
%``Requester'' and ``Approver'' just to avoid using the same terms as Paxos, or
%is there some semantically significant distinction?}
%\jiyong{I think leader election is a more general term than selection in the community.}

\paragraph{Requester}
The goal of leader election is to set a new term, give one node exclusive access to the
entire system, and synchronize every node's local state such that the
global state is prepared to be updated.

Using function definitions in \pref{fig:basic-state}, we define the template function for leader election as follows:
\begin{small}
$$
\begin{array}{l}
	\ldrfunction (nid: \nodeid) (g_{pre}\ g_{post}:  \gldrfunction)
	(tx: \sendfunction)(rx: \recvfunction)(l_1  :\networklog ) := \\
\ \ \ \mbox{\textbf{let}} \ l_2 := g_{pre}\ nid \ l  \ \mbox{\textbf{in}}\ \mbox{\textbf{let}}\ l_3 := tx(nid, \replay(l_2 \mdoubleplus l_1)[nid]) \ \mbox{\textbf{in}}  \\
\ \ \ \mbox{\textbf{let}} \ l_4 := rx(nid, \replay(l_3 \mdoubleplus l_2 \mdoubleplus l_1)[nid]) \
  \mbox{\textbf{in}}\\ 
\ \ \  \mbox{\textbf{let}} \ l_5 := g_{post}\ nid \ (l_4 \mdoubleplus l_3 \mdoubleplus l_2 \mdoubleplus l_1) \ \mbox{\textbf{in}} \ l_5. \\
\end{array}
$$
\end{small}
where $\replay(l)[nid]$ is a projection from the global state generated by the log replay function to 
the local stated of $nid$.
%\wolf{what does the $[nid]$ part of the $\replay(l)[nid]$ notation mean?}
%\jieung{projection from the global state to the local state mapped with $nid$}
%\lucas{It means lookup the local state belonging to $nid$ in the global state
%obtained from the replay function. What I find more confusing is why both
%$\sendfunction$ and $\recvfunction$ take an entire local state as an argument.
%This is highly unintuitive, and if correct needs explanation. Not only that, but
%$\ldrfunction$ is implicitly either monadic or stateful. In this otherwise
%purely functional context, this needs to be explicitly acknowledged or changed.
%One way or another, I think we need a better explanation of what state
%$\sendfunction$ changes.}
%\jiyong{I think abstracting out some details, if it does not obstruct explaining
%the proof or other parts of this paper, will increase the readability.}

%Intuitively, the function tries to get a control for the distributed object as well as
%updates the local distributed object mapped with the node id as an up-to-date status.
%The up-to-date status implies that the status that guarantees the strong consistency.
%The function also updates the round number too, which are
%encapsulated in the local state.

This function captures the common form of a leader election scheme, which sends requests to
and gets approval from a set of authoritative nodes.
First, $g_{pre}$ changes the local state of a candidate node to prepare
for a leader election and generates corresponding request messages.
Network messages are sent and received using $tx$ and $rx$, which define system-specific
communication patterns between the requester and the approvers.
Finally, $g_{post}$ determines the final outcome of the
leader election and executes any actions that are necessary before becoming the leader.

The table on the left side of \pref{fig:process-flow} lists concrete examples of the parameters of
$\ldrfunction$. For example, multi-Paxos requires a majority vote from the acceptors to
become the leader, a two-phase commit protocol that uses a membership service may delegate
a transaction coordinator selection to this service, and Lamport's lock requires all node's
approval to acquire the lock. Note that the transaction coordinator and the lock holder are
regarded as leaders. The typical post-election operations are checking whether the
previous leader left the system in a consistent state and making the state consistent
if necessary. 

\paragraph{Approver}
The other part of leader election is how the approvers handle requests.
This typically involves some protocol-specific check of the request's
term number, followed by a local state update, and then an acknowledgement
containing the new state. We express this pattern as follows:
\begin{small}
$$
\begin{array}{l}
	\ldrhfunction (nid: \nodeid) (g:  \gldrfunction)
	(tx: \sendfunction)(rx: \recvfunction)(l_1  :\networklog ) := \\
\ \ \ \mbox{\textbf{let}} \ l_2 := rx(nid, \replay(l_1)[nid]) \
  \mbox{\textbf{in}} \\
\ \ \ \mbox{\textbf{let}}\ l_3  := g\ nid \ (l_2 \mdoubleplus l_1) \ \mbox{\textbf{in}} \\
\ \ \ \mbox{\textbf{let}} \ l_4 :=  tx(nid, \replay(l_3 \mdoubleplus l_2 \mdoubleplus l_1)[nid]) \ \mbox{\textbf{in}} \ l_4.
   \\
\end{array}
$$
\end{small}

Clearly, to prove any interesting properties of $\ldrfunction$ or $\ldrhfunction$
the system-specific state update functions must be restricted in some way.
For example, the term number modified by protocol-specific functions in $\ldrfunction$
need to increase monotonically in most leader-based distributed systems.
Additionally, $g$ in $\ldrhfunction$ should not touch the value inside
the distributed object in order to separate the leader election and the commit operation.
We discuss more about those requirements that the parameters must satisfy and what properties we
are able to prove in \pref{sec:prove-safety-with-witness}.


\subsection{Commit}
\label{subsec:commit}

\paragraph{Requester}
The commit function has a similar format to the leader election function, but
includes an object update function $u : \updatefunc$:
%One another key part of leader based distributed systems is a operation function.
%The way to operate the change usually depends on the system and the current status.
%the internal definition of the commit function can be defined as follows:
\begin{small}
$$
\begin{array}{l}
\opfunction (nid: \nodeid) (u : \updatefunc) (g_{pre}\ g_{post}:  \gopfunction)
	(tx: \sendfunction)\\
	(rx: \recvfunction)
	(l_1  :\networklog ) := \\
\ \ \ \mbox{\textbf{let}} \ l_2 := g_{pre}\ u \ l_1  \ \mbox{\textbf{in}}\   l_3 := tx(nid, \replay(l_2 \mdoubleplus l_1)[nid]) \ \mbox{\textbf{in}}  \\
\ \ \ \mbox{\textbf{let}} \ l_4 := rx(nid, \replay(l_3 \mdoubleplus l_2 \mdoubleplus l_1)[nid]) \
	\mbox{\textbf{in}} \\
\ \ \ \mbox{\textbf{let}} \ l_5 := 	 g_{post}\  u \ (l_4 \mdoubleplus l_3 \mdoubleplus l_2 \mdoubleplus l_1) \ \mbox{\textbf{in}}\ l_5.
\end{array}
$$
\end{small}
The object update function defines how the system updates the local object and
the version number depending on the local state of each node.
The update function is encapsulated in the message that is sent to the target node
via the send function ($tx : \sendfunction$).

Examples of how $\opfunction$ function is used are found on the right side
of \pref{fig:process-flow}. The number of recipients of the commit request differs depending
on the system; it could be all nodes in the system or just a few targeted nodes.
For the protocols that require more than 2 rounds in their commit phases, such as 
two-phase commit, $\opfunction$ should be called multiple times and $g_{pre}$ and
$g_{post}$ should have a flag to handle different functions that correspond to
each round. 

%\jiyong{Can we do 2 round protocols for commit using this format? Sending execution 
%function seems like a limiting factor.}
%\jieung{we can always send an identity function in the specification even though the low level 
%Intuitively, the operation function tells how the distributed object can actually be manipulated.
%The function looks similar to the leader election, but the biggest difference is the existence
%of update function $\updatefunc$.
%The update function defines how it update the distributed object and the sequence number
%depending on the local state itself.
%The update function is encapsulated in the packet message when the $\sendfunction$ send
%the message.
%Then the specification can be used with any arbitrary
%update function to specify the behavior of the protocol.
\paragraph{Approver}
When a requester triggers a commit operation, the approvers need to
handle the operation and update the local view of the distributed object if
necessary. Unlike $\opfunction$, $\ophfunction$ does not take an update function as an argument
because it is already in the message from the requester.
To retrieve the update function from the received message $\ophfunction$ requires
an auxiliary function, $\mathrm{proj}_{f}$, which extracts the update function from the network log.
\ignore{\lucas{I fixed the above paragraph a bit, but the last part is problematic. I
don't know what this sentence means: All those behaviors are also hidden in the
$\gldrfunction$ as follows:}}
\vspace{-0.2em}
\begin{small}
$$
\begin{array}{l}
	\ophfunction (nid: \nodeid) (g:  \gopfunction)
	(tx: \sendfunction)(rx: \recvfunction)(l_1  :\networklog ) := \\
\ \ \ \mbox{\textbf{let}} \ l_2 := rx(nid, \replay(l_1)[nid]) \
  \mbox{\textbf{in}} \\
\ \ \ \mbox{\textbf{let}} \ u := \mathrm{proj}_{f} \ (l_2 \mdoubleplus l_1, nid) \
  \mbox{\textbf{in}} \\
\ \ \ \mbox{\textbf{let}}\ l_3  := g\ u\ nid \ (l_2 \mdoubleplus l_1) \ \mbox{\textbf{in}} \\
\ \ \ \mbox{\textbf{let}} \ l_4 := tx(nid, \replay(l_3 \mdoubleplus l_2 \mdoubleplus l_1)[nid]) \ \mbox{\textbf{in}}  \ l_4.
   \\
\end{array}
$$
\end{small}
\vspace{-1em}

\ignore{
\subsection{Proofs with Generic Specification}
\label{subsec:generic-spec-proof}


\wolf{this section seems more like an introduction to witnesses than a discussion about proofs with the generic specification}
The generic specification provides a template for leader-based distributed protocols and hides unnecessary details from users.
It also provides a guideline for proving functional correctness when connecting the specification with a low-level implementation written in C or Assembly.

It, however, needs further items to prove safety properties of distributed systems efficiently.
Based on CCAL, and our generic specification, 
the state of the distributed system is modeled as a network history, which we can refer the 
status of other nodes.
The network history, however, usually contains unnecessary events in it due to the network failure assumptions, packet reordering or duplication, that most distributed systems have to consider. 
Therefore, having more precise, locally built, and globally consistent data structure to aid the safety proof of the distributed system is desired. 
In this sense, we introduce the witness and witness-passing. 
}


\ignore{
\subsection{Properties}
\label{subsec:properties}
With following the generic specification, 
\sysname{} guarantees that those operations will not be interfered by other operations, 
even though they generate multiple events while their operations. 

\begin{theorem}[Interference-Free]

\end{theorem}
Regardless of the detail in the function definition 


The generic specification can be used to verify common properties of
leader-based distributed systems. For instance, linearizability~\cite{herlihy90}
is a property that can guarantee that the system is making state transitions
without exposing any anomalous state.
\lucas{This doesn't sound like what I understand linearizability to be. I was
under the impression that linearizability is satisfied when the challenges and
responses in a history of events can be reordered into a correct, consistent
sequential history. I don't know what it means to ``expose anomalous state''.}

If we can safely assume that object update function $\updatefunc$ is applied by
a single valid leader and the function call makes atomic state changes, this
will lead to easy verification of the linearizability properties.
\lucas{Are there multiple linearizability properties? Maybe we should elaborate,
or simply stick to ``property''?}

However, constructing a proof for linearizability requires evidence for the
assumptions. Therefore, generic specifications alone are not sufficient for the
verification of common properties of leader-based distributed systems.
\lucas{Could we be more specific about which assumptions we mean?}

\jiyong{I think this subsection should talk about what properties are proved for
free if we use \sysname{} rather than saying that the generic specification is
insufficient to verify the common properties. So we should say the generic spec
requires X, Y, and Z, and this makes the proof possible. We should also sketch
the proof as well.}
\lucas{Ah, yes. This is a good point, Ji-Yong. I think you're right that we
could emphasize what we can do rather than what we can't.}
%The high level specification with proper constraints about update function
%can be used for verifying generic properties of
%leader-based distributed systems.
%
%For instance, linearizabilty~\needcite{linearizability paper}
% is a common feature that may be required in
%all leader based distributed systems.
%
%Let's assume that
%all $\updatefunc$ in the previous section
%guarantees that the sequential number and the committed sequential number monotonically.
%With this constraint,
%verifying that
%each commit function
%guarantees linearizability.
%
%However,
%providing the proof object about linearizability
%also requires us to connect all commit function applications as well as all leader election functions.
%In this sense,
%only providing
%generic specifications are not sufficient
%for the verification of leader-based distributed systems.
}

%\begin{figure}
{\small
\raggedright
\textbf{Quorum definition}  \\
	\vspace{0.05in}
$
\begin{array}{llll}
\accsset & := & \set{\nodeid} & \mbox{(All voters)}\\
\voters & := & \set{\nodeid} & \mbox{(All supportive voters)}\\
\isquorums_{ldr} (ca: \accsset) (v: \voters) &:& \mathbb{B} & \mbox{(Leader election quorum)} \\
\isquorums_{op} (ca: \accsset) (v: \voters)  &:&  \mathbb{B} &  \mbox{(Commit quorum)}\\
\end{array}
$
\\
	\vspace{0.05in}
\textbf{Leader election witness}  \\
	\vspace{0.05in}
$
\begin{array}{llll}
\ldrwitness &:=&  \set{\termnum \mapsto list\ (\dstate, \voters, \accsset, \isquorums_{ldr})} & \\
\end{array}
$
\\
	\vspace{0.05in}
\textbf{Commit operation witness}  \\
	\vspace{0.05in}
$
\begin{array}{llll}
\opwitness &:=& \set{\termnum \mapsto list\ (\dstate, \updatefunc, \voters,
	\accsset, \isquorums_{op})}& 
\end{array}
$
\\
} % end of \small
\vspace{-1em}
\caption{Witness Definiton}
\label{fig:witness-definition}
\vspace{-1.5em}
\end{figure}

\vspace{-0.5em}

\section{Witness Passing}
\label{sec:witness-passing}

The generic specification provides a template for leader-based distributed
system and hides unnecessary details from users. It also provides a guideline 
for proving functional correctness when connecting the specification with a 
low-level implementation written in C or Assembly.
It, however, needs further items to prove safety properties of distributed 
systems efficiently.  
\sysname{} models the state of the distributed system as a network history, 
which we can refer the status of other nodes.
The network history, however, contains unnecessary events in it due to
the network failure assumptions, packet reordering or duplication, that  
distributed systems consider. Therefore, having a more precise, 
locally built, and globally consistent data structure to aid the safety proof 
of the distributed system is desired. Hence, we propose the witness.   

The witness is a logical data structure that constantly grows to contain a global
history of the distributed system state changes that are meaningful to the
latest state. A witness grows by being passed between nodes through the network.
Information in a witness is never replaced or deleted, but a witness can
be discarded if the operation it is associated with fails.

\ignore{
The witness can be built in a modular way by extending generic specifications, 
especially the log replay function and the network message definition. 
The witness can be used regardless of network failure assumptions because it is
simply an extra ghost information that is attached to messages to help 
verify crucial properties of distributed systems.
}

While \sysname{}'s template functions mainly use witnesses to verify
linearizability of state changes and the leader election by composing the 
witnesses for commit operations and leaders, the underlying systems can
build their own witnesses to verify system-specific properties (e.g. immutability
of Paxos in \pref{sec:background}).

We first formally define witnesses as they are used in the template functions
and then show how we integrate them into the specifications described in \pref{sec:specs-for-leader-based-system}.

\subsection{Witness Definitions} 
\label{subsec:witness-definition}
Witnesses in \sysname{} are associated with the two template functions.
We view the leadership witness $\ldrwitness$ as a history of ownership transfers
of a distributed object and the commit witness $\opwitness$ as
a logical sequence of updates to the object. While the information added to a
witness is created by all nodes in the system, the witness construction only
happens at the requester side. Witness-related definitions are outlined in
\pref{fig:witness-definition}.

The witness for the leader election function ($\ldrwitness$) contains a map
from the term numbers ($\termnum$) to a set of witness elements.
Each element contains a list of local node state ($\dstate$), node ids
of all of the voters in the leader election ($\accsset$), node ids of the voters who voted
for the requester ($\voters$), and a function to determine the success of the
election ($\isquorums_{ldr}$), which makes a decision by checking for the existence of a
quorum. The necessary quorum size for a successful election varies
by protocol; e.g., Raft requires $F + 1$ votes out of $2F + 1$
nodes while the Lamport lock needs $N$ votes out of $N$ nodes.

The witness for the commit function ($\opwitness$) describes how the distributed
object has evolved from the initial to the latest state by keeping a
history of all intermediate values of $\sigma$ and the update function
$\updatefunc$ that is applied for each update. $\opwitness$ also contains
the other fields in $\ldrwitness$; however, the necessary quorum size for $\isquorums_{op}$
can be different than that for $\isquorums_{ldr}$. For example,
two-phase commit as described in \pref{fig:process-flow} uses
a membership service for leader election, but relies on resource managers to
commit data.

$\isquorums_{ldr}$ and $\isquorums_{op}$, which determine whether the leader 
election and operation are successful, generalize how leader-based distributed 
systems make atomic state transitions. We make the assumption that the underlying
protocol allows only one quorum that allows the state change for an operation 
that multiple nodes may try to execute concurrently. This is one of the key
requirements that a system must meet to use \sysname{}'s proof template.

The process of building a witness is modular and simple. A requester sends
requests and approvers send back acknowledgements with existing witnesses that
show how they reached their latest state.
The requester can simply construct a new witness and concatenate it with the
received witness without having to inspect it.
The new composite witness is then sent back to the approvers at the point where
the approvers' state is updated.

\subsection{Specifications with Witness} 
\label{subsec:specifications-with-witness}

\begin{figure}
{\small
\raggedright
\textbf{State definition with witness}\\
	\vspace{0.05in}
$
\begin{array}{ll}
\dstate_\witness:= (\dstate \times \ldrwitness \times \opwitness)  & \mbox{(Local state)} \\
\gstate_\witness := \set{\nodeid \mapsto \dstate_\witness} & \mbox{(Global state)}\\
\packet_\witness:= \packet~\vert~(\packet \times  \witness_{ldr} \times \witness_{op}) & \mbox{(Network event)}\\
{\networklog}_{\witness}:= \mbox{list}\ \packet_\witness & \mbox{(Network log)}\\
\replay_\witness : {\networklog}_{\witness} \rightarrow  \gstate_\witness& \mbox{(Log replay function)}\\
\end{array}
$
\\
	\vspace{0.05in}
\textbf{Witness building routine in} ${\replay}_{\witness}$  \\
	\vspace{0.05in}
$
\begin{array}{ll}

find\_ca_{op} : \termnum \rightarrow \networklog \rightarrow \networklog \rightarrow  \set{\nodeid}\\
find\_vtr_{op} : \termnum \rightarrow \networklog \rightarrow \networklog \rightarrow  \set{\nodeid}\\
find\_ca_{ldr} : \termnum \rightarrow \networklog \rightarrow \networklog \rightarrow  \set{\nodeid}\\
find\_vtr_{ldr} : \termnum \rightarrow \networklog \rightarrow \networklog \rightarrow  \set{\nodeid}\\
\\
\conwcons \ (nid : \nodeid)\ (rn: \termnum)\  (w_l :\witness_{ldr})\ (w_o : \witness_{op})\ (l_{pre}\ l_{rcv}: \networklog)\\ (f_{q} :  \isquorums_{op}) := \\
\ \ \ \ \mbox{\textbf{let}}\ ca := (find\_ca_{ldr} \ rn \ l_{pre} \ l_{rcv})\  \mbox{\textbf{in}} \\ 
\ \ \ \ \mbox{\textbf{let}}\ vtr := (find\_vtr_{ldr} \ rn \ l_{pre} \ l_{rcv})\  \mbox{\textbf{in}} \\ 
\ \ \ \ \mbox{\textbf{if}} \ f_{q} \ ca \ vtr \ \mbox{\textbf{then}}  \\
\ \ \ \ \ \ \mbox{\textbf{match}} \ w_l[rn] \ \mbox{\textbf{with}} \\
\ \ \ \ \ \ \ \ \vert~nil \Rightarrow w_o \\
\ \ \ \ \ \ \ \  \vert~((rn', \_, \_), \_, \_, \_ )::\_  \Rightarrow w_o / \langle w_o[rn] := w_o[rn'] \rangle \\
\ \ \ \ \ \ \mbox{\textbf{end}}\\
\ \ \ \ \  \mbox{\textbf{else}} \ w_o.
\\
\ldrwcons \ (nid : \nodeid)\ (rn: \termnum)\ (w_l :\witness_{ldr})\ (w_o : \witness_{op})
(l_{pre}\ l_{rcv}: \networklog)\\  (f_{q} :  \isquorums_{ldr})\ (f_{o} :  \isquorums_{op}) := \\
 \ \ \ \ \mbox{\textbf{let}}\ ca := (find\_ca_{ldr} \ rn \ l_{pre} \ l_{rcv})\  \mbox{\textbf{in}} \\ 
 \ \ \ \ \mbox{\textbf{let}}\ vtr := (find\_vtr_{ldr} \ rn \ l_{pre} \ l_{rcv})\  \mbox{\textbf{in}} \\ 
\ \ \ \ \mbox{\textbf{if}} \ f_{q} \ ca \ vtr \ \mbox{\textbf{then}}  \\
\ \ \ \ \ \ \mbox{\textbf{let}}\ l := l_{rcv}~\verb!++!~l_{pre} \ \mbox{\textbf{in}} \\
\ \ \ \ \ \ \mbox{\textbf{let}}\ w_l' := w_l/\langle w_l[rn]  := ((\replay(l)[nid]),f, vtr, ca, f_{q}) ::w_l[rn])  \rangle \ \mbox{\textbf{in}}\\
\ \ \ \ \ \  \mbox{\textbf{let}}\ w_o' := \conwcons\  nid\ rn\ w_l\ w_o \ l_{pre} \ l_{rcv} \ f_o \ \mbox{\textbf{in}}\  (w_l', w_o')\\
%\ \ \ \ \ \ \ \ \ \ \mbox{\textbf{match}} \ w_l[rn] \ \mbox{\textbf{with}} \\
%\ \ \ \ \ \ \ \ \ \ \vert~nil \Rightarrow  (w_l', w_o)\\
%\ \ \ \ \ \ \ \ \ \ \vert~((rn', \_, \_), \_, \_, \_ )::\_  \Rightarrow (w_l', w_o / \langle w_o[rn] := w[rn'] \rangle \\
%\ \ \ \ \ \ \ \ \ \ \mbox{\textbf{end}} \\ 
\ \ \ \ \mbox{\textbf{else}}  \ (w_l, w_o). 
\\
\opwcons (nid : \nodeid) \ (rn: \termnum)\ (f: \updatefunc)\ (w :\witness_{op})
 (l_{pre}\ l_{rcv}: \networklog)\\ (f_{q} :  \isquorums_{op}) := \\
\ \ \ \ \mbox{\textbf{let}}\ ca := (find\_ca_{op} \ rn \ l_{pre} \ l_{rcv})\  \mbox{\textbf{in}} \\ 
\ \ \ \ \mbox{\textbf{let}}\ vtr := (find\_vtr_{op} \ rn \ l_{pre} \ l_{rcv})\  \mbox{\textbf{in}} \\ 
\ \ \ \ \mbox{\textbf{if}} \ f_{q} \ ca \ vtr \ \mbox{\textbf{then}}  \\
\ \ \ \ \ \  w/\langle w[rn] := ((\replay(l_{rcv}~\verb!++!~l_{pre})[nid]),f, vtr, ca, f_{q}) ::w[rn])  \rangle\\
\ \ \ \ \mbox{\textbf{else}}  \ w. \\
\end{array}
$
}% end of \small
\vspace{-1em}
\caption{Witness Construction}
\label{fig:witness-construction}
\vspace{-1em}
\end{figure}


Based on the witness definition, we extend the state definitions of
leader-based systems to utilize witnesses and discuss routines to construct
witnesses (\pref{fig:witness-construction}).

Each node's local state is extended to store leader and commit witnesses ($\dstate_\witness)$.
The definition of the global state with witnesses is also updated accordingly.
In order to pass witnesses from one node to another,
network events must also contain witnesses in addition to the usual events.
When the leader election function or the commit operation function successfully 
updates the local state of the distributed node, the associated witness should be
attached to the new state. To do so, the log replay
function should internally call a witness building routine ($\ldrwcons$ or $\opwcons$) at the end of the
leader election and the commit functions.

\pref{fig:witness-construction} shows methods for constructing witnesses.
At the end of $\ldrfunction$, $\ldrwcons$ first constructs the information about the
voters ($ca:\accsset$) and ($vtr:\voters$) and, if the requester succeeds in becoming the leader,
it attaches this evidence to $\ldrwitness$. Then $\conwcons$ is called against
the nodes that processes the commit operation to gather the witness for
$\opwitness$\footnote{Set of nodes that elect leaders and commit operations
can be different. E.g., if the system uses external membership services,
the set of nodes can be different, whereas multi-Paxos and Raft use an identical
set for both purposes.}.
$\conwcons$ copies the witnesses from the previous term into the current one.

%Up to this point it only
%touches the witness regarding leader election and then as a leader it probes the 
%it extracts the witness, which represents the current state of the system, from 
%the approvers.
%
%The witness building in the leader election does not need to know 
%all details about the distributed object manipulation.
%When the leader has been elected, however, it first needs to build the proper leader witness and 
%also needs to copy the previous witness mapped with the old term number $rn'$ to
% the new term number $rn$. 
% Copying the previous witness does not require calculating all the previous writes. 
% It, however, automatically guarantees the consistency of the distributed object manipulation 
% via leader election when the leader election and the commit operation function guarantees a 
% certain property described in the ~\ref{subsec:prove-safety-with-witness}. 

$\opwcons$ is the analogous function for $\opfunction$.
The leader starts from a witness with the complete history of the system state
that was collected using $\conwcons$ during leader election.
Upon a successful call to $\opfunction$, the leader creates a new witness by
adding a witness entry, which contains the update function $f$ that is to be used
to update the existing state.

Witnesses are stored in the local state, and are communicated over the network.
This means failed communications can
cause certain nodes to fall behind with updating their witnesses.
However, as long as the system operates without violating the failure 
assumptions, the system state eventually becomes consistent and reasoning 
about the protocol is not affected by outdated nodes.

%\begin{figure}
% \includegraphics[scale=0.05]{figs/witness_table.jpg}
%\caption{Witness Example}
%\label{fig:witness-example}
%\end{figure}


%Those witness building functions are write-only functions, 
%which does not remove any previous committed values from the data at all. 
%Figure~\ref{fig:witness-example} shows the witness building example with arbitrary leader
%based protocol. 
%In the example, 
%leader is elected when the term numbers are $1$, $3$, and $7$. 
%The example shows that building the leader election witnesses
%is isolated from the log update via commit functions. 
%The commit witness, however, needs an additional information 
%to connect the previous commit histories that are associated with 
%other term numbers. 
%In this sense, when the new leader is elected with the term number $3$, the new election 
%copies the old committed witnesses associated with $1$ to $3$ without knowing the 
%detail of the commits. 
%Similarly, the election with the term number $7$ also copies the committed 
%witness value from $3$ to the committed witness associated with $7$.
%By doing so, the committed witness value contains all valid committed history 
%in a compositional way. 


%The witness also contains few simple witnesses 
%\jieung{need to change the following ones as a formal rules}

Witnesses impose multiple invariants in them.
For example, all elements inside both leader election and commit operation witnesses 
always contain the valid voters and candidates that satisfy the quorum as follows:
\vspace{-0.2em}

\begin{theorem}[Witness Invariant]\label{thm:witness-invariant}
Assume that for a network log $l$, the resulting state of replaying the log $l$ 
	is $gst$, which can be represented as $\replay_\witness(l) = gst$. 
For all $nid$, if $gst[nid] = (\_, \_, \_, w_l, w_o)$, then $w_l$ and $w_o$ satisfy the following properties:
\begin{enumerate}
\item $\forall\ vtr\ ca\ f_q, \ (\_, vtr, ca, f_q) \in w_l \rightarrow f_q(ca, vtr) = true$
\item $\forall\ vtr\ ca\ f_q, \ (\_, vtr, ca, f_q) \in w_o \rightarrow f_q(ca, vtr) = true$
\end{enumerate}
\end{theorem}
\vspace{-0.2em}

\begin{proof}
The proof is straightforward with the witness construction definition.
\end{proof}
\vspace{-0.2em}

\begin{theorem}[Specification Refinement]\label{thm:spec-refine}
With the refinement relation $R$ between $\dstate$ and $\dstate_\witness$, 
the template functions satisfy the following properties:
1) $\ldrfunction  \sqsubseteq_{R} {\ldrfunction}_{\witness} $; 
2) $\ldrhfunction  \sqsubseteq_{R} {\ldrhfunction}_{\witness} $; 
3) $\opfunction  \sqsubseteq_{R} {\opfunction}_{\witness} $; and
4) $\ophfunction  \sqsubseteq_{R} {\ophfunction}_{\witness}$.
\end{theorem}
\vspace{-0.2em}

\begin{proof}
Proof is straightforward. 
Since both $\dstate$ and $\dstate_\witness$ contain the same fields aside from
the witnesses, the refinement relation $R$ for each field will be identity.
Thus, the simulation proof for these fields are straightforward. 
The witness can be easily constructed from the witness construction functions
in \pref{fig:witness-construction}, which do not touch fields other
than the witness. 
Hence, proving the refinement relation between the two specifications is
straightforward.
\end{proof}

Similarly, proving that providing a generic specification for distributed
systems implies providing a generic specification with witnesses for distributed
system is also straightforward.



%\section{Common Property Proof}
\label{chapter:witnesspassing:sec:prove-safety-with-witness}

Proving a safety property of generic specifications using witnesses is similar to showing that invariants of the witness hold.
%Besides the basic invariant including Theorem~\ref{thm:witness-invariant},
%other assumptions about protocol specific functions can be defined and used to 
%prove other crucial properties of distributed systems. 
This section illustrates the proof of one of the common desirable properties
of a leader-based distributed system, linearizability.
Linearizability guarantees that if a quorum of approvers accepts a request $c$, 
all the subsequently issued requests reflect the result of $c$. 
The proof of linearizability can take advantage of the witness. 

\begin{lemma}
All issued requests have a unique identifier.
\end{lemma}

This lemma enables us to distinguish all requests. To achieve this goal the 
update function $u$, which is an argument of $\opfunction$ should fulfill a 
property that monotonically increases the version number to distinguish all
requests:
\begin{center}
$\forall \ (st \ st' : (\seqnum * \dsvalue)) : \ u \ st = st' \rightarrow \text{fst }st
<_{\seqnum}  \text{fst }st'$,
\end{center}
where $<_{\seqnum}$ is a comparison relation of two version numbers.


Another requirement to build a linearizable leader-based distributed system is to make sure that an elected leader can get the system's latest consistent
state. Such a state is represented as a witness with the most extended list, and this witness includes the entire history of all committed requests. 
It requires the check for witnesses to figure out the witness that contains the longest the committed witness. If there are two committed witnesses from two approvers, choose the witness that includes longer uncommitted witness. It only requires the check for the last index value in the uncommitted witness if there are elements in the uncommitted witness.


This proof is related to generic specifications, $\ldrfunction$ and $\ldrhfunction$, 
and the witness construction specifications, $\ldrwcons$ and $\opwcons$.
First, the $g$ function in $\ldrhfunction$ has to satisfy that  
it always returns the current status to the requester.
Second, the $g_{post}$ function has to choose the value mapped with the highest version number among valid acknowledgments. 
Third, the $\ldrwcons$ function should be called with the proper write witness, which is the witness mapped with the highest version number among all witnesses from  approvers (in a set of quorum).
With the given constraints, the following two lemmas can be proved.

\begin{lemma}[Election consistency]\label{lemma:chapter:witnesspassing:election-consistency}
The leader election does not miss any committed operation.
\end{lemma}

\begin{lemma}[Leader witness] 
The leader always contains the up-to-date commit witness that is represented as
witness with the largest number of entries.
\end{lemma}

\begin{proof}
Proving the election consistency is providing that system-specific functions satisfy the conditions that we illustrated. However, satisfying these
conditions are common requirements for system-specific functions in
our approach and thus if we assume that the conditions already hold for all
system-specific functions, the proofs of the lemmas are straightforward.
\end{proof}

With these lemmas, proving  linearizability is finally available.

\begin{theorem}[linearizability]\label{theorem:chapter:witnesspassing:linear-spec}
The commit witness is linearizable.
\end{theorem}

\begin{proof}
Our witness contains all the linearizable history that is matched by the current local state, 
and the witness in all consistent nodes is the same due to our witness passing mechanism. 
Therefore, linearizability is captured by our witness.
\end{proof}

Since a leader election collects the leader witness with the largest
size using $\ldrwcons$, the linearizability of leader election, and thus uniqueness of the leader
for each term can be proved in a similar way. 

%\section{Representing the Network}
\label{chapter:witnesspassing:sec:low-level-implementation}


Previous sections have shown how we provide general specifications
that can accommodate a variety of distributed systems.  To assure correctness of a concrete implementation, a distributed system verification also needs to
provide evidence that actual running codes refine high-level generic
specifications. While mere functional correctness is relatively well-understood,
it is not sufficient for distributed system verification. The
system's behavior concerning the network must be taken into account. 
The previous chapter addresses this issue and provides a network model only for one specific protocol, $\WOR$. 
In here, we have generalized our asynchronous network model to make it be used for multiple protocols by filling out the necessary information. 




\subsection{Network Model}
\label{chapter:witnesspassing:subsec:network-model}

\begin{figure}
\begin{small}
\raggedright
$$
\begin{array}{llll}
\chid & := & \mathbb{Z} & \mbox{(channel id)} \\
\msg & : & Type & \mbox{(message)} \\
\mathrm{gmsg} & : & Type & \mbox{(ghost message)} \\
\mathrm{Esnd} & := & \nodeid \times \msg \\
\mathrm{Erecv} & := & \nodeid \times \msg \\
\mathrm{eType} & := & \mathrm{Esnd} + \mathrm{Erecv} + \nodeid + \mathrm{gmsg} \\
\mathrm{E}_{net} & := & \chid \times \nodeid \times \mathrm{eType} \\
\end{array}
$$
\end{small}
\caption{Network Definitions.}
\label{fig:chapter:witnesspassing:net-defs}
\end{figure}


A network model here is one instantiation of a log in
 Section~\ref{chapter:witnesspassing:sec:specs-for-leader-based-system}. Our network is defined as a list of
network events, and key definitions are found in Figure~\ref{fig:chapter:witnesspassing:net-defs}. A
network event ($\mathrm{E}_{net}$) represents an interaction between a node and
a greater distributed system of which it is part. This $\mathrm{E}_{net}$ is a
low-level instantiation of $\packet$ from Figure~\ref{fig:chapter:witnesspassing:basic-state}.

At the lowest logical level, there are four kinds of events: 1) send, 2) receive,
3) timeout, and 4) ghost. A send event represents an interaction in which one
node sends data to another node and registers corresponding send events in the
network log. A receive event represents the moment in time when a node begins
processing data that arrived from another node (not the moment of the arrival
itself). A timeout event represents the interaction in which a node decides it
has not received an expected communication from another node. Ghost events do
not correspond to any concretely observable network communication, but can
greatly simplify local reasoning about the global state of the distributed
system by explicitly demarcating global state transitions in the network log.

All network events carry a channel identifier ($\chid$), the purpose of which is
to logically separate network events into disjoint subsets for separate
reasoning and refinement. Assigning different channel identifiers for each
protocol and each verification layer allows for the composition of protocols and
for network reductions (see Section~\ref{chapter:witnesspassing:subsec:connection}).

Network events also carry a source node ID, which denotes the node that
instigated the event. For any given datagram sent from node $a$ to node $b$, the
network log would eventually contain a send event with source $a$ and either a
receive event or a timeout event also with source $a$. The source of a ghost
event is the ID of the node that observed the distributed system transition from
one state to another. \ignore{(for example, a Paxos proposer observing that it has
received responses from a majority of acceptors, meaning that the system as a
whole has come to consensus).} Send, receive, and timeout events carry a
destination node ID indicating the intended recipient of the datagram to which
they correspond. Send and receive events carry a message ($\msg$), the payload
of the communication.

A local node's view of the network log grows over time by the action of two
primitives - send and receive. The low-level send primitive adds a single send
event to the network log. There is no acknowledgment and send always succeeds.
The low-level receive primitive conceptually collects data received from other
nodes, but in addition to receive and timeout events, populates the network log
with all intermediate events including events not observable by the
concrete local node. For reasoning purposes, receive consults a network oracle
to fill in the log. The oracle is constrained by predicates that model only
relevant forms of network failure,(which can be specified per protocol), and
verification of a protocol will involve quantifying over all such valid oracles
to show the correctness of the local node's behavior and the greater distributed
system.


\subsection{Connection with Generic Specifications}
\label{chapter:witnesspassing:subsec:connection}

\begin{figure}
\begin{center}
\includegraphics[scale=0.35]{figs/witnesspassing/network_reduction.pdf}
\end{center}
\caption{Network Reduction in Distributed Systems.}
\label{fig:chapter:witnesspassing:network-reduction}
\end{figure}

\paragraph{Network Reduction}
The network model in concert with $\ccalname$ permits a logical refinement step in
which a complex view of the network log is reduced to a simpler one for easier
reasoning. Channel ids and projection functions allow these network
reductions to be applied independently to disjoint subsets of the network log.
This facilitates clear separation of protocol verification and aids
horizontal composition.

Two example network reductions are shown in Figure~\ref{fig:chapter:witnesspassing:network-reduction}. In
reduction (a), a two-step reduction simplifies a network log in which node $P1$ issues three independent send events. The first step rearranges the three send
events so that they are adjacent in the network log. The second step
consolidates the three send events into a single broadcast event. These
reductions do not involve a change in the behavior of the system (the first because
the send events are independent of the intervening network log, and the second
because a broadcast has the same semantics as individual sends). Each reduction
introduces a new channel and disallows an old one (the channel id is the first
number in brackets in each event in the figure).

The separation of network reductions by channel enables verification of a
protocol by parts and of several protocols in a single system. Since network
reductions can be focused by channel, verification of different protocols is
separable and thus composable.


\subsubsection{Building Global State Transition Systems}

\paragraph{Log Replay Function}
Specifications that we have described in
 Section~\ref{chapter:witnesspassing:sec:specs-for-leader-based-system}
 are based on the network log and network
log replay function. To connect low-level specifications with these log-oriented
specifications, a refinement from a direct update on local state is necessary. 
 $\ccalname$  showed how this can
be done with CPU schedules, and the same technique applies straightforwardly to
the network log.

A high-level global state calculated by a replay function contains local
states for each node. A network oracle, and by extension of local states, can
be constrained with protocol-specific assumptions for rely-guarantee reasoning.


%\vspace{-0.7em}

\section{Multi-Paxos with Witness}
\label{sec:multipaxos-with-witness}

This section describes how we modeled and verified multi-Paxos using 
\sysname{}. We present the systems specific specifications and how the
specification connects to the template to verify linearizability and soundness
of the leader. 

\paragraph{Data type}
Concrete data types in Figure~\ref{fig:basic-state} are instantiated:
\begin{itemize}[leftmargin=*]
\item $\nodeid$ and $\termnum$ are $\mathbb{Z}$ type values.
\item $\seqnum$ is a pair of two values in $\mathbb{Z}$ type. The first value represents 
the \textit{index} in the multi-Paxos log that is being written, and the 
second represents the last \textit{committed index} where all log entires upto this
index contain immuatable states chosen by the acceptors.
\item $\dsvalue$ is an array of $\mathbb{Z}$ type values.
\vspace{-0.2em}
\end{itemize}

\paragraph{Leader election}
To model the multi-Paxos leader election function based on $\ldrfunction$,
two protocol specific functions in the requester side are instantiated.
\begin{itemize}[leftmargin=*]
\item $g_{pre} (l)$ adds a ghost packet that adds 1 to the term number.
\begin{small}
$$g_{pre} (l) := nid.[\mbox{\textbf{inc}} \ \replay(l)[nid].rn]::l$$
\end{small}
\item $g_{post} (l)$ collects acknowledgement messages and checks for the
	quorum.
\begin{small}
\vspace{-0.2em}
$$g_{post} (l) := \mbox{\textbf{let}}\ pkts := filter_{pkt}(\replay(l)[nid].rn, nid)\ \mbox{\textbf{in}} \ cal\_qrm(pkts)$$
\end{small}
\vspace{-0.3em}
\end{itemize}

\paragraph{commit}
Similar to the leader election function, the protocol specific functions in the requester side are instantiated. 
\begin{itemize}[leftmargin=*]
\item $g_{pre} (l, u)$ is an empty function, which does nothing.
\item $g_{post} (l, u)$ collects acknowledgement messages, checks for the
	quorum, and updates the local status by applying the update function
		$u$.
\begin{small}
$
\begin{array}{c}
 g_{post} (l, u) :=  \mbox{\textbf{let}}\ pkts := filter_{pkt}(\replay(l)[nid].rn, nid) \ \mbox{\textbf{in}}\\
\mbox{\textbf{if}} \ OK(cal\_qrm(pkts)) \ \mbox{\textbf{then}} \ nid.[u]::l\ \mbox{\textbf{else}}\ l \\
\end{array}
$
\end{small}
\vspace{-0.5em}
\end{itemize}

\paragraph{Update function}
The object update function is relatively simple, 
which adds index and committed index numbers by 1 and applying changes to the
state:
$
\begin{array}{l}
u (st : \dsvalue) :=\\
\ \ \ \mbox{\textbf{match}} \ st \ \mbox{\textbf{with}}\\
	\ \ \ \ \ \vert~(\_, (idx, cidx), dsval) \\
	\ \ \ \ \ \ \  \Rightarrow ((idx + 1, cidx + 1), dsval[idx + 1] := uf_{\dsvalue}\ dsval[idx])\\
\ \ \ \mbox{\textbf{end}}.\\
\end{array}
$
Here, $u_{\dsvalue}$ is a function that is specific to the system hosted 
by multi-Paxos.
%
%\paragraph{Linearizability and Prefix Existence}
%
%As discussed in Sect.~\ref{subsec:prove-safety-with-witness}, 
%the linearizability proof of multipaxos can automatically derived when 
%the instances of protocol specific functions satisfy it. 
%
%\begin{theorem}[Prefix Existance]
%
%\end{theorem}

\begin{theorem}[Linearizabilty of MultiPaxos]
Multipaxos is  linearizable.
\end{theorem}

\begin{proof}
Our implementation satisfies all the requirements in Sect.~\ref{sec:prove-safety-with-witness}. 
Therefore, proving the linearizability of MultiPaxos is straightforward with Thm.\ref{thm:linear-spec}.
\end{proof}

%\section{Examples}
\label{chapter:witnesspassing:sec:examples}

\jieung{We hope to add more details later.}

We demonstrate the flexibility of \sysname{} by applying it to other leader-based distributed systems, Raft~\cite{raft}, two-phase commit~\cite{distsys}, and Lamport's distributed
lock~\cite{lamportclock}.

\subsection{Raft with Witness}
\label{chapter:witnesspassing:subsec:raft-with-witness}

Raft implements state machine replication similar to our multi-Paxos example and
has few commonalities. They use the same leader election scheme that elects
the leader based on the majority vote at each term and commits a new entry based on
successful write/accept to majority of nodes/acceptors. The key differences are, 
the Raft leader can make progress during network partition without confirming the 
commit of older requests, the Raft leader must have the longest log size, and 
the leader always checks and keeps consistency of all nodes during each write.
Compare to multi-Paxos, $\gldrfunction$ of Raft requires a simple additional
condition check for the log length before voting for the leader and this barely
changes the witness structure. However, Raft leader being able to make progress
during network partition makes $\gopfunction$ to be more complicated to
reason about. However, the linearization point of Raft is when a leader
successfully writes to the majority of nodes. Therefore, we can subdivide
$\gopfunction$ to try making state changes without confirming a successful
commit and if a node with missing log entry is found, switch to recovery mode
to retry an old request. The recovery mode can use the update function almost same 
as the update function in Section~\ref{chapter:witnesspassing:sec:multipaxos-with-witness}. It, however, 
requires a special case in the $g_{post}$ function to handle the recovery case, 
which does not update the leader's version number and the witness.  
as well as a quorum definition because it is between one approver and a leader.
\begin{center}
$
\begin{array}{lll}
u (st : \dsvalue) &:=&  \mbox{\textbf{match}} \ st \ \mbox{\textbf{with}}\\
&& \vert~(\_, (idx, cidx), dsval) \Rightarrow ((idx + 1, cidx + 1), dsval[idx + 1] := uf_{\dsvalue}\ dsval[idx])\\
&&\ \mbox{\textbf{end}}.\\
\end{array}
$
\end{center}
In either case, Raft protocol returns success only to a
linearizable write, which is an append to the last successful write. The
combination of the log length indicator changes the committed index and this
operation can be encoded into to post action of $\gopfunction$. Indeed,
because Raft employs a strong notion of a leader, the leader already contains
enough information about the global state of the system which is as much
information as our typical witness structure. This information can be given to
the template for the linearizability proof. 


\subsection{Two-phase Commit with a Membership Service}
\label{chapter:witnesspassing:subsec:two-phase-commit-with-a-membership-service}

Two-phase commit protocol itself typically does not include a membership
change scheme and we assume that there is a membership service which can makes one of the
node in the system to become the new transaction manager in case a transaction 
manager fails. 

A membership service can be a Zookeeper-like~\cite{zookeeper} system that is implemented on top
of multi-Paxos, but we assume a single front-end node for such system as the
approver that processes the leader election request. Thus, the leader election is 
as simple as sending the request to a single node and getting an acknowledgement back. 
However, data of the system is stored in the resource managers so the leader election and
commit operation happens in different set of nodes. Committing data requires sending
and receiving acknowledgements from all resource nodes that the transaction
touches.  Two-phase commit requires unanimous commit vote to commit a transaction, so the
logic in the requester side is simply making sure all requests are approved by
the resource nodes. 

\subsection{Lamport Lock with Witness}
\label{chapter:witnesspassing:subsec:lamport-lock-with-witness}

Lamport's lock can be thought of as a leader-based system where the owner of
the lock is the leader.
Each node maintains a logical clock that is incremented every time it sends or
receives a message and is included in sent messages.
Nodes also keep a priority queue of request messages ordered by timestamp.
A node initiates a request for the lock by adding the request to its own queue
and broadcasting it to every other node.
Upon receiving a request the node adds it to the queue and sends a timestamped
acknowledgement.
The lock is acquired when a node observes that its own request is at the front
of its queue and it has received a message from every other node with a later
timestamp.
At this point the node that owns the lock is guaranteed to have exclusive
access to the shared state and is free to modify it.
To release the lock it removes its request from its queue and broadcasts a
message telling other nodes to do the same.
To model it, $\dsvalue$ contains a bit ($lock$) to indicate whether the node holds the lock or not.
$\ldrfunction$ acquires the lock and $\opfunction$ with the specific update function 
releases the lock as follows:
\begin{center}
$
\begin{array}{lll}
u (st : \dsvalue)&:=& \mbox{\textbf{match}} \ st \ \mbox{\textbf{with}}\\
&&\vert~(\_, idx, dsval) \Rightarrow (idx, dsval.lock := false)\\
 && \mbox{\textbf{end}}.\\
\end{array}
$
\end{center}

%
\section{Evaluation}
\label{chapter:mcslock:sec:evaluation}

\begin{figure}
\begin{minipage}{\linewidth}
\noindent
\begin{multicols}{2}
\lstinputlisting[numbers = left, language = C]{source_code/mcslock/palloc_example.c}
\lstinputlisting[numbers = left]{source_code/mcslock/sharedeventtype.v}
\end{multicols}
\end{minipage}
\caption{$\pallocfunc$ Example.}
\label{fig:chapter:mcslock:palloc-example}
\end{figure}

\subsection{Clients}

The verified $\mcsname$ lock code is used by multiple clients in the $\certikos$ (in Chapter~\ref{chapter:certikos})
system. To be practical the design should require as little extra work
as possible compared to verifying non-concurrent programs, both to
better match the programmer's mental model, and to allow code-reuse
from the earlier, single-processor version of $\certikos$.

For this reason, we don't want our machine model to generate an event
for every single memory access to shared memory. Instead we use what
we call a \emph{push/pull memory model} in Section~\ref{chapter:ccal:sec:interface-calculus}.
To recall it,
 A CPU that wants to access shared memory first generates a ``$\pull$''
event, which declares that that CPU now owns a particular block of
memory. After it is done it generates a ``$\push$'' event, which
publishes the CPU's local view of memory to the rest of the system. In
this way, individual memory reads and writes are treated by the same
standard operational semantics as in sequential programs, but the
state of the shared memory can still be replayed from the log.  The
$\push$/$\pull$ operations are logical (generate no machine code) but
because the replay function is undefined if two different CPUs try to
pull at the same time, they force the programmer to prove that
programs are well-synchronized and race-free. Like we did for atomic
memory operations, we extend the machine model at the lowest layer by
adding logical primitives, e.g. $\releaseshared$ which takes a
memory block identifier as an argument and adds a
$\mcsomeme{\codeinmath{(l:list Integers.Byte.int)}}$ event to the log, where the byte list is a
copy of the contents of the shared memory block when the primitive was
called.

When we use $\acquirereleaselock$ we
need a lock to make sure that only one CPU pulls, so we begin
by defining combined functions $\acquirelockfunc$ which
takes the lock (with a bound of 10) and then pulls, and
$\releaselockfunc$ which pushes and then releases the
lock. The specification is similar to $\mcspasshlockspec$,
except it appends \emph{two} events.

Similar to Section~\ref{chapter:mcslock:sec:liveness-atomicity}, logs for different
layers can use different types of pull and push events.
Figure~\ref{fig:chapter:mcslock:palloc-example} (right) shows the events for the
$\pallocfunc$ function (which uses a lock to protect the page
allocation table). The lowest layer in the palloc-verification adds
$\mcsOMEME$ events, while higher layers instead add
($\mcsoateev{(a:\ \mcsalloctabletype)}$) events, where the relation between logs
uses the same relation as between raw memory and abstract
$\mcsalloctabletype$ data. Therefore, we write wrapper functions
$\acquirereleaselockATspec$, where the
implementation just calls $\acquirereleaselock$
with the particular memory block that contains the allocation table,
but the specification adds an $\mcsOATEev$ event.
This refinement step, which changes the log replay function to compute
allocation tables instead of byte lists, is
specific to the $\pallocfunc$ function.

\begin{figure}

\lstinputlisting{source_code/mcslock/release_lock_AT_spec_short.v}
\lstinputlisting{source_code/mcslock/palloc_spec_short.v}

\caption{$\pallocfunc$ Specification.}
\label{fig:chapter:mcslock:palloc-spec}
\end{figure}

We can then ascribe a low-level functional specification
$\pallocpspec$ to the $\pallocfunc$ function. As shown
in Figure~\ref{fig:chapter:mcslock:palloc-spec}, this is decomposed into three parts, the
acquire/release lock, and the specification for the critical
section. The critical section spec is exactly the same in a sequential
program: it does not modify the log, but instead only affects the $\mcsalloctable$ field in the abstract data.

Then in a final, pure refinement step, we ascribe a high-level atomic
specification $\lpallocspec$ to the $\pallocfunc$
function. In this layer we no longer have any lock-related events at
all, a call to $\pallocfunc$ appends a single
$\mcsopalloce$ event to the log. This is when we see the
proof obligations related to liveness of the locks.
Specifically, in order to prove the downwards refinement, we need to
show that the call to $\pallocpspec$ doesn't return
$\None$, so we need to show that $\HCalLock\codeinmath{ l'}$ is
defined, so in particular the bound counter must not hit zero.
By expanding out the definitions, we see that
$\pallocpspec$ takes a log $\codeinmath{l}$ as its initial global log
and generates the result,
$\rellock\codeinmath{::}\mcsoateev{(\mcsalloctable\ adt)}\codeinmath{::} \mcstshared{\mcsopullev}\codeinmath{::}\waitlock{10}\codeinmath{::l}.$
The initial bound is 10, and there are two shared memory events, so the
count never goes lower than 8. If a function modified more than one
memory block there would be additional push- and pull-events, which
could be handled by a larger initial bound.

Like all kernel-mode primitives in $\certikos$, the $\pallocfunc$ function is
total: if its preconditions are satisfied it always returns. So
when verifying it, we show that all loops inside the critical section
terminate. Through the machinery of bound numbers, this guarantee is
propagated to the the while-loops inside the lock implementation:
because all functions terminate, they can know that other CPUs will
make progress and add more events to the log, and because of the
bound number, they cannot add push/pull events forever. On the other
hand, the framework completely abstract away how long time (in microseconds) elapses
between any two events in the log.

\subsection{Code reuse} 
The same
$\acquirereleaselock$ specifications can be
used for all clients of the lock. The only proofs that need to be done
for a given client is the refinement into abstracted primitives like
$\releaselockATspec$ (easy if we already have a sequential
proof for the critical section), and the refinement proof for the
atomic primitive like $\lpallocspec$ (which is very
short). We never need to duplicate the thousands of lines of proof
related to the lock algorithm itself.

\subsection{Using more than one lock}

The layers approach is particularly nice when verifying code that uses more than one
lock. To avoid deadlock, all functions must acquire the locks in the
same order, and to prove the correctness the ordering must be
included in the program invariant. We \emph{could} do such a
verification in a single layer, by having a single log with different
events for the two locks, with the replay function being undefined if
the events are out of order. But the layers approach provides a
better way. Once we have ascribed an atomic specification to
$\pallocfunc$, as above, all higher layers can use it
freely without even knowing that the $\pallocfunc$ implementation
involves a lock (Note that the lock is not re-exported from the
$\pallocfunc$ layer, and if it was the proof of the atomic
specification would not go through.)  For example, some function in a
higher layer could acquire a lock, allocate a page, and release the
lock; in such an example the the order of the layers provides an order
on the locks implicitly.

\subsection{Proof Effort}


As an evaluation, we do not count the total lines of code in $\coq$ for our entire 
$\mcsname$ Lock module due to the two following reasons. First, our $\mcsname$ Lock implementation 
is a part of $\certikos$ (in Chapter~\ref{chapter:certikos}). Therefore, our $\mcsname$ Lock module also contains several definitions 
and proofs that are totally irrelevant to $\mcsname$ Lock verification. 
This implies that counting the total lines of code for $\mcsname$ Lock module has a 
high possibility of misinterpretation due to the lines of code for those definitions and proofs.
Second, we intensively use contextual refinement approach to 
build the whole system rather than focusing on verifying the correctness and 
liveness of $\mcsname$ Lock. Therefore, our proof efforts are mainly focus on proving 
$\mcsname$ Lock that is able to be easily combined with multiple client codes 
rather than the efficient lock verification itself.  

Among the whole proofs, the most challenging parts are the proofs for starvation 
freedom theorems like Theorem~\ref{thm:chapter:mcslock:mcs_wait_lock_exist}, 
and the functional correctness proofs for $\mcsacquire$ 
and $\mcsrelease$ functions
in Section~\ref{chapter:mcslock:subsec:atomicoperation}.
The total lines of codes for starvation freedom is 2.5K lines, 0.6K lines for specifications, 
and 1.9k lines for proofs. This is because of the subtlety of those proofs. 
To prove the starvation freedom theorems and show the evidence of loop termination,
lots of lemmas are required to express
state changes by replaying the log. 
When $\QSCalLock\codeinmath{(l)=}\Some\codeinmath{(c1,c2,b,q,s,t)}$
and $\codeinmath{q=}\nil$, 
for instance, the mechanized proof for $\codeinmath{s=}\emptyset$ 
and $\codeinmath{t=}\nil$ is necessary. It looks trivial in the hand-written proofs, 
but requires multiple lines of codes in the mechanized proof. 

The total lines of codes for the low-level functional correctness
of two main C functions, $\mcsacquire$ and $\mcsrelease$, are 3.2K lines,  
0.7K lines for specifications, and 2.5K lines for proofs.
It is much bigger than other code correctness proofs for while-loops in $\certikos$, which we will
discuss in Chapter~\ref{chapter:certikos},
because these loops do not have any explicit decreasing value.
One another big part in our $\mcsname$ Lock proofs is the proofs for 
Theorem~\ref{thm:chapter:mcslock:machine-state-refinement} and the lines of code for this part is 
approximately 5K lines. The log replay function ($\calmcslock$) always 
return the whole $\mcsname$ Lock values ($\lockabsloc$) related 
to the  $\codeinmath{mcs\_lock}$ structure defined in Figure~\ref{fig:chapter:mcslock:mcs_lock}. 
In this sense, we always have to give the exact values for all memory 
chunks and prove the correspondence between the memory and the abstract 
data even the event associated with reading values (\eg, $\getnext$).
Hence, those proofs contain a lot of duplicate proofs for the memory access. 
However, they are quite straightforward and easy to produce. 
On top of that, we strongly believe 
that they can be easily reduced by introducing mechanized user-defined tactics later. 

As can be seen from these line counts, proofs about concurrent programs
have a huge ratio of lines of proof to lines of C code.
If we tried to directly verify shared objects that use locks to 
perform more complex operations, like thread scheduling
and inter-process communication, a monolithic proof  
would become much bigger than the current one, and would be quite
unmanageable. The modular lock specification is essential here.

By contrast, the proofs for them in $\certikos$ are quite tractable, 
because the proofs for the locks are modular, re-usable, and can 
be combined with other client-part proofs like we have briefly 
mentioned earlier in this Section.
Therefore, we believe that our approach is a promising way to 
show the correctness of large systems that use shared objects with mutex protection. 


% END CONTENTS

%\vspace{-0.7em}

\section{Related Work and Conclusion}
\label{sec:related}

\para{Model of leader-based distributed system}
\sysname{}'s leader-based distributed system model, which is based on the two functions
$\ldrfunction$ and $\opfunction$, is inspired by CASPaxos~\cite{caspaxos},
which implements an atomic shared object. CASPaxos uses prepare and accept
phases similar to those in Paxos but it can repeatedly apply a compare-and-swap
functions to the stored state instead of setting it just once. While CASPaxos is a system implementation
to atomically update a distributed object, we use the implementation style of
CASPaxos to build a generic specification for leader-based distributed systems.
Our specification also resembles the high-level specification of a state machine
replication protocol, which Lamport generalized~\cite{generalizedconsensus}.
While Lamport mainly models multi-Paxos and consensus, \sysname{} models and
verifies common properties of a generic leader-based distributed system,
which includes multi-Paxos and state machine replication.
%
%\para{Witness}
%\jiyong{ATTN Jieung: Write about witness and ghost states}
%The witness of \sysname{} is a ghost state that is often used for system
%specification and verification~\needcite{ghost state}~\cite{ghoststateorigin}...
%\lucas{I am VERY NOT SURE about this citation. Another paper (Higher-Order Ghost
%State) cites the paper as a possible origin of the use of ghost state, saying
%``[Ghost state] is a fixture of Hoare logics since the work of Owicki and Gries
%[29] in the 1970s''.}
\vspace{-0.2em}

\para{Verification of linearizability and leadership}
Verifying linearizability~\cite{herlihy90} for concurrent objects has been
studied for decades. Methodologies to simplify linearizability have been
proposed mostly in a concurrent programming context~\cite{Elmas10tacas,
Liang13pldi,Gotsman12concur,Viktor10CAV} and the linearizability of Raft and
multi-Paxos has been verified~\cite{cppraft, ironfleet}. The linearizability
and leader-soundness proofs of \sysname{} build on top of exiting work on
linearizability: the key insight for the proof is to base the reasoning on
an atomic step of each operation.  Yet, we generalize the reasoning of linearizability
to the leader-based distributed system and create a reusable proof template.

%The template extracts
%necessary states from the witness that is created by the lower level
%specification of concrete systems
%of a generic leader-based distributed system and provides a proof template based on
%abstract states which are represented as the witness.
%The witness encodes atomic steps backed by a quorum, which approves state
%changes, and the validity of the leader who is initiating the atomic steps.
%Our template guarantees the linearizability of any leader-based system that
%satisfies the constraints of the witness. Using a similar witness structure, our
%template automatically verifies the soundness of the leader based on
%linearizability of leader election.
\vspace{-0.2em}

\para{Distributed system verification}
Approaches to verify distributed systems have been explored actively over the
past few years. IronFleet~\cite{ironfleet} annotates functions with pre- and
post-conditions to automatically prove the correctness of the code with an SMT (satisfiability
modulo theories) solver and proves a refinement relation between the code and
protocol proofs in different layers. Taube et al.~\cite{modular} studied adding
decidability for verifying distributed systems using SMT solvers.
The work surrounding SMT solvers has a philosophical difference with our verification approach.
\ignore{\wolf{this seems like more a technical difference where we trade expressiveness for weaker automation}}
We use high-order logic and CCAL, which have higher expressiveness but less automation,
while SMT-solver-based work requires encoding higher-order concepts into
first-order logic for better automation. While a relatively large portion of the
verification can be automated, the SMT-solver-based approach has some limitations.
For example, certain higher-order properties (e.g. network reduction
in IronFleet) are not always encodable or verifiable in first order logic.

Verdi~\cite{verdi} presents a distributed system verification tool chain
where developers specify and implement a system using a functional language
embedded in Coq while assuming a perfect network model. The system can then automatically convert
it into a system that handles a more realistic network and failure model. DISEL~\cite{disel}
studies how to verify and horizontally compose different distributed protocols.
It verifies the protocols in separation and uses send-hooks to restrict the
interference among protocols. \sysname{} assumes a realistic network to begin with
and it not only allows both horizontal and vertical composition of verified distributed
protocols but also supports combined reasoning of protocols by using witnesses.
\vspace{-0.2em}

\para{Paxos verification}
A huge body of work exists on verifying the Paxos protocol~\cite{paxos}.
Lamport provided a proof sketch at the time of proposing the protocol~\cite{paxosmadesimple}
and Lampson attempted to distill Paxos into its core components by creating a very
high-level Abstract Paxos~\cite{Lampson2001} and showing how variants of
Paxos can be derived from it. Additional efforts were made to divide Paxos
into simpler components~\cite{dpaxos, sdpaxos} to find a reusable framework for
proving its variants. Our witness-based approach also attempts
to find a reusable framework for these proofs, but instead of decomposing Paxos
further, we make the key implicit invariants more explicit by passing around
the information needed to prove them.

Padon et al. verified high-level specifications of many variants of Paxos
and multi-Paxos while proposing a method to specify the protocols using
a decidable first-order logic~\cite{paxosepr}. Verification of Paxos
variants was made possible by reusing the specification and proof of vanilla
Paxos. \sysname{} on the other hand can reuse the proof of a Paxos instance
due to CCAL to prove variants of Paxos using proof objects and verifies both the
C code and the specification of the system based on witness.
\vspace{-0.2em}

\para{Certified Abstraction Layers}
The first formal presentation of how to use certified abstraction layers to
verify large systems was given by \citet{deepspec} and later the framework was
extended to support concurrency (CCAL)~\cite{concurrency}.
One of the key strengths of this approach over other verification frameworks
is its support for contextual refinement over multiple layers.
Based on the contextual refinement, \sysname{} extends the CCAL framework with
realistic and simplified network models, composable witnesses,
and proof templates for leader-based distributed systems in modular layers.


%Woos et al. presented the verification of Raft~\cite{cppraft}. The properties
%verified for Raft includes linearizability and soundness of the leader and
%the code written in Coq translates to executable OCaml code. \sysname{}
%delegates verification of such Raft properties common to leader-based distributed
%systems to the template and the final executable code extracted from
%\sysname{} is in assembly, which is generated from CompCert C compiler and is more
%optimized than the OCaml code.


%\topic{Paxos Verification and Deconstruction}
%Within the category of distributed system verification, there is a significant amount of work focusing specifically on Paxos.
%Lamport's first paper on the protocol~\cite{paxos}, and his second attempt to explain it more clearly~\cite{paxosmadesimple}
%present the basic algorithm and give paper proofs of the safety properties.
%Since then, many variations have been developed such as Disk Paxos~\cite{diskpaxos}, Egalitarian Paxos~\cite{epaxos},
%and Vertical Paxos~\cite{vertpaxos}.
%Oftentimes, although these variations seem similar to the original protocol, it is not possible to reuse the original
%proof of the safety properties and a significant amount of work is required to re-prove them.
%Lampson attempted to distill Paxos into its core components by creating a very high-level Abstract Paxos~\cite{Lampson2001}
%and showing how other variants can be derived from it.
%Some still felt that this did not get at the essence of the algorithm because at least two works since then \cite{dpaxos, sdpaxos}
%have studied other ways of dividing Paxos into simpler components such that proofs of the protocol can be made more modular.
%Our write-witness-passing approach also attempts to find a reusable framework for these proofs,
%but instead of decomposing Paxos further, we make the key implicit invariants more explicit by passing around
%the information needed to prove them.

%This enables proofs to be done in a thread-local (or node-local) manner using an environment context to
%capture the behavior of the rest of the world.
%These proofs can then be linked together to obtain a strong correctness theorem for the entire system.
%
%Our witness benefits greatly from using CCAL.
%By decomposing distributed systems into layers, we reduce the amount of time and effort needed to
%prove functional correctness.
%By treating each distributed node as a separate thread, we can use the environment context to
%prove properties in a local context.
%Combining this with the witness then allows us to bring in information about the global state when necessary.
%We also use the environment context to model a realistic, non-deterministic network.
%Another advantage afforded by CCAL is that we can lift our safety proofs to higher layers via contextual refinement.
%This allows us to implement and verify a system once, and then reuse it as a component in various distributed applications.


%Approaches to verify distributed systems have been explored actively over the
%past few years. IronFleet~\cite{ironfleet} annotate functions with pre- and
%post-conditions to automatically prove the code with an SMT (satisfiability
%modulo theories) solver. It requires refinement proofs among three layers:
%implementation, distributed protocol, and high level specifications. However,
%the proof of network model is not fully machine-checkable whereas all \sysname{}
%proofs are machine-checked using Coq.
%Verdi~\cite{verdi} presents a distributed system
%verification tool chain, where developers specify and implement a system using
%a functional language embedded in Coq while assuming a perfect network model.
%A transformer in the tool chain then automatically converts the system into one
%that can handle a more realistic network and fault model. The strong network
%assumption in Verdi made it difficult to verify Raft-like systems that
%inherently assumes weak network, but \sysname{} starts with a realistic network
%assumtions and verifies multi-Paxos and Raft.
%DISEL studies how to verify and compose different distributed protocols.
%It verifies protocols in separation uses a send-hook to restrict the
%interference among protocols. \sysname{} verifies different protocols in
%separate layers in isolation and then combine them not only horizontally
%as DISEL does but also vertically, and the witness structures allow reasoning
%about multiple protocols (e.g. leader election and operations) at the same time.
%While the main goal of IronFleet, Verdi, and DISEL is to generalize the
%distributed system verification environment, \sysname{} not only contributes to
%the same goal by proposing a composable withness and a network model but also
%contributes to a specialized verification of the leader-based distributed
%system.




\ignore{ %%%

\para{Distributed System Verification} IronFleet~\cite{ironfleet} is built on the Dafny language,
which allows the developer to annotate functions with pre- and post-conditions that are automatically proved
by an underlying SMT (satisfiability modulo theories) solver.
Dafny code can be compiled into C\# and eventually into assembly~\cite{ironclad}.
The verification of distributed protocols in IronFleet requires refinement proofs among three layers:
implementation, distributed protocol, and high level specifications.
The verification of distributed protocols is not fully machine-checked because
it requires hand written proofs to show that the atomic protocol step of the implementation
is equivalent to the interleaved protocol steps in the real world.
Our framework, on the other hand, enables fully machine-checkable verification and starts from a completely interleaved asynchronous network model.

Verdi~\cite{verdi} presents a distributed system verification tool chain,
where developers specify and implement a system using a functional language embedded in Coq.
The initial step of the development assumes a perfect network model, and later a transformer
in the tool chain automatically converts the system into one that can handle a more realistic network and fault model.
The tool chain can then extract executable OCaml code from the transformed program.
However, this approach can have limitations when the system to be verified is inherently designed for faulty network models, as is the case for Raft or Paxos.
Our framework is able to handle such systems by starting with very weak network assumptions.
Additionally, we verify C code and link it with a verified compiler~\cite{compcert} so our TCB does not include any compilers or runtime environments.

DISEL~\cite{disel} is another distributed system verification framework developed
by overlapping authors of Verdi. DISEL enables multiple distributed protocols
to be composed and verified in a system.
DISEL allows protocols that touch disjoint states to be verified
separately and later composed by using a send-hook
that restricts the interfering behaviors among protocols.
Similarly, our framework can verify different protocols in separate layers in isolation and then combine them not only horizontally as DISEL does,
but also vertically to hide verification details from the higher layers while preserving verified properties.

Brisk~\cite{canonical} is a tool that can automatically verify the absence of deadlocks in certain distributed systems
by synthesising the canonical sequentialization of a program.
This is very related to network reductions (discussed in Section \ref{subsec:link-with-low-level-code-verification})
in that it relies on the fact that program behavior sometimes remains the same even when certain network events are rearranged.
In particular, this is true if a program satisfies the symmetric non-determinism condition, which means
that every receive either has a unique corresponding send, or choosing any of multiple possible matching sends results in the same state.
Although Brisk can synthesise canonical sequentializations automatically, it has some limitations compared to manual proofs
of network reductions.
In particular, it cannot handle error cases from an unreliable network such as duplicated packets or timeouts.

\para{Paxos Verification and Deconstruction}
Within the category of distributed system verification, there is a significant amount of work focusing specifically on Paxos.
Lamport's first paper on the protocol~\cite{paxos}, and his second attempt to explain it more clearly~\cite{paxosmadesimple}
present the basic algorithm and give paper proofs of the safety properties.
Since then, many variations have been developed such as Disk Paxos~\cite{diskpaxos}, Egalitarian Paxos~\cite{epaxos},
and Vertical Paxos~\cite{vertpaxos}.
Oftentimes, although these variations seem similar to the original protocol, it is not possible to reuse the original
proof of the safety properties and a significant amount of work is required to re-prove them.
Lampson attempted to distill Paxos into its core components by creating a very high-level Abstract Paxos~\cite{Lampson2001}
and showing how other variants can be derived from it.
Some still felt that this did not get at the essence of the algorithm because at least two works since then \cite{dpaxos, sdpaxos}
have studied other ways of dividing Paxos into simpler components such that proofs of the protocol can be made more modular.
Our write-witness-passing approach also attempts to find a reusable framework for these proofs,
but instead of decomposing Paxos further, we make the key implicit invariants more explicit by passing around
the information needed to prove them.

} %%% end of ignore

%Many efforts have been recently made to formally verify distributed systems with machine checkable verification tools. IronFleet~\cite{ironfleet} and Verdi~\cite{verdi} are distributed system verification frameworks that use distributed consensus as a target exampmle for verification. The protocol layer of IronFleet is equivalent to the \globalstate{} of our framework where all proof about distributed protocols take place. In part, due to the limitation of verification tool that IronFleet uses, the network model required pencil and paper proof to show that an arbitrarily interleaved network model refines the IronFleet's network model. Verdi verifies distributed systems under an idealized network model, and presents transformations that preserve correctness to a weaker network model. However, this approach had limitations to fully verify systems that are inherently designed for weak network models. Our framework starts from a fully asynchronous network which is verfied only using machine checkable tools.

%unreliable network to begin with. While both papers propose a systematic way to verify a standalone distributed
%system, we employ a modular layer-based verification approach to enable extensible verification, where the
%proofs can be reused and connected with new verified application layers in the stack.

%It is well known that modularity leads to ease of verification. DISEL~\cite{disel} verifies independent distributed protocols in isolation and horizontally combines them. Taube et al.~\cite{modular} explores modularity for automated distributed system verification. A modularity based Paxos verification~\cite{dpaxos,sdpaxos} was explored but in pencil and paper proofs.

%Prior work has examined a layered storage system verification for crash safety~\cite{vijay,fscq, pushbuttonfs}
%Prior work has examined layered logical storage stacks to simplify storage system
%verification for crash safety~\cite{fscq, pushbuttonfs,vijay}.
%WOR shares the same insight about modularity,
%but leverages contextual refinement to provide incremental and extensible verification;
%enables both vertical and horizontal composition of layers; and verifies correctness
%in a concurrent and distributed environment.

%Contextual refinement encapsulates proofs in each layer, enables both vertical and
%horizontal composition of layers, and facilitates verifying WOR layer correctness against a
%high-order concurrent and distributed environment, which is simply passed as a context.
%Our verification approach can reason about different protocols independently in separate layers;
%our layered verification encapsulates the verification details and enable horizontal and vertical compositions.

%Formal verification plays a key role for guaranteeing
%the correctness of security features~\cite{vale, komodo, ironclad, expressos}.
%While WOR's proof does not focus on security, adding security features to the system
%and guaranteeing the security properties across WOR and application layers is a direction for future work.

%We uses the same CCAL approach~\cite{deepspec, concurrency} as CertiKOS~\cite{certikos:osdi16}. While CertiKOS first demonstrated the
%power of CCAL by verifying an entire OS, our verification showed that combination of CCAL the witness passing apprach provides a powerful framework for distributed system verification.

%FSCQ filesystem~\cite{fscq} uses a chain of modules to verify .
%While WOR layers mostly depend on only one layer, FSCQ modules often dependend
%on multiple modules for verification. Yggdrasil~\cite{pushbuttonfs} also introduces layers to verify a filesystem.
%While Yggdrasil verifies each layer's implementation refines the specification,
%the refinement relation between layers

%There have been many efforts to verify distributed systems in the systems and
%programming languages communities.
%The approaches that have been used have different strengths,
%weaknesses and philosophies.


%A push button verification approach, which is based on the Z3 SMT solver,
%was used to verify a filesystem~\needcite{} and an OS~\needcite{}.
%Its primary benefit is a low verification burden: careful writing of the specification for
%the SMT solver can automate the verification process
%and SMT solver yields a counter example for a failed verification.
%Push button verification mostly focuses on the functional correctness proofs and its tool chain
%that compiles the code to C or binary is not fully verified.
%The push button approach currently does not support verifying concurrency nor concurrent distributed systems.

\ignore{
\para{Certified Abstraction Layers}
The first formal presentation of how to use certified abstraction layers to verify large systems
was given by \citet{deepspec}.
\citet{concurrency} then showed how the framework could be extended to support concurrency.
One of the key strengths of this approach over other verification frameworks
is its support for contextual refinement.
This enables proofs to be done in a thread-local (or node-local) manner using an environment context to
capture the behavior of the rest of the world.
These proofs can then be linked together to obtain a strong correctness theorem for the entire system.

Our write-witness-passing framework benefits greatly from using CCAL.
By decomposing distributed systems into layers, we reduce the amount of time and effort needed to
prove functional correctness.
By treating each distributed node as a separate thread, we can use the environment context to
prove properties in a local context.
Combining this with write-witness-passing then allows us to bring in information about the global state when necessary.
We also use the environment context to model a realistic, non-deterministic network.
Another advantage afforded by CCAL is that we can lift our safety proofs to higher layers via contextual refinement.
This allows us to implement and verify a system once, and then reuse it as a component in various distributed applications.
}
%\para{Certified Abstraction Layers} \citet{deepspec}
%presented the first formal account of certified abstraction layers and
%showed how to apply layer-based techniques to build certified system
%software. The layer-based approach differs from Hoare-style program
%verification~\cite{hoare69,reynolds02,boogie05,nanevski06} in several
%significant ways. First, it uses the termination-sensitive forward
%simulation techniques~\cite{Lynch95,compcert} and proves a stronger
%contextual correctness property rather than simple partial or total
%correctness properties (as done for Hoare logics).
%%%%%%
%Second, the overlay interface of a certified layer object completely
%removes the internal concrete memory block (for the object) and
%replaces it with an abstract state suitable for reasoning; this
%abstract state differs from auxiliary or ghost states (in Hoare
%logic) because it is actually used to define the semantics of the
%overlay abstract machine and the corresponding contextual refinement
%property.
%%%%%%
%Third, as we move up the abstraction hierarchy by composing more
%layers, each layer interface provides a new programming language that gets
%closer to the specification language---it can call primitives at
%higher abstraction levels while still supporting general-purpose
%programming in C and assembly.

% We follows the same layer-based methodologies. Each time
% we introduce a new concrete concurrent object implementation, we
% replace it with an abstract atomic object in its overlay
% interface. All shared abstract states are represented as a single
% global log, so the semantics of each atomic method call would need to
% {\em replay} the entire global log to find out the return value.  This
% seemingly ``inefficient'' way of treating shared atomic objects is
% actually great for compositional specification. Indeed, it allows us
% to apply game-semantic ideas and define a general semantics that
% supports parallel layer composition.

% \para{Abstraction for Concurrent Objects}
% \citet{herlihy90} introduced {\em linearizability} as a key technique
% for building abstraction over concurrent objects. Developing
% concurrent software using a stack of shared atomic objects has since
% become the best practices in the system
% community~\cite{Herlihy08book,ospp11}. Linearizability is quite
% difficult to reason about, and it is not until 20 years later that
% \citet{filipovic10} showed that linearizability is actually equivalent
% to a termination-insensitive version of the contextual refinement
% property. \citet{Gotsman12concur} showed that such equivalence also
% holds for concurrent languages with ownership
% transfers~\cite{ohearn:concur04}.  Liang et al.~\cite{liang13,lili16} showed that linearizability plus various
% progress properties~\cite{Herlihy08book} for concurrent objects is
% equivalent to various termination-sensitive versions of the contextual
% refinement property. These results convinced us that we should prove
% termination-sensitive (contextual) simulation when building certified
% concurrent layers as well.

\ignore{
\wolf{I'm not sure if this is so relevant to this work. And if it is then I'm not sure how best to rewrite it to make that clear}
\para{RGSim and LiLi}
Building contextual refinement proofs
for concurrent programs (and program transformations) is challenging.
Liang~{et~al.}~\cite{RGSim,Liang14lics,lili16,xu16} developed the
Rely-Guarantee-based Simulation (RGSim) that can support both parallel
composition and  contextual refinement of concurrent
objects. Our contextual simulation proofs between two concurrent
layers can be viewed as an instance of RGSim if we extend RGSim with
auxiliary states such as environment contexts and shared logs. This
extension, of course, is the main innovation of our new compositional
layered model. Also, all existing RGSim systems are limited to reasoning
about atomic objects at one layer; their client program context cannot
be the method body of another concurrent object, so they cannot
support the same general vertical layer composition as our work does.

\citet{lili16} also developed a program logic called LiLi that can
directly prove both the linearizability and starvation-freedom (or
deadlock-freedom) properties. Their ``rely'' conditions are specified
over shared states only, so they cannot express temporal properties. To
prove progress, they have to introduce a separate temporal ``rely''
condition called {\em definite actions}.  This made it difficult to
provide a standalone (total) specification for each lock acquire
method.  Indeed, all examples in their paper are code fragments that
must acquire a lock, then perform critical-section tasks, and then release the
lock. In contrast, our environment context can specify the full
strategies (i.e., both the past and the future events) of all
environment threads and the scheduler, so we can readily impose
temporal invariants over the environment. Within each thread-modular
layer $L[t]$, we can show that each lock acquire primitive (e.g., for
ticket locks) always returns as long as its environment is cooperative
(e.g., always releases its acquired lock), even if $t$ itself may not
be cooperative.
In other words, the termination of $t$'s lock acquire
operation does not depend on whether $t$ itself will release the lock
after first acquiring it.
}

%\para{Treatment of Parallel Composition}
%Most concurrent languages (including those used by RGSim) use a
%parallel composition command $(C_1 \| C_2)$ to create and terminate
%new threads.  In contrast, we provide thread spawn and join
%primitives, and assign every new thread a unique ID (e.g., $t$, which
%must be a member of the full thread-ID domain set $D$). Parallel layer
%composition in our work is always done over the whole program $P$ and over
%all members of $D$. This allows us to reason about the current
%thread's behaviors over the environment's full strategies (i.e., both
%past and future events). Even if a thread $t$ is never
%created, the semantics for running $P$ over $L[t]$ is still well
%defined since it will simply always query its environment context to
%construct a global log.

% \para{Program Logics for Shared-Memory Concurrency}
% A large body of new program
% logics~\cite{ohearn:concur04,brookes:concur04,feng07:sagl,vafeiadis:marriage,LRG,verifast,gotsman13,Turon13popl,Turon13icfp,nanevski13,nanevski14,sergey15,sergey15pldi,pinto14,iris15,civl15,pinto16,xu16}
% have been developed to support modular verification of shared-memory
% concurrent programs. Most of these follow Hoare-style logics so they
% do not prove the same strong contextual simulation properties as RGSim
% and our layered framework do. Very few of them (e.g.,~\cite{pinto16})
% can reason about progress properties. Nevertheless, many of these
% logics support advanced language features such as high-order functions
% and sophisticated non-blocking synchronization, both of which will be
% useful for verifying specific concurrent objects within our layered
% framework. Our use of a global log is similar to the use of compositional
% subjective history traces~\cite{sergey15}; the main difference is
% again that our environment context can talk about both past and future
% events but a history trace can only specify past events.

% Both CIVL~\cite{civl15} and FCSL~\cite{sergey15pldi} attempt to build
% proofs of concurrent programs in a ``layered'' way, but their notions
% of layers are different from ours in three different ways: (1) they do
% not provide formal foundational contextual refinement proofs of
% linearizability as shown by \citet{filipovic10} and \citet{liang13};
% (2) they do not address the liveness properties; (3) they have not be
% connected to any verified compilers.

%\para{Compositional CompCert}
%\citet{stewart15} developed a new compositional extension of the
%original CompCert compiler~\cite{compcert} with the goal of providing
%thread-safe compilation of concurrent Clight programs.  Their
%interaction semantics also treats all calls to synchronization
%primitives as external calls. Their compiler does not support a layered
%ClightX language as our CompCertX does, so they cannot be used
%to build concurrent layers as shown in Fig.~\ref{fig:arch}.

% \para{Game Semantics} Even though we have used
% game-semantic concepts (e.g., strategies) to describe our
% compositional semantics, our concurrent machine and the layer simulation is still defined using
% traditional small-step semantics.  This is in contrast to several past
% efforts~\cite{ghica08,nishimura13,rideau11,abramsky99} of modeling
% concurrency in the game semantics community which use games to
% define the semantics of a complete language. Modeling higher-order
% sequential features as games is great for proving full abstraction,
% but it is still unclear how it would affect large-scale
% verification as done in the certified software community.  We
% believe there are great potential synergies between the two communities
% and hope our work will promote such interaction.

\ignore{
\wolf{this would be mostly redundant with the first two paragraphs if we made it about distributed system verification}
\para{Other Verification Works} There has been a large body
of recent work on  program verification.
seL4~\cite{klein2009sel4,klein14},
CertiKOS~\cite{certikos-osdi16}, Verve~\cite{hawblitzel10},
and Ironclad~\cite{ironclad}. None of these works have addressed the
issues on concurrency with fine-grained locking. Very recently,
\citet{xu16} developed a new verification framework based on RGSim
and Feng~{et~al.}'s program logic~\cite{feng08:aim} for reasoning
about interrupts; they have successfully verified many key modules
(in C) in the $\mu$C/OS-II kernel, though so far, they have not proved
any progress properties.

\jieung{The following parts are the thing that we have dumped from the OSDI18 paper. we have to
rephrase the previous part and the following part as well as add some distributed system verification works -
Especially, PLDI18, OOPSLA17a, OOPSLA17b, (POPL18, ESOP18 - DISEL related), (CPP, PLDI - verdi)}
}

%\para{Distributed systems} A number of abstractions similar to the WOR exist in theoretical distributed systems,
% including sticky registers~\cite{stickyregister}, consensus objects~\cite{herlihy1991wait}, and the Paxos register~
% \cite{li2007paxos}; however, these are abstractions for theoretical reasoning, rather than for programming or code
% verification. Other theoretical work points out the link between fault-tolerant atomic commit and consensus~
% \cite{frolund2001implementing, hadzilacos1990relationship}. SWMR (single writer many reader) registers support a
% single writer (which can write multiple times) and many readers; they can  be used to implement a WOR using a
% protocol like Disk Paxos~\cite{diskpaxos}.

%A number of abstractions similar to the WOR exist in theoretical distributed systems, including sticky registers~
%\cite{stickyregister}, consensus objects~\cite{herlihy1991wait}, and the Paxos register~\cite{li2007paxos}; however,
%these are abstractions for theoretical reasoning, rather than for programming or code verification. Other theoretical
%work points out the link between fault-tolerant atomic commit and consensus~\cite{frolund2001implementing,
%hadzilacos1990relationship}. SWMR (single writer many reader) registers are atomic registers that support a single
%writer (which is allowed to write multiple times) and many readers. SWMRs can be used to implement a WOR using a
%protocol like Disk Paxos~\cite{diskpaxos}.

%SWMR (single writer many reader) registers are atomic registers that support a single writer (which is allowed to write
%multiple times) and many readers. SWMRs can be used to implement a WOR using a protocol like Disk Paxos~
%\cite{diskpaxos}.

%The Paxos Register~\cite{li2007paxos} -- this looks clearly different from what we do, but how precisely?\\
%The Paxos Register~\cite{li2007paxos} presents the classic single-degree non-Byzantine and Byzantine Paxos
%algorithms using the analogy of a ``register". Here, a register corresponds to the
%value stored in the acceptors, and reading and writing a register corresponds,
%respectively, to the prepare and accept phases of the original algorithm.
%Place in context with Fast Paxos~\cite{fastpaxos}, Disk Paxos~\cite{diskpaxos}, Cheap Paxos~\cite{cheappaxos}. Are
%these different implementations of a WOR?

%Distributed applications often use services that embed consensus or replication protocols,
%such as Chubby~\cite{chubby} and Zookeeper~\cite{zookeeper}.
% WOR supports a more primitive abstraction compared to these services.
%Horus~\cite{horus} is a modular stack for group communication that led to a verification effort called Ensemble~\cite{ensemble}.
%Distributed transaction systems~\cite{janus, tapir} often
%combine transaction protocols with consensus protocols,
%`opening the Paxos box' to implement different optimizations.
%These could conceivably be implemented over the WOR API in the same manner as the optimizations
%in Section \ref{sec:wormtx}.
%The WOR APIs could conceivably be used to implement similar optimizations without rewiring the Paxos protocol.
%%%%%%%%%%%%%%%%%%%%%%%%%%%%%%%%%%%%%%%%%%%%%%%%%%%%%%%%%%%%%%%

%\vspace{-0.2em}

\paragraph{Conclusions}

We have presented \sysname{}, a framework for verifying the key safety
properties of leader-based distributed systems. \sysname{} provides the common
template for modeling and specifying leader-based distributed systems and the
template-based proof of linearizability of state update and leader election. 
\sysname{} uses the witness, a novel logical data structure that keeps track of the
history of global state changes and invariants, to facilitate the 
reasoning about the system and the proof. Using \sysname{}, we verify
leader-based distributed systems including multi-Paxos, and compare the system
characteristics based on the \sysname{} template. Thanks to the underlying
certified abstraction layer verification approach that \sysname{} uses,
verified properties of \sysname{} can be connected from the high-level
specification to the C and assembly implementation. 



%As far as we know, our approach is the first work which provides vertical 
%and horizontal compositions on distributed system verification as well as 
%the capability of the full proof linking low-level implementation written in C 
%or Assembly with the high-level representation of the system.

%As far as we know, our example of Paxos verification
%is the first work to mechanically prove the functional correctness
%and safety of a distributed protocol written in
%a low-level programming language such as C or Assembly.


\chapter{Related Work}
\label{chapter:related}

\section{Certified Abstraction Layers}

Our CCAL toolkit follows the layer-based compositional proof approach proposed by Gu \etal~\cite{deepspec}.
They proposed certified abstraction layers (CAL) for sequential programs and 
verified certified system software to show the applicability of the approach.
CAL has many different aspects with
 hoare-style program verification~\cite{hoare69,reynolds02,boogie05,nanevski06}.

First, CAL proves a contextual correctness property by using the termination-sensitive forward simulation 
techniques~\cite{Lynch95,compcert}, 
which is stronger than simple partial or total correctness properties guaranteed by Hoare logic style verification.
Besides, Its overlay interface of a certified object thoroughly disables 
the concrete memory block (of the object)  for the future usage 
and replaces the memory with an abstract state;
this abstract state is suitable for high-level reasoning such as proving security properties and progress properties (when it is extended to support
concurrency as CCAL does). 
It also differs from ghost or auxiliary states in hoare-style program verification in terms of 
its usage.
The abstract state is a part of an entire state of the transition machine; 
thus it can be used to define the semantics of the overlay abstract machine
as well as the corresponding contextual refinement property.
Lastly,  building layers enable us to provide a new state transition machine and a new programming language that gets close to the specification language. 
Higher layers replace more low-level memory blocks and function implementations with 
the corresponding abstract states and primitives,
and can call primitives at higher abstraction levels while  
supporting general-purpose
programs written in C and assembly.

Those differences are also the same in the comparison between CCAL in Chapter~\ref{chapter:ccal} and 
hoare-sytle program verification. 
When introducing a new concrete concurrent object implementation in our layers, 
we always replace a set of events generated by the object with an atomic event; thus we replace the implementation with an abstract atomic object in its overlay interface. 
In CCAL, 
all shared abstract states are represented as a single global log,
and atomic method calls always have to 
replay the entire global state and find the return value to refer to the current status of shared abstract states.
This global log approach treats all shared atomic objects with a single log.
Thus this  is seemingly an inefficient way of treating multiple shared atomic objects,
but great for providing compositional 
It allows us to apply game semantics ideas, therefore
is required for supporting parallel layer composition.




\section{Linearizability}
Linearizabilty~\cite{herlihy90}  is a well-known safety condition for concurrent objects 
and has been studied for decades. 
Developing concurrent software using a stack of shared atomic objects has since
become the best practice in the system
community~\cite{Herlihy08book,ospp11}. 
The original definition of linearizability instrumented programs
to record a global history of method-invocation and method-return
events. However, that's not a convenient theorem statement when
verifying client code.
In this sense,
Linearizability is considered as a quite 
difficult problem to reason about, and it is until 20 years when 
Filipovic \etal~\cite{filipovic10} showed that linearizability is actually equivalent
to a termination-insensitive version of the contextual refinement.
Followed by that work, methodologies to simplify linearizability have been
proposed mostly in a concurrent programming context~\cite{Elmas10tacas,
Liang13pldi,Gotsman12concur,Viktor10CAV}.
Among them, Gotsman \etal~\cite{Gotsman12concur} showed that the equivalence between linearizability and a termination-insensitive version of the contextual refinement property also
holds for concurrent languages with ownership transfers~\cite{ohearn:concur04}.  
Liang \etal~\cite{liang13,lili16} also showed that linearizability with various
progress properties~\cite{Herlihy08book} for concurrent objects is
equivalent to multiple termination-sensitive versions of the contextual refinement property. 
These results convinced us that 
termination-sensitive (contextual) simulation is the proper property to show
when building certified
concurrent layers as well.

Some
authors have presented mechanized verification of linearizability
(\eg~ \cite{DGLMQueue,DerrickSW11}),
They, however, are not directly on executable
code, but on abstract transition system models.
The formulation in CCAL is closer to Derrick \etal~\cite{DerrickSW11}, who prove a simulation
to a history of single atomic actions modifying abstract state.  
Some mechanized proofs in the distributed system area show 
the linearizability of Raft and
multi-Paxos~\cite{cppraft, ironfleet}.
%
% Linearizability for distributed systems - let's decide weather adding that part or not after finish chapter witnesspassing
Linearizability
and leader-soundness proofs in Chapter~\ref{chapter:witness-passing} build on top of existing work on
linearizability: the key insight for the proof is to base the reasoning on
an atomic step of each operation.  Yet, we generalize the reasoning of linearizability
to the leader-based distributed system and create a reusable proof template.
The template extracts
necessary states from the witness that is created by the lower level
specification of concrete systems
of a generic leader-based distributed system and provides a proof template based on
abstract states which are represented as the witness.
The witness encodes atomic steps backed by a quorum, which approves state
changes, and the validity of the leader who is initiating the atomic actions.
Our template guarantees the linearizability of any leader-based system that
satisfies the constraints of the witness. Using a similar witness structure, our
model automatically verifies the soundness of the leader based on
linearizability of leader election.




\section{Program Logics for Shared-Memory Concurrency}
Multiple program  logics~
\cite{cap10, ohearn:concur04,brookes:concur04,feng07:sagl,vafeiadis:marriage,LRG,verifast,gotsman13,Turon13popl,Turon13icfp,nanevski13,nanevski14,
sergey15,sergey15pldi,pinto14,iris15,civl15,pinto16,xu16}
have been proposed for 
modular verification of shared-memory concurrent programs. 
Among them, most modern separation-style concurrent logics~
\cite{cap10,Turon13popl,sergey15pldi,pinto14,iris15,pinto16} do
not prove the same strong termination-sensitive contextual simulation
properties as our work does,
while very few of them (e.g.,~\cite{pinto16})
can reason about progress properties.
On the other hand,
RGSim~\cite{RGSim} as well as our layered framework prove the same strong contextual simulation properties.

Many of these program logics~\cite{Turon13popl,iris15}, however, support 
higher-order functions 
and sophisticated non-blocking synchronization,
which our work does not address.
Both of which will be
useful for verifying specific concurrent objects within our layered
framework. 
Our use of a global log is similar to the use of compositional
subjective history traces, a history trace specifies past events, proposed by Sergey \etal~\cite{sergey15},
Our environment context, however, can talk about both past and future events rather than only past events.
Total-TaDA~\cite{pinto16} can be used to prove
the total correctness of concurrent programs but it has not been
mechanized in any proof assistant and there is no formal proof that
its notion of liveness is precisely equivalent to Helihy's notion of
linearizability and progress properties for concurrent
objects~\cite{Herlihy08book}. 
Going
beyond safety, one also wants to prove a progress property such as
wait-freedom~\cite{herlihy91:waitfree} or (in our case)
starvation-freedom~\cite{Herlihy08book}.

Two previous works, CIVL~\cite{civl15} and FCSL~\cite{sergey15pldi},
propose the way to build and prove concurrent programs in a \textit{layered} way like CCAL does. 
However, their layers differ from CCAL layers in three aspects;
1) their approaches do not support contextual refinement proofs of linearizability like Filipovic \etal~\cite{filipovic10} and Liang \etal~\cite{liang13} does;
2) they are lack of connections with any verified compilers; and
3) they do address liveness properties as we have shown in MCS lock verification.




\section{Treatment of Parallel Composition}

\para{RGSim and LiLi.}
Building contextual refinement proofs for concurrent programs and supporting parallel composition for concurrent program proofs 
are challenging.
Liang \etal~\cite{RGSim,Liang14lics,lili16, liang:2017} 
proposes several approaches from Rely-Guarantee-based Simulation (RGSim) 
that support parallel
composition and  contextual refinement of concurrent
objects.
The contextual simulation proof between two concurrent layers in CCAL 
is an instance of RGSim variance; the extended version of RGSim by adding auxiliary states including environmental contexts and shared logs. 
They are the main ingredients of our CCAL framework 
to build our new  compositional
layered model.
All existing RGSim systems are limited to reasoning
about atomic objects at one layer.
Since their client program context cannot 
be the method body of another concurrent object, 
they cannot
support the vertical layer composition what our CCAL toolkit supports.
Their recent work \etal~\cite{liang:2017} presents a way for specifying and verifying the progress of concurrent objects with partial methods, but the mechanized proof is out of their research scope. 

LiLi (Linearizability and Liveness) is a program logic based on RGSim
that can
directly prove both the linearizability and starvation freedom (or
deadlock-freedom) properties together. 
Their ``rely'' conditions are specified
over shared states only, so they cannot express temporal properties. 
To prove progress, they have to introduce a separate temporal ``rely''
condition called {\em definite actions}.  This made it difficult to
provide a standalone (total) specification for each lock acquire
method.  Indeed, all examples in their paper are code fragments that
must acquire a lock, then perform critical-section tasks, and then release the
lock. In contrast, our environmental context can specify the full
strategies (i.e., both the past and the future events) of all
environment threads and the scheduler, so we can readily impose
temporal invariants over the environment. Within each thread-modular
layer $L[t]$, we can show that each lock acquire primitive (e.g., for
ticket locks) always returns as long as its environment is cooperative
(\eg~ always releases its acquired lock), even if $t$ itself may not
be cooperative.
In other words, the termination of $t$'s lock acquire
operation does not depend on whether $t$ itself will release the lock
after first acquiring it.

Most concurrent languages use a
parallel composition command $(C_1 \| C_2)$ to create and terminate
new threads.  
Our approach is slightly different from most previous works.
We provide thread spawn and assign every new thread a unique ID $t$,
which is not reusable and must be a member of the full thread-ID domain set $D$.
As we have shown in Chapter~\ref{chapter:linking},
our parallel layer composition happens in specific layers and it is
always done over the whole program $P$ and over
all members of $D$. 
This difference allows us to reason about the current
thread's behaviors over the environment's full strategies (\ie~ both
past and future events).
Our composition treats the semantics for running $P$ over $L[t]$ is still well
defined even if a thread $t$ is never
created, because it will always query its environment context to
construct a global log.

% add View paper %

\para{Extending CompCert and Verified Compilation.} 

Stewart {\etal}~\cite{stewart15} extends the original CompCert compiler~\cite{compcert}  
to support compositional thread-safe compilation of concurrent Clight programs. 
They introduce their interaction semantics, which treats
synchronization
primitive calls as external calls.
They, however, does not support a layered ClightX language as we support in our CompCertX.  
Thus, building concurrent layers is impossible based on their work.
Kang \etal~\cite{hur16} and Ramananandro \etal~\cite{ramananandro:2015} also modified CompCert compiler to support separate compilation and composition, 
but they do not support concurrency.  
Other works on the verified compilation~\cite{Lochbihler10esop, Sevcik11popl, zhao:2013, kang:2018} does not support concurrent and/or compositionality. 


\para{Game Semantics.} 

We have used
game-semantic concepts such as strategies to describe our
compositional semantics, 
but CCAL still uses traditional small-step semantics for its concurrent machine and layer simulation, which differs from 
past works~\cite{ghica08,nishimura13,rideau11,abramsky99}  for concurrency modeling in the game semantics community.
They use games to
define the semantics of a complete language, 
and modeling higher-order sequential features as games is excellent for proving full abstraction. 
However, it is unclear how it is applied to the large-scale verification in the certified software community.


\section{OS Kernel Verification} 

A large body
of recent work including
seL4~\cite{klein2009sel4,klein14},
Verve~\cite{hawblitzel10}, Hyperkernel~\cite{hyperkernel}, and Komodo~\cite{komodo} are 
addressed OS kernel verification.
None of these works, however, do not handle the issues on concurrency with fine-grained locking.
Xu \etal~\cite{xu16} developed a new verification framework by facilitating RGSim
and Feng~\etal's program logic~\cite{feng08:aim} for reasoning
about interrupts;.
They verified many key modules (in C) 
in the $\mu$C/OS-II kernel, but they have not proved
any progress properties or proof linking properties like we do.
Other verified systems~\cite{klein2009sel4,hawblitzel10,ironclad},
are single-threaded, or use a per-core big kernel lock.


The Verisoft team used VCC~\cite{leinenbach09} to verify spinlocks in a
hypervisor by directly postulating a Hoare logic rather than building
on top of an operational semantics for C, and only proved properties
about the low-level primitives rather than the full functionality of
the hypervisor. By contrast, CertiKOS deals with the problem of
formulating a specification in a way that can be used as one layer
inside a large stack of proofs. As for CertiKOS itself, while we
discussed the ``local'' verification of a single module, other papers
explain how to relate the log and context to a more realistic
nondeterministic machine model~\cite{certikos:osdi16}, how to
``concurrently link'' the per-CPU proofs into a proof about the full
system~, and how this extends to multiple threads per
CPU~\cite{concurrency}.
CertiKOS is an end-to-end verified concurrent system showing that its
assembly code indeed ``implements'' (contextually simulates) the
high-level specification.

CertiKOS is an end-to-end verified concurrent system showing that its
assembly code indeed ``implements'' (contextually simulates) the
high-level specification.


\section{Other Work on the MCS Algorithm}





As far as we know, two other efforts apply formal verification methods
to the MCS algorithm.  Ogata and Futatsugi~\cite{ogata:mcs-lock} used the UNITY program logic to 
develop a mechanized proof for the MCS algorithm.
They work, however, is not with executable code, but with an abstract transition system. 
Their correctness proof works by refinement like we do (between a fine-grained
and a more atomic spec) but they directly prove backward
simulation.

One difference is that Ogata and Futatsugi's proof is
done using a weaker fairness assumption. They assume ``every CPU gets
scheduled infinitely often'', while we require a maximum scheduling
period ($F$ in Section~\ref{sec:liveness-atomicity}).  This is because
we write our specification of \lstinline$wait_lock$ as a Coq function
defined by recursion on a natural number, and all Coq functions must
be total. So although our ultimate theorem only states that the method
terminates ``eventually'', as an intermediate lemma we need to prove
an explicit natural number bound on when a given call to
\lstinline$wait_lock$ will finish.  We could avoid this by e.g. using
Coq's facilities to define functions by well-founded recursion, and
making the termination measure $M_i$ take ordinal instead of number
values, but in practice assuming a fixed $F$ seems like a reasonable
model of multi-core concurrency.

The other MCS Lock verification we know of is by Liang and
Feng~\cite{lili16}, who define a program logic LiLi to prove
liveness and linearizability properties and verify the MCS algorithm
as one of their examples.  The LiLi proofs are done on paper, so they
can omit many ``obvious'' steps, and they work with a simple
while-loop language instead of C. Many of the concepts in our proof
are also recognizable in theirs. The state of their concrete
  programs includes a pointer $\mathrm{\texttt{tail}}$ and nodes
  $\mathrm{\texttt{Node}}(\mathrm{\texttt{busy}}, \mathrm{\texttt{next}}, \mathrm{\texttt{ThrdID}})$.
In their invariant and precondition they use specificational variables
$\mathrm{\textit{ta}}$ and $\mathrm{\textit{tb}}$ (like \texttt{la} in
Sec.~\ref{subsec:atomicoperation}), $\mathrm{\textit{tl}}$ and $S$ (like $q$
and $s$ in Sec.~\ref{sec:representation-ghost}). Their
  ``wellformed lock'' predicate $\mathrm{\textsf{lls}}$ includes our
  tail-soundness and next-correctness properties, so in order to prove
  that the invariant is preserved they need essentially the same
  lemmas as in Section~\ref{sec:representation-ghost}. and their
termination measure $f(\mathrm{G})$ includes the length of
$\mathit{tl}$ and the size of $S$ (like $M$ in
Sec.~\ref{sec:liveness-atomicity}. On the other hand, the fairness
constant makes no appearance in $f(\mathrm{G})$, because fairness
assumptions are implicit in their inference rules.

A big difference between our work and LiLi is our emphasis on
modularity.  Between every two lines of code of a program in LiLi, you
need to prove all the different invariants, down to low-level data
representation in memory. The specification takes the form of a single
pre- and post-condition which involves concepts at many level of
abstraction. For example, unfolding the definition of the measure $f$,
we find not only $\mathit{tl}$, but also the tail-pointer $p$, and
eventually the lock-array $\mathit{ta}$. In our development, these
concerns are in different modules which can be completed by different
programmers.  Similarly, we aim to produce a stand-alone specification
of the lock operations. In the LiLi example, the program being
verified is an entire ``increment'' operation, which takes a lock,
increments a variable and releases the lock. The pre/post-conditions
of the code in the critical section includes the low-level
implementation invariants of the lock, and the fact the lock will
eventually be released is proved for the ``increment'' operation as a
whole. Our locks are specified using \emph{bound} numbers, so they can
be used by many different methods.

Apart from modularity, one can see a more philosophical difference
between our CCAL approach and program logics such as LiLi.  Liang
and Feng are constructing a program logic which is tailor-made
precisely to reason about liveness properties under fair
scheduling. To get a complete mechanized proof for a program in that
setting would require mechanizing not only the proof of the program
itself, but also the soundness proof for the logic, which is a big
undertaking. Other parts of the program will favor other kinds of
reasoning, for example many researchers have studied program logics
with inference rules for reasoning about code \emph{using} locks. One
of the achievements of the CertiKOS style of specification is its
flexibility, because the same model---a transition system with data
abstraction and a log of events---works throughout the OS kernel. When
we encountered a feature that required thinking about liveness and
fairness, we were able to do that reasoning without changing the
underlying logical framework.


\section{Distributed system verification}

Approaches to verify distributed systems have been explored actively over the
past few years. IronFleet~\cite{ironfleet} annotates functions with pre- and
post-conditions to automatically prove the correctness of the code with an SMT (satisfiability
modulo theories) solver and proves a refinement relation between the code and
protocol proofs in different layers.

Verdi~\cite{verdi} is a distributed system verification tool chain
where developers specify and implement a system using a functional language
embedded in Coq while assuming a perfect network model. 
The system can then automatically convert
it into a system that handles a more realistic network and failure model,
and provides the refinement property of the system behavior on different network and failure models.
%They facilitate Verdi to verify Raft.~\cite{cppraft} 
%Using the framework, they show the correctness and the linearizability of their own Raft implementation written in Coq.

DISEL~\cite{disel}
studies how to verify and horizontally compose different distributed protocols.
It verifies the protocols in separation and uses send-hooks to restrict the
interference among protocols. Throne assumes a realistic network to begin with
and it not only allows both horizontal and vertical composition of verified distributed
protocols but also supports combined reasoning of protocols by using witnesses.

Taube et al.~\cite{modular} studied adding
decidability for verifying distributed systems using SMT solvers.
The work surrounding SMT solvers has a philosophical difference with our verification approach.
We use high-order logic and CCAL, which have higher expressiveness but less automation,
while SMT-solver-based work requires encoding higher-order concepts into
first-order logic for better automation. While a relatively large portion of the
verification can be automated, the SMT-solver-based approach has some limitations.
For example, certain higher-order properties (e.g. network reduction
in IronFleet) are not always encodable or verifiable in first order logic.


\section{Paxos Verification and Deconstruction}
A huge body of work exists on verifying the Paxos protocol~\cite{paxos}.
Lamport provided a proof sketch at the time of proposing the protocol~\cite{paxosmadesimple}
and Lampson attempted to distill Paxos into its core components by creating a very
high-level Abstract Paxos~\cite{Lampson2001} and showing how variants of
Paxos can be derived from it. Additional efforts were made to divide Paxos
into simpler components~\cite{dpaxos, sdpaxos} to find a reusable framework for
proving its variants. Our witness-based approach also attempts
to find a reusable framework for these proofs, but instead of decomposing Paxos
further, we make the key implicit invariants more explicit by passing around
the information needed to prove them.

Padon et al. verified high-level specifications of many variants of Paxos
and multi-Paxos while proposing a method to specify the protocols using
a decidable first-order logic~\cite{paxosepr}. Verification of Paxos
variants was made possible by reusing the specification and proof of vanilla
Paxos. Throne on the other hand can reuse the proof of a Paxos instance
due to CCAL to prove variants of Paxos using proof objects and verifies both the
C code and the specification of the system based on witness.


Woos et al. presented the verification of Raft~\cite{cppraft}. The properties
verified for Raft includes linearizability and soundness of the leader and
the code written in Coq translates to executable OCaml code. Throne
delegates verification of such Raft properties common to leader-based distributed
systems to the template and the final executable code extracted from
Throne is in assembly, which is generated from CompCert C compiler and is more
optimized than the OCaml code.


Within the category of distributed system verification, there is a significant amount of work focusing specifically on Paxos.
Lamport's first paper on the protocol~\cite{paxos}, and his second attempt to explain it more clearly~\cite{paxosmadesimple}
present the basic algorithm and give paper proofs of the safety properties.
Since then, many variations have been developed such as Disk Paxos~\cite{diskpaxos}, Egalitarian Paxos~\cite{epaxos},
and Vertical Paxos~\cite{vertpaxos}.
Oftentimes, although these variations seem similar to the original protocol, it is not possible to reuse the original
proof of the safety properties and a significant amount of work is required to re-prove them.
Lampson attempted to distill Paxos into its core components by creating a very high-level Abstract Paxos~\cite{Lampson2001}
and showing how other variants can be derived from it.
Some still felt that this did not get at the essence of the algorithm because at least two works since then \cite{dpaxos, sdpaxos}
have studied other ways of dividing Paxos into simpler components such that proofs of the protocol can be made more modular.
Our write-witness-passing approach also attempts to find a reusable framework for these proofs,
but instead of decomposing Paxos further, we make the key implicit invariants more explicit by passing around
the information needed to prove them.

This enables proofs to be done in a thread-local (or node-local) manner using an environment context to
capture the behavior of the rest of the world.
These proofs can then be linked together to obtain a strong correctness theorem for the entire system.

Our witness benefits greatly from using CCAL.
By decomposing distributed systems into layers, we reduce the amount of time and effort needed to
prove functional correctness.
By treating each distributed node as a separate thread, we can use the environment context to
prove properties in a local context.
Combining this with the witness then allows us to bring in information about the global state when necessary.
We also use the environment context to model a realistic, non-deterministic network.
Another advantage afforded by CCAL is that we can lift our safety proofs to higher layers via contextual refinement.
This allows us to implement and verify a system once, and then reuse it as a component in various distributed applications.


\section{Model of leader-based distributed system}
Our distributed system verification focuses on leader-based distributed system model, 
which is based on the two functions,
leader election and update functions, is inspired by CASPaxos~\cite{caspaxos},
which implements an atomic shared object. CASPaxos uses prepare and accept
phases similar to those in Paxos but it can repeatedly apply a compare-and-swap
functions to the stored state instead of setting it just once. While CASPaxos is a system implementation
to atomically update a distributed object, we use the implementation style of
CASPaxos to build a generic specification for leader-based distributed systems.
Our specification also resembles the high-level specification of a state machine
replication protocol, which Lamport generalized~\cite{generalizedconsensus}.
While Lamport mainly models multi-Paxos and consensus,Throne models and
verifies common properties of a generic leader-based distributed system,
which includes multi-Paxos and state machine replication.






\chapter{Limitations, Future Work, and Conclusions}
\label{chapter:conclusion}

\para{Conclusions}
We have developed write-witness-passing, a simple but powerful methodology for
verifying the key safety properties of distributed systems.
This approach can also be easily combined with
the concurrent verification framework~\cite{concurrency}.
Together they provide the tools
to build and verify distributed systems in a scalable and extensible way with 
less human effort required.
As far as we know, our example of Paxos verification
is the first work to mechanically prove the functional correctness
and safety of a distributed protocol written in
a low-level programming language such as C or Assembly.
We believe this is a promising approach that can be applied to many other distributed systems.
In the future we hope to use write-witness-passing to create mechanized proofs of some of the systems we
describe in Section \ref{sec:witness-passing-semantics-with-paxos-variants} as well as other
complicated protocols such as PBFT \cite{pbft}.



\backmatter

\bibliography{refs}
% for your own sake, use a bibtex file, so all of the numbering of references will be done
% automatically.

\end{document}
