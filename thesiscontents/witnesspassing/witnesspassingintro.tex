\jieung{Check the notations for list app and cons (and others) - those things should be consistent with the operators in previous sections}

The previous chapter (Chapter~\ref{chapter:wormspace}) illustrates that our verification work on distributed systems, 
a different kind of concurrent programs with multicore/multithreaded concurrent programs. 
It shows 
that $/ccalname$ can be applied not only to the concurrency by local machines
but also to the concurrency in the cluster of multiple machines in distributed protocols. 
By using the approach, we also verified the safety property of $\WOR$, immutability. 
However, verifying a single distributed protocol is insufficient to build large-scale distributed applications because they use multiple distributed systems as their bases.

Proving the correctness of the applicaiton requires the verification of the underlying system, but even verifying a single distributed system is
still very challenging. In this paper, we focus on one of the most commonly
used families of distributed systems, the leader based distributed system, and study the standard properties to build a verification framework.
We build a generic specification to model leader-based distributed systems, and based on
the specification we create a proof template that can verify the common
desirable properties of the leader based system: linearizability and soundness of
the leader election. 
Our framework uses a novel logical data structure called witness which keeps track of the history of state changes in distributed nodes.
The witness can be used to reason about different distributed functions and can
be composed to verify the properties of the entire distributed system. We provide
multi-Paxos as the primary example and compare it with
how Raft, two-phase commit, and other protocols are modeled and verified under
our framework. 
We demonstrate that once systems are specified to satisfy the proof template's requirement, the witness data structure delivers necessary
information to the proof template and common properties to leader-based systems are proved for free. 






\jieung{In this chapter, I copied and pasted and slightly revised our PLDI submisison.  But, I will update this part continuously}