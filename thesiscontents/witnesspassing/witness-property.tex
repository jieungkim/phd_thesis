\section{Common Property Proof}
\label{chapter:witnesspassing:sec:prove-safety-with-witness}

Proving the safety property of generic specifications using witnesses is similar to showing that invariants of the witness hold.
%Besides the basic invariant including Theorem~\ref{thm:witness-invariant},
%other assumptions about protocol specific functions can be defined and used to 
%prove other crucial properties of distributed systems. 
This section illustrates the proof of one of the common desirable properties
of a leader-based distributed system, linearizability.
Linearizability guarantees that if a quorum of approvers accepts a request $c$, 
all the subsequently issued requests reflects the result of $c$. 
The proof of linearizability can take advantage of the witness. 

\begin{lemma}
All issued requests have a unique identifier.
\end{lemma}

This lemma enables us to distinguish all requests. To achieve this goal the 
update function $u$, which is an argument of $\opfunction$ should fulfill a 
property that monotonically increases the version number to distinguish all
requests:
\begin{center}
$\forall \ (st \ st' : (\seqnum * \dsvalue)) : \ u \ st = st' \rightarrow \text{fst }st
<_{\seqnum}  \text{fst }st'$,
\end{center}
where $<_{\seqnum}$ is a comparison relation of two version numbers.


Another requirement to build a linearizable leader-based distributed system is to make sure that the elected leader can get the latest consistent
state of the system. Such a state is represented as a witness with the most extended list, and this witness includes the entire history of all committed requests. 
It requires the check for witnesses to figure out the witness that contains the longest the committed witness. If there are two committed witnesses from two approvers, choose the witness that includes longer uncommitted witness. It only requires the check for the last index value in the uncommitted witness if there are elements in the uncommitted witness.


This proof is related to generic functions, $\ldrfunction$ and $\ldrhfunction$, 
and the witness construction functions, $\ldrwcons$ and $\opwcons$.
First, the $g$ function in $\ldrhfunction$ has to satisfy that  
it always returns the current status to the requester.
Second, the $g_{post}$ function has to choose the value mapped with the highest version number among the valid acknowledgments. 
Third, the $\ldrwcons$ function should be called with the proper write witness, which is the witness mapped with the highest version number among all witnesses from the approvers (in a set of quorum).
With the given constraints, the following two lemmas can be proved.

\begin{lemma}[Election consistency]\label{lemma:chapter:witnesspassing:election-consistency}
The leader election does not miss any committed operation.
\end{lemma}

\begin{lemma}[Leader witness] 
The leader always contains the up-to-date commit witness that is represented as
witness with the largest number of entries.
\end{lemma}

\begin{proof}
Proving the election consistency is providing that system-specific functions satisfy the conditions that we illustrated. However, satisfying these
conditions are common requirements for system-specific functions in
our approach and thus if we assume that the conditions already hold for all
system-specific functions, the proofs for the lemmas are straightforward.
\end{proof}

With these lemmas, proving the linearizability can be proved.

\begin{theorem}[linearizability]\label{theorem:chapter:witnesspassing:linear-spec}
The commit witness is linearizable.
\end{theorem}

\begin{proof}
Our witness contains all the linearizable history that is matched with the current local state, 
and the witness in all consistent nodes are the same due to our witness passing mechanism. 
Therefore, linearizability is captured by our witness.
\end{proof}

Since the leader election collects the leader witness with the largest
size using $\ldrwcons$, the linearizability of leader election, and thus uniqueness of the leader
for each term can be proved in a similar way. 
