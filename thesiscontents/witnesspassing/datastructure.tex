
\section{Specifications for Leader Based Systems}
\label{chapter:witnesspassing:sec:specs-for-leader-based-system}

\sysname{} specifies leader-based distributed systems with a generic
specification and a system-specific specification. 
The generic specification defines the high-level behavior of the leader-based 
distributed system and the system-specific specification, which is encapsulated 
within the generic specification describes the low-level behaviors. 

The generic specification obviates the need for the user to write specifications
of essential safety properties of the leader-based system and instead enables them to
focus on writing the system-specific parts. The
generic and system-specific specifications are designed in a modular fashion and
can be composed to form a full specification of a leader-based distributed system.

Modeling the leader-based system requires two kinds of actors in the system:
\textit{requesters} and \textit{approvers}. 
Requesters seek permission to perform actions, which approvers either authorize or deny.
This separation need not be physical as in some protocols, such as Raft, one node can act
both as a requester and approver simultaneously.
The generic specification for both requesters and approvers 
works as a template for writing system-specific
specifications and for common safety property proofs.
Safety properties can be verified over the generic specification once,
and the system-specific specification, which meets the pre-defined template in
the generic specification, is automatically guaranteed to be safe. The safety
properties that \sysname{} guarantees are linearizability of state update and
soundness of leader selection.

\begin{figure}
\includegraphics[width=0.95\textwidth]{figs/witnesspassing/overviewfig}
\caption{Leader-based Distributed System Process Flow: the process flow can be
	expressed using template functions and system-specific functions can
	be plugged into the template. Examples show multi-Paxos, 
	Raft~\cite{raft}, two-phase commit with external membership service, 
	and Lamport's distributed lock~\cite{lamportclock}.}
\label{fig:chapter:witnesspassing:process-flow}
\end{figure}

The generic specification includes two template functions for both actors that describe the
abstract behavior of a leader-based distributed system:
\begin{itemize}
	\item $\ldrfunction$ (the requester's spec) and  $\ldrhfunction$ (the approver's spec) specify how a leader is elected, and
	\item $\opfunction$  (the requester's spec) and $\ophfunction$ (the approver's spec)  specify how distributed states in the system are
		updated.
\end{itemize}

The operations of a leader based system can be described using these functions
as illustrated in Figure~\ref{fig:chapter:witnesspassing:process-flow}.
We first define the definitions that are necessary to understand the
functions and then explain both functions in detail.

\subsection{State Definition}
\label{subsec:state-definition}

\begin{figure}
\begin{small}
\raggedright

\textbf{Local State}\\
$$
\begin{array}{lllllllll}
\nodeid &:& Type & \mbox{(Node ID)} & & \termnum &:& Type & \mbox{(Term num)}^* \\
\seqnum &:& Type & \mbox{(Version num)}^* & &\dsvalue &:& Type & \mbox{(Local object)}\\
\dstate &:=& (\termnum  \times \seqnum \times \dsvalue) & \mbox{(Local state)} & & & & & \\
\end{array}
$$
\\
\raggedleft

* : Total order required

\raggedright
\textbf{Global State}\\
$$
\begin{array}{lllllllll}
\fullset &:=& \set{\nodeid} & \mbox{(Full node ID set)} & & \gstate &:=& \set{ \nodeid \mapsto \dstate ~\vert~ \forall \nodeid \in \fullset} & \mbox{(Global state)}\\
\end{array}
$$
\\
\textbf{Network} \\
$$
\begin{array}{llll}
\packet &: & Type & \mbox{(Network event)}\\
\networklog &:=& \mbox{list}\ \packet & \mbox{(Network log)}\\
\replay &:& \networklog \rightarrow  \gstate& \mbox{(Log replay function)}\\
\end{array}
$$
\\
\textbf{Protocol-specific Functions}  \\
$$
\begin{array}{llll}
	\updatefunc & := & \dstate \rightarrow (\seqnum \times \dsvalue) &
	\mbox{(Obj update function)} \\
\gldrfunction & := & \nodeid \rightarrow \networklog \rightarrow \networklog  & \mbox{(Protocol functions)} \\
\gopfunction & := & \nodeid \rightarrow \updatefunc \rightarrow \networklog \rightarrow \networklog  & \mbox{(Protocol functions)} \\

\sendfunction & := & \nodeid \rightarrow \dstate \rightarrow \networklog & \mbox{(Send function)}\\
\recvfunction & := & \nodeid \rightarrow \dstate \rightarrow \networklog & \mbox{(Recv function)}\\
\end{array}
$$
\end{small}
\vspace{-1em}
\caption{Basic Definitions}
\label{fig:chapter:witnesspassing:basic-state}
\vspace{-1.5em}
\end{figure}

Figure~\ref{fig:chapter:witnesspassing:basic-state} presents the necessary definitions for the
generic specification.
Nodes must have unique identifiers of type $\nodeid$, which can be any type with decidable equality.
A term number of type $\termnum$ acts as a logical temporal indicator in the system.
Term numbers do not need to correspond to any physical clock,
but they are used to resolve ordering conflicts between operations within the system,
so they must be totally ordered.
Nodes also contain a version number of type $\seqnum$, which is associated with the sequence of updates
of the local object $\dsvalue$.
A local object is one node's partial view of the global distributed object.
Since each node's local state may not be exactly in sync globally due to network or
node failures, the version number indicates how up-to-date a node's state is.
\jieung{remove the next sentence?}
Nodes may also contain additional system-specific state inside their distributed object.

The global state $\gstate$ is the collection of all the local states hosted by the
entire system, which is composed of the set of distributed nodes $\fullset$.
There is a partial map from each node in the distributed system ($\fullset$)
to its local state ($\dstate$), which is a tuple consisting of a term number,
a version number, and a local object.

As mentioned in the previous chapter, 
distributed systems usually have one shared resource, the network. 
Nodes in a distributed system communicate via the network. 
Inspired by our concurrent verification framework in Chapter~\ref{chapter:ccal},
the global state $\gstate$ can be reconstructed by
applying the log replay function $\replay$ to the
network log $l : \networklog$, which is a list of network events ($\packet$).
The effect of a call to either $\tplldrfunction$ or $\tplopfunction$ by a node is atomic 
and it cannot be interfered by other operations, 
but function calls by other nodes can be interleaved.
The interleaving is expressed by our network model, described in Section~\ref{chapter:witnesspassing:sec:low-level-implementation} in detail.
Our framework keeps track of each function call
within the network log and the replay function $\replay$ computes the
global state based on the network log.
Finally, \sysname{}'s generic specification is parameterized by some protocol-specific
functions that we explain later in this section.


\subsection{Leader Election}
\label{chapter:witnesspassing:subsec:leader-election}

\subsubsection{Requester}
The goal of leader election is to set a new term, give one node exclusive access to the
entire system, and synchronize every node's local state such that the
global state is prepared to be updated.

Using function definitions in Figure~{fig:chapter:witnesspassing:basic-state}, we define the template function for leader election as follows:
$$
\begin{array}{l}
	\ldrfunction (nid: \nodeid) (g_{pre}\ g_{post}:  \gldrfunction)
	(tx: \sendfunction)(rx: \recvfunction)(l_1  :\networklog ) := \\
\ \ \ \ \ \ \ \mbox{\textbf{let}} \ l_2 := g_{pre}\ nid \ l  \ \mbox{\textbf{in}}\ \mbox{\textbf{let}}\ l_3 := tx(nid, \replay(l_2 \mdoubleplus l_1)[nid]) \ \mbox{\textbf{in}}  \\
\ \ \ \ \ \ \ \mbox{\textbf{let}} \ l_4 := rx(nid, \replay(l_3 \mdoubleplus l_2 \mdoubleplus l_1)[nid]) \
  \mbox{\textbf{in}}\\ 
\ \ \ \ \ \ \ \mbox{\textbf{let}} \ l_5 := g_{post}\ nid \ (l_4 \mdoubleplus l_3 \mdoubleplus l_2 \mdoubleplus l_1) \ \mbox{\textbf{in}} \ l_5. \\
\end{array}
$$
where $\replay(l)[nid]$ is a projection from the global state generated by the log replay function to 
the local stated of $nid$.

This function captures the common form of a leader election scheme, which sends requests to
and gets approval from a set of authoritative nodes.
First, $g_{pre}$ changes the local state of a candidate node to prepare
for a leader election and generates corresponding request messages.
Network messages are sent and received using $tx$ and $rx$, which define system-specific
communication patterns between the requester and the approvers.
Finally, $g_{post}$ determines the final outcome of the
leader election and executes any actions that are necessary before becoming the leader.

The table on the left side of Figure~\ref{fig:chapter:witnesspassing:process-flow} lists concrete examples of the parameters of
$\ldrfunction$. For example, multi-Paxos requires a majority vote from the acceptors to
become the leader, a two-phase commit protocol that uses a membership service may delegate
a transaction coordinator selection to this service, and Lamport's lock requires all node's
approval to acquire the lock. Note that the transaction coordinator and the lock holder are
regarded as leaders. The typical post-election operations are checking whether the
previous leader left the system in a consistent state and making the state consistent
if necessary. 

\subsubsection{Approver}
The other part of leader election is how the approvers handle requests.
This typically involves some protocol-specific check of the request's
term number, followed by a local state update, and then an acknowledgement
containing the new state. We express this pattern as follows:
$$
\begin{array}{l}
	\ldrhfunction (nid: \nodeid) (g:  \gldrfunction)
	(tx: \sendfunction)(rx: \recvfunction)(l_1  :\networklog ) := \\
\ \ \ \ \ \ \ \mbox{\textbf{let}} \ l_2 := rx(nid, \replay(l_1)[nid]) \
  \mbox{\textbf{in}} \\
\ \ \ \ \ \ \ \mbox{\textbf{let}}\ l_3  := g\ nid \ (l_2 \mdoubleplus l_1) \ \mbox{\textbf{in}} \\
\ \ \ \ \ \ \ \mbox{\textbf{let}} \ l_4 :=  tx(nid, \replay(l_3 \mdoubleplus l_2 \mdoubleplus l_1)[nid]) \ \mbox{\textbf{in}} \ l_4.
   \\
\end{array}
$$
Clearly, to prove any interesting properties of $\ldrfunction$ or $\ldrhfunction$
the system-specific state update functions must be restricted in some way.
For example, the term number modified by protocol-specific functions in $\ldrfunction$
need to increase monotonically in most leader-based distributed systems.
Additionally, $g$ in $\ldrhfunction$ should not touch the value inside
the distributed object in order to separate the leader election and the commit operation.
We discuss more about those requirements that the parameters must satisfy and what properties we
are able to prove in Section~\ref{chapter:witnesspassing:sec:prove-safety-with-witness}.

\subsection{Commit}
\label{chapter:witnesspassing:subsec:commit}

\subsubsection{Requester}
The commit function has a similar format to the leader election function, but
includes an object update function $u : \updatefunc$:

$$
\begin{array}{l}
\opfunction (nid: \nodeid) (u : \updatefunc) (g_{pre}\ g_{post}:  \gopfunction)
	(tx: \sendfunction)
	(rx: \recvfunction)
	(l_1  :\networklog ) := \\
\ \ \ \ \ \ \ \mbox{\textbf{let}} \ l_2 := g_{pre}\ u \ l_1  \ \mbox{\textbf{in}}\   l_3 := tx(nid, \replay(l_2 \mdoubleplus l_1)[nid]) \ \mbox{\textbf{in}}  \\
\ \ \ \ \ \ \ \mbox{\textbf{let}} \ l_4 := rx(nid, \replay(l_3 \mdoubleplus l_2 \mdoubleplus l_1)[nid]) \
	\mbox{\textbf{in}} \\
\ \ \ \ \ \ \ \mbox{\textbf{let}} \ l_5 := 	 g_{post}\  u \ (l_4 \mdoubleplus l_3 \mdoubleplus l_2 \mdoubleplus l_1) \ \mbox{\textbf{in}}\ l_5.
\end{array}
$$
The object update function defines how the system updates the local object and
the version number depending on the local state of each node.
The update function is encapsulated in the message that is sent to the target node
via the send function ($tx : \sendfunction$).

Examples of how $\opfunction$ function is used are found on the right side
of Figure~\ref{fig:chapter:witnesspassing:process-flow}. The number of recipients of the commit request differs depending
on the system; it could be all nodes in the system or just a few targeted nodes.
For the protocols that require more than 2 rounds in their commit phases, such as 
two-phase commit, $\opfunction$ should be called multiple times and $g_{pre}$ and
$g_{post}$ should have a flag to handle different functions that correspond to
each round. 

\subsubsection{Approver}
When a requester triggers a commit operation, the approvers need to
handle the operation and update the local view of the distributed object if
necessary. Unlike $\opfunction$, $\ophfunction$ does not take an update function as an argument
because it is already in the message from the requester.
To retrieve the update function from the received message $\ophfunction$ requires
an auxiliary function, $\mathrm{proj}_{f}$, which extracts the update function from the network log.
$$
\begin{array}{l}
	\ophfunction (nid: \nodeid) (g:  \gopfunction)
	(tx: \sendfunction)(rx: \recvfunction)(l_1  :\networklog ) := \\
\ \ \ \ \ \ \ \mbox{\textbf{let}} \ l_2 := rx(nid, \replay(l_1)[nid]) \
  \mbox{\textbf{in}} \\
\ \ \ \ \ \ \ \mbox{\textbf{let}} \ u := \mathrm{proj}_{f} \ (l_2 \mdoubleplus l_1, nid) \
  \mbox{\textbf{in}} \\
\ \ \ \ \ \ \ \mbox{\textbf{let}}\ l_3  := g\ u\ nid \ (l_2 \mdoubleplus l_1) \ \mbox{\textbf{in}} \\
\ \ \ \ \ \ \ \mbox{\textbf{let}} \ l_4 := tx(nid, \replay(l_3 \mdoubleplus l_2 \mdoubleplus l_1)[nid]) \ \mbox{\textbf{in}}  \ l_4.
   \\
\end{array}
$$


