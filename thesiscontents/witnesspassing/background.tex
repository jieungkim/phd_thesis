\section{Leader-Based Distributed Systems}
\label{chapter:witnesspassing:sec:leader-based-distributed-systems}


We define a leader-based distributed system with a set of nodes 
$\fullset$ that are connected over the network and host a distributed shared state/object
$\dobj$. At each monotonically increasing term $\termnum$, only one leader 
$L \in \fullset$ can make changes to $\dobj$. There can be many ways to elect
$L$, though the  election protocol must assign at most one $L$ per $\termnum$,
or the system should include a conflict resolution scheme to only apply
state changes by at most one $L$ at $\termnum$.
Update operations can fail during execution, but the effect on $\dobj$ should be atomic. 
For example, a $\multipaxos$ system can reach a
state where a $\paxos$ instance did not reach a consensus but contains an
incomplete state change proposal from $\gstate$ to $\gstate^\prime$.
Then, $\protocolfunction$ should ensure that this incomplete state is either safely abandoned (in favor of retaining state $\gstate$)
or completed (in favor of adopting $\gstate^\prime$ in its entirety)
before executing any subsequent operations. 
Our approach only focuses on the leader-based systems that can be modeled by this approach.
We further formally model the elements of a leader-based distributed system.

\section{MultiPaxos}
\label{chapter:witnesspassing:multipaxos}


We use the fixed-configuration multi-decree $\paxos$ protocol (a simplified variant
of the protocol commonly referred to as $\multipaxos$)~\cite{rvrpaxos}
as an example protocol to explain how our approach facilitates modeling and
verifying leader-based distributed systems. $\multipaxos$ is a
state machine replication protocol. In $\multipaxos$, a system's state changes
are chosen by synod consensus of a $\paxos$ instance and replicated across acceptor nodes.
The $\paxos$ instance consists of a group of acceptor nodes, while the $\paxos$ protocol ensures that a state chosen by the instance is immutable.

Each $\paxos$ instance requires a prepare and an accept phase to choose a new state. 
In this system, a proposer sends the request and acceptors accept the new state. 
A successful prepare requires a quorum (majority) of acceptor acknowledgments, and the state is successfully chosen only if a quorum of acceptors have accepted the state. 
For an acceptor to accept a state, the proposer's prepare
request should be the last request (ordered by a round number $r$)
that the acceptor has received.
Once a state is chosen by the $\paxos$ instance, 
proposers attempting to propose a new state to the same instance are obligated to propose the already-chosen state; 
this makes the chosen state immutable. 
If there are contending proposers attempting to propose a new state to a $\paxos$ instance that does not have a chosen state, 
the proposers can indefinitely race with each other by overwriting the prepare request.

As an optimization for successive proposals, $\multipaxos$ elects a leader (dedicated proposer) that can propose new state changes to multiple $\paxos$ instances as long as no other proposer attempts to prepare to write. However, the original $\paxos$ paper~\cite{paxos, paxosmadesimple} 
primarily focuses on the operation of a single Paxos instance, and developers are left to choose their implementation of $\multipaxos$.


Our $\multipaxos$ example employs the scheme described below:
\begin{itemize}
	\item {\textbf{Leader election: }} whenever the leader at $\termnum$ is
		suspected to be dead, a proposer increments $\termnum$ to $\termnum'$ and
		sends a vote request to the acceptors.
		If a proposer receives a majority of votes with a $\termnum'$ tag it
		becomes the leader for round $\termnum'$.
	\item {\textbf{Preparation of new leader: }}  the new leader checks and completes any partially completed tasks by the previous leader and determines the tail of the log  $t$. 
The leader batch then prepares the log entries ($\paxos$ instances) that come after $t$ with a round number  $r = \termnum'$.
	\item {\textbf{Proposing new states: }} the leader sends proposals
		to $\paxos$ instance $t+1$ whenever new state changes need to be chosen.
		The leader ensures   that $t$ is incremented only after the
		request at $t+1$ is successfully chosen (the leader retries if the request fails) to prevent gaps  in the log.
\end{itemize}


\section{Global vs Local Reasoning}
\label{chapter:witnesspassing:sec:global-local-reasoning}


One of the major challenges of distributed system verification is that it requires reasoning regarding global invariants 
 (\ie, properties that hold for the entire distributed state rather than just
one node's local state).
However, relations between local and global invariants are not straightforward. 
As such, one must trace the network history back in time and consider multiple nodes’ states simultaneously in order to reason globally. 
This is further complicated if one assumes a network that can have failures such as duplicated or lost messages. 
We attempt to reduce the amount of global reasoning by logically gathering sufficient information in the local 
state to prove the desired properties. We call this gathered information a \textit{witness}, and it is helpful to first observe a simple example while we provide a formal treatment. 

\begin{figure}
\begin{minipage}{\linewidth}
\noindent
\begin{multicols}{2}
  \lstinputlisting[numbers = left, language=C, mathescape=true, escapeinside={(*}{*)},
  deletekeywords={struct, prop}, morekeywords={such,that,forall,in,null,to}]{source_code/witnesspassing/paxos_spec.c}
\end{multicols}
\end{minipage}
\caption{Paxos with Witnesses. Witness Extensions in \bfseries{Bold}.}
\label{fig:chapter:witnesspassing:paxos-witness}
\end{figure}

Figure~\ref{fig:chapter:witnesspassing:paxos-witness} 
contains pseudo-code for single-decree $\paxos$, which has the same routine as the pseudo-code in Figure~\ref{fig:chapter:multipaxos:paxos-pseudocode};
 however, evidence of distributed object manipulations exists with the witness.
 The parts in bold are necessary additions to augment $\paxos$ with witnesses. These additions are purely logical in that they do not appear in the actual implementation. 
 It is also clear that the overall behavior of the protocol does not depend on the witnesses; 
 they exist purely to aid in reasoning regarding the protocol. By having witnesses and assuming a well-formedness property, certain safety invariants become more obvious.
 
For example, one of the primary safety properties of $\paxos$ is immutability, 
which implies that it is safe for an acceptor to write a value $v$ when either no value is currently written in a majority of acceptors or $v$ is already written in a majority of acceptors. 
In Figure~\ref{fig:chapter:witnesspassing:paxos-witness}, it is evident that in order for a proposer to write a value system-wide, 
it must have the highest round number that a majority of acceptors have seen. 
It is also evident that whenever an acceptor writes a value, it exactly matches the one in the witness. 
Then, by inspecting the promises stored in the witnesses, we can demonstrate that every value in the witnesses originated from a proposer with the highest round number among the majority of acceptors. 
As such, we can know that the witnesses contain the history of all written values. In light of this fact, we create a sketch a proof of immutability.
