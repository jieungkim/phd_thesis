\section{Leader-Based Distributed Systems}
\label{chapter:witnesspassing:sec:leader-based-distributed-systems}

\jieung{\sysname{} - let's change the name later}

We define a leader-based distributed system as a system with a set of nodes
$\fullset$ that are connected over the network and host a distributed shared state/object
$\dobj$. At each monotonically increasing term $\termnum$, at most one leader
$L \in \fullset$ can make changes to $\dobj$. There can be many ways to elect
$L$, but the election protocol must assign at most one $L$ per $\termnum$,
or the system should include a conflict resolution scheme to only apply
state changes by at most one $L$ at $\termnum$.
Update operations can fail during execution, but the effect on $\dobj$ should be atomic. 
For example, a $\multipaxos$ system can reach a
state where a $\paxos$ instance did not reach a consensus but contains an
incomplete state change proposal from $\gstate$ to $\gstate^\prime$.
Then $\protocolfunction$ should make sure that this incomplete state is either safely abandoned (in favor of retaining state $\gstate$)
or completed (in favor of adopting $\gstate^\prime$ in its entirety)
before executing any subsequent operations. 
Our approach only focuses on the leader-based systems that can be modeled by this approach.
We further formally model the elements of a leader-based distributed system.

%The constraints of state updates, which are specified in $\protocolfunction$ in \sysname{},
%are that update operations can fail during execution, but the effect on $\dobj$ should be atomic. For example, a $\multipaxos$ system can reach a
%state where a $\paxos$ instance did not reach a consensus but contains an
%incomplete state change proposal from $\gstate$ to $\gstate^\prime$.
%Then $\protocolfunction$ should make sure that this incomplete state is either safely abandoned (in favor of retaining state $\gstate$)
%or completed (in favor of adopting $\gstate^\prime$ in its entirety)
%before executing any subsequent operations. 
%\sysname{} only focus on the leader-based systems that can be modeled by this approach.
%We further formally model the elements of a leader-based distributed system.

\section{MultiPaxos}
\label{chapter:witnesspassing:multipaxos}


We use the fixed-configuration multi-decree $\paxos$ protocol (a simplified variant
of the protocol commonly referred to as $\multipaxos$)~\cite{rvrpaxos}
as an example protocol to explain how our approach facilitates modeling and
verifying leader-based distributed systems. $\multipaxos$ is a
state machine replication protocol. In $\multipaxos$, a system's state changes
are chosen by synod consensus of a $\paxos$ instance and replicated across acceptor nodes.
The $\paxos$ instance consists of a group of acceptor nodes and the $\paxos$ protocol
ensures that a state chosen by the instance is immutable.

Each $\paxos$ instance requires a prepare and an accept phases to choose a new
state. A proposer sends the request and acceptors accept the new state.
A successful prepare requires a quorum (majority) of acceptor acknowledgments
and the state is successfully chosen only if a quorum of acceptors
accepted the state. For an acceptor to accept a state, the proposer's prepare
request should be the last request (ordered by a round number $r$)
that the acceptor has received. Once a state is chosen by the $\paxos$ instance,
proposers trying to propose new state to the same instance are obligated to
propose the already-chosen state; this makes the chosen state immutable.
If there are contending proposers trying to propose a new
state to a $\paxos$ instance, which does not have a chosen state, the proposers
can indefinitely race with each other by overwriting the prepare request.

As an optimization for successive proposals, $\multipaxos$ elects a
leader (dedicated proposer), which can propose new state changes to multiple $\paxos$
instances for as long as no other proposer attempts to prepare to write. However, the
original $\paxos$ paper~\cite{paxos, paxosmadesimple} mostly focuses on the operation
of a single $\paxos$ instance
and developers are left to choose their implementation of $\multipaxos$.


Our $\multipaxos$ example employs the scheme that is described as below:
\begin{itemize}
	\item {\textbf{Leader election: }} whenever the leader at $\termnum$ is
		suspected to be dead, a proposer increments $\termnum$ to $\termnum'$ and
		sends a vote request to the acceptors.
		If a proposer receives a majority of votes with a $\termnum'$ tag it
		becomes the leader for round $\termnum'$.
	\item {\textbf{Preparation of new leader: }} the new leader checks and completes any
		partially finished tasks by the previous leader and figures out the
		tail of the log $t$. Then the leader batch prepares the log entries
		(Paxos instances) that come after $t$ with a round number $r = \termnum'$.
	\item {\textbf{Proposing new states: }} the leader sends proposals
		to $\paxos$ instance $t+1$ whenever new state changes need to be chosen.
		The leader makes sure that $t$ is incremented only after the
		request at $t+1$ is successfully chosen (retries if the
		request fails) to prevent holes in the log.
\end{itemize}


\section{Global vs Local Reasoning}
\label{chapter:witnesspassing:sec:global-local-reasoning}


One of the challenges of distributed system verification is that it requires reasoning about
global invariants, (\ie, properties that hold for the entire distributed state rather than just
one node's local state).
However, relations between local and global invariants are not straightforward 
So,  one must trace the network history back in time and consider multiple
nodes' states at once to reason globally.
It is further complicated if one assumes a network that can have failures such as duplicated
or lost messages.
We attempt to reduce the amount of global reasoning by logically gathering sufficient
information in the local state to prove the desired properties.
We call this gathered information a \textit{witness}, and while we provide a formal treatment, it is helpful first to see a simple example.


\begin{figure}
\begin{minipage}{\linewidth}
\noindent
\begin{multicols}{2}
  \lstinputlisting[numbers = left, language=C, mathescape=true, escapeinside={(*}{*)},
  deletekeywords={struct, prop}, morekeywords={such,that,forall,in,null,to}]{source_code/witnesspassing/paxos_spec.c}
\end{multicols}
\end{minipage}
\caption{Paxos with Witnesses. Witness extensions in \bfseries{bold}.}
\label{fig:chapter:witnesspassing:paxos-witness}
\end{figure}

Figure~\ref{fig:chapter:witnesspassing:paxos-witness} contains pseudo-code for single-decree $\paxos$,
which has the same routine with the pseudo-code in Figure~\ref{fig:chapter:multipaxos:paxos-pseudocode},
but with the witness, the evidence of distributed object manipulations.
The parts in bold are the necessary additions to augment $\paxos$ with witnesses.
These additions are purely logical in that they do not show up in the actual implementation.
It is clear also that the overall behavior of the protocol does not depend on the witnesses;
they exist purely to aid in reasoning about the protocol.
By having the witnesses and assuming a well-formedness property certain safety invariants are
made more obvious.

For example, one of the primary safety properties of $\paxos$ is immutability, which says that it
is safe for an acceptor to write a value $v$ when either no value is currently written in a
majority of acceptors or $v$ is already written in a majority of acceptors.
Looking at Figure~\ref{fig:chapter:witnesspassing:paxos-witness} we can see that in order for a proposer to write a value system-wide,
it must have the highest round number that a majority of acceptors have seen.
It can also be shown that whenever an acceptor writes a value, it exactly matches the one in the witness.
Then by inspecting the promises stored in the witnesses, we can show that every value in the witnesses
came from a proposer with the highest round number among a majority of acceptors.
Therefore we know that the witnesses contain a history of all written values.
We use this fact to sketch a proof of immutability.
%
%\begin{proof}
%Consider a proposer in phase 2a with a witness $wit$ and a value $val$.
%Proceeding by induction on the length of $wit$, the base case is trivial because no witnesses
%means no value is written so it is safe to write anything.
%In the inductive case we have a value $old$ that was the last write in round $rnd$.
%If $val$'s round number is $rnd'$ then it must be the case that $rnd' >= rnd$ because
%the acceptors would have rejected the messages otherwise.
%If $rnd' = rnd$ then $val = old$ because round numbers are unique and only one value can be proposed
%per round, so we are done.
%If $rnd' > rnd$ then that implies there was a write in a round after $rnd$, which contradicts our
%assumption that $rnd$ was the latest write.
%Therefore, for any acceptor in phase 2b, the value in the witness is always safe to write and since
%it matches the value in the message from the proposer, immutability is preserved.
%\end{proof}

%\jieung{Check the correctness of these tables!!}
%
%\begin{figure}
%\begin{center}
%%\includegraphics[width=.45\textwidth,page=1,trim=0 120 0 50,clip]{figs/witnesspassing/construct_witness}
%	\includegraphics[width=.7\textwidth, page=1]{figs/witnesspassing/witnessspmp}
%\end{center}
%\caption{Single-Paxos Witnesses: fields are (round number, value) (promises omitted)}
%\label{fig:chapter:witnesspassing:paxos-witness-table}
%\vspace{-1em}
%\end{figure}
%
%\begin{figure}
%\begin{center}
%%\includegraphics[width=.45\textwidth,page=2,trim=40 90 0 0,clip]{figs/witnesspassing/construct_witness}
%	\includegraphics[width=.7\textwidth, page=2]{figs/witnesspassing/witnessspmp}
%\end{center}
%\caption{$\multipaxos$ Witnesses: fields are (cell-index, round number, value) (promises omitted)}
%\label{fig:chapter:witnesspassing:multipaxos-witness-table}
%\vspace{-1em}
%\end{figure}
%
%Another strength of witnesses is their compositionality.
%Figure~\ref{fig:chapter:witnesspassing:paxos-witness-table} shows the witnesses for two instances of $\paxos$.
%The second write on each $\paxos$ instance keeps track of witnesses for each
%instance: writes at round 3 in both instances shows that value $v_{1}$ and
%$v_{2}$ originally came from writes at round 1 and round 2 respectively. 
%If we want to extend our proofs about single-Paxos to $\multipaxos$, we can construct
%a function that composes the witnesses in the manner shown in Figure~\\ref{fig:chapter:witnesspassing:multipaxos-witness-table}.
%The composed witness is almost equivalent to what is actually stored in the sequence of
%$\paxos$ instances in $\multipaxos$, which shows step-by-step state changes.
%Even in a leader-based distributed system that does not maintain an explicit log 
%to record state changes, the composed witness can provide the entire history of
%state changes that can be used to reason about the system. 

