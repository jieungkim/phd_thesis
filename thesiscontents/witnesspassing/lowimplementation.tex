\section{Representing the Network}
\label{chapter:witnesspassing:sec:low-level-implementation}


Previous sections have shown how we provide general specifications
that can accommodate a variety of distributed systems.  To assure correctness of a concrete implementation, a distributed system verification also needs to
provide evidence that actual running codes refine high-level generic
specifications. While mere functional correctness is relatively well-understood,
it is not sufficient for distributed system verification. The
system's behavior concerning the network must be taken into account. 
The previous chapter addresses this issue and provides a network model only for one specific protocol, $\WOR$. 
In here, we have generalized our asynchronous network model to make it be used for multiple protocols by filling out the necessary information. 




\subsection{Network Model}
\label{chapter:witnesspassing:subsec:network-model}

\begin{figure}
\begin{small}
\raggedright
$$
\begin{array}{llll}
\chid & := & \mathbb{Z} & \mbox{(channel id)} \\
\msg & : & Type & \mbox{(message)} \\
\mathrm{gmsg} & : & Type & \mbox{(ghost message)} \\
\mathrm{Esnd} & := & \nodeid \times \msg \\
\mathrm{Erecv} & := & \nodeid \times \msg \\
\mathrm{eType} & := & \mathrm{Esnd} + \mathrm{Erecv} + \nodeid + \mathrm{gmsg} \\
\mathrm{E}_{net} & := & \chid \times \nodeid \times \mathrm{eType} \\
\end{array}
$$
\end{small}
\caption{Network Definitions.}
\label{fig:chapter:witnesspassing:net-defs}
\end{figure}


A network model here is one instantiation of a log in
 Section~\ref{chapter:witnesspassing:sec:specs-for-leader-based-system}. Our network is defined as a list of
network events, and key definitions are found in Figure~\ref{fig:chapter:witnesspassing:net-defs}. A
network event ($\mathrm{E}_{net}$) represents an interaction between a node and
a greater distributed system of which it is part. This $\mathrm{E}_{net}$ is a
low-level instantiation of $\packet$ from Figure~\ref{fig:chapter:witnesspassing:basic-state}.

At the lowest logical level, there are four kinds of events: 1) send, 2) receive,
3) timeout, and 4) ghost. A send event represents an interaction in which one
node sends data to another node and registers corresponding send events in the
network log. A receive event represents the moment in time when a node begins
processing data that arrived from another node (not the moment of the arrival
itself). A timeout event represents the interaction in which a node decides it
has not received an expected communication from another node. Ghost events do
not correspond to any concretely observable network communication, but can
greatly simplify local reasoning about the global state of the distributed
system by explicitly demarcating global state transitions in the network log.

All network events carry a channel identifier ($\chid$), the purpose of which is
to logically separate network events into disjoint subsets for separate
reasoning and refinement. Assigning different channel identifiers for each
protocol and each verification layer allows for the composition of protocols and
for network reductions (see Section~\ref{chapter:witnesspassing:subsec:connection}).

Network events also carry a source node ID, which denotes the node that
instigated the event. For any given datagram sent from node $a$ to node $b$, the
network log would eventually contain a send event with source $a$ and either a
receive event or a timeout event also with source $a$. The source of a ghost
event is the ID of the node that observed the distributed system transition from
one state to another. \ignore{(for example, a Paxos proposer observing that it has
received responses from a majority of acceptors, meaning that the system as a
whole has come to consensus).} Send, receive, and timeout events carry a
destination node ID indicating the intended recipient of the datagram to which
they correspond. Send and receive events carry a message ($\msg$), the payload
of the communication.

A local node's view of the network log grows over time by the action of two
primitives - send and receive. The low-level send primitive adds a single send
event to the network log. There is no acknowledgment and send always succeeds.
The low-level receive primitive conceptually collects data received from other
nodes, but in addition to receive and timeout events, populates the network log
with all intermediate events including events not observable by the
concrete local node. For reasoning purposes, receive consults a network oracle
to fill in the log. The oracle is constrained by predicates that model only
relevant forms of network failure,(which can be specified per protocol), and
verification of a protocol will involve quantifying over all such valid oracles
to show the correctness of the local node's behavior and the greater distributed
system.


\subsection{Connection with Generic Specifications}
\label{chapter:witnesspassing:subsec:connection}

\begin{figure}
\begin{center}
\includegraphics[scale=0.35]{figs/witnesspassing/network_reduction.pdf}
\end{center}
\caption{Network Reduction in Distributed Systems.}
\label{fig:chapter:witnesspassing:network-reduction}
\end{figure}

\paragraph{Network Reduction}
The network model in concert with $\ccalname$ permits a logical refinement step in
which a complex view of the network log is reduced to a simpler one for easier
reasoning. Channel ids and projection functions allow these network
reductions to be applied independently to disjoint subsets of the network log.
This facilitates clear separation of protocol verification and aids
horizontal composition.

Two example network reductions are shown in Figure~\ref{fig:chapter:witnesspassing:network-reduction}. In
reduction (a), a two-step reduction simplifies a network log in which node $P1$ issues three independent send events. The first step rearranges the three send
events so that they are adjacent in the network log. The second step
consolidates the three send events into a single broadcast event. These
reductions do not involve a change in the behavior of the system (the first because
the send events are independent of the intervening network log, and the second
because a broadcast has the same semantics as individual sends). Each reduction
introduces a new channel and disallows an old one (the channel id is the first
number in brackets in each event in the figure).

The separation of network reductions by channel enables verification of a
protocol by parts and of several protocols in a single system. Since network
reductions can be focused by channel, verification of different protocols is
separable and thus composable.


\subsubsection{Building Global State Transition Systems}

\paragraph{Log Replay Function}
Specifications that we have described in
 Section~\ref{chapter:witnesspassing:sec:specs-for-leader-based-system}
 are based on the network log and network
log replay function. To connect low-level specifications with these log-oriented
specifications, a refinement from a direct update on local state is necessary. 
 $\ccalname$  showed how this can
be done with CPU schedules, and the same technique applies straightforwardly to
the network log.

A high-level global state calculated by a replay function contains local
states for each node. A network oracle, and by extension of local states, can
be constrained with protocol-specific assumptions for rely-guarantee reasoning.

