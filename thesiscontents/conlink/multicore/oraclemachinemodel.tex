


\subsection{Oracle Machine Model: Multicore Machine Model with Hardware Scheduler}
\label{chapter:linking:subsec:multicore-machine-model-with-hardware-scheduler}


\begin{figure}
\noindent\fbox{state: }
$$
\begin{array}{lll}
\localstatekwd &:=& \localstate{(ps: \privatestate)}{(coin: \booltype)}\\
\localstatepool{S : \set{\ztype}} & := &  \ztype \rightharpoonup \localstatekwd\\
& & \hfill  (\forall i . i \in S \rightarrow \exists . (ps : \localstatekwd) (i, ps) \in \privatestatepool{S}) \wedge   (\forall j . j \notin S  \rightarrow (j, \_) \in \localstatepool{S}) \\

%\mcstatekwd & := & \mcstateconkwd :~ \ztype \rightarrow \set{i \mapsto \localstatekwd~\vert~ i \in \coreset } \rightarrow \mclog \rightarrow \hstatekwd\\
\mcstatekwd & := & \mcstate{(curid: \ztype)}{(lsp: \localstatepool{\coreset})}{(log: \mclog)}\\
\end{array}
$$

\noindent\fbox{oracle step: $(oracle:\mcoracle{\coreset}{\mclog}{\mcevent}) :  \mcstatekwd \rightarrow \cctracekwd \rightarrow \mcstatekwd \rightarrow \mcprop$}
\begin{mathpar}
\inferrule[progress]
{ \getpstsome{\localstatekwd}{curid}{lsp}{(\localstate{ps}{\bfalse})}\\
\hardwarelocalstep{curid'}{(\localview{ps}{l})}{(\localview{ps'}{l'})}\\
\setpst{\localstatekwd}{curid'}{\localstate{ps'}{\btrue}}{lsp}{lsp'}}
{\oraclestep{oracle}{(\mcstate{curid}{lsp}{l})}{\ccemptytrace}{(\mcstate{curid}{lsp'}{(\listappoppo{l'}{l})})}}

\inferrule[yield]
{  l_0 = \listconsoppo{\yieldbackev{curid}}{l} \\
l' = \listconsoppo{\yieldev{curid'}}{l_0} \\
\getpstsome{\localstatekwd}{curid'}{lsp}{(\localstate{ps}{\btrue})}\\
\setpst{\localstatekwd}{curid'}{\localstate{ps}{\bfalse}}{lsp}{lsp'}\\
\oget{\coreset}{\mclog}{\mcevent}{l_0}{oracle}{{\yieldbackev{curid}}}}
{\oraclestep{oracle}{(\mcstate{curid}{lsp}{l})}{\ccemptytrace}{(\mcstate{curid}{lsp'}{l'})}}
\end{mathpar}
\caption{Oracle Machine Model Syntax and Semantics.}
\label{fig:chapter:conlink:multicore-machine-with-hardware-oracle}
\end{figure}


As discussed in Section~\ref{chapter:ccal:subsec:concurrent-layer-with-environment}, 
one hidden component in multicore machine model is a logical participant, hardware scheduler. 
The hardware scheduler builds a specific scheduling strategy for the single game that the verification wants to consider at each time.  
To define this model, 
we extend the private state definition by adding a boolean value, which tells us whether the CPU needs to perform its local hardware step or call hardware scheduler to give out the control. 
First, the \textsf{Progress} rule is for local evaluation of each CPU. It uses the same evaluation rule (hardware local step) with the previous multicore machine model, 
but change the boolean value from false to true to indicated that the CPU served its evaluation at this turn. 
Second, the \text{Yield} rule is for the hardware scheduler.
Similarly, it adds two hardware scheduler events into the log, but the difference in here with the previous multicore machine is it queries the environmental context with the full CPU Set and the current global log. 
Then, the oracle will return a single event, which tells the next event that will perform its evaluation. 

We also prove the simulation of each machine via using backward simulation (upward-forward simulation -- if the target language
has a single step evaluation, then the source language has corresponding multiple steps.) 
in $\compcert$. 
Directly proving backward simulation, however, requires us to prove the progress property of the source language, which is bothersome. 
In this sense, $\compcert$ provide the way to show the equivalence of forward simulation and the backward simulation when 
two languages are deterministic.
We could not take advantage of this conversion in here due to the non-deterministic behavior of our bottom level machine,
multicore machine model. 
Therefore, we directly need to prove the backward refinement proof in here to inject the proof in the backward simulation form of $\compcert$ later. 


\begin{figure}
%\noindent\fbox{variables for match relation}
%
%$$
%\begin{array}{lllr}
%start\_cpu & : & \ztype & \mbox{(starting CPU of the system)}\\
%hw\_o & : & \mcoracle{\coreset}{\mclog}{\mcevent} &  \mbox{(hardware scheduler oracle)}\\
%\end{array}
%$$

\noindent\fbox{local match state: $\ztype \rightarrow \ztype \rightarrow \privatestate \rightarrow \localstatekwd \rightarrow \mcprop$}

\begin{mathpar}
\inferrule[match local neq]
{(curid = i \rightarrow b = \btrue) \\
(curid \neq i \rightarrow b = \bfalse) }
{\matchlocal{curid}{i}{ps}{(\localstate{ps}{b})}}
\end{mathpar}

\noindent\fbox{match state : $\mcstatekwd \rightarrow \hstatekwd \rightarrow \mcprop$}

\begin{mathpar}
\inferrule[match state]
{(\forall \ i \ ps \ ls \ . \ i \in \coreset \rightarrow \getpstsome{\privatestate}{i}{psp}{ps} \rightarrow\\\\
\getpstsome{\localstatekwd}{i}{lsp}{ls} \rightarrow  \matchlocal{curid}{i}{ps}{ls})\\\\
 l \neq \listconsoppo{\yieldev{\_}}{\_}\\
 curid \in \coreset \\
 \validlog{start\_cpu}{l} \\
 \getcuridfromlog{start\_cpu}{l}{curid}}
 {\matchstatehstate{(\mcstate{curid}{lsp}{l})}{(\hstate{curid}{psp}{l})}}
\end{mathpar}

\caption{State Match Relation: the relation between the state of the oracle machine model and the state of the multicore machine model.}
\label{fig:chapter:conlink:match-relation-multicore-oracle-steps}
\end{figure}

The most challenging part in connecting those two different machine models is defining the refinement relation between two states of those languages. 
Figure~\ref{fig:chapter:conlink:match-relation-multicore-oracle-steps} shows the 
refinement relation for those two states with relying on one variable, $start\_cpu$, that indicates the CPU that first starts the system. 
The local state of the oracle machine differs from that of the multicore machine model by having a boolean indicator 
to distinguish the evaluation that the machine has to take next time.
The local state match definition tells that except the current running CPU, all the indicator has to have a false value,
which implies that they can evaluate their own local step after they get an evaluation control via a scheduling result.
 With this local step, 
defining the match relation between two full states are available.
 It tells that 
 two states are valid matched states when all of their local states (for all CPUs) are in a proper relation as well as the global log and the current running CPU ID are well-formed. 


%
%\begin{hypothesis}[valid$\_$cpuid]
%\begin{tabular}{P{0.95\textwidth}}
%$cpuid \in \coreset$
%\end{tabular}
%\end{hypothesis}
%
%\begin{hypothesis}[valid$\_$oracle$\_$def]
%\begin{tabular}{P{0.95\textwidth}}
%$\validoraclenoeq{cpuid}{\coreset}{hw\_o}$
%\end{tabular}
%\end{hypothesis}

With the definition, providing the proof for the refinement between two machines is possible,
which is stated in Lemma~\ref{lemma:chapter:conlink:mc-hw-refines-oracle}.
\begin{lemma}[one$\_$step$\_$hw$\_$refines$\_$oracle]
\label{lemma:chapter:conlink:mc-hw-refines-oracle}
Let's assume that there is a valid start CPU ID for the multicore machine model, 
$start\_cpu$ ($start\_cpu \in \coreset$), and a hardware oracle $hw\_o$,  
which is an environmental context for the oracle machine model that contains the full focused set ($\coreset$) in it.
When the hardware oracle is valid ($ \validoraclenoeq{start\_cpu}{\coreset}{hw\_o}$) we can show the following fact:
\begin{center}
\begin{tabular}{P{0.95\textwidth}}
$
\begin{array}{l}
\forall \ s_l \ s_l' \ t \ s_h . \ \hardwarestep{start\_cpu}{s_l}{t}{s_l'} \rightarrow  \matchstatehstate{s_h}{s_l} \rightarrow \\
\ \ \ \ \ \ \ \ \ \ \ \ \ \ \ \ \exists\ s_h' . \  \ccplusstep{\oraclestep{hw\_o}{s_h}{t}{s_h'}} \wedge  \matchstatehstate{s_h'}{s_l'}\\
\end{array}
$
\end{tabular}
\end{center}
\end{lemma}

The purpose of our framework is providing all the necessary proofs that are required 
for users to use the backward simulation theorems in $\compcert$ to link the all steps together. 
When using the backward refinement proof for that purpose, however, we has to provide the progress property of the 
source language, which is an oracle machine in this case. 
The progress property is stated in Lemma~\ref{lemma:chapter:conlink:mc-hw-refines-oracle-progress}.

\begin{lemma}[match$\_$state$\_$implies$\_$one$\_$step]
\label{lemma:chapter:conlink:mc-hw-refines-oracle-progress}

Let's assume that there is a valid start CPU ID for the multicore machine model, 
$start\_cpu$ ($start\_cpu \in \coreset$), and a hardware oracle $hw\_o$,  
which is an environmental context for the oracle machine model that contains the full focused set ($\coreset$) in it.
When the hardware oracle is valid ($ \validoraclenoeq{start\_cpu}{\coreset}{hw\_o}$) we can show the following fact:
\begin{center}
\begin{tabular}{P{0.95\textwidth}}
$
\begin{array}{l}
\forall \ curid \ lsp \ l \ curid' \ hsp \ l' . \ \matchstatehstate{(\mcstate{curid}{lsp}{l})}{(\hstate{curid'}{hsp}{l'})} \rightarrow \\
\ \ \ \ \ \ \ \ \ \ \ \ \ \ \ \  \exists \ curid'' \ lsp' \ l'  . \ \\
\ \ \ \ \ \ \ \ \ \ \ \ \ \ \ \  \ \ \ \ccstarstep{\oraclestep{hw\_o}{(\mcstate{curid}{lsp}{l})}{\ccemptytrace}{(\mcstate{curid''}{lsp}{l'})}} \wedge\\
\ \ \ \ \ \ \ \ \ \ \ \ \ \ \ \  \ \ \ l' = \listconsoppo{\yieldbackev{curid''}}{\listconsoppo{\yieldev{curid}}{l}} \wedge\\
\ \ \ \ \ \ \ \ \ \ \ \ \ \ \ \  \ \ \ (\forall \ ps  . \ \getpstsome{\localstatekwd}{curid}{lsp}{(\localstate{ps}{\btrue})} \rightarrow \\
\ \ \ \ \ \ \ \ \ \ \ \ \ \ \ \  \ \ \ \ \ \ \ \ \ \  \ \ \setpst{\localstatekwd}{curid}{(\localstate{ps}{\bfalse})}{lsp}{lsp'} \\
\end{array}
$
\end{tabular}
\end{center}
\end{lemma}
This progress property requires the validity of the result event generated by the hardware oracle, which implies
that the returned value is exactly matched with the single instance of the evaluation performed 
by the multicore machine model, which is not available for us to prove. 
In this sense, we  provide Hypothesis~\ref{hypo:chapter:conlink:mc-hw-refines-oracle-hypo}.

\begin{hypothesis}[related$\_$hw$\_$step$\_$hw$\_$oracle]
\label{hypo:chapter:conlink:mc-hw-refines-oracle-hypo}
Let's assume that there is a valid start CPU ID for the multicore machine model, 
$start\_cpu$ ($start\_cpu \in \coreset$), and a hardware oracle $hw\_o$,  
which is an environmental context for the oracle machine model that contains the full focused set ($\coreset$) in it.
When the hardware oracle is valid ($ \validoraclenoeq{start\_cpu}{\coreset}{hw\_o}$) we assume the following fact:
\begin{center}
\begin{tabular}{P{0.95\textwidth}}
$
\begin{array}{l}
\matchstatehstate{(\mcstate{curid}{lsp}{l})}{(\hstate{curid}{psp}{l})} \rightarrow \\
\exists\ ev .\ \oget{\coreset}{\mclog}{\mcevent}{\listconsoppo{\yieldev{curid}}{l}}{hw\_o}{ev} \wedge \\
\ \ (\forall \ curid'' . \ ev = \yieldbackev{curid''} \rightarrow\\
 \ \ \  \forall \ curid' \ ps \ st .\  \hardwarelocalstep{curid'}{(\localview{ps}{(\listconsoppo{\yieldbackev{curid'}}{\listconsoppo{\yieldev{curid}}{l}})})}{st} \rightarrow \\
 \ \  \ curid' = curid'') \\
\end{array}
$
\end{tabular}
\end{center}
\end{hypothesis}

Providing the evidence about building an oracle with satisfying the hypothesis is possible, 
but connecting the proof with the $\compcert$ style proof is a future work of our framework. 


%\noindent\fbox{hypothesis} 
%
%\begin{enumerate}
%\item valid$\_$cpuid: $cpuid \in \coreset$
%\item valid$\_$oracle$\_$def: $\validoraclenoeq{cpuid}{\coreset}{hw\_o}$
%\item related$\_$hw$\_$step$\_$hw$\_$oracle : \\
%$
%\begin{array}{l}
%\matchstatehstate{(\mcstate{curid}{lsp}{l})}{(\hstate{curid}{psp}{l})} \rightarrow \\
%\ \ \ \exists\ ev .\ \oget{\coreset}{\mclog}{\mcevent}{\listconsoppo{\yieldev{curid}}{l}}{hw\_o}{ev} \wedge \\
%\ \ \ \ \ \ \forall \ curid'' . \ ev = \yieldbackev{curid''} \rightarrow\\
%\ \ \ \ \ \ \ \ \  \forall \ curid' \ ps \ st . \\
%\ \ \ \ \ \ \ \ \ \ \ \ \hardwarelocalstep{curid'}{(\localview{ps}{(\listconsoppo{\yieldbackev{curid'}}{\listconsoppo{\yieldev{curid}}{l}})})}{st} \rightarrow\\
%\ \ \ \ \ \ \ \ \ \ \ \ \ \ \ curid' = curid'' \\
%\end{array}
%$
%\end{enumerate}
%
%\noindent\fbox{top lemmas}
%
%\begin{enumerate}
%\item one$\_$step$\_$hw$\_$refines$\_$oracle : \\
%$
%\begin{array}{l}
%\forall \ s_l \ s_l' \ t \ s_h . \ \hardwarestep{curid}{s_l}{t}{s_l'} \rightarrow  \matchstatehstate{s_h}{s_l} \rightarrow \\
%\ \ \ \ \ \ \ \ \ \ \ \ \ \ \ \ \exists\ s_h' . \  \ccplusstep{\oraclestep{hw\_o}{s_h}{t}{s_h'}} \wedge  \matchstatehstate{s_h'}{s_l'}\\
%\end{array}
%$
%\item match$\_$state$\_$implies$\_$one$\_$step: \\
%$
%\begin{array}{l}
%\forall \ curid \ lsp \ l \ curid' \ hsp \ l' . \ \matchstatehstate{(\mcstate{curid}{lsp}{l})}{(\hstate{curid'}{hsp}{l'})} \rightarrow \\
%\ \ \ \ \ \ \ \ \ \ \ \ \ \ \ \  \exists \ curid'' \ lsp' \ l'  . \ \\
%\ \ \ \ \ \ \ \ \ \ \ \ \ \ \ \  \ \ \ \ccstarstep{\oraclestep{hw\_o}{(\mcstate{curid}{lsp}{l})}{\ccemptytrace}{(\mcstate{curid''}{lsp}{l'})}} \wedge\\
%\ \ \ \ \ \ \ \ \ \ \ \ \ \ \ \  \ \ \ l' = \listconsoppo{\yieldbackev{curid''}}{\listconsoppo{\yieldev{curid}}{l}} \wedge\\
%\ \ \ \ \ \ \ \ \ \ \ \ \ \ \ \  \ \ \ (\forall \ ps  . \ \getpstsome{\localstatekwd}{curid}{lsp}{(\localstate{ps}{\btrue})} \rightarrow \\
%\ \ \ \ \ \ \ \ \ \ \ \ \ \ \ \  \ \ \ \ \ \ \ \ \ \  \ \ \setpst{\localstatekwd}{curid}{(\localstate{ps}{\bfalse})}{lsp}{lsp'} \\
%\end{array}
%$
%\end{enumerate}
%
%
%
%Based on the relation we prove two key lemmas; 
%we prove that one step in the multicore machine model always has the corresponding multiple steps of multicore machine with hardware scheduler oracle (in lemma 1). 
%When proving the backward simulation in $\compcert$, providing the evidence about the progress property of the oracle machine, which is stated in the lemma 2. 
%Proving that lemmas require that there will always be a matched environmental context that is matched with the same scheduling sequence of the possible execution of the program in multicore machine model (among all non-deterministic interleaved cases of execution). 
%This is stated in hypothesis 3. 
%
%
