

\subsection{Concurrent (Environmental) Machine Model}
\label{chapter:linking:subsec:concurrent-machine-model}

The concurrent (environmental) machine model is a language that can run programs with a partially focused set in it,
and it is associated with the concurrent layer interface in Section~\ref{chapter:ccal:subsec:concurrent-layer-interface}. 
This machine model does not provide the interface for us to build layers on top of it (\eg, $\compcertx$ and its machine model), but it provides the key connection between our oracle machine model (and thus our multicore machine model) and a single CPU machine model
by treating parts of the system as an environment of the currently focused system. 

\begin{figure}

\noindent\fbox{environmental state: }

$$
\begin{array}{lll}
\envstatekwd_{[A : \set{\ztype}]}& := & \envstate{A}{(curid: \ztype)}{(lsp: \localstatepool{A})}{(log: \mclog)}\\
\end{array}
$$

\noindent\fbox{back ID function:}
$$
\backidkwd (curid : \ztype) (e : \mcevent):= \left\{\begin{array}{lr}
curid' & \text{for } e = \yieldbackev{curid'} \\
\schedid &   \text{for } e =\yieldev{\_} \\
curid & \text{Otherwise}
        \end{array} \right.
 $$

\noindent\fbox{env step: ${}_{[A : \set{\ztype}]} (oracle:\mcoracle{A}{\mclog}{\mcevent}) :  \envstatekwd_{[A : \set{\ztype}]} \rightarrow \cctracekwd \rightarrow \envstatekwd_{[A : \set{\ztype}]} \rightarrow \mcprop$}

\begin{mathpar}
\inferrule[progress]
{ \getpstsome{\localstatekwd}{curid}{lsp}{(\localstate{ps}{\bfalse})}\\
\hardwarelocalstep{curid'}{(\localview{ps}{l})}{(\localview{ps'}{l'})}\\
\setpst{\localstatekwd}{curid'}{\localstate{ps'}{\btrue}}{lsp}{lsp'}}
{\envstep{A}{oracle}{(\envstate{A}{curid}{lsp}{l})}{\ccemptytrace}{(\envstate{A}{curid}{lsp'}{(\listappoppo{l'}{l})})}}

\inferrule[yield]
{  l_0 = \listconsoppo{\yieldbackev{curid}}{l} \\
l' = \listconsoppo{\yieldev{curid'}}{l_0} \\
\getpstsome{\localstatekwd}{curid'}{lsp}{(\localstate{ps}{\btrue})}\\
\setpst{\localstatekwd}{curid'}{\localstate{ps}{\bfalse}}{lsp}{lsp'}\\
\oget{A}{\mclog}{\mcevent}{l_0}{oracle}{\yieldbackev{curid}}}
{\envstep{A}{oracle}{(\envstate{A}{curid}{lsp}{l})}{\ccemptytrace}{(\envstate{A}{curid}{lsp'}{l'})}}

\inferrule[skip]
{  l' = \listconsoppo{e}{l} \\
\getpstnone{\localstatekwd}{curid'}{lsp}\\
\oget{A}{\mclog}{\mcevent}{l}{oracle}{e} \\
\backid{curid}{e}{curid'}}
{\envstep{A}{oracle}{(\envstate{A}{curid}{lsp}{l})}{\ccemptytrace}{(\envstate{A}{curid'}{lsp}{l'})}}
\end{mathpar}
\caption{Environmental Machine Model Syntax and Semantics.}
\label{fig:chapter:conlink:multicore-env-step-machine}
\end{figure}

Figure~\ref{fig:chapter:conlink:multicore-env-step-machine} contains the state definition, 
auxiliary functions, and step relations of the partial machine. 
The state definition is almost the same as the state definition in Figure~\ref{fig:chapter:conlink:multicore-machine-with-hardware-oracle},
which is the state for all cores.
However, it has a different state definition with our oracle machine in the sense the local state pool can be parameterized by any sets of CPUs, the focused set for the machine.
Other CPUs outside of the focused set ($D - A$) and the hardware scheduler are treated as the environment of the machine. 
To handle the steps by the environment of the machine, the step relation has the \textsf{SKIP} rule, 
which queries the oracle to update the global log when the currently running CPU is not in the set $A$.
This rule is  a key to connecting the full machine semantics with the local machine semantics. 

\subsubsection{Concurrent Machine Model with the Full CPU Set}

%thus we treat this proof is a linking proof (for parallel composition)
Because the set for this concurrent machine model can be a set that is precisely the same as the full CPU set, 
we first show that the program runs with the oracle machine refines the concurrent machine model with full CPU set. 
The refinement corresponds to the parallel composition rule for all CPUs. It provides that the transition on the full machine semantics can refine the transition on the collection of multiple instances in the concurrent program (Lemma~\ref{lemma:chapter:conlink:mc-oracle-env}) based on the 
match relation between two states defined in Figure~\ref{fig:chapter:conlink:parallel-composition-for-multicore} (with the 
environmental variable $start\_cpu$, which is a start CPU ID for the system's evaluation).
The relation is basically saying that all values in both states are identical when keeping the valid evaluation flag as well as the valid log condition. 
With the definition, the proof of this lemma is straightforward, because  evaluations of both machines are identical. 

\begin{figure}

%\noindent\fbox{variables for match relation}
%
%$$
%\begin{array}{lllr}
%cpuid & : & \ztype & \mbox{(starting CPU of the system)}\\
%hw\_o & : & \mcoracle{\coreset}{\mclog}{\mcevent} &  \mbox{(hardware scheduler oracle)}\\
%\end{array}
%$$
%
%\noindent\fbox{match state : $\envstatekwd_{[\coreset]} \rightarrow \mcstatekwd \rightarrow \mcprop$}

\begin{mathpar}
\inferrule[match state]
{curid \in \coreset \\ 
\validlog{start\_cpu}{l} \\
l \neq \listconsoppo{\yieldev{\_}}{\_}\\
\getcuridfromlog{start\_cpu}{l}{curid}\\
(\forall \ ps \ b \ tid  . \ tid \neq curid \rightarrow \getpstsome{\localstatekwd}{tid}{lsp}{(\localstate{ps}{b})} \rightarrow b = \bfalse)\\
(\forall \ ps \ b  . \ \getpstsome{\localstatekwd}{curid}{lsp}{(\localstate{ps}{b})} \rightarrow b = \bfalse \rightarrow \\\\
\exists \ e \ l' . \ l = \listconsoppo{e}{l'} \wedge e = \yieldbackev{curid})} 
{\matchestatestate{(\envstate{\coreset}{curid}{lsp}{l})}{(\mcstate{curid}{lsp}{l})}}
\end{mathpar}

%
%\noindent\fbox{hypothesis}
%
%\begin{enumerate}
%\item valid$\_$oracle$\_$def: $\validoraclenoeq{cpuid}{\coreset}{hw\_o }$
%\end{enumerate}
%
%\noindent\fbox{top lemmas}
%
%\begin{enumerate}
%\item one$\_$step$\_$env$\_$refines$\_$oracle : \\
%$
%\begin{array}{l}
%\forall \ s_h \ s_h' \ t \ s_l . \ \envstep{\coreset}{hw\_o }{s_h}{t}{s_h'} \rightarrow  \matchestatestate{s_h}{s_l} \rightarrow \\
%\ \ \ \ \ \ \ \ \ \ \ \ \ \ \ \ \exists\ s_l' . \  \ccplusstep{\oraclestep{hw\_o }{s_l}{t}{s_l'}} \wedge  \matchstatehstate{s_h'}{s_l'}\\
%\end{array}
%$
%\end{enumerate}
\caption{State Match Relation: the relation between the state of the environmental  machine model (with a full CPU set) and the state of the oracle machine model.}
\label{fig:chapter:conlink:parallel-composition-for-multicore}
\end{figure}
%
%
%\begin{hypothesis}[valid$\_$oracle$\_$def]
%\begin{tabular}{P{0.95\textwidth}}
%$\validoraclenoeq{cpuid}{\coreset}{hw\_o }$\\
%\end{tabular}
%\end{hypothesis}

\begin{lemma}[one$\_$step$\_$env$\_$refines$\_$oracle]
\label{lemma:chapter:conlink:mc-oracle-env}
Let's assume  there is a valid start CPU ID,
$start\_cpu$ ($start\_cpu \in \coreset$), and a hardware oracle, $hw\_o$,  
which is an environmental context for the oracle machine model that contains the full focused set ($\coreset$).
When the hardware oracle is valid ($\validoraclenoeq{start\_cpu}{\coreset}{hw\_o}$) we can show the following fact:
\begin{center}
\begin{tabular}{P{0.95\textwidth}}
$
\begin{array}{l}
\forall \ s_h \ s_h' \ t \ s_l . \ \envstep{\coreset}{hw\_o }{s_h}{t}{s_h'} \rightarrow  \matchestatestate{s_h}{s_l} \rightarrow \\
\ \ \ \ \ \ \ \ \ \ \ \ \ \ \ \ \exists\ s_l' . \  \ccplusstep{\oraclestep{hw\_o }{s_l}{t}{s_l'}} \wedge  \matchstatehstate{s_h'}{s_l'}\\
\end{array}
$\\
\end{tabular}
\end{center}
\end{lemma}

Figure~\ref{fig:chapter:conlink:parallel-composition-for-multicore} shows the full context of the proof, which is relatively easy after defining the proper refinement relation. 

\subsubsection{Concurrent Machine Model with a Single CPU}


However, when  focusing on a single CPU,
multiple things have to be considered to connect the environmental machine that uses the full CPU set  
with the environmental machine that uses a single-focused set.
It plays a crucial role 
for our multicore linking because it proves that the multiple concrete steps  in the machine with the full CPU set can
 refine the corresponding evaluation of multiple environmental steps in the machine with a single CPU.
The proof is sophisticated due to the mismatch between those two  states (even though they are 
using the same concurrent machine model).
To prove this theorem, 
we first introduce one auxiliary function defined in Figure~\ref{fig:chapter:conlink:auxiliary-function-for-full-cpus-and-a-single-cpu},
which measures the progress of the evaluation by the environment. 

The evaluation on the machine with the full focused set does not evaluate the program with its $\textsf{SKIP}$ rule because all private states are associated with  all CPU IDs in the core set. 
In this sense, the evaluation of the machine with the full-core set always generates two adjacent events regarding its scheduling. 
On the other hand, the evaluation of the machine with a single focused set uses 
$\textsf{SKIP}$ rules when the currently running CPU is not in the domain (is not equal to the focused CPU ID.)
This implies that one-step evaluation in the machine with the full focused set 
can be associated with the two-step oracle queries in the machine with a single focused CPU ID.

To work around this issue  using the forward-to-backward simulation proof technique, 
we introduce $\mcmeasurekwd$ as well as $\mcstatemeasurekwd$ which implies that 
the currently running CPU ID is a hardware scheduler when the given log ends with the $\yieldevkwd$ event. This means  we need to pause the evaluation on the machine with the full focused set to match the evaluation in both machines.

\begin{figure}
\noindent\fbox{auxiliary function - decreasing index:}
\begin{center}
\begin{tabular}{l}
$
   \mcmeasurekwd (l : \mclog)  : \nattype :=
     \left\{\begin{array}{lr}
        O & \text{for }  l = \listconsoppo{\yieldev{\_}}{\_} \\
        S \ O & \text{otherwise}\\
        \end{array} \right.$ \\
$ \mcstatemeasurekwd (A : \set{\ztype}) (est : \envstatekwd_{[A]}) : \nattype := \mcmeasurenoeq{l} \ \ \ \ \ \ \ \ \ \ (\text{for } est = \envstate{A}{\_}{\_}{l}) $\\
$ \localstepevkwd (ev : \mcevent) : \optiondef\ \mclog := 
     \left\{\begin{array}{lr}
        \mathrm{Some} \ (\listconsoppo{ev}{\nulllist}) & \text{for }  ev = \acqev{\_}{\_} \\
                                        & \vee  ev= \relev{\_}{\_}{\_} \\
                                        & \vee ev= \atomicev{\_}{\_}{\_} \\
        \mathrm{Some} \ \nulllist & \text{for }  ev = \yieldev{\_} \\
	   \mathrm{None} & \text{otherwise}\\
        \end{array} \right. $ \\
\end{tabular}        
\end{center}
%  Definition measure (l : Log) : nat := 
%    match lastEvTy l with 
%    | Some YIELDTY => O
%    | _ => S O 
%    end.
%  
%  Definition state_measure (A : ZSet) (est : estate (A := A)) :=
%    match est with 
%    | EState _ _ l => measure l 
%    end.
%
%  Definition local_step_ev (ev : event) : option Log := 
%    match ev with 
%    | EACQ _ _ => Some (ev::nil)
%    | EREL _ _ _ => Some (ev::nil)
%    | EATOMIC _ _ _ => Some (ev::nil)
%    | EYIELD _ => Some nil
%    | _ => None
%    end.
\caption{Auxiliary Functions:  auxiliary functions for the refinement proof between
the enviornmental machine model with a single-focused set and a full CPU set.}
\label{fig:chapter:conlink:auxiliary-function-for-full-cpus-and-a-single-cpu}
\end{figure}


By using those auxiliary functions, we define the refinement relation between two machine states in Figure~\ref{fig:chapter:conlink:refinement-relation-for-full-cpus-and-a-single-cpu}.

\begin{figure}
\noindent\fbox{match state : $\envstatekwd_{[\set{cpuid}]} \rightarrow  \envstatekwd_{[\coreset]}  \rightarrow \mcprop$}

\begin{mathpar}
\inferrule[match state S]
{\mcmeasure{l}{S\ O} \\
curid \in \coreset \\
\validlog{cpuid}{l}\\
\getcuridfromlog{cpuid}{l}{curid}\\
\getpstnoeq{\localstatekwd}{cpuid}{lsp'} = \getpstnoeq{\localstatekwd}{cpuid}{lsp}\\
(curid = cpuid \rightarrow \forall \ ps \ b . \ \getpstsome{\localstatekwd}{cpuid}{lsp}{(\localstate{ps}{b})} \rightarrow\\\\
b = \bfalse \rightarrow \exists \ e \ l' . \ l = \listconsoppo{e}{l'} \wedge e = \yieldbackev{curid})\\\\
(\forall \ tid  . \ tid \neq curid \rightarrow \forall \ ps \ b . \ \getpstsome{\localstatekwd}{tid}{lsp}{(\localstate{ps}{b})} \rightarrow b = \bfalse)\\\\
(curid \neq cpuid \rightarrow l = \listconsoppo{\yieldbackev{\_}}{\_} \rightarrow\\\\
\forall \ ps \ b . \ \getpstsome{\localstatekwd}{curid}{lsp}{(\localstate{ps}{b})} \rightarrow b = \bfalse)\\\\
(curid \neq cpuid \rightarrow l \neq \listconsoppo{\yieldbackev{\_}}{\_} \rightarrow\\\\
\forall \ ps \ b . \ \getpstsome{\localstatekwd}{curid}{lsp}{(\localstate{ps}{b})} \rightarrow b = \btrue)}
{\matchestateestate{\set{cpuid}}{(\envstate{\set{cpuid}}{curid}{lsp'}{l})}{(\envstate{\coreset}{curid}{lsp}{l})}}

\inferrule[match state O]
{\mcmeasure{l'}{O} \\
\getcuridfromlognoeq{cpuid}{l} \in \coreset \\
\validlog{cpuid}{l'}\\
l' = \listconsoppo{\yieldev{curid}}{l} \\ 
curid' = \schedid\\
\getcuridfromlog{cpuid}{l}{curid} \\
\getpstnoeq{\localstatekwd}{cpuid}{lsp'} = \getpstnoeq{\localstatekwd}{cpuid}{lsp}\\
\getpstnone{\localstatekwd}{(\getcuridfromlognoeq{cpuid}{l})}{lsp'}\\
\oget{\set{cpuid}}{\mclog}{\mcevent}{l}{si\_o}{\yieldev{curid}}\\
(\forall \ tid  . \ tid \neq curid \rightarrow \forall \ ps \ b . \ \getpstsome{\localstatekwd}{tid}{lsp}{(\localstate{ps}{b})} \rightarrow b = \bfalse)\\\\
(curid \neq cpuid  \rightarrow \forall \ ps \ b  . \ \getpstsome{\localstatekwd}{curid}{lsp}{(\localstate{ps}{b})} \rightarrow\\\\
b = \bfalse \rightarrow  l = \listconsoppo{\yieldbackev{\_}}{\_})}
{\matchestateestate{\set{cpuid}}{(\envstate{\set{cpuid}}{curid'}{lsp'}{l'})}{(\envstate{\coreset}{curid}{lsp}{l})}}

\end{mathpar}
\caption{State Match Relation: the relation between two states of the environmental  machine model (with a single-focused CPU and a full CPU set).}
\label{fig:chapter:conlink:refinement-relation-for-full-cpus-and-a-single-cpu}
\end{figure}

%\noindent\fbox{variables for match relation:}


%\begin{figure}
%\noindent\fbox{hypothesis}
%
%\begin{enumerate}
%\item valid$\_$cpuid: $cpuid \in \coreset$
%\item valid$\_$oracle$\_$cond: $\validoraclenoeq{cpuid}{\set{cpuid}}{si\_o}$
%\item related$\_$single$\_$oracle$\_$hw$\_$oracle$\_$def : \\
%$
%\begin{array}{l}
%\forall \ l \ ev  . \ \validlog{cpuid}{l} \rightarrow l = \listconsoppo{\yieldev{\_}}{\_} \rightarrow\\
%\ \ \ \ \oget{\set{cpuid}}{\mclog}{\mcevent}{l}{si\_o}{ev} \rightarrow \oget{\coreset}{\mclog}{\mcevent}{l}{hw\_o}{ev} \\
%\end{array}
%$
%\item relate$\_$single$\_$oracle$\_$concrete$\_$step$\_$def: \\ 
%$
%\begin{array}{l}
%\forall \ curid' \ curid \ lsp' \ lsp \ l' \ l \ l\_res \ ps \ ev  . \\
%\ \ \ \matchestateestate{\set{cpuid}}{(\envstate{\set{cpuid}}{curid'}{lsp'}{l'})}{(\envstate{\coreset}{curid}{lsp}{l})} \rightarrow\\
%\ \ \ \getcuridfromlognoeq{cpuid}{l} \neq cpuid \rightarrow \\
%\ \ \ \getpstsome{\localstatekwd}{\getcuridfromlognoeq{cpuid}{l}}{lsp}{(\localstate{ps}{false})} \rightarrow \\
%\ \ \ \oget{\set{cpuid}}{\mclog}{\mcevent}{l}{si\_o}{ev} \rightarrow \\
%\ \ \ \localstepev{ev}{l\_res} \rightarrow l = \listconsoppo{\yieldbackev{\_}}{\_} \rightarrow \\
%\ \ \ \exists\ (ps' : \privatestate) . \ \hardwarelocalstep{(\getcuridfromlognoeq{cpuid}{l})}{(\localview{ps}{l})}{(\localview{ps'}{l\_res})} \\
%\end{array}
%$
%\end{enumerate}
%
%
%\noindent\fbox{top lemmas}
%
%\begin{enumerate}
%\item one$\_$step$\_$env$\_$refines$\_$oracle : \\
%$
%\begin{array}{l}
%\forall \ s_h \ s_h' \ t \ s_l. \ \envstep{\set{cpuid}}{si\_o}{s_h}{t}{s_h'} \rightarrow  \matchestateestate{\set{cpuid}}{s_h}{s_l} \rightarrow \\
%\ \ \ \ \ \ \ \ \ \ \ \ \ \ \ \ (\exists \ s_l' .\ (\ccplusstep{\envstep{\coreset}{hw\_o}{s_l}{t}{s_l'}} \wedge\\
%\ \ \ \ \ \ \ \ \ \ \ \ \ \ \ \ \ \ \ \ \ \ \ \ \  \matchestateestate{\set{cpuid}}{s_h'}{s_l'}) \vee \\ 
%\ \ \ \ \ \ \ \ \ \ \ \ \ \ \ \ (\mcstatemeasurenoeq{\set{cpuid}}{s_h'} < \mcstatemeasurenoeq{\set{cpuid}}{s_h} \wedge t = \ccemptytrace \wedge \\ 
%\ \ \ \ \ \ \ \ \ \ \ \ \ \ \ \ \matchestateestate{\set{cpuid}}{s_h'}{s_l})\\
%\end{array}
%$
%\end{enumerate}
%\caption{Refinement between a single core and a full core set over environmental step machines}
%\label{fig:chapter:conlink:cpu-env-to-env-theorem}
%\end{figure}



%\begin{hypothesis}[valid$\_$cpuid]
%\begin{tabular}{P{0.95\textwidth}}
%$cpuid \in \coreset$
%\end{tabular}
%\end{hypothesis}
%
%\begin{hypothesis}[ valid$\_$oracle$\_$cond]
%\begin{tabular}{P{0.95\textwidth}}
%$\validoraclenoeq{cpuid}{\set{cpuid}}{si\_o}$
%\end{tabular}
%\end{hypothesis}
%


Using the refinement relation, 
we formally state the refinement lemma between two machines in Lemma~\ref{lemma:chapter:conlink:one-env-refines-env},
which basically says that 
multiple evaluations in the environmental machine with a single-focused CPU matches with certain multiple evaluations in the environmental machine with a full-core set.

\begin{lemma}[one$\_$step$\_$env$\_$refines$\_$env]
\label{lemma:chapter:conlink:one-env-refines-env}
Let's assume that there is a valid CPU ID $cpuid$ (\ie, a focused CPU as well as a start CPU of the system
$cpuid \in \coreset$),  a hardware oracle $hw\_o$ (\ie, an environmental context for the environmental machine model that contains the full focused set, $\coreset$), 
and a single-core oracle $si\_o$ (\ie, an environmental context for the environmental machine model that contains a singleton CPU, $cpuid$.)
When the single oracle is valid ($ \validoraclenoeq{cpuid}{\set{cpuid}}{si\_o}$) we can show the following fact:
\begin{center}
\begin{tabular}{P{0.95\textwidth}}
$
\begin{array}{l}
\forall \ s_h \ s_h' \ t \ s_l. \ \envstep{\set{cpuid}}{si\_o}{s_h}{t}{s_h'} \rightarrow  \matchestateestate{\set{cpuid}}{s_h}{s_l} \rightarrow \\
\ \ \ \ \ \ \ \ \ \ \ \ \ \ \ \ (\exists \ s_l' .\ (\ccplusstep{\envstep{\coreset}{hw\_o}{s_l}{t}{s_l'}} \wedge \matchestateestate{\set{cpuid}}{s_h'}{s_l'}) \vee \\ 
\ \ \ \ \ \ \ \ \ \ \ \ \ \ \ \ (\mcstatemeasurenoeq{\set{cpuid}}{s_h'} < \mcstatemeasurenoeq{\set{cpuid}}{s_h} \wedge t = \ccemptytrace \wedge \\ 
\ \ \ \ \ \ \ \ \ \ \ \ \ \ \ \ \matchestateestate{\set{cpuid}}{s_h'}{s_l})\\
\end{array}
$
\end{tabular}
\end{center}
\end{lemma}

Proving this lemma relies on two oracles 
as well as an assumption about the correctness of each CPU's private evaluation;
hardware-scheduling events in two oracles are exactly matched with each other (Hypothesis~\ref{hypothesis:chapter:conlink:related-single-oracle-hw-oracle}), so
the corresponding private evaluation in the machine with the full-core set
will always succeed (will not get stuck) when we have a strategy for an environmental step in the machine with a single CPU
(Hypothesis~\ref{hypothesis:chapter:conlink:related-single-oracle-concrete-step}).

\begin{hypothesis}[related$\_$single$\_$oracle$\_$hw$\_$oracle$\_$def]
\label{hypothesis:chapter:conlink:related-single-oracle-hw-oracle}
Let's assume  there is a valid CPU ID $cpuid$
($cpuid \in \coreset$),  a hardware oracle $hw\_o$,
and a single-core oracle $si\_o$.
When the single oracle is valid ($ \validoraclenoeq{cpuid}{\set{cpuid}}{si\_o}$) we assume the following fact:
\begin{center}
\begin{tabular}{P{0.95\textwidth}}
$
\begin{array}{l}
\forall \ l \ ev  . \ \validlog{cpuid}{l} \rightarrow l = \listconsoppo{\yieldev{\_}}{\_} \rightarrow\\
\ \ \ \ \oget{\set{cpuid}}{\mclog}{\mcevent}{l}{si\_o}{ev} \rightarrow \oget{\coreset}{\mclog}{\mcevent}{l}{hw\_o}{ev} \\
\end{array}
$
\end{tabular}
\end{center}
\end{hypothesis}


\begin{hypothesis}[relate$\_$single$\_$oracle$\_$concrete$\_$step$\_$def]
\label{hypothesis:chapter:conlink:related-single-oracle-concrete-step}
Let's assume  there is a valid CPU ID $cpuid$ ($cpuid \in \coreset$),  a hardware oracle $hw\_o$,
and a single-core oracle $si\_o$.
When the single oracle is valid ($ \validoraclenoeq{cpuid}{\set{cpuid}}{si\_o}$) we assume the following fact:
\begin{center}
\begin{tabular}{P{0.95\textwidth}}
$
\begin{array}{l}
\forall \ curid' \ curid \ lsp' \ lsp \ l' \ l \ l\_res \ ps \ ev . \\
\ \ \matchestateestate{\set{cpuid}}{(\envstate{\set{cpuid}}{curid'}{lsp'}{l'})}{(\envstate{\coreset}{curid}{lsp}{l})} \rightarrow\\
\ \ \getcuridfromlognoeq{cpuid}{l} \neq cpuid \rightarrow \\
\ \ \getpstsome{\localstatekwd}{\getcuridfromlognoeq{cpuid}{l}}{lsp}{(\localstate{ps}{false})} \rightarrow \\
\ \ \oget{\set{cpuid}}{\mclog}{\mcevent}{l}{si\_o}{ev} \rightarrow \localstepev{ev}{l\_res} \rightarrow l = \listconsoppo{\yieldbackev{\_}}{\_} \rightarrow \\
\ \ \exists\ (ps' : \privatestate) . \ \hardwarelocalstep{(\getcuridfromlognoeq{cpuid}{l})}{(\localview{ps}{l})}{(\localview{ps'}{l\_res})} \\
\end{array}
$
\end{tabular}
\end{center}
\end{hypothesis}

Providing the evidence about building an oracle that satisfies the hypothesis is possible, 
but connecting the proof with the $\compcert$ style proof is a future work for our framework. 

%
%$$
%\begin{array}{lllr}
%cpuid & : & \ztype & \mbox{(starting and the focused CPU of the system)}\\
%hw\_o & : & \mcoracle{\coreset}{\mclog}{\mcevent} &  \mbox{(hardware scheduler oracle)}\\
%si\_o & : & \mcoracle{\set{cpuid}}{\mclog}{\mcevent} &  \mbox{(oracle for a single cpu)}\\
%\end{array}
%$$
