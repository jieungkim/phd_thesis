

\subsection{Single core machine model}
\label{chapter:linking:subsec:cpu-local-layer-interface}

The bottom level of multithreaded linking, a cpu local layer has a similar state definition with $\compcert$. 
They follows the definition of $\compcertx$, 
which is an extension of $\compcert$ and is defined as
\begin{lstlisting}[language=C]
Inductive state `{memory_model_ops: Mem.MemoryModelOps mem}: Type :=
  | State: regset -> mem -> state.
\end{lstlisting}
where the memory, $mem$, contains a memory itself (a partial map from a identifier to the value), an abstract data (which is a collection of abstract data representations), as well as
a global log.
In other words, 
$mem$ in the state is precisely a tuple consists of those three elements.
With those definitions, the evaluation transition relation has a type
\begin{lstlisting}[language=C]
  Inductive step (ge: genv): state -> trace -> state -> Prop :=
\end{lstlisting}
which change the state by the transition with recording $\compcert$ style trace. 

The step relation contains five rules, 
but we explain only three rules in this section, which are the rules 
that are sufficient for us to explain our thread-linking framework.

\begin{lstlisting}[language=C]
    | exec_step_internal:
        forall b ofs f i rs m rs' m',
        rs PC = Vptr b ofs ->
        Genv.find_funct_ptr ge b = Some (Internal f) ->
        find_instr (Int.unsigned ofs) f.(fn_code) = Some i ->
        exec_instr ge f i rs m = Next rs' m' ->
        step ge (State rs m) E0 (State rs' m')
\end{lstlisting}
The first rule is for internal instruction revaluation rules, 
that defines the rules for bunch of possible instructions including 
memory load and stores. 
\begin{lstlisting}[language=C]
    | exec_step_external:
        forall b ef args res rs m t rs' m',
        rs PC = Vptr b Int.zero ->
        Genv.find_funct_ptr ge b = Some (External ef) ->
        extcall_arguments rs m (ef_sig ef) args ->
        external_call' (fun _ => True) ef ge args m t res m' ->
      rs' = (set_regs (loc_external_result (ef_sig ef)) res (undef_regs (CR ZF :: CR CF :: CR PF :: CR SF :: CR OF :: nil) (undef_regs (map preg_of destroyed_at_call) rs))) #PC <- (rs RA) #RA <- Vundef ->
        forall STACK:
        forall b o, rs ESP = Vptr b o ->
                    (Ple (Genv.genv_next ge) b /\ Plt b (Mem.nextblock m)),
        forall SP_NOT_VUNDEF: rs ESP <> Vundef,
        forall RA_NOT_VUNDEF: rs RA <> Vundef,
        step ge (State rs m) t (State rs' m')
\end{lstlisting}
the next one is for external call transition rules, which is
the evaluation rules for external function calls.
\begin{lstlisting}[language=C]
    | exec_step_prim_call:
        forall b ef rs m t rs' m',
          rs PC = Vptr b Int.zero ->
          Genv.find_funct_ptr ge b = Some (External ef) ->
          primitive_call ef ge rs m t rs' m' ->
          step ge (State rs m) t (State rs' m').
\end{lstlisting}
The third is a primitive call that is also similar to the external calls,
but all information is hidden in the specifications. 
\jieung{1. Do we actually need to explain it? I think we do not  need to do}
\jieung{If we explain it, we need to explain primitive calls more} 
\begin{lstlisting}
  Inductive initial_state {F V} (p: AST.program F V): state -> Prop :=
    | initial_state_intro m0:
        Genv.init_mem p = Some m0 ->
        let rs0 :=
          (Pregmap.init Vundef)
          # PC <- (symbol_offset (Genv.globalenv p) (prog_main p) Int.zero)
          # ESP <- Vzero in
        initial_state p (State rs0 m0).
\end{lstlisting}
The initial state of this semantics is define 
set the initial stack pointer as the initial value, which is zero,
and set the program counter as a function pointer for the main function in the program. 

As we have discussed,
the first step is introducing scheduling primitives in our CPU local layers..
In terms of artifacts, 
The language model that we have used in CPU-local layers is $\compcertx$\ propposed by \cite{deepspec}, and 
The machine state of it is
\begin{center}
$st_{\AsmLM} = (\regs, (m, adt))$
\end{center}
where $\regs$ is a register set, $m$ is a memory, and $adt$ is an abstract data.
To use this existing tool as much as possible, we encapsulate the idea of 
our concurrent machine mode in this machine state, 
mostly in the abstract data ($adt$) among the components.
For instance, when looking at the state of the first example (1) in Fig.~\ref{fig:chapter:conlink:threadlinking}, the state is defined as $st = (\rho, m, a, l)$, but both $a$ and $l$ are actually two components of $adt$ in our implementation level.
By successfully integrating our framework with $\compcertx$, we do not need to modify the language semantics and machine model
to build CPU local layers.
For example, the external call rule in $\compcertx$\ $\AsmLM$ machine, 
which is also used for $\yield$ and $\sleep$ functions,  
is defined as follows:
\begin{center}
$c, \oracle \vdash_{\AsmLM} \sstepr{\spec_{_{id}}}{args}{\regs, m, adt}{\textit{res}\cup \{\}}{\regs',  m', adt'}$
\end{center}
