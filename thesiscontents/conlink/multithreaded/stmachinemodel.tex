\subsection{Single-threaded Machine Model}
\label{chapter:linking:subsec:single-threaded-machine-model}

Like we show in Figure~\ref{fig:chapter:conlink:introduce-single-threaded-machine-model-with-tasm}, 
another optimization is available and necessary 
to connect the program runs on the $\ieasmmach$ with  a single focused set to
the program runs with the per-thread machine model, $\hasmmach$, which shares the 
same state definition and transition rules with $\lasmmach$. 
In this manner,
we introduce $\tasmmach$, a machine model
which simplifies the machine state definition as  well as
merges multiple operational style environmental context queries into a single big-step style 
environmental context query.  
This part has a similarity with the corresponding part of our multicore-linking framework, 
that introduces $\singlebigstepkwd$ and $\singlebigtwostepkwd$ in Section~\ref{chapter:linking:subsec:global-log-optimization}.



%
%
%before connecting multithreaded machine model,  $\ieasmmach$, with a single thread 
%and our per-thread layer interface, $\hasmmach$, to minimize the complexity of our multithreaded linking. 
%They are similar to the work what we did in  Section~\ref{chapter:linking:subsec:single-core-machine-model} and
%Section~\ref{chapter:linking:subsec:global-log-optimization},
%%
%
%When the focused thread set in $\ieasmmach$ is a single thread, then the machien itself is 
%already a single threaded machine model. 
%It, however, has a huge differences between the model that we want to introduce ($\compcertx$ like model).


\begin{figure}
\begin{lstlisting}[language=C]
Fixpoint get_env_log (n: nat) (curid: Z) (l: Log) :=
  match n with
  | O => None
  | S n => if zeq (proc_id (uRData l)) curid then Some l
                else get_env_log n curid (Single_Oracle l:: l)
  end.
\end{lstlisting}
\begin{center}
Auxiliary Function for Big-step Oracle Query in $\tasmmach$.
\end{center}

\begin{lstlisting}[language=C]
| texec_step_external_yield:
  forall b ef (rs0 rs<@$'$@>: regset) (m m<@$'$@>: mem) curid s l l<@$'$@> l<@$''$@> l<@$'''$@> nb d e,
   // check the current thread ID
    proc_id (uRData l) = curid ->
    // check the program counter
    rs0 PC = Vptr b Int.zero ->
    Genv.find_funct_ptr ge b = Some (External ef) ->
    <@$\cdots$@>
    // update the log using [get_env_log]. limit is for a fairness assumption
    l<@$'$@> = LEvent curid (LogYield (Mem.nextblock m))::l ->
    get_env_log limit curid l<@$'$@> = Some l<@$''$@> ->
    l<@$'''$@> = LEvent curid LogYieldBack::l<@$''$@> ->
    rs<@$'$@> = <@$\cdots$@> // flush the values in the registers and set the new PC value
    // check the primitive ID (it should be [thread_yield])
    match ef with 
      | EF_external id _ => if peq id thread_yield then True else False
      | _ => False end ->
	  <@$\cdots$@>
      tstep ge (TState curid rs0 m d l) E0 (TState curid rs<@$'$@> m<@$'$@> d l<@$'''$@>)
\end{lstlisting}
\begin{center}
Evaluation Rule for Yield in  $\tasmmach$.
\end{center}
\caption{Evaluation Rule (Yield) in $\tasmmach$.}
\label{fig:chapter:conlink:yield-rule-in-tasm}
\end{figure}


The simplified state definition for our intermediate machine model ($\tasmmach$) is
\begin{lstlisting}[language=C]
Inductive tstate: Type := | TState: Z -> regset -> mem -> dproc -> Log -> tstate.
\end{lstlisting}
which is a tuple of five elements. 
Those elements represent a current thread ID, 
a register value of the thread, a local memory, a private abstract state, and a global log for shared objects. 
The state definition is identical to the each thread's view of the global state in $\ieasmmach$ an $\easmmach$ in the previous section. 
This way,
this machine can also use the same abstract layer definition, $\TLink$, 
as well as similar but simplified transition rules as those of the two multithreaded machine models.
However, the machine
 collects the environmental context queries to form a big-step style transition for scheduling primitives. 
 For example, 
 the transition rule for $\yield$ is defined in Figure~\ref{fig:chapter:conlink:yield-rule-in-tasm} (b),
 which always returns the same thread ID as the result of the rule 
 by using 
 an auxiliary function, \lstinline$get_env_log$, in  Figure~\ref{fig:chapter:conlink:yield-rule-in-tasm} (a).
 The function   merges multiple environmental context queries into a single step with a fairness assumption
 of the threads' evaluation, which is stated as \lstinline$limit$ in line 11 of  Figure~\ref{fig:chapter:conlink:yield-rule-in-tasm} (b).

We also provide 
the theorem to connect the program evaluation on $\tasmmach$
with the program evaluation on $\ieasmmach$,
which users can freely use without any additional cost.

\begin{lemma}[Refinement Between $\ieasm$ and $\tasm$]
\label{lemma:chapter:conlink:ieasm-refines-tasm}
Suppose  $\ThreadConf$ is a multithreaded-linking environment that contains thread configurations, auxiliary functions, and properties for them. 
Additionally, let's assume  
 $tid$ is a valid thread ID in a certain CPU, 
 $\oracle^{\TLink}_{tid}$ is a valid
environmental context for a thread $tid$, 
and $\codeinmath{Ctxt}$ is a
 context program that runs on top of layer $\TLink$.
 Then we can state that
 \begin{center}
\begin{tabular}{c}
$\semwmachine{\ThreadConf, \PLayer{\TLink}{tid}{\oracle^{\TLink}_{cid}}}{\codeinmath{Ctxt}}{\ieasmmach} \refines_{R_{\mathcal{M}{[\tasm, \ieasm]}}} \semwmachine{\ThreadConf, \PLayer{\TLink}{tid}{\oracle^{\TLink}_{tid}}}{\codeinmath{Ctxt}}{\tasmmach}$\\
\end{tabular}
\end{center}
when $R_{\mathcal{M}{[\tasm, \ieasm]}}$ is a refinement relation between two machine models with the  $\TLink$ layer.
\end{lemma}

Similar to Lemma~\ref{lemma:chapter:conlink:easm-refines-iieasm} and Lemma~\ref{lemma:chapter:conlink:easm-refines-single-ieasm},
Lemma~\ref{lemma:chapter:conlink:ieasm-refines-tasm} depends on a thread configuration ($\ThreadConf$) and an abstract layer definition ($\TLink$).
However, it does not depend on other abstract properties such as $\AbsRelC$; thus,
using it does not require us to introduce any additional concrete definitions. 



%
%\begin{lstlisting}[language=C]
%  Inductive tstep (ge: genv) : tstate -> trace -> tstate -> Prop :=
%\end{lstlisting}

%\begin{lstlisting}[language=C]
%    | texec_step_internal:
%        forall b ofs f i (rs: regset) m m' d d' ds' rs' curid l,
%          proc_id (uRData l) = curid ->
%          rs PC = Vptr b ofs ->
%          Genv.find_funct_ptr ge b = Some (Internal f) ->
%          find_instr (Int.unsigned ofs) f.(fn_code) = Some i ->
%          lastEvType l <> Some LogYieldTy ->
%          exec_instr ge f i rs (m, (uRData l,d)) = Next rs' (m', (ds',d')) ->
%          tstep ge (TState curid rs m d l) E0
%                  (TState curid rs' m' d' l)
%\end{lstlisting}
%
%\begin{lstlisting}[language=C]
%  | texec_step_external:
%      forall b ef args res (rs: regset) m m' d d' ds' t rs' curid  l l',
%        proc_id (uRData l) = curid ->
%        rs PC = Vptr b Int.zero ->
%        Genv.find_funct_ptr ge b = Some (External ef) ->
%        extcall_arguments rs m (ef_sig ef) args ->
%        lastEvType l <> Some LogYieldTy ->
%        external_call' (mem:= mwd (cdata PData)) (fun _ => True) ef ge args (m, (uRData l, d)) t res (m', (ds',d')) ->
%        rs' = (set_regs (loc_external_result (ef_sig ef)) res 
%                        (undef_regs (CR ZF :: CR CF :: CR PF :: CR SF :: CR OF :: nil)
%                                    (undef_regs (map preg_of destroyed_at_call) rs))) 
%                #PC <- (rs RA) #RA <- Vundef ->
%        forall NON_YIELD: match ef with
%                            | EF_external id _ => 
%                              if peq id thread_yield then False
%                              else if peq id thread_sleep then False
%                                   else
%                                     if has_event id then
%                                       exists largs choice,
%                                         val2Lval_list args largs  /\
%                                         choice_check id largs (uRData l) d = choice /\ 
%                                         l' = (LEvent curid (LogPrim id largs choice (snap_func d)) :: l) 
%                                     else
%                                       l' = l
%                            | _ => l' = l
%                          end,
%                          forall STACK:
%                          forall b o, rs ESP = Vptr b o ->
%                                      (Ple (Genv.genv_next ge) b /\ Plt b (Mem.nextblock m)),
%                          forall SP_NOT_VUNDEF: rs ESP <> Vundef,
%                          forall RA_NOT_VUNDEF: rs RA <> Vundef,
%                            tstep ge (TState curid rs m d l) t 
%                                  (TState curid rs' m' d' l')
%\end{lstlisting}

%\begin{lstlisting}[language=C]
%    | texec_step_prim_call:
%        forall b ef (rs: regset) m m' d d' ds' t rs' curid l l',
%          proc_id (uRData l) = curid ->
%          rs PC = Vptr b Int.zero ->
%          Genv.find_funct_ptr ge b = Some (External ef) ->
%          lastEvType l <> Some LogYieldTy ->
%          primitive_call (mem:= mwd (cdata PData)) ef ge rs (m, (uRData l, d)) t rs' (m', (ds', d')) ->
%          forall NON_YIELD: match ef with
%                              | EF_external id _ => 
%                                if peq id thread_yield then False
%                                else if peq id thread_sleep then False
%                                     else
%                                       if has_event id then
%                                         l' = (LEvent curid (LogPrim id nil 0 (snap_func d)) :: l)
%                                       else
%                                         l' = l
%                              | _ => l' = l
%                            end,
%          tstep ge (TState curid rs m d l) t 
%                (TState curid rs' m' d' l')
%\end{lstlisting}
%

%\begin{lstlisting}[language=C]
%    | texec_step_external_sleep:
%        forall b ef (rs0 rs': regset) (m m': mem) curid s l l' l'' l''' nb d e i,
%          proc_id (uRData l) = curid ->
%          rs0 PC = Vptr b Int.zero ->
%          Genv.find_funct_ptr ge b = Some (External ef) ->
%          stencil_matches s ge ->
%          l' = LEvent curid (LogSleep (Int.unsigned i) (Mem.nextblock m)
%                                      (sync_chpool_check thread_sleep ((Lint i)::nil) (uRData l) d))::l ->
%          lastEvType l <> Some LogYieldTy ->
%          get_env_log limit curid l' = Some l'' ->
%          last_op l'' = Some e ->
%          state_check thread_sleep (Lint i::nil) l d ->
%          l''' = LEvent curid LogYieldBack::l'' ->
%          rs' = (undef_regs (CR ZF :: CR CF :: CR PF :: CR SF :: CR OF :: nil)
%                            (undef_regs (map preg_of destroyed_at_call) rs0)) 
%                  # EAX <- Vundef #PC <- (rs0 RA) #RA <- Vundef ->
%          forall
%%            (Hnextblock: getLogEventNB (e) = Some nb)
%%            (Hargs: extcall_arguments rs0 m (mksignature (Tint:: nil) None cc_default) (Vint i:: nil))
%%            (NON_YIELD: match ef with
%%                          | EF_external id _ => 
%                            if peq id thread_sleep then True
%                            else False
%                          | _ => False
%%                        end)
%%%            (LIFT_NEXTBLOCK: mem_lift_nextblock m (Pos.to_nat (nb) - Pos.to_nat (Mem.nextblock m) % nat) = m'),
%%%            tstep ge (TState curid rs0 m d l) E0
%                  (TState curid rs' m' d l''').
%\end{lstlisting}

%   Definition tfinal_state : tstate -> int -> Prop :=
%     fun _ _ => False.
%   
%   Definition init_dproc  :=
%     if zeq main_thread current_thread then main_init_dproc else nomain_init_dproc.
%
%   Inductive tinitial_state {F V} (p: AST.program F V): tstate -> Prop :=
%   | tinitial_state_intro: 
%       forall m0 m1 rs0,
%         Genv.init_mem p = Some m0 ->
%         init_mem_lift_nextblock m0 = m1 ->
%         initial_thread_state (Genv.globalenv p) current_thread init_log = Some rs0 ->
%         tinitial_state p (TState current_thread rs0 m1 (thread_init_dproc current_thread) init_log).
%\end{lstlisting}
%
%\begin{lstlisting}[language=C]
%%  Lemma star_step_get_env_log:
%%    forall ge rs d m curid l l',
%%      get_env_log limit curid l = Some l' ->
%%      star iestep ge
%%           (IEState (proc_id (uRData l))
%%                    (ZMap.set curid (Running rs) (ZMap.init Environment))
%%                    m
%%                    (ZMap.set curid (Some d) (ZMap.init None)) l) E0
%%           (IEState curid
%%                    (ZMap.set curid (Running rs) (ZMap.init Environment))
%%                    m
%%                    (ZMap.set curid (Some d) (ZMap.init None)) l').
%%%  Proof.
%%\end{lstlisting}
%%
%%\begin{lstlisting}[language=C]
%  Inductive match_tstate_iestate: (ZMap.t ThreadState) -> (mem * mem) -> (ZMap.t (option dproc)) ->
%                                 regset -> mem -> dproc -> Log -> Z -> Prop :=
%  | MATCH_TSTATE_IESTATE:
%      forall (rs: regset) rsm (m: mem) (mp : mem * mem) (d: dproc) (dp : ZMap.t (option dproc)) curid l
%             (RSM: rsm = ZMap.set curid (Running rs) (ZMap.init Environment))
%             (Hcurid: proc_id (uRData l) = current_thread)
%             (MP: m = fst mp)
%             (DP: dp = ZMap.set curid (Some d) (ZMap.init None)),        
%        match_tstate_iestate rsm mp dp rs m d l curid.
%
%  Theorem one_step_Asm_T2E:
%    forall ge mp m m' curid curid' rsm rs rs' l l' dp d d' t
%           (TStep: TAsm.tstep ge (TState curid rs m d l) t (TState curid' rs' m' d' l'))
%           (Match: match_tstate_iestate rsm mp dp rs m d l curid),
%    exists rsm' mp' dp',
%      plus IIEAsm.iestep ge (IEState curid rsm mp dp l) t
%           (IEState curid' rsm' mp' dp' l')
%      /\ match_tstate_iestate rsm' mp' dp' rs' m' d' l' curid'.
%\end{lstlisting}

%\subsection{Intermediate Thread-Local Machine Model}\label{subsec:tasm}
%
%Now, we have already gotten a single threaded machine model, which gives us a full isolation.
%However, $\EAsmM{}$ itself is quite different with $\AsmLM$ in terms of its state definition and evaluation rules,
%so it is hard for us to show the refinement relation between this $\EAsmM{}$ and $\AsmLM$ directly. 
%Especially, yield back and environmental steps in $\EAsmM{}$ does not match well with our $\AsmLM$ rules. 
%If we keep those operational style strategy query evaluation, it is hard for us to define a single step behavior of 
%scheduling primitives in our thread-local layer interface.
%To bridge the gap between $\EAsmM{}$ and $\AsmLM$,
%we have introduced one more intermediate machine model, $\TAsmM{[c, \curthread]}$. 
%In here, the machine is always parameterized by a fixed thread id, which we will always notate as $\curthread$. 
%And using this variable, we define this machine's state as
%\begin{small}
%\[
%st_{TAsm} = (\curthread, \ \regs_{\curthread}, \dproc_{\curthread}, \SLog) 
%\]
%\end{small}
%, which has a shared log and only one private data for the thread.
%In this machine, all other threads' steps should be replaced by environmental steps
%as our single-threaded concurrent machine model, $\EAsmM{[c, \{current\_thread\}]}$, does.
%In addition to that, to define a single step behavior of scheduling primitives in our thread-local machine, a big-step style strategy query function has 
%been introduced as
%\begin{small}
%\[{\small
%\begin{array}{l}
%\oracle^{t} \vdash \getenvlog\ (limit :\mathbb{N})\ (l: \SLog ):  \SLog  \\
%\ \ \ \ \ \ \ \ \ \ \ \ \ \ \ \ \ \ \ \ \ \ \ \ \ \ \ \ \ \ :=
%\begin{cases}
%\ \ \mathrm{None}\hfill (\mathrm{when} \ limit = 0) \\
%\ \ \mathrm{Some} \ l \hfill (\mathrm{when} \ \procid{\updatefun_{init}(l)} = \curthread) \\
%\ \ \oracle^{t} \vdash \getenvlog((limit - 1), ((\oracle^{t}(\{\curthread\})(l))::l))\ \ \ \ \  \ \ \ \ \ 
%\hfill (\mathrm{Otherwise})\\
%\end{cases} 
%\end{array}
%}\]
%\end{small}
%, which queries the strategy iteratively until the current thread gets the evaluation control again.
%
%Using the definition, $\TAsmM{[c, current\_thraed]}$ can merge multiple strategy queries during yield call as a single one, 
%and the yield evaluation rule for $\TAsmM{[c, current\_thraed]}$ is defined as
%\begin{small}
%\begin{mathpar}
%\inferrule{
%\updatefun_{init}(l) = ds \\
%\procid(ds) = \curthread \\ 
%\statecheck(\yield, [],ds,\snapfun(d)) \\ 
%f_{\dproc}(curid) = Some\ d \\ 
%\nextblockfun(m) = nb \\ 
%l' = (\curthread , ([], nb, \snapfun(d)))::l \\
%\oracle^{t} \vdash \getenvlog(limit, l') = l'' \\
%last\_event(l) = ev \\
%ev = (\_, (nb, \_)) \vee ev = (\_, (\_, nb, \_))  \\
%\liftnextblock{m}{nb - \nextblock{m}} = m' \\
%l''' = (\curthread , \yieldbackunit)::l''\\
%%\oracle, c\vdash \sstepr{\spec_{id}}{[largs]}{\regs, m, ds, d}{\textit{res}\cup \{\}}{\regs',  m', ds', d'}
%}{
%\curthread, \oracle^{t}, c, \oracle \vdash_\codeinmath{TAsm} \sstep{\yield}{[]}{\curthread, \regs, m', l, d}
% {\{ \}, \curthread, \regs', m', l', d}
%}
%\end{mathpar}
%%\vspace{-5px}
%\end{small}%
%Intuitively, this rule first updates the current shared log by adding one yield event 
%and perform the strategy query using the log.
%If the current strategy returns a log as its result,
%this yield rule updates the machine state using
%the information that the machine has gotten from the querying. 
%Note that the currently-running thread information and a private abstract data do not change at all 
%during this yield call. 
%From $\TAsmM{}$, all machines are thread-local, 
%and scheduling primitives will not perform any context switching at all. 
%Since the differences between $\TAsmM{}$ and $\EAsmM{}$ are only in the state definition 
%and the way to handle yield and sleep evaluations, 
%we can easily prove the following theorem by doing case analysis:
%
%\begin{theorem}[$\TAsmM{}$ refines $\EAsmM{}$]
%\begin{small}
%$$\ltyp{(\{\curthread\}, \oracle^{t}, c, \oracle  \vdash_{\EAsmM{}} \mathrm{PH})}{}{\varnothing}
%{(\curthread, \oracle^{t}, c, \oracle \vdash_{\TAsmM{}} \mathrm{PH})}$$
%\end{small}
%\label{theorem:tasm-refines-easm}
%\end{theorem}
%
%With the given simulation relation, $\simrel(st_{\TAsmM{}}, st_{\EAsmM{}})$, we also need to show the following two cases.
%
%$\proofcase{initial\_state}$ 
%The proof for this case requires additional lemmas.
%This is because in $\TAsmM{}$, a shared log of the initial state is already $\SLog_{init}$ of the current thread.
%In the case of $\EAsmM{}$, however, a shared log of the initial state is always $nil$. 
%Therefore, we need to prove the statement, which is when we have a $init\_st_\TAsmM{}$ and a $init\_st_\EAsmM{}$, there always exist valid more than zero steps in $\EAsmM{}$ and the state $st_{\EAsmM{}}$ which satisfies
%$$ \{\curthread\}, \oracle^{t}, c, \oracle\vdash_{\TAsmM{}} init\_st_{\EAsmM{}} \ni^{*} st_{\EAsmM{}} \wedge \simrel(init\_st_{\TAsmM{}}, st_{\EAsmM{}})$$
%Intuitively, the lemma implies that we need to apply $\EAsmM{}$'s evaluation rules to its initial state for environment threads until we get $\SLog_{init}$ as a shared log of the state.
%
%Proofs for this lemma requires us to work with the induction on $\SLog_{init}$ for the current thread.
%If the initial log is $nil$, the initial state of $\EAsmM{}$ will be directly matched with the initial state of $\TAsmM{}$, which is $\simrel(init\_st_{\TAsmM{}}, init\_st_{\EAsmM{}})$.
%In the inductive case, we use the well-formedness condition of our $\SLog_{init}$. 
%This enforces that the initial log should be 
%either $nil$ or only contain events generated by other threads except the last one $yieldBack$ event as well as 
%the log should have a valid event that creates the current thread at some point.
%This well-formedness condition is not a magic in our proof because it can be removed when we link all threads together 
%(we already show the way to link multiple threads in the previous section).
%Then, we know that all events except the last $yieldBack$ are
% generated by other threads in our environment, and cannot contain
%any events raised by the current thread. 
%Since the current $\EAsmM{}$ that we are considering is a single threaded machine parameterized by
%a singleton set, $\{\curthread\}$, we know that all evaluation to generate the exactly same log with the initial log (for $\curthread$) on $\EAsmM{}$ should be on environment step of $\EAsmM{}$ except the last one $yieldBack$ rule after the current 
%thread get a control for its evaluation. 
%Therefore, we can provide the $st_{\EAsmM{}}$ which satisfies
%$$\{\curthread\}, \oracle^{t}, c, \oracle \vdash_{\TAsmM{}} init\_st_{\EAsmM{}} \ni^{*} st_{\EAsmM{}} \wedge \simrel(init\_st_{\TAsmM{}}, st_{\EAsmM{}})$$
%
%
%$\proofcase{one\_step\_refinement}$
%Next step is showing the fact that when $\simrel(st_{\TAsmM{}}, st_{\EAsmM{}})$ and 
%when we have one step evaluation on $\TAsmM{}$, 
%which is $\curthread, \oracle^{t}, c, \oracle \vdash_{\TAsmM{}} st_{\TAsmM{}} \ni st'_{\TAsmM{}} $, 
%then there exists a $st'_{\EAsmM{}}$ that satisfies 
%$$\{\curthread\}, \oracle^{t}, c, \oracle \vdash_{\TAsmM{}} st_{\EAsmM{}} \ni^{+} st'_{\EAsmM{}} \wedge \simrel(st'_{\TAsmM{}}, st'_{\EAsmM{}})$$
%
%Proving this property requires case analysis on $\TAsmM{}$ evaluation rules.
%
%For all cases except scheduling rules in $\TAsmM{}$ (\textit{\ie} $\yield$ and $\sleep$ rules), 
%we can provide a next state in $\EAsmM{}$ ($st'_{\EAsmM{}}$) directly, which satisfies $\{\curthread\}, \oracle^{t}, c,\oracle \vdash_{\EAsmM{}} st_{\EAsmM{}} \ni st'_{\EAsmM{}} \wedge \simrel(st'_{\TAsmM{}}, st'_{\EAsmM{}})$.
%
%In case of the $\yield$ evaluation rule in $\TAsmM{}$, let's assume that 
%a shared log in $st_{\TAsmM{}}$ is $l$, and $\oracle^{t} \vdash \getenvlog{}$ returns a shared log $l''$ with an arbitrary big number $limit$ and the 
%given argument $l'$ (\textit{\ie} $l' = (\curthread , ([], nb, \snapfun(d)))::l$) when $d$ is a 
%private state of $st_{\TAsmM{}}$ and $nb$ is a next available block of the memory in $st_{\TAsmM{}}$). 
%With this result, a shared log in $st'_{\TAsmM{}}$ will be $(\curthread , \yieldbackunit)::l''$.
%Then, by its definition, there always exists a valid $l_{added}$ which satisfies $l'' = l_{added} {+\!\!+} l'$.
%Therefore, we can prove this case by doing an induction on $l_{added}$. 
%If the log ($l_{added}$) is $nil$, then, we simply match this one $\yield$ evaluation rule 
%(that updates the log as $l'''$) with one step $yield$ and one step $yieldBack$ rules in $\EAsmM{}$.
%Therefore, we can provide a valid $st'_{\EAsmM{}}$ 
%which satisfies $\{\curthread\}, \oracle^{t}, c,\oracle \vdash_{\EAsmM{}} st_{\EAsmM{}} \ni^{+} st'_{\EAsmM{}} \wedge \simrel(st'_{\TAsmM{}}, st'_{\EAsmM{}})$.
%In the inductive case, $\EAsmM{}$ needs to perform an evaluation with its environment rule which will add one event 
%into its shared log at every time. 
%Other rules will not be applied to generate the next $\EAsmM{}$ state because all threads except
%$\curthread$ are in the state of $\mathrm{Environment}$ now. 
%Therefore, we also can provide a valid new state, $st'_{\EAsmM{}}$ that satisfies 
%$\{\curthread\}, \oracle^{t}, c,\oracle \vdash_{\EAsmM{}} st_{\EAsmM{}} \ni^{+} st'_{\EAsmM{}} \wedge \simrel(st'_{\TAsmM{}}, st'_{\EAsmM{}})$ and which is doing its evaluation by applying one $\yield$ rule, multiple environment 
%rules, and one $yieldBack$ rule of $\EAsmM{}$ on $st_{\EAsmM{}}$.
%
%In the case of the $\sleep$ evaluation rule in $\TAsmM{}$, we also can provide a valid $st'_{\EAsmM{}}$ that satisfies 
%$\{\curthread\}, \oracle^{t}, c, \oracle\vdash_{\EAsmM{}} st_{\EAsmM{}} \ni^{+} st'_{\EAsmM{}} \wedge \simrel(st'_{\TAsmM{}}, st'_{\EAsmM{}})$ by using the similar approach with the $\yield$ case.
%
%
%\subsection{Thread-Local Machine Model}\label{subsec:hasm}
%By applying the whole process on $\CSched$ that we have mentioned,
%we finally can define thread-local layer interface  discussed in Sec.~\ref{subsec:phthreadlayer}.
%The machine model for thread-local layer interface is almost same with the machine model of CPU-local layer interface, which is $\AsmLM$.
%Therefore, we are able to 
%utilize the whole power of $\compcertx$\ and build thread-local layers both written in C and in Assembly with the guarantee that those layers preserve the same behavior in our CPU-local machine.
%This machine model, however, has two differences with  $\AsmLM$.
%First, as mentioned in Sec.~\ref{subsec:phthreadlayer}, the evaluation semantics
%for $\yield$ and $\sleep$ is totally changed as no-op like evaluations.
%This difference is already dealt with multiple steps of refinement with several different machine models. 
%In this sense, no additional steps are required to handle this issue.
%Second, the machine allows a dynamic initial state for each thread, and this dynamically assigned 
%initial state should be satisfied by our system invariant.
%To resolve this challenge, 
%Our thread-local layer interface provides one abstract definition that will be
%instantiated with a concrete layer and a data structure implementation later.
%
%The abstract function, $\composedata$, has a type of
%\begin{small}
%\[
%\composedata : \mathbb{Z} \rightarrow \dshare \rightarrow \dproc \rightarrow adt
%\]
%\end{small}
%which gets the current thread id, a shared data, and a private data as its arguments 
%and returns a \textit{abstract data} for $\AsmHM$ with the current thread id.
%With this function, an initial state of a thread-local layer interface with the current thread id ($\curthread$) will be 
%as
%\begin{small}
%\[
%\composedata(\curthread, \ \updatefun_{init}(\SLog_{init}), init\_dproc(\curthread))
%\]
%\end{small}
%This definition also gives consistency between our multithreaded concurrent machine model 
%and this thread-local machine model.
%When looking at the initial state definition for $\EAsmM{}$ in Sec.~\ref{subsec:fulleasm},
%finding the similarity between both definitions is straightforward.
%For the guarantee about preserving our system invariant in the initial state, we only need to prove that 
%the calculated initial state satisfies our invariant.
%
%With the all stated ingredients during multiple previous sections, we can prove the following refinement theorem:
%\begin{theorem}[$\AsmHM$ Refines $\TAsmM{}$]
%\begin{small}
%Assume that the simulation relation between $\TAsmM{}$ and $\AsmHM$ is $\simrel(st_{\AsmHM}, st_{\TAsmM{}})$.
%In addition, suppose that there exists a layer definition, $\mathrm{PH}$, that satisfies an abstract relation, 
%$\absrelt(\simrel(st_{\AsmHM}, st_{\TAsmM{}}), \TSched, \mathrm{PH})$. Then,
%$$\ltyp{(\curthread, \oracle^{t}, c, \oracle \vdash_{\TAsmM{}} \mathrm{PH})}
%{\absrelt(\simrel(st_{\AsmHM}, st_{\TAsmM{}}), \TSched, \mathrm{PH})}
%{\varnothing}{(\curthread, \oracle^{t}, c, \oracle \vdash_{\AsmHM} \TSched)}$$
%\end{small}
%\label{theorem:hasm-refines_tasm}
%\end{theorem}%
%
%$\proofcase{initial\_state}$ We show that there exists a valid initial state of $\TAsmM{}$ which satisfies our simulation relation
%$\simrel$ with an initial machine state of $\AsmHM$, which is stated as follows:
%$$\simrel(init\_st_{\AsmHM}, init\_st_{\TAsmM{}})$$
%This proof relies on the abstract relation $\absrelt(\simrel(st_{\AsmHM}, st_{\TAsmM{}}), \TSched, \mathrm{PH})$,
%but this case is quite trivial in its essence
%because we construct $init\_st_{\AsmHM}$ using a $init\_st_{\TAsmM{}})$ and 
%the $\composedata$ function.
%
%$\proofcase{one\_step\_refinement}$
%Next step is showing the fact that when $\simrel(st_{\AsmHM}, st_{\TAsmM{}})$ and when we have one step evaluation on $\AsmHM$, which is $\oracle, \ c, \oracle^{t}, \curthread \vdash_{\AsmHM} st_{\AsmHM} \ni st'_{\AsmHM} $, 
%then there exists a $st'_{\TAsmM{}}$ that satisfies 
%$$\curthread, \oracle^{t}, c, \oracle \vdash_{\TAsmM{}} st_{\TAsmM{}} \ni st'_{\TAsmM{}} \wedge \simrel(st'_{\AsmHM}, st'_{\TAsmM{}})$$
%
%Proving this property requires case analysis on $\AsmHM$ evaluation rules, and it  
%also heavily relies on the abstract definition,  $\absrelt(\simrel(st_{\AsmHM}, st_{\TAsmM{}}), \TSched, \mathrm{PH})$.
%
%For all cases except the external call evaluation rule in $\AsmHM$, 
%we can provide a valid $st'_{\TAsmM{}}$ directly, which satisfies $\curthread, \oracle^{t}, c, \oracle \vdash_{\TAsmM{}} st_{\TAsmM{}} \ni st'_{\TAsmM{}} \wedge \simrel(st'_{\AsmHM}, st'_{\TAsmM{}})$.
%
%For the external call case, we need additional case analysis.
%When the primitive is neither $\yield$ nor $\sleep$, then the matched
%evaluation rule in $\TAsmM{}$ will also be the external evaluation rule. 
%Then, we can provide a next state, $st'_{\TAsmM{}}$, that satisfies $\curthread, \oracle^{t}, c, \oracle \vdash_{\TAsmM{}} st_{\TAsmM{}} \ni st'_{\TAsmM{}} \wedge \simrel(st'_{\AsmHM}, st'_{\TAsmM{}})$.
%
%If the primitive is one of $\yield$ and $\sleep$, we match the evaluation with the $\yield$ evaluation rule in $\TAsmM{}$ or 
%the $\sleep$ evaluation rule in $\TAsmM{}$, respectively.
%Then, we can also provide a valid $st'_{\TAsmM{}}$ that satisfies $\curthread, \oracle^{t}, c, \oracle \vdash_{\TAsmM{}} st_{\TAsmM{}} \ni st'_{\TAsmM{}} \wedge \simrel(st'_{\AsmHM}, st'_{\TAsmM{}})$.
%
%Now, we can finally prove that our thread-local machine refines our CPU-local machine and our fully linked thread-local machine refines our 
%CPU-local machine. 
%The first property gives us a source to build thread-local layers and the second one guarantee our thread-local machine behavior is consistent with 
%
%\ignore{
%\begin{theorem}[$\AsmHM$ refines $\AsmLM$]
%\begin{small}
%Assuming two machines, $\AsmHM$ and $\AsmLM$. Then we can show that
%$$\ltyp{(c, \oracle \vdash_{\AsmLM} \CSched)}
%{}
%{\varnothing}{(\curthread, \oracle^{t}, c, \oracle \vdash_{\AsmHM} \TSched)}$$
%when $\curthread$ is a member of threads on CPU $\ThreadConf$, and when we have valid abstract relations
%$\absrel$ and $\absrelt$ with the abstract intermediate layer $\mathrm{PH}$ on $\EAsmM{}$ and $\TAsmM{}$.
%\end{small}
%\label{theorem:tasm_refines_lasm}
%\end{theorem}%
%
%By using Theorem~\ref{theorem:easm_refine_lasm},~\ref{theorem:full-easm-refines-single-easm},~\ref{theorem:tasm-refines-easm}~\ref{theorem:hasm-refines_tasm}, we can show that the theorem is correct.
%}
%The second theorem in here is more detailed version of Theorem~\ref{theorem:thread-full-compose} by exposing several hidden definitions to simplify the theorem. 
%
%\begin{theorem}[Merged $\AsmHM$ refines $\AsmLM$]
%\begin{small}
%Assuming two machines, $\AsmHM$ and $\AsmLM$ and $\fullthreadset$ contains all available threads on CPU $\ThreadConf$.
%. Then we can show that
%$$\ltyp{(c, \oracle \vdash_{\AsmLM} \CSched)}
%{}
%{\varnothing}{\join_{tid \in \fullthreadset} (\curthread, \oracle^{t}, c, \oracle \vdash_{\AsmHM} \TSched)}$$
%when we have valid abstract relations $\absrel$ and $\absrelt$ with the abstract intermediate 
%layer $\mathrm{PH}$ on $\EAsmM{}$ and $\TAsmM{}$.
%\end{small}
%\label{theorem:tasm_refines_lasm}
%\end{theorem}%
%
%By using Theorem~\ref{theorem:easm_refine_lasm},~\ref{theorem:thread-full-compose},~\ref{theorem:tasm-refines-easm}~\ref{theorem:hasm-refines_tasm}, we can show that the theorem is correct.
%
%
