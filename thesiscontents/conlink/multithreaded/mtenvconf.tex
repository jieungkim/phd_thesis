
\subsection{Multithreaded Environment Configuration}
\label{chapter:linking:subsec:multithreaded-env-configuration}

This section provides  necessary definitions 
to build intermediate machine models in our multithreaded-linking framework
(\ie, abstract state types for the $\TLink$ layer , thread configurations, and auxiliary functions),
which includes a function that calculates a dynamic initial state for each thread.
These definitions are used to formally define a multithreaded machine model in Figure~\ref{fig:chapter:conlink:multithreaded-machine-model-easm}.
%multithreaded environment definitions that we explained in Section~\ref{chapter:linking:subsec:multithreaded-linking-overview}.
%
%Similar to the multicore machine model, we provide configurations 
%for mutlithreaded linking.

\subsubsection{Data Type} 

\begin{figure}
\begin{lstlisting}[language=C, morekeywords={Class}]
Class SingleData := {
  dshare: Type; // shared data    
  processor_id : dshare -> Z;  // a CPU ID is a part of shared data
  proc_id: dshare -> Z; // the current thread ID is is a part of shared data
  dproc: Type; // private data for each thread
  <@$\cdots$@>}.
\end{lstlisting} 
\caption{Abstract State Types for $\TLink$.}
\label{fig:chapter:conlink:abs-state-types-for-tlink}
\end{figure}

To divide a CPU private state and a global log into the proper set of private states and a shared log, we introduce new abstract data types in \lstinline$SingleDAta$ in Figure~\ref{fig:chapter:conlink:abs-state-types-for-tlink}. 
There are two--$\codeinmath{dshare}$ and $\codeinmath{dproc}$--which represent a shared datum and a private datum for each thread, respectively. 
There are not many restrictions for those types,
but shared data contain the information about the 
CPU ID as well as the currently running thread ID to provide 
\lstinline$processor_id$ and \lstinline$proc_id$ functions. 


\subsubsection{Thread Machine Memory Model}

% algebraic memory model
Multithreaded linking also requires the composition and decomposition of a memory,
which logically divides a single memory into a set of memories associated with threads.
Our system neither allows ownership transfers for the same memory region among threads nor assumes shared memory; thus, we can safely assume that the memory can be divided into multiple disjoint memories that are owned by the designated threads in the system. 
However, threads share several resources in the kernel via primitive calls that provide the way to manipulate them with restricted manners. 
Based on this assumption, 
we define an algebraic memory model--the extended memory model--by adding two key relations from the $\compcertx$'s memory model.
Those two key relations are 
\begin{lstlisting}[language=C]
mem_disjoint_union : mem -> mem -> mem -> Prop;
mem_lift_nextblock : mem -> nat -> mem;
\end{lstlisting}
The memory in $\compcertx$ is distinguished by its block identifier and an offset for each block. 
The type of this block identifier is a natural number,
and we restrict the memory operation of each thread to distinguished blocks.  
The first \lstinline$mem_disjoint_union m1 m2 m3$  shows that 
the $m3$ is the union of $m1$ and $m2$ when $m1$ and $m2$ 
are disjoint from each other. 
The second, \lstinline$mem_lift_nextblock$,
provides the operation that makes the adjustment on the memory possible. 



\subsubsection{Thread Configuration}

\begin{figure}
\begin{lstlisting}[language=C, morekeywords={Class}]
Class ThreadsConfigurationOps := {      
  current_CPU_ID: Z; // The ID for the CPU we're looking at. 
  current_thread: Z; // The thread we're looking at in high-level machines
  non_current_thread_list: list Z;  // The other threads on this CPU 
  full_thread_list := current_thread :: non_current_thread_list; // The full list
  main_thread: Z; // The main thread on this CPU 
  limit: nat // A log look-back limit used in TAsm 
  }.
\end{lstlisting}
\caption{Thread Configurations.}
\label{fig:chapter:linking:thread-configurations}
\end{figure}


Thread linking also requires multiple configurations about the environment of the system, which are defined in  
\lstinline$ThreadConfigurationOps$ in Figure~\ref{fig:chapter:linking:thread-configurations}.
For example, \lstinline$current_CPU_ID$ is a CPU ID that is 
associated with the current local layer.
Layer $\CSched$ is associated with \lstinline$current_CPU_ID$, 
and all threads associated with the $\TLink$ and $\TSched$ layers 
are also run with  \lstinline$current_CPU_ID$.
The \lstinline$current_thread$ is a variable for the focused thread ID that is associated with our per-thread machine.
In this sense, the layer above $\TSched$ is parameterized by this thread ID (\lstinline$current_thread$). 
The full thread list is a set of thread identifiers which run on top of the current CPU ID. 
It is a union of the current thread identifier, a singleton set, 
and non-current threads. 
There are no restrictions on the size of the full thread set even though 
the thread pool can be bounded in the real implementation, which differs from the full CPU set in our multicore linking configuration.
The \lstinline$main_thread$ represents the main thread (which initializes the system) of the CPU. 
This main thread may differ from the current thread identifier.

\begin{figure}
\begin{lstlisting}[language=C]
Inductive LogEventUnit :=
  | LogYield (n: positive)
  | LogSleep (i: Z) (n: positive) <@$\cdots$@>
  | LogPrim (id: ident) (args: list lval) (choice : Z) (dprocSnap : privDataSnap) 
  | LogYieldBack.
Inductive LogEvent := | LEvent : Z -> LogEventUnit -> LogEvent.
\end{lstlisting}
\begin{center}
(a) Event and Log Definitions for Machine Models in the Multithreaded Linking Framework.
\end{center}
\begin{lstlisting}[language=C, morekeywords={Class}]
Class SingleOracleOps := {
  init_log: Log; // initial log for each thread
  Single_Oracle : Log -> LogEvent; // environmental context for multithreaded linking
  <@$\cdots$@> }.
\end{lstlisting}
\begin{center}
(b) Initial Log and Environmental Context Definitions.
\end{center}
\caption{Event, Log, and Environmental Context Definitions for Multithreaded Linking}
\label{fig:chapter:linking:event-and-log-for-multithreaded-linking}
\end{figure}


New types of  events and a log  are also necessary to build our multithreaded linking framework. 
Similar to the multicore linking case, 
shared resources are represented as a log--a list of event--which can be applied to any concrete layer definitions for 
$\CSched$, $\TLink$, and $\TSched$.
The definition of those events and the log is in Figure~\ref{fig:chapter:linking:event-and-log-for-multithreaded-linking} (a).
Our framework allows two scheduling primitives, 
$\yield$ and $\sleep$. 
Those two primitives use their designated events to keep the information in the global log. 
The yield event does not need to contain any information; it just needs to give the control
to another thread (among the threads in the same CPU). 
However, the sleep event is usually related to the conditional variable and also a shared resource among CPUs.
In this sense, we have to memorize the conditional variable that the sleep call uses ($i$).
Other primitives that access the shared resources trigger the primitive event. 
The event memorizes the primitive name, the argument for the primitive name,
and the snapshot of the private data of the thread.
The private data snapshot is taken  when the primitive has been invoked. 
They are not always necessary, but some primitives may be required to contain the information.
For example, the acquire lock and the release lock primitives have to memorize the snapshot of the (abstract) private data when they have been invoked, which has similarities with the $\push/\pull$ operations of our multicore memory model. 
The last event, \lstinline$LogYieldBack$, is to memorize the moment to figure out when each thread re-achieves the evaluation control. 

There are also a few definitions that depends on event and log definitions, which are presented in Figure~\ref{fig:chapter:linking:event-and-log-for-multithreaded-linking} (b).
The environmental context \lstinline$Single_Oracle$  for our multithreaded linking
then can be defined as a function that gets the current log and returns the event. 
In addition,
each thread has its initial log for the evaluation. 
For example, as shown in Figure~\ref{fig:chapter:conlink:multithreaded-machine-model-easm},
the initial log of thread 2 must include the proper spawn event as well as yield and yield-back events
when thread 2 starts its evaluation after thread 1 spawns  thread 2 and gives the control to it.

\subsubsection{Auxiliary Functions}

\begin{figure}
\begin{lstlisting}[language=C, morekeywords={Class}]
  Class SingleOracle  <@$\cdots$@> := {
      update : dshare -> Log -> dshare;
      uRData (l: Log) : dshare := update (init_dshare) l;
      has_event: ident -> bool;      
      <@$\cdots$@>
      thread_init_dproc : Z-> dproc; // calculate each thread's initial private state
      // connect two different types of states to build <@$\color{red} \hasmmach$@>
      PHThread2TCompose : dshare -> dproc -> RData; 
      // provide the initial state for each thread that uses  <@$\color{red} \hasmmach$@>
      thread_init_rdata (tid: Z) :=
        PHThread2TCompose (uRData init_log) (thread_init_dproc tid);
      // calculate each thread's initial register value
      prim_thread_init_pc {F V} (ge: Genv.t F V): ident -> list lval -> option val }.
\end{lstlisting}
\caption{Auxiliary Functions in Multithreaded Linking Framework.}
\label{fig:chapter:conlink:multithreaded-linking-aux-functions}
\end{figure}

Based on those definitions, we provide multiple auxiliary functions 
for  per-thread machines to calculate necessary information in Figure~\ref{fig:chapter:conlink:multithreaded-linking-aux-functions}.
Among them,  \lstinline$update$ and \lstinline$uRData$ functions are  declarations of a replay function for the global log. 
The \lstinline$has_event$ function is to check whether the primitive calls must trigger the event.
We also require the snapshot functions for the events.
It also shows two important abstract function declarations--\lstinline$thread_init_dproc$ and \lstinline$prim_thread_init_pc$--that calculate the initial abstract data as well as the initial program counter value of the thread. 
Two other functions--\lstinline$thread_init_rdata$ \lstinline$PHThread2TCompose$--are for providing the proper initial state for the programs that run with $\hasmmach$, 
which requires a translation from the result of \lstinline$thread_init_dproc$   to adjust the differences between
the abstract state type in the per-thread machine model and the abstract state type in the
intermediate machine models in our multithreaded linking. 

\subsubsection{Properties on Thread Configurations and Auxiliary Functions}


\begin{figure}
\begin{lstlisting}[language=C, , morekeywords={Class}]
Class SingleOracleP <@$\cdots$@> := {
  CPU_ID_valid: forall l, processor_id (uRData l) = current_CPU_ID;
  null_log_proc_id: proc_id (uRData nil) = main_thread;
  init_log_proc_id: proc_id (uRData init_log) = current_thread;
  all_cid_in_full_thread_list: forall l, In (proc_id (uRData l)) full_thread_list;
  update_proc_id:  forall d1 d2 d3 l a, update d1 l =  d2 ->
      getLogEventType a = LogOtherTy ->
      update d1 (a :: l) =  d3 -> proc_id d3 = proc_id d2;
  update_processor_id:
    forall d1 d2 d3 l a, update d1 l = d2 -> 
      update d1 (a::l) = d3 -> processor_id d2 = processor_id d3;   
  prim_thread_init_pc_symbols_preserved:
    forall {F V} (ge1 ge2: Genv.t F V) f vargs v,
      (forall i, Genv.find_symbol ge1 i = Genv.find_symbol ge2 i) ->
      prim_thread_init_pc ge1 f vargs = Some v ->
      prim_thread_init_pc ge2 f vargs = Some v
 <@$\cdots$@>      
 }.
\end{lstlisting}
\begin{center}
(a) Properties of Auxiliary Functions.
\end{center}

\begin{lstlisting}[language=C, , morekeywords={Class}]
Class SingleOracleLinkP  <@$\cdots$@> := {
  <@$\cdots$@>
  oracle_event_yield_back:
    forall l, lastEvType l = Some LogYieldTy ->
      exists i, Single_Oracle l = LEvent i LogYieldBack;
  oracle_init_log:
    forall (l_p: Log) (e: LogEvent) (l: Log),
      init_log = l_p ++ e :: l -> Single_Oracle l = e;
  init_log_property:
    init_log = nil \/
    exists l e, init_log = LEvent (proc_id (uRData l)) LogYieldBack :: l /\
      proc_id (uRData l) <> proc_id (uRData nil) /\ last_op l = Some e /\
      getLogEventNB e <> None /\
      forall e l1 l2, l = l1 ++ e :: l2 ->
        proc_id (uRData l2) <> 
        proc_id (uRData init_log) }.
\end{lstlisting}
\begin{center}
(b) Properties of an Environmental Context and an Initial Log.
\end{center}
\caption{Properties of Thread Configurations and Auxiliary Functions in the Multithreaded-Linking Framework.}	
\label{fig:chapter:conlink:multithreaded-configuration-props}
\end{figure}

Of course, those definitions must satisfy multiple properties.
Those properties are necessary for us to prove the multiple properties that the machine must follow (\eg, decidability on their initial state and the property of their log construction).
Figure~\ref{fig:chapter:conlink:multithreaded-configuration-props} shows several properties related to thread configurations, auxiliary functions, and initial logs.
For example, 
 \lstinline$CPU_ID_valid$ (Figure~\ref{fig:chapter:conlink:multithreaded-configuration-props} (a))
tells us that all the scheduling results of this CPU 
will not change the CPU ID. 
Also, the initial log for each thread satisfies a certain property--\lstinline$oracle_init_log$ and \lstinline$init_log_property$
in Figure~\ref{fig:chapter:conlink:multithreaded-configuration-props} (b)--which provides the well-formed initial log for us to be required to show the 
refinement between the intermediate machines. 
Most of them can be instantiated with our concrete definition of the configurations, auxiliary functions,
and abstract thread definitions. 

%\begin{figure}
%$$
%\begin{array}{llllllll}
%(\textit{Update}) & \updatefun & \in & \dshare \rightarrow \SLog \rightarrow \dshare &
%(\textit{Construct}) & \updatefun_{init} \ l& := & \updatefun\ \dshare_{init} \ l  \ \ \ \ (l \in \SLog) \\
%(\textit{StateCheck}) & \statecheck & \in &
%\multicolumn{5}{l}{
% \threadfunid \rightarrow \largs \rightarrow \dshare \rightarrow \dsnap \rightarrow \mathrm{Prop}
% }
% \\
% (\textit{EventCheck}) & \haseventfun & \in & \threadfunid \rightarrow bool  &
%(\textit{Reg})  &\regs & ::= & \mathsf{EIP} \ | \ \mathsf{EAX} \ | \  \cdots \\
%(\textit{ThreadState}) & \threadstate & \in &
%\multicolumn{5}{l}{
% \{\mathrm{Environment}, \ \mathrm{Available}, \ \mathrm{Running}\ \regs \} 
%  }
%\end{array}
%$$
%\caption{Auxiliary Definitions for multithreaded linking}
%\end{figure}



%
%Some 
%We also requires the restrictions about the form of the initial log (and the log) 
%and \lstinline$SingleOracleLinkP$  contains the constraints
%the properties about the log that 
%the log has to  satisfy them. 
%\jieung{I think exposing all those functions in here is not a good idea. I need to rearrange them.
%we can defer the explanation of some of them}

%\begin{lstlisting}
%Record high_level_invariant (abd: RData) := 
%\end{lstlisting}
%
%\begin{lstlisting}
%
%  Class SingleOracleProp `{thread_conf_prf: ThreadsConfiguration}:=
%    {
%
%      (* assumptions about update functions - proc_id and processor_id preserve property *)
%      update_proc_id:
%        forall d d' d'' l a,
%          update d l =  d' ->
%          getLogEventType a = LogOtherTy ->
%          update d (a :: l) =  d'' ->
%          proc_id d'' = proc_id d';
%
%      update_processor_id:
%        forall d d' d'' l a,
%          update d l = d' ->
%          update d (a::l) = d'' ->
%          processor_id d'' = processor_id d';
%
%    (** This checks whether the given primitive call initializes any
%        thread. If so, returns the thread's id and initial register state. *)
%
%      init_data_low_level_invariant n:
%        low_level_invariant n (thread_init_rdata current_thread);
%
%      init_data_phthread_high_level_invariant:
%        PHThreadInvariant.high_level_invariant (thread_init_rdata current_thread);
%       
%      prim_thread_init_pc_global:
%        forall {F V} (ge: Genv.t F V) f vargs v,
%          prim_thread_init_pc ge f vargs = Some v ->
%          (exists id ofs, symbol_offset ge id ofs = v);
%
%      prim_thread_init_pc_symbols_preserved:
%        forall {F V} (ge1 ge2: Genv.t F V) f vargs v,
%          (forall i, Genv.find_symbol ge1 i = Genv.find_symbol ge2 i) ->
%          prim_thread_init_pc ge1 f vargs = Some v ->
%          prim_thread_init_pc ge2 f vargs = Some v
%    }.
%\end{lstlisting}

%\begin{lstlisting}[language=C]
%Class ThreadsConfiguration `{s_oracle : SingleOracle}: Prop := {
%  CPU_ID_valid: forall l, processor_id (uRData l) = current_CPU_ID;
%  main_valid: In main_thread full_thread_list;
%  null_log_proc_id: proc_id (uRData nil) = main_thread;
%  main_thread_val: main_thread = current_CPU_ID + 1;
%  current_CPU_ID_range: 0 <= current_CPU_ID < Constant.TOTAL_CPU;
%  dev_handling_cid_constraint: dev_handling_cid = main_thread;
%  init_log_proc_id: proc_id (uRData init_log) = current_thread;
%  valid_thread_list: forall i, In i full_thread_list -> 
%    Constant.TOTAL_CPU < i < Constant.num_proc \/ i = main_thread;
%  all_cid_in_full_thread_list: forall l, In (proc_id (uRData l)) full_thread_list;
%  full_thread_list_prop: NoDup full_thread_list }. 
%\end{lstlisting}
%\begin{lstlisting}[language=C]
%  Class SingleOracle  `{single_data: SingleData} `{single_oracle_ops : SingleOracleOps}  
%        `{threads_conf_ops : ThreadsConfigurationOps} := {
%      update : dshare -> Log -> dshare;
%      has_event: ident -> bool;      
%      uRData (l: Log) : dshare := update (init_dshare) l;      
%      choice_check: ident -> list lval -> dshare -> dproc -> Z;
%      sync_chpool_check: ident -> list lval -> dshare -> dproc ->
%        option SyncChanPool;
%      snap_func : dproc -> privDataSnap;
%      snap_rev_func : privDataSnap -> dproc; 
%      thread_init_dproc : Z-> dproc;
%      state_checks (name : ident) (largs: list lval) (l : Log) 
%        (pdp: ZMap.t (option dproc)) :=
%        match ZMap.get (proc_id (uRData l)) pdp with 
%          | Some pd => state_check name largs l pd
%          | _ => True
%        end;
%      PHThread2TCompose : option Kctxt -> dshare -> dproc -> RData;      
%      thread_init_rdata (tid: Z) :=
%        PHThread2TCompose
%          None
%          (uRData init_log)
%          (thread_init_dproc tid);
%      prim_thread_init_pc {F V} (ge: Genv.t F V): ident -> list lval -> option val }.
%\end{lstlisting}


%\begin{figure}
%\[
%\begin{array}{llll llll}
%(\textit{ThreadSet}) & \threadset & \in & \{Z\} &
%(\textit{FullThreadSet}) & \fullthreadset & \in & \threadset \\
%(\textit{InitLog}) & \SLog_{init} & \in & \SLog &
%(\textit{InitNb}) & \blocknum_{init} & \in & \blocknum \\
%(\textit{EnvContext}) & \oracle^{t} & \in & \threadset \rightarrow \varphi^{\threadset} &
%(\textit{Strategy}) & \varphi^{\threadset} & \in & \SLog \rightarrow \Sevent \\
%\end{array}
%\]
%\caption{Thread Configurations }
%\end{figure}

%\begin{lstlisting}[language=C]
%Class ThreadsConfigurationOps := {      
%    current_CPU_ID: Z; // The ID for the CPU we're looking at. (necessary?) 
%    current_thread: Z; // The thread we're looking at in high-level machines
%    dev_handling_cid : Z; // specific thread for handling device drivers 
%    vm_handling_cid : Z; // specific thread for handling this CPU's virtual machine 
%    non_current_thread_list: list Z;  // The other threads on this CPU 
%    full_thread_list := current_thread :: non_current_thread_list; // The full list
%    main_thread: Z; // The main thread on this CPU 
%    limit: nat // A log look-back limit used in TAsm  }.
%\end{lstlisting}

%
%\begin{figure}
%$$
%\begin{array}{llll llll}
%(\textit{SharedData}) & \dshare & : & \toptype &
%(\textit{PrivateData}) &  \dproc & : & \toptype \\
%(\textit{InitSharedData}) & \dshare_{init} & : & \dshare &
%(\textit{ProcessorID}) & \cpuid & : & \dshare \rightarrow \ztype\\
%(\textit{ThreadID}) & \procid & : & \dshare \rightarrow \ztype&
%(\textit{BlockNum}) & \blocknum & : & \nattype\\
%(\textit{Mem}) & m & : & loc \rightarrow val &
%(\textit{NextBlock}) & \nextblockfun & : & m \rightarrow  \nattype \\
%\end{array}
%$$
%\caption{Data Types and Auxiliary Functions Related to Data Types}
%\end{figure}
%
%\begin{figure}
%\[
%\begin{array}{llllllll}
%(\textit{Lval}) & \lval & \in & Int \cup Ptr  &
%(\textit{Largs}) & \largs & \in & list \ \lval  \\
%(\textit{Ident}) & \threadfunid & \in & \{\yield,\ \sleep,\ \cdots\}&
%(\textit{SnapShot}) & \dsnap & \in & Type  \\
%(\textit{SnapShotFun}) & \snapfun & \in &  \dproc \rightarrow \dsnap &
%(\textit{SleepEventUnit}) & \primevunit{sleep} & \in & Int \times \blocknum \times \dsnap \\
%(\textit{YieldEventUnit}) & \primevunit{yield} & \in & \blocknum \times \dsnap &
%(\textit{PrimEventUnit}) & \primevunit{\threadfunid} & \in & \threadfunid \times \largs \times \dsnap \\
%(\textit{EventUnit}) &  \primevunit{} & \in &
%\multicolumn{5}{l}{
% \{\primevunit{sleep},\ \primevunit{sleep},\  \primevunit{ident},\  \yieldbackunit \} 
% }
% \\
%(\textit{Event}) &  \Sevent & \in & tid \times \primevunit{} &
%(\textit{Log}) &  \SLog & \in & list\ \Sevent \\
%\end{array}
%\]
%\caption{Event and Log for Mutlithreaded Linking Framework.}
%\end{figure}
