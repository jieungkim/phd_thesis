
\subsection{Multithreaded Environment Configuration}
\label{chapter:linking:subsec:multithreaded-env-configuration}

Similar to the multicore machine model, we provide configurations 
for mutlithreaded linking. 
We have defined the abstract type of 
shared and private states, 
definitions for the global log, 
auxiliary functions,
and properties for those auxiliary definitions. 

\subsubsection{Data Type} 
To divide the CPU private state and the log into the proper set of private state and the shared log, we introduce 
multiple abstract data types for the multithreaded machine, which are defined as follows:
\begin{lstlisting}[language=C]
Class SingleData :=
  {
    dshare: Type;
    init_dshare: dshare;
    processor_id : dshare -> Z;
    proc_id: dshare -> Z;
    dproc: Type;
    main_init_dproc: dproc;
    nomain_init_dproc: dproc
  }.
\end{lstlisting} 
there are two abstract data types, $\codeinmath{dshare}$ and $\codeinmath{dproc}$, 
which represent shared data and private data, respectively. 
There are not many restrictions for those types in this definiton, 
but shared data has the information about the CPU id as well as the currently running process ID (thread ID) in it. 
In addition to that, 
the private data has two different kinds of initial states. 
The first is for the main thread, which starts at the beginning, 
and initialize the whole data structures for the whole  threads (for example, the main and idle process 
in operating systems)
On the other hand, the second is for the other threads, 
who are usually children of main process. 



\subsubsection{Thread Machine Memory Model}

% algebraic memory model

\jieung{When writing paper, need to mention that it is in CCAL instead of mentioning it in the paper}
\jieung{Please take a loop at the CCAL paper for better explanation} 
\jieung{Please add drawings}
Multithreaded linking also requires the composition and the decompoisiton of the memory,
which divides a single memory into multiple set of memroy. 
Our system does not allow ownership transfers for the bare memories, 
we could safety assume that the memory can be divided into 
multiple disjoint memories that are owned by the designated threads in the system. 
Based on the assumption, 
we have built algebraic memory model, the extended version of memory model by adding two key relations from $\compcertx$'s memory model.
Those two key relations are 
\begin{lstlisting}[language=Caml]
      mem_disjoint_union : mem -> mem -> mem -> Prop;
      mem_lift_nextblock : mem -> nat -> mem;
\end{lstlisting}
The memory in $\compcertx$ is 
distinguished by its block identifier and a offset for each block. 
The type of block identifier is a natural number,
and we restricts that each thread has distinguished blcok  
The first \lstinline$mem_disjoint_union m1 m2 m3$  shows that 
the $m3$ is the union of $m1$ and $m2$ when $m1$ and $m2$ 
are disjoint from each other. 

The second, \lstinline$mem_lift_nextblock$ 
provides the operation that makes 
the adjustment on the memory possible. 
\jieung{need figure!! Let's first write something and clean up and revise later}


\subsubsection{Thread Configuration}
\begin{lstlisting}[language=C]
Class ThreadsConfigurationOps := {      
    current_CPU_ID: Z; // The ID for the CPU we're looking at. (necessary?) 
    current_thread: Z; // The thread we're looking at in high-level machines
    dev_handling_cid : Z; // specific thread for handling device drivers 
    vm_handling_cid : Z; // specific thread for handling this CPU's virtual machine 
    non_current_thread_list: list Z;  // The other threads on this CPU 
    full_thread_list := current_thread :: non_current_thread_list; // The full list
    main_thread: Z; // The main thread on this CPU 
    limit: nat // A log look-back limit used in TAsm  }.
\end{lstlisting}

\begin{lstlisting}[language=C]
Class ThreadsConfigurationOps := {      
    current_CPU_ID: Z; // The ID for the CPU we're looking at. (necessary?) 
    current_thread: Z; // The thread we're looking at in high-level machines
    non_current_thread_list: list Z;  // The other threads on this CPU 
    full_thread_list := current_thread :: non_current_thread_list; // The full list
    main_thread: Z; // The main thread on this CPU 
    limit: nat // A log look-back limit used in TAsm  }.
\end{lstlisting}

Thread linking also requires multiple configurations about 
the environment of the system. 
For example, \lstinline$current_CPU_ID$ is the CPU ID that is 
associated with the current local layer.
Layer $\CSched$ is associated with \lstinline$current_CPU_ID$, 
and all threads associated with $\TLink$ and $\TSched$ layers 
are also run with  \lstinline$current_CPU_ID$.
The \lstinline$current_thread$ 
is a variable for the focused thread ID For the thread local layer. 
The layer above $\TSched$ is parameterized by the thread id (\lstinline$current_thread$). 
The full thread list is a set of thread identifiers 
which run on top of the current CPU ID. 
It is a union of the current thread identifier, a singleton set, 
and a non current threads. 
There are no restrictions on the size of the full thread set in here,
differently with the full CPU set in multicore linking configuration.
the main thread is a main thread (which initializes the system) of the CPU. 
This may differs from the current thread identifier or same with the current thread identifier.
the limit is for the fairness assumptions. 
It is necessary for the composition in our systems, 
because of all queries to extract the other threas' behavior via the associated environmental context 
should return in a finite number of events generated by the environmental context. 

Similar to the multicore linking case, 
shared resources are represented as a log, \lstinline$Definition Log := list LogEvent.$,  a list of event.
The event definition is 
\begin{lstlisting}[language=C]
// encode the argument in the log event
Inductive LogEventUnit :=
| LogYield (n: positive)
| LogSleep (i: Z) (n: positive) (syncch : option AuxStateDataType.SyncChanPool)
| LogPrim (id: ident) (args: list lval) (choice : Z) (dprocSnap : privDataSnap) 
| LogYieldBack.

Inductive LogEvent :=| LEvent : Z -> LogEventUnit -> LogEvent.
\end{lstlisting}
Our framework allows 
two scheduling primitives, 
yield and sleep. 
Those two primitives use their designated events to keep the information in the global log. 
Yield event does not need to contain many information in it, it just needs to give the control
to an another thread (among the therad in the same CPU). 
The sleep event, however, is usually related to the conditional variable and a shared resource among CPUs..
In this sense, we have to memorize the conditional variable that the sleep call uses ($i$),
and the snapshot of the shared resource, $AuxStateDataType.SyncChanPool)$.  \jieung{need to change the definition for general representation}
Other primitives that access the shared resources trigger the primitive event. 
They memories the primitive name, the argument for the primitive name. 
% The choice check is - it does not need to be.... 
The private data snapshot is a snapshot for the private data when the primitive has been invoked. 
They are not always necessary, but some primitives may be required to be memorized. 
For example, the acquire lock and the release lock primitives 
has to memorize the snapshot of the (abstract) private data when they have been invoked. 
The last event, \lstinline$LogYieldBack$ is the memorize the 
moment to figure out when each thread re-achieve the evaluation control. 
\jieung{Need to explain it more}

\begin{lstlisting}[language=C]
Class SingleOracleOps := {
  init_log: Log;
  Single_Oracle : Log -> LogEvent;
  init_nb : positive  }.
\end{lstlisting}
The environmental context \lstinline$Single_Oracle$ 
then can be defined as a function that gets the current log and returns the event. 
In addition to that, 
each thread has to have its initial log 
for the evaluation. 
For example,
when the thread 2 starts its evaluation after 
thread 1 and 3 has been spawned and scheduled. 
the thread 2 has to cotain all those information in its initial log.

\subsubsection{Auxiliary Functions}

Based on the configurations, we provide multiple auxiliary functions 
for thread local machines to calculate necessary information (Figure~\ref{fig:chapter:conlink:multithreaded-linking-aux-functions}.)
\begin{figure}
\begin{lstlisting}[language=C, morekeywords={Class}]
  Class SingleOracle  `{single_data: SingleData} `{single_oracle_ops : SingleOracleOps}  
        `{threads_conf_ops : ThreadsConfigurationOps} := {
      update : dshare -> Log -> dshare;
      has_event: ident -> bool;      
      uRData (l: Log) : dshare := update (init_dshare) l;      
      choice_check: ident -> list lval -> dshare -> dproc -> Z;
      sync_chpool_check: ident -> list lval -> dshare -> dproc ->
        option SyncChanPool;
      snap_func : dproc -> privDataSnap;
      snap_rev_func : privDataSnap -> dproc; 
      thread_init_dproc : Z-> dproc;
      state_checks (name : ident) (largs: list lval) (l : Log) 
        (pdp: ZMap.t (option dproc)) :=
        match ZMap.get (proc_id (uRData l)) pdp with 
          | Some pd => state_check name largs l pd
          | _ => True
        end;
      PHThread2TCompose : dshare -> dproc -> RData;      
      thread_init_rdata (tid: Z) :=
        PHThread2TCompose (uRData init_log) (thread_init_dproc tid);
      prim_thread_init_pc {F V} (ge: Genv.t F V): ident -> list lval -> option val }.
\end{lstlisting}
%\begin{lstlisting}[language=C]
%  Class SingleOracle  `{single_data: SingleData} `{single_oracle_ops : SingleOracleOps}  
%        `{threads_conf_ops : ThreadsConfigurationOps} := {
%      update : dshare -> Log -> dshare;
%      has_event: ident -> bool;      
%      uRData (l: Log) : dshare := update (init_dshare) l;      
%      choice_check: ident -> list lval -> dshare -> dproc -> Z;
%      sync_chpool_check: ident -> list lval -> dshare -> dproc ->
%        option SyncChanPool;
%      snap_func : dproc -> privDataSnap;
%      snap_rev_func : privDataSnap -> dproc; 
%      thread_init_dproc : Z-> dproc;
%      state_checks (name : ident) (largs: list lval) (l : Log) 
%        (pdp: ZMap.t (option dproc)) :=
%        match ZMap.get (proc_id (uRData l)) pdp with 
%          | Some pd => state_check name largs l pd
%          | _ => True
%        end;
%      PHThread2TCompose : option Kctxt -> dshare -> dproc -> RData;      
%      thread_init_rdata (tid: Z) :=
%        PHThread2TCompose
%          None
%          (uRData init_log)
%          (thread_init_dproc tid);
%      prim_thread_init_pc {F V} (ge: Genv.t F V): ident -> list lval -> option val }.
%\end{lstlisting}
\caption{Auxiliary Functions in Multithreaded Linking Framework}
\label{fig:chapter:conlink:multithreaded-linking-aux-functions}
\end{figure}
Among them,  \lstinline$update$ and \lstinline$uRData$ functions are  declarations of a replay function for the global log. 
The has event function is for check whether the primitive calls has to trigger the event or not. 
we also requires the snapshot functions for the events as we have seen in the above.
And two important abstract function declarations 
are \lstinline$thread_init_dproc$ and \lstinline$prim_thread_init_pc$ functions 
that calculate the initial abstract data for the thread as well as the initial program counter value 
for the thread. 
\jieung{Need to remove some of them}
Later, when we want to link the concrete layer definitions using our multithreaded linking library, users need to define them.

\subsubsection{Auxiliary Properties}

%\begin{lstlisting}[language=C]
%Class ThreadsConfiguration `{s_oracle : SingleOracle}: Prop := {
%  CPU_ID_valid: forall l, processor_id (uRData l) = current_CPU_ID;
%  main_valid: In main_thread full_thread_list;
%  null_log_proc_id: proc_id (uRData nil) = main_thread;
%  main_thread_val: main_thread = current_CPU_ID + 1;
%  current_CPU_ID_range: 0 <= current_CPU_ID < Constant.TOTAL_CPU;
%  dev_handling_cid_constraint: dev_handling_cid = main_thread;
%  init_log_proc_id: proc_id (uRData init_log) = current_thread;
%  valid_thread_list: forall i, In i full_thread_list -> 
%    Constant.TOTAL_CPU < i < Constant.num_proc \/ i = main_thread;
%  all_cid_in_full_thread_list: forall l, In (proc_id (uRData l)) full_thread_list;
%  full_thread_list_prop: NoDup full_thread_list }. 
%\end{lstlisting}

\begin{lstlisting}[language=C, morekeywords={Class}]
Class ThreadsConfiguration `{s_oracle : SingleOracle}: Prop := {
  CPU_ID_valid: forall l, processor_id (uRData l) = current_CPU_ID;
  main_valid: In main_thread full_thread_list;
  null_log_proc_id: proc_id (uRData nil) = main_thread;
  main_thread_val: main_thread = current_CPU_ID + 1;
  current_CPU_ID_range: 0 <= current_CPU_ID < Constant.TOTAL_CPU;
  init_log_proc_id: proc_id (uRData init_log) = current_thread;
  valid_thread_list: forall i, In i full_thread_list -> 
    Constant.TOTAL_CPU < i < Constant.num_proc \/ i = main_thread;
  all_cid_in_full_thread_list: forall l, In (proc_id (uRData l)) full_thread_list;
  full_thread_list_prop: NoDup full_thread_list }. 
\end{lstlisting}


Those are the property about the auxiliary functions. 
They are usually for the minimum conditions of 
thread configurations. 
For example, 
the \lstinline$CPU_ID_valid$ tells that all the scheduling result of this CPU 
will not change the CPU ID. 
in addition, 
the main thread always have to be a member of a thread set. 
Those properties also have to be instantiated later with the concrete definitions.

%\begin{lstlisting}
%Record high_level_invariant (abd: RData) := 
%\end{lstlisting}
%
%\begin{lstlisting}
%
%  Class SingleOracleProp `{thread_conf_prf: ThreadsConfiguration}:=
%    {
%
%      (* assumptions about update functions - proc_id and processor_id preserve property *)
%      update_proc_id:
%        forall d d' d'' l a,
%          update d l =  d' ->
%          getLogEventType a = LogOtherTy ->
%          update d (a :: l) =  d'' ->
%          proc_id d'' = proc_id d';
%
%      update_processor_id:
%        forall d d' d'' l a,
%          update d l = d' ->
%          update d (a::l) = d'' ->
%          processor_id d'' = processor_id d';
%
%    (** This checks whether the given primitive call initializes any
%        thread. If so, returns the thread's id and initial register state. *)
%
%      init_data_low_level_invariant n:
%        low_level_invariant n (thread_init_rdata current_thread);
%
%      init_data_phthread_high_level_invariant:
%        PHThreadInvariant.high_level_invariant (thread_init_rdata current_thread);
%       
%      prim_thread_init_pc_global:
%        forall {F V} (ge: Genv.t F V) f vargs v,
%          prim_thread_init_pc ge f vargs = Some v ->
%          (exists id ofs, symbol_offset ge id ofs = v);
%
%      prim_thread_init_pc_symbols_preserved:
%        forall {F V} (ge1 ge2: Genv.t F V) f vargs v,
%          (forall i, Genv.find_symbol ge1 i = Genv.find_symbol ge2 i) ->
%          prim_thread_init_pc ge1 f vargs = Some v ->
%          prim_thread_init_pc ge2 f vargs = Some v
%    }.
%\end{lstlisting}
\begin{lstlisting}[language=C, , morekeywords={Class}]
Class SingleOracleP `{thread_conf_prf: ThreadsConfiguration}:= {
  update_proc_id:
    forall d1 d2 d3 l a, update d1 l =  d2 ->
      getLogEventType a = LogOtherTy ->
      update d1 (a :: l) =  d3 -> proc_id d3 = proc_id d2;
  update_processor_id:
    forall d1 d2 d3 l a, update d1 l = d2 -> 
      update d1 (a::l) = d3 -> processor_id d2 = processor_id d3;
       
  prim_thread_init_pc_global:
    forall {F V} (ge: Genv.t F V) f vargs v,
      prim_thread_init_pc ge f vargs = Some v ->
      (exists id ofs, symbol_offset ge id ofs = v);

  prim_thread_init_pc_symbols_preserved:
    forall {F V} (ge1 ge2: Genv.t F V) f vargs v,
      (forall i, Genv.find_symbol ge1 i = Genv.find_symbol ge2 i) ->
      prim_thread_init_pc ge1 f vargs = Some v ->
      prim_thread_init_pc ge2 f vargs = Some v }.
\end{lstlisting}
We also defined multiple properties related to 
update functions, which we defined in the
\lstinline$SingleOracleP$ contains properties about auxiliary functions.
The  first is about the process ID update. 
It tells that the process ID will not be updated unless we evaluate 
scheduling events, yield or sleep events. 
The second one is about the CPU ID, which implies that 
the CPU ID is always remained as same. 
Remeber that the log in here is a log
for multithreaded linking, which implies that 
one event in here contains the information for multiple events in the previous definitions. 
Other two are for the properties of initial program counter value for each 
process.
We support a compositional compilation, and thus 
we do not have the full information of the context programs.
We however, requires the condition about the program counter of each threads' initial state,
becauseo of the safety condition as well as its uniqueness.


\begin{lstlisting}[language=C, , morekeywords={Class}]
Class SingleOracleLinkP `{thread_conf_prf: ThreadsConfiguration}:= {
  Oracle_le_nb: 
  forall l, (last_nb l <= last_nb (Single_Oracle l :: l)) % positive;

  oracle_event_yield_back:
    forall l, lastEvType l = Some LogYieldTy ->
      exists i, Single_Oracle l = LEvent i LogYieldBack;

  oracle_init_log:
    forall (l_p: Log) (e: LogEvent) (l: Log),
      init_log = l_p ++ e :: l -> Single_Oracle l = e;
      
  init_log_property:
    init_log = nil \/
    exists l e, init_log = LEvent (proc_id (uRData l)) LogYieldBack :: l /\
      proc_id (uRData l) <> proc_id (uRData nil) /\ last_op l = Some e /\
      getLogEventNB e <> None /\
      forall e l1 l2, l = l1 ++ e :: l2 ->
        proc_id (uRData l2) <> 
        proc_id (uRData init_log) }.
\end{lstlisting}

We also requires the restrictions about the form of the initial log (and the log) 
and \lstinline$SingleOracleLinkP$  contains the constraints
the properties about the log that 
the log has to  satisfy them. 
\jieung{I think exposing all those functions in here is not a good idea. I need to rearrange them.
we can defer the explanation of some of them}
