 We propose $\ccalname$ in Chapter~\ref{chapter:ccal} and
provide an example to demonstrate how it can be used for concurrent program verification--the MCS Lock verification--in Chapter~\ref{chapter:mcs-lock}.
However, previous chapters do not mention the final piece of concurrent program verification, 
which involves combining the verified multiple proof instances of the program to form the claim that the full concurrent program guarantees correctness. Notably, this is a parallel composition in concurrent program verification. 
Section~\ref{chapter:ccal:subsec:linking}
informally mentions how a formal parallel composition rule can be provided to more than two disjoint-focused sets of program instances in concurrent programs.
However, the chapter does not provide any formal rules or theorems for this parallel composition. 
Thus, the present chapter focuses on issues related to providing compositions in concurrent program verification for both multicore and multithreaded cases.


This linking process has a considerable cost for multiple reasons. 
First, concurrent linking requires machine syntax and semantics for the entire program as well as for partial concurrent programs 
(or a machine that can run both). 
Our tool, $\compcertx$, is designed as a powerful tool to build abstraction layers with single-threaded (or sequential-like) programs written in both C and assembly by using the variance of the verified compiler $\compcert$ as well as its linking library (described in Section~\ref{chapter:ccal:subsec:linking}). However, $\compcertx$ only includes its own local state (\ie, per-CPU or per-thread) as its machine state and achieves interleaved behaviors of concurrent programs via environmental contexts that contain the abstracted view of other system components. 
This isolation also has advantages in terms of re-using the verified compiler and efficiently building concurrent programs without directly handling complex interactions; however, $\compcertx$ is not suitable for parallel composition. 
In this sense, providing a parallel composition requires us to employ a machine model for full participants of the concurrent program (i.e., a multicore machine model and/or a multithreaded machine model) that has a close connection with our machine model for  local layer interfaces, $\compcertx$.

Second, the linking proof must be as generic as possible due to its high cost and complexity.
Some portions of the linking proofs need to depend on the program implementation (such as a softwarescheduler implementation or a process creation), its logical environment (such as the number of threads), or a physical environment (such as the number of CPUs.) 
Notably, these dependencies should be minimized for reusability.

Third, the linking proof must handle not only interactions between shared resources but also a side effect to local states via shared-resource interactions.
Especially in a system that allows dynamic resource allocation, a thread spawn usually allocates maximum resources for a newly created process
 (which is typically determined by its parent process) and sets an initial program counter for it. 
 Instead of allowing local state updates from other threads, which raises the contradiction of the local state definition, 
 our linking framework requires a method of calculating the proper side effect on each thread’s local state from shared-data updates when necessary. 
 During process creation, each thread requires the framework to set its proper dynamically allocated values when it first starts its evaluation, which can occur much later than its spawning time point.

We overcome these issues by introducing multiple environment configurations, auxiliary functions, abstract intermediate languages, and proofs between those languages. 
We also apply them to the linking techniques for the existing verified concurrent operating systems. 
Notably, no previous works have exposed these challenges or offered empirical studies on linking for concurrent program verification--a point discussed later in this chapter.

%
%
% informally mentions how to provide a formal parallel-composition rule to more than two disjoint-focused sets of program instances in concurrent programs.
%However, the chapter does not provide any formal rules or theorems for this parallel composition in that chapter.
%%It, however, overlooked and missed multiple challenges in the composition when it is related to real programs,
%% which need to deal with details of programs, 
%%such as a complex interaction between private state and shared state in the actual system
%%and connecting the proof composition with the appropriate existing and efficient tool for verification (\ie, a $\compcert$ compiler to use C language for program verification).
%Thus, this chapter focuses on  issues related to providing compositions in concurrent program verification,
%both for multicore and multithreaded cases.


%%when compared to building layers for local reasoning (CPU-local layers or thread-local layers)
%This linking process has a considerable cost for  multiple reasons.
%First, concurrent linking requires machine syntax and semantics 
%for the whole program as well as for partial concurrent programs (or a machine that can run both).
%% the source language of our verified programs.
%%To help the development of the system software, 
%%the  tool that can support both high-level languages (C) as well as low-level ones (Assembly) and 
%%can compose the programs written by them is necessary.
%Our tool, $\compcertx$, is designed as a 
%%The previous work, $\compcertx$, the extension of $\compcert$
%powerful tool to build abstraction layers with single-threaded (or sequential-like) programs
%written in both C and assembly by using the variance of the verified compiler $\compcert$ as well as its linking library described in Section~\ref{chapter:ccal:subsec:linking}. 
%However, $\compcertx$ only includes its own local state (\ie, per-CPU or per-thread) as its machine state and achieves interleaved behaviors of concurrent programs via environmental contexts which contain the abstracted view of of other components in the system.
%This isolation also has advantages in terms of re-using the verified compiler and efficiently  building  concurrent programs without directly handling complex interactions, but $\compcertx$ is not 
%suitable for parallel composition.
%In this sense, providing a parallel  composition requires us to 
% employ a machine model for full participants of the concurrent program  (\ie, a multicore machine model and/or a multithreaded machine model)
% that has a close connection with our machine model for local layer interface, $\compcertx$.
%%
%
%Providing those machine models also has to closely related to the 
%source language that we want to use for our verified programs;
%thus it implies that
%the concurrent machine model has to be rich enough for us to connect it with our local layer interfaces (in Section~\ref{chapter:ccal:subsec:local-layer-interface})
%what we are usually working on the program verification.


%
%With this variance of $\compcert$,
%building the sequential like programs (and providing local layer interface) discussed in Section~\ref{chapter:ccal:subsec:local-layer-interface} 
%is possible in a scalable and practical way;
%but they are not for parallel compositions.

%In this sense, at each time when providing parallel proof composition, 
%we should give the machine model that runs multiple instances of programs (\ie, a multicore machine model and a multithreaded machine model.) 
%
%Second,  
%the linking proof has to be generic as much as possible due to its high cost and complexity. 
%Some parts of  linking proofs need to depend
%on the program implementation (such as a software-scheduler implementation or a process creation),
% its logical environment (such as a number of threads),
% or a physical environment (such as a number of CPUs.)
%Those dependencies should be minimized for reusability.


%Third, 
%the  linking proof 
%needs to handle not only interaction between shared resources but also a side effect to local states via shared-resource interactions. 
%Especially in a system that allows a dynamic resource allocation,
%a thread spawn usually allocates maximum resources for a newly created process (which is typically determined by its parent process) and sets an initial program counter for it.
%Instead of allowing local-state updates from other threads, which raise the contradiction of the local-state definition, 
%our linking framework needs a way to calculate the proper side effect on each thread's local state from shared-data updates when necessary. 
%During process creation, each thread requires the framework to set its proper dynamically allocated values when it first starts its evaluation,
%the time which can occur quite later than the time when it was spawned. 

%%It requires the case when a thread updates other threads' local state,
%%which raises the contradiction of the local state definition. 

%In some cases, those resources include a private state that does not share with other threads at all. 

%We formally define (sequential consistency) concurrent machine models.

%Therefore, we formally define (sequential consistency) multicore machine model that is reusable for multiple machines (such as ARM and $\intelmachine$),  
%as well as a multithreaded machine model with a capability of using software schedulers. 


%First, providing the refinement between the whole program (with multicore/multithreaded machine models) means 
%that we actually need to define those machine models. 
%Second, we also need to hide the nondeterminism as soon as possible 
%to facilitate the forward-to-backward simulation techniques 
%in $\compcert$ in our 
%multicore/multithreaded machine models,
%which provides the tool for us to connect all programs runs on the all layers (or other machines) together. 
%Third, 
%optimization in the global log or environmental context query are necessary to provide the 
%them. 
%we have to introduce multiple intermediate languages, 
%prove refinement between those languages, 
%and connect those languages with the assembly machine model that can be used with CompCert. 


%
%
%We overcome those issues by introducing multiple environment configurations, auxiliary functions, abstract intermediate languages,
%and proofs between those languages.
%We also apply them to the linking techniques for the existing verified concurrent operating systems. 
%None of the previous works expose those challenges or offer empirical studies on linking for the concurrent program verification,
% we discuss in this chapter.
%%This chapter is not aiming to introduce a new theory, 
%%but provide a valuable empirical study that 
%about the composition of large-scale concurrent program verification that many readers are interested in it.


\section{Overview}
\label{chapter:conlink:sec:overview}

% do we need to add it for the paper version?
%\begin{figure}
%\caption{Example of Linking}
%\label{fig:conlink:overview-example}
%\end{figure}


% concurrent layer  - the proof
$\ccalname$
has proven to be a useful and efficient tool for building and verifying concurrent programs with the layered structure, as discussed in Chapters~\ref{chapter:ccal} and~\ref{chapter:mcs-lock}.
For example, our $\mcsname$ Lock verification provides the following theorem:
\begin{center}
$\ltyp{\PLayer{\mmcsbootfull}{cid}{\oracle_{mcs}}}{R_{mcs}}{\codeinmath{M}_{\codeinmath{mcslock}}}{\PLayer{L_{\mhmcslockopfull}}{cid}{\oracle^{lk}_{cid}}}$,
\end{center}
which means that ``any context program plus the implementation of MCS Lock ($\codeinmath{M}_{\codeinmath{mcslock}}$) runs on top of 
 a  layer $\mmcsbootfull$ with a CPU ID $cid$ and an environmental context $\oracle^{mcs}_{cid}$ \textit{contextually refines}
 the same context program runs on the of a  layer $\mhmcslockopfull$ with a CPU ID $cid$ and an environmental context $\oracle^{lk}_{cid}$.''
However, the proof does not consider a program that runs with all CPUs in the system, 
so it is a partial-correctness statement. 
To guarantee the full correctness of our program, we must provide a judgment that connects program runs on the full concurrent machine with the program on this local layer interface. In the case of our $\mcsname$ Lock operation, the judgment is as follows:
\begin{definition}[Multicore Linking for MCS Lock]
\label{definition:conlink:con-linking-for-mcs-lock}
For any program $ \codeinmath{P}$, our MCS Lock implementation $\codeinmath{M}_{\codeinmath{mcslock}}$, a full-core set on the system ($\coreset$), and a valid environmental context for CPU  $cid$ in $\coreset$ ($\oracle^{\codeinmath{mcs}}_{cid}$), the concurrent linking for MCS Lock is 
\begin{center}
$\forall \codeinmath{P}. \ \semwmachine{\mmcsbootfull}{\codeinmath{M}_{\codeinmath{mcslock}} \oplus \codeinmath{P}}{\hardwarestepkwd} \refines_{R_{comp}} \sum_{i \in D}\  (\semwmachine{\PLayer{\mmcsbootfull}{i}{\oracle^{\codeinmath{mcs}}_cid}}{\codeinmath{M}_{\codeinmath{mcslock}} \oplus \codeinmath{P}}{\lasmmach})$
\end{center}
where  $\hardwarestepkwd$ is a multicore machine model, and 
$\lasmmach$ is a machine model for $\compcertx$.
\end{definition}

Definition~\ref{definition:conlink:con-linking-for-mcs-lock} 
provides evidence regarding the formal connection between a program runs with a single local machine ($\compcertx$) in a proper environmental context and a program runs on top of a multicore machine model $\hardwarestepkwd$ with all CPUs. 
Generalizing the theorem is possible with any arbitrary layers, even any concurrent and local machine models with satisfying desired properties. This judgement is provided as follows:
\begin{definition}[Concurrent Linking]
\label{definition:conlink:concurrent-linking}
For any program $ \codeinmath{P}$, a full focused set for the system ($D$), a valid environmental context for the layer with the singleton focused set in $D$ ($\oracle^{\codeinmath{local}}_{i}$), and a layer $L_{\codeinmath{full}}$ and $L_{\codeinmath{local}}$ related with a certain 
refinement relation $R_{comp}$, we define the concurrent linking as follows:
\begin{center}
$\forall \codeinmath{P}. \ \semwmachine{L_{\codeinmath{full}}[D]}{\codeinmath{P}}{\codeinmath{mach}_{\codeinmath{full}}} \refines_{R_{link}} \semwmachine{\PLayer{L_{\codeinmath{local}}}{i}{\oracle^{\codeinmath{local}}_{i}}}{\codeinmath{P}}{\codeinmath{mach}_{\codeinmath{local}}}$
\end{center}
,where $\codeinmath{mach}_{\codeinmath{local}}$ is a machine model that runs the layer with a singleton focused set (such as a machine model of $\compcertx$).
\end{definition}
{\noindent}
In addition, proofs regarding the compositionality of our verification are highly desirable to demonstrate that multiple instances within it rely on consistent assumptions that can be ruled out via other instances in the same system. Formally, it is presented as follows:
\begin{definition}[Parallel Composition]
\label{definition:conlink:parallel-composition}
For any program $ \codeinmath{P}$, a full focused set for the system ($D$), a valid environmental context for the layer with the singleton focused set in $D$ ($\oracle^{\codeinmath{local}}_{i}$), and a layer $L_{\codeinmath{full}}$ and $L_{\codeinmath{local}}$ related to a certain 
refinement relation $R_{comp}$, we define the parallel composition as follows:
\begin{center}
$\forall \codeinmath{P}. \ \semwmachine{L_{\codeinmath{full}}[D]}{\codeinmath{P}}{\machkwd_{\codeinmath{full}}} \refines_{R_{comp}} \sum_{i \in D}\  (\semwmachine{\PLayer{L_{\codeinmath{local}}}{i}{\oracle^{\codeinmath{local}}_i}}{\codeinmath{P}}{\machkwd_{\codeinmath{local}}})$
\end{center}
where  $\codeinmath{mach}_{\codeinmath{full}}$ is a machine model that runs the layer with the full focused set (without any environmental contexts) and
$\codeinmath{mach}_{\codeinmath{local}}$ is a machine model that runs the layer with a singleton focused set (such as a machine model of $\compcertx$).
\end{definition}


\begin{figure}
\jieung{Need to revise those figures. If I will add the linking with concrete layers in the next section, I need to simplify those figures}
\begin{center}
\includegraphics[width=0.9\textwidth, page=1]{figs/conlink/overview}\newline
(a) Multicore Linking Structure
\end{center}
\begin{center}
\includegraphics[width=0.9\textwidth, page=2]{figs/conlink/overview}\newline
(b) Multithreaded Linking Structure
\end{center}
\textbf{Note:} We discuss the components in the grey boxes in the next chapter, but we include them in this figure to promote improved understanding.
\caption{Concurrent Linking Structure.}
\label{fig:chapter:conlink:concurrent-linking-structure}
\end{figure}

%\begin{mathpar}
%\inferrule{L[i] }{\jieung{Need to add composition rule}\\
%\jieung{the PCOMP rule in the ccal paper is not supported at all in our framework.}}
%\end{mathpar}
%\begin{mathpar}
%\inferrule{
%\ltyp{\PLayer{L_1}{A}{\oracle_A}}{R}{M}{\PLayer{L_2}{A}{\oracle_A'}} \\
%\ltyp{\PLayer{L_1}{B}{\oracle_B}}{R}{M}{\PLayer{L_2}{B}{\oracle_B'}
%}\\
%\Compose{\PLayer{L_1}{A}{\oracle_A}}{\PLayer{L_1}{B}{\oracle_B}}{\PLayer{L_1}{A \cup B}{\oracle_C}} \\
%\Compose{\PLayer{L_2}{A}{\oracle_A}}{\PLayer{L_2}{B}{\oracle_B}}{\PLayer{L_2}{A \cup B}{\oracle_C}}
%}{\ltyp{\PLayer{L_1}{A \cup B}{\oracle_C}}{R}{M}{\PLayer{L_2}{A \cup B}{\oracle_C'}}}
%\end{mathpar}
These proofs are based on the compositionality and consistency of multiple strategies in partial programs (discussed in Section~\ref{chapter:ccal:sec:ccal-overview}).
This notion appears simple as a high-level idea, but it is not simple in reality due to the aforementioned challenges. 
To connect the high-level idea with the real concurrent linking framework that addresses multiple challenges, 
we introduce multiple configurations, auxiliary functions, and intermediate languages for linking. In the case of multicore linking 
(Figure~\ref{fig:chapter:conlink:concurrent-linking-structure} (a)),
our purpose is to use the framework not only for one single-concurrent programs run on a specific machine model, 
but also for other programs with additional hardware models that satisfy the generic hardware setting properties. 
Based on the abstract hardware setting, we built multiple layers of languages to refine a nondeterministic multicore machine model into  local layer interfaces that can use the $\compcertx$ compiler as well as hide the behavior of other instances in other CPUs. 
Specifically, it requires us to introduce partial machines with an environmental context that can run a program with a focused CPU set and introduce a per-CPU machine that only runs one CPU on it (but differs from $\compcertx$’s machine). 
We must also prove the refinement proofs between all those intermediate machine models that can be injected to $\compcert$'s backward-simulation theorems to guarantee the connection between this concurrent linking framework and $\ccalname$ local layers via contextual refinement proofs. 
Thereafter, the multicore linking template can be used for any layers alongside a sequential machine model in $\compcertx$ when they satisfy certain properties related to the abstract hardware setting.



\jieung{need to revise the following part. It's quite ambiguous now}
Most claims in multicore linking (Figure~\ref{fig:chapter:conlink:concurrent-linking-structure} (b)) 
remain nearly identical. However, claims are usually located in the middle of the system software stack, which already contains multiple primitives in the layer. 
This requires us to handle some software and machine-dependent properties when implementing the linking framework as well as proving linking theorems. It also requires us to handle the mutual exclusiveness of private states, including memories, register values, and private fields in abstract data (such as page table allocations). 
In addition, it also requires us to assign proper initial states for all threads in the system if we want to prove that the composition of multithreaded linking has the same behavior as the multithreaded program runs on the 
per-CPU layer. 
To handle these, we also introduce multiple intermediate machine models that can parameterize abstract layer as well as thread configurations, 
and we use various auxiliary functions and auxiliary properties related to the abstract layer (and the top-most layer in the per-CPU machine as well as the bottom-most layer in the per-thread  machine). 
With the linking framework provided, users must fill out the layer definition, configurations, and generic properties to link the layer for each thread and CPU.


In summation, we provide a generic linking framework that minimizes user effort as much as possible. This chapter focuses on the linking framework, and we discuss the applicability of this framework by adapting it to large-scale concurrent program verification--a concurrent operating system--in the next chapter.



%\ignore{ 
%\section{CCAL Review}
%\jieung{This is only for papers}
%
%CCAL, the extention of is an extension of the  Certified Abstraction Layer proposed by~\cite{deepspec}, 
%provides a state transition machine for certified concurrent program.
%With the variance of $\compcert$, that gets the layer as its parameter, 
%programs can directly run on the state transition machines and can be formaly connect to the 
%other layers 
%with layer calculus that the tool provides. 
%
%It also supports 
%compositional reasoning for a single instance of 
%It is a collection of layers, and each layer 
%is a abstract machine that can run C and assembly programs on it. 
%
%Key definitions of CCAL is defined in Fig.~\ref{fig:key-definitions-of-ccal}. 
%Each layer is is a set of 
%
%
%The definition of a  CCAL over the instance identifier $i$ (\textit{i.e.} CPU ID or thread ID)
%is a tuple with four elements, notated as $L[i,\oracle] := (\Layer, \Rely, \Guard)$.
%The first component $\Layer$ contains a partial map from
%identifiers to transition specifications
%($\Layer := \set{\primid \mapsto \primspec{\primid}}$ where $ \primspec{\primid}$ is
%the specification of the primitive $\primid$).
%The state of the layer can be interpreted as a pair
%of the private abstract state for the instance $i$ ($lst_i$), and
%a log of events ($l$) that represents the history of the atomic operations ($st_i := (lst_i, l)$).
%The local state consists of multiple machine-dependent concrete definitions such as register and memory values (,
%as well as abstract objects that correspond to
%certain regions of memory through some relation.
%For specifications of primitives that only touch local state, the transition is straightforward.
%Specifications of primitives that contain network transition primitives, on the other hand,
%must use the environment context for the instance $i$ ($\oracle_i$)
%to model the behavior of other instances.
%For example, the specification of a function that broadcasts a message from a proposer to a set of acceptors must query the environment context
%between each send to learn how the environment has changed.
%The other two components of the layer definition, $\Rely$ and $\Guard$,
%provide invariants about the network and the distributed system
%following the approach of previous work on rely/guarantee systems~\cite{RGSim, LRG}.
%The invariants in $\Rely$ and $\Guard$ are complementary to each other.
%Each layer must contain evidence that all of the local transitions satisfy the conditions in $\Rely$.
%Conversely, we can restrict the behavior of other instances by using assuming the conditions in $\Rely$ hold.
%One example of a rely/guarantee rule in Paxos concerns the relation between the round number and value in a Phase 1b message.
%It is true that the value is $\bot$ if and only if the round number is $\bot$.
%To prove that this invariant always holds it is enough to show that
%for layer $L'[i]$, every transition of instance $i$ will satisfy it.
%Having satisfied the guarantee, we can rely on the fact that the invariant will hold for all other single instances in the system.
%
%
%
%$/ccalname$ also provides a way to build a layer on top of another using a program module $M$, which consists of code written in C or assembly.
%The predicate $L'[i] \vdash_R  M : L[i]$ indicates that the layer implementation $M$, built on top of the interface $L'[i]$, rigorously implements
%the layer $L[i]$ with the two layers related by $R$.
%CCAL can compile these C modules using the CompCertX certified compiler~\cite{deepspec},
%which is a modified version of CompCert~\cite{compcert}.
%This, combined with the {\em contextual} correctness property,
%lets us define contextual refinement over abstraction layers with the ability to compile layers.
%A certified layer converts any {\em safe} client program $P$ running on top of $L'[i]$ into one that has the
%same behavior but runs on top of $L[i]$ by compiling the abstract
%primitives in $L[i]$ into their implementation in $M$.
%If we use ``$\sem{L[i]}{\cdot}$'' to denote the behavior of the layer machine based on
%$L[i]$, the correctness property of ``$\ltyp{L'[i, \oracle']}{R}{M}{L[i, \oracle]}$'' is written
%formally as ``$\forall{}P.\sem{L'[i, \oracle']}{P\oplus{}M} \refines_R \sem{L[i, \oracle]}{P}$''
%where $\oplus$ denotes a linking operator over programs $P$ and $M$ and 
%the relation ($\refines_R$) is formally defined as a forward
%simulation~\cite{Lynch95,leroy09,Milner71,Park81} with the (simulation) relation $R$.
%
%$\ccalname$ compiles these modules with the $\compcertx$ certified compiler~\cite{deepspec},
%which is a modified version of $\compcert$~\cite{deepspec, compcert}.
%The {\em implements} relation ($\refines_R$) is formally defined as a forward
%simulation~\cite{Lynch95,leroy09,Milner71,Park81} with the (simulation) relation $R$.
%This guarantees that if $L'[i] \vdash_R  M : L[i]$ holds,
%the behaviors allowed by layer $L[i]$ simulate those allowed by $L'[i]$.
%
%Certified layers enforce a {\em contextual} correctness property as well.
%A certified layer converts any {\em safe} client program $P$ running on top of $L'[i]$ into one that has the
%same behavior but runs on top of $L[i]$ by compiling the abstract
%primitives in $L[i]$ into their implementation in $M$.
%If we use ``$\sem{L[i]}{\cdot}$'' to denote the behavior of the layer machine based on
%$L[i]$, the correctness property of ``$\ltyp{L'[i]}{R}{M}{L[i]}$'' is written
%formally as ``$\forall{}P.\sem{L'[i]}{P\oplus{}M} \refines_R \sem{L[i]}{P}$''
%where $\oplus$ denotes a linking operator over programs $P$ and $M$.
%
%The implements relation also applies to the environment context and the network.
%Formally,\\
%$\begin{array}{l}
%\forall \varphi_l \in \oracle'_i, \exists \varphi_h,  \varphi_h \in \oracle \wedge R_{\oracle', \oracle}(\varphi_l , \varphi_h) \\
%\hfill (\mbox{where} \ L'[i] = (\_,  \oracle'_i, \_, \_) \ \mbox{and} \ L[i] = (\_,  \oracle, \_, \_))
%\end{array}
%$\\
%This allows us to simplify our view of the possible network behaviors by showing that certain reductions refine the interleaved pattern.
%}
