\section{Multithreaded Linking}
\label{chapter:linking:sec:multithreaded-linking}


The purpose of building a framework for multithreaded linking is similar to  that of our multicore part,
provide the thread-local interface by hiding other threads' behavior. 
It, however, differs from our multicore linking framework, of which the  context and the configuration 
are simpler than those of multithreaded linking part due to multiple reasons.
First, when looking at the multithreaded linking, 
systems already have to provide 
scheduling related primitives (\ie, yield, sleep, create)
as well as other system services 
that are necessary to implement those primitives. 
For example, 
thread (process) creation usually related to the 
resource management,
such as designating the maximum capability of the memory usage 
for the newly created thread. 
Also, 
local memory areas for multiple threads should be 
mutually exclusive from others even though 
they are not distinguished from others before the system introduces the notion of threads.
In addition to that, 
several parts of the thread's private state,
which are hidden from other threads,
have to be initialized appropriately when the thread starts its evaluation.
The problem with this initialization is other threads cannot modify the thread's private state even though these initializations are dynamically triggered by
other threads, usually the thread's parent. 

Multithreaded linking is also related to 
multiple challenges based on the environment of the system.
We assume a simple but generic environment. 
Our system contains a full thread set per one CPU,
which contains arbitrary active or available threads in it.
Those threads are associated with their identifiers, and they do not need to be a sequence of consecutive numbers.
For example, when the range of
the thread identifier in the system is from 0 to 1023, 
then, CPU 1 may be able to run threads that have thread IDs divided by 8 (8, 16, \etc). 
%Each thread associated with the 
%designated thread ID. 
It is possible for threads to use the same identifier again, but we avoid this re-usage of the same number because it increases complexity without any remarkable benefits. 
We, however, assume that the thread set contains non-consecutive unbounded numbers,
which gives freedom of mapping any newly created thread with a new thread identifier. 
In addition to that, 
we allow dynamic initial states for each thread. 
For example, each thread has a maximum capacity of page numbers based on the argument that its parent 
provides.
Simpler systems may not allow this dynamic allocation for each thread,
but many systems assign an appropriate initial state for each thread 
according to the information when the thread has been spawned.

Using those assumptions on the environment,
our purpose of  multithreaded linking is providing a thread local layer interface that hides 
the local behavior of other threads in the system by resolving the following for 
main challenges:
(1) hide context switching between threads; 
(2) construct the environmental context for each thread and show the relationship between the CPU-local machine evaluation with each thread's evaluation based on the environmental context;
(3) prove compositionality of multiple per-thread machines; and 
(4) provide the same (almost similar) interface for per-thread layers 
for users to build layers on top of per-thread machines. 
Similar to the multicore linking, 
we propose a framework for multithreaded linking by introducing multiple intermediate languages to 
solve those challenges easily.



\subsection{Multithreaded Linking Overview}
\label{chapter:linking:subsec:multithreaded-linking-overview}



%\jieung{need to explain more about those challenge}
%\jieung{Expecially about context switching - page 20 of concurrent linking slide}
\begin{figure}
\begin{center}
\includegraphics[width=\textwidth, page=2]{figs/conlink/concurrent_linking}
%\jieung{Need to redraw this figure to add only one env-context for thread local layers}
\caption{Intermediate Languages and Their Relations in Multithreaded Linking Framework.}
\label{fig:chapter:linking:multithreaded-linking-structure}
\end{center}
\end{figure}

Figure~\ref{fig:chapter:linking:multithreaded-linking-structure} shows the overall structure of our multithreaded linking framework and the abstract definitions that the framework relies on them.
Numbers in the figure indicate the places that we deal with the challenges we have mentioned above.

\subsubsection{$\CSched$ layer with $\lasmmach$} 


The figure shows that the bottom level of multithreaded linking is a layer $\CSched$ associated with the focused CPU $cid$ and the relevant environmental context $\oracle_{cid}$, 
which uses the machine model $\lasmmach$, 
the machine model of $\compcertx$. 
The minimum restriction on the layer $\CSched$ is necessary, 
which is that the layer has to contain scheduling primitives to support software scheduler, yield and sleep.
Formally, 
when the layer $\CSched[cid, \oracle_{cid}] = (\Layer_{\CSched}, \ignorechar, \ignorechar)$,
we relies on the assumption about $\Layer_{\CSched}$, 
$(\threadyieldfunc, \ignorechar) \in \Layer_{\CSched}$ as well as 
$(\threadsleepfunc, \ignorechar) \in \Layer_{\CSched}$. 


%The bottom level of multithreaded linking, a cpu local layer has a similar state definition with $\compcert$. 
%They follows the definition of $\compcertx$, 
%which is an extension of $\compcert$ and is defined as
%\begin{lstlisting}[language=C]
%Inductive state `{memory_model_ops: Mem.MemoryModelOps mem}: Type :=
%  | State: regset -> (mem :mwd AbsData) -> state.
%\end{lstlisting}
The state definition of the bottom layer interface follows that of $\lasmmach$ is
\begin{lstlisting}[language=C]
Inductive state `{memory_model_ops: Mem.MemoryModelOps mem}: Type :=
  | State: regset -> (mem :mwd LATAOps) -> state.
\end{lstlisting}
that is a slightly extended state definition from $\compcert$ by modifying the memory 
into the memory that can be parameterized by an abstract data (\ie, $(mem :mwd LATAOps)$).
The abstract data also contains the global log in it, which shows the sequence of all operations on shared objects in the system. 
The purpose of using this state definition for per-CPU local layer interface is reusability of $\compcertx$, which could be used to verify sequential programs. 
With the underlying components of the state definition, 
it's easy for us to translate the definition into a simple form, $state_{\lasm} = (lst, l)$,
when the first element ($lst$) is a local state for the CPU, which contains 
the register values ($\regs$), a local copy of the memory ($mem$), as well as a private abstract data ($abs$),
and the second one is a global log for shared objects.
%With those definitions, the evaluation transition relation has a type
%% hide GE
%%\begin{lstlisting}[language=C]
%%  Inductive step (ge: genv): state -> trace -> state -> Prop :=
%%\end{lstlisting}
%\begin{lstlisting}[language=C]
%  Inductive step: state -> trace -> state -> Prop :=
%\end{lstlisting}
%which change the state by the transition with recording $\compcert$ style trace,
%which is 
%\begin{center}
%$\codeinmath{Mach}_{\lasm}: st_{\lasm} \rightarrow \cctracekwd \rightarrow st_{\lasm} \rightarrow \mcprop$
%\end{center}
%
The transition rule of the machine is then a relation on two states, the initial state to the result state, 
and each primitive in the layer defines (including $\CSched$) its transition by providing its specifications as follows:
\begin{center}
$\PLayer{\CSched}{cid}{\oracle_{cid}}(\Layer)(id)
 \vdash_{\lasm}  \sstepr{\spec_{id}}{args}{lst, l}{\textit{res}\cup \{\}}{lst', l'}$
\end{center}
{\noindent}when $\spec_{id}$ is a specification for primitive $id$,
$args$ is the list of arguments for the primitive call, and $\textit{res}\cup \{\}$ defines the 
return type of the primitive call.


Specifications for scheduling primitives in the layer also follows the same form.
For example, the key parts of the specification for the $\yield$ primitives is
\begin{mathpar}
\inferrule[Yield Rule]
{ lst = (\regs, m, adt)\\
lst' = (\regs', m', adt')\\
m = m' \\
get\_curid(l) = cid\\
l' = \listcons{\listapp{l}{(\oracle_{cid}\ cid\  l)}}{(cid \ YIELD)}\\
get\_curid(l') = cid'\\
adt.kctxt[curid'] = \regs'\\
adt' = adt/[adt.kctxt := adt.kctxt/[cid = \regs]]}
{\PLayer{\CSched}{cid}{\oracle_{cid}}(\Layer)(\yield)
 \vdash_{\lasm}  \sstepr{\spec_{\yield}}{\{\}}{lst, l}{\{\}}{lst',  l'}}
\end{mathpar}
which modifies the shared state by first querying the environmental context to update the global information 
regarding scheduling and adding an $\codeinmath{yield}$ event.
It, on the other hand, also modifies the local state of the CPU, 
which saves the old register values (kernel context - $\regs$ into the kernel context pool - 
$adt' = adt/[adt.kctxt := adt.kctxt/[cid = \regs]]$) 
and restores the kernel context ($\regs'$) for the next thread on the CPU that will start its evaluation.

\subsubsection{$\TLink$ layer with intermediate machines}

Introducing $\TLink$ with intermediate state machine models for multithreaded linking requires multiple steps. 
The first step is
dividing a single CPU state into a set of thread states as well as a shared state among threads.
Simply, the state of $\lasmmach$ ($state_{\lasm} = (lst, l)$) need to be divided into 
a set of thread states as well as a shared log, such as ($(\set{lst_{thread}}, l)$).
Since intermediate machine models in our multithreaded linking framework possibly use a different data type definition from that of a per-CPU machine, 
directly using the specification in  $\CSched$ is not possible. 
In this sense, 
we requires a layer definition $\TLink$ (\ie, $\PLayer{\TLink}{\ignorechar}{\ignorechar} = (\Layer_{\TLink}, \ignorechar, \ignorechar)$) which has the same domain on the primitive set with that of the $\CSched$ layer.
Primitive specifications in the $\TLink$ layer are similar to hardware local semantics in our multicore linking framework,
which defines how the primitive call modifies a thread's state that performs the evaluation,
a \textit{single} thread state and a log as follows:
\begin{center}
$\PLayer{\TLink}{\ignorechar}{\ignorechar}(\Layer)(id)
 \vdash_{\ignorechar}  \sstepr{\spec_{id}}{args}{lst_{thread}, l}{\textit{res}\cup \{\}}{lst_{thread}', l'}$
\end{center}
Scheduling primitives, however, cannot be specified in this manner because it involves changes on multiple thread local states. 
 In this sense, it models those rules directly on our intermediate machine models 
 instead of the layer definition.
 Formally, the relation on the primitive domain between two $\TLink$ and $\CSched$ layers 
 is in Definition~\ref{definition:chapter:conlink:same-domain-sched-tlink}
 Multithreaded linking with the newly defined state also 
requires machine models that facilitate the state as well as the layer definition ($\TLink$) with the state.

\begin{definition}[Primitive Domain in $\TLink$ and $\CSched$]
\label{definition:chapter:conlink:same-domain-sched-tlink}
\begin{tabular}{P{0.95\textwidth}}
$\forall id,\ id \neq \threadyieldfunc \wedge id \neq \threadsleepfunc \rightarrow 
 \ (id, \ignorechar) \in \Layer_{\CSched} \leftrightarrow (id, \ignorechar) \in  \Layer_{\TLink}.$\\
\end{tabular}
\end{definition}


\begin{figure}
\begin{tabular}{P{0.95\textwidth}}
\includegraphics[width=0.9\textwidth, page=3]{figs/conlink/concurrent_linking}\\
(a) Introduce Multithreaded Machine Model: divide local states\\
\\
\includegraphics[width=0.9\textwidth, page=4]{figs/conlink/concurrent_linking}\\
(b) Introduce Multithreaded Machine Model: introduce the environmental context\\
\\
\includegraphics[width=0.9\textwidth, page=5]{figs/conlink/concurrent_linking}\\
(c) Introduce Multithreaded Machine Model: add thread state ($\codeinmath{TS}$)
\end{tabular}
\caption{Introduce Multithreaded Machine Model: $\easmmach$}
\label{fig:chapter:conlink:multithreaded-machine-model-easm}
\end{figure}


Figure~\ref{fig:chapter:conlink:multithreaded-machine-model-easm} shows
the first step of our multithreaded linking by providing 
a $\easmmach$, a multithreaded machine model.
Figure~\ref{fig:chapter:conlink:multithreaded-machine-model-easm} (a)
shows a simple overview how we divide a single state for $\lasmmach$  ($\codeinmath{state}_{\lasm} := (lst_{\codeinmath{cpu}}, log)$)
into a state, ($\codeinmath{state}_{\easm[{\codeinmath{T}_{[1]}}]} := (tid, mem, \set{ti \mapsto lst_{ti}}, log)$),
that contains a global state for the CPU ($tid$, $mem$, and $log$) as well as 
multiple thread private states ($\set{ti \mapsto lst_{ti}}$).
A set of thread private states is a partial map from a thread id to its private states designed as a tuple that contains
a register set, as well as an abstract state that may have a different type of an abstract state in $state_{\lasm}$. 
We, however, do not explicitly divide a per-CPU memory into the multiple sets of memories since 
it adds the complexity of the machine definition as well as providing the 
refinement theorem between machine models.
Physically dividing the memory also elements the future possibility of our work, which provides fancy ownership resolutions
for the memory operations.
Instead of it, the framework provides the memory model 
that guarantees the mutual exclusion if we show the mutual exclusion of each thread's memory usage, which we name it an algebraic memory model (more details for this memory model is in Section~\ref{chapter:linking:subsec:multithreaded-env-configuration}.)
With this new state definition, 
we could match the evaluation of the program with $\CSched$ on $\lasmmach$ (Figure~\ref{fig:chapter:conlink:multithreaded-machine-model-easm} (a) (1)) with  the evaluation of the program with $\TLink$ on $\easmmach$
(Figure~\ref{fig:chapter:conlink:multithreaded-machine-model-easm} (a) (2)) 
by replacing the scheduling evaluation that performs context switching in the local state $\codeinmath{state}_{\lasmmach}$
into  the scheduling evaluation that simply changes the currently running thread ID and 
adds the proper log to memorize it in the state, $\codeinmath{state}_{\easmmach}$.
Building a per-thread machine model t for each thread 
is replacing the strategy of other threads with the proper environmental context as shown in Figure~\ref{fig:chapter:conlink:multithreaded-machine-model-easm} (b).
Thread linking, however, requires a little bit more fancy resolution to solve another challenge which differs from the case of multicore linking, building a dynamic initial state for each thread.
Figure~\ref{fig:chapter:conlink:multithreaded-machine-model-easm} (c) 
shows the case, which illustrates why we need to distinguish the thread state in a fine-grained way. 






\begin{figure}
\begin{center}
\includegraphics[width=0.9\textwidth, page=6]{figs/conlink/concurrent_linking}
\end{center}
\caption{Parallel Composition in Multithreaded Linking Framework}
\label{fig:chapter:conlink:parallel-composition-in-easm}
\end{figure}

With this model, 
providing the correctness of parallel composition is in Figure~\ref{fig:chapter:conlink:parallel-composition-in-easm},
which shows that 
It requires the match 


showing that the program evaluation on a per-CPU machine 









\begin{figure}
\begin{center}
\includegraphics[width=0.9\textwidth, page=7]{figs/conlink/concurrent_linking}
\end{center}
\caption{Introduce Single Threaded Machine Model.}
\label{fig:chapter:conlink:introduce-single-threaded-machine-model-with-iiasm}
\end{figure}



\begin{figure}
\begin{center}
\includegraphics[width=0.9\textwidth, page=8]{figs/conlink/concurrent_linking}
\end{center}
\caption{Introduce Single Threaded Machine Model with Environmental Context Query Optimization.}
\label{fig:chapter:conlink:introduce-single-threaded-machine-model-with-tasm}
\end{figure}



\subsubsection{$\TSched$ layer with $\hasmmach$} 



\begin{figure}
\begin{center}
\includegraphics[width=0.9\textwidth, page=9]{figs/conlink/concurrent_linking}
\end{center}
\caption{Per-thread Machine Model ($\hasmmach$).}
\label{fig:chapter:conlink:per-thread-machine-model}
\end{figure}

By following all those steps, 
we provide the machine model for per-thread layer interfaces
and  Figure~\ref{fig:chapter:conlink:per-thread-machine-model} shows 
the key feature of the machine model.
The machine state of the per-thread machine model $\codeinmath{state}_{\hasmmach}$ is identical
 to the machine state of the per-CPU machine model, $\codeinmath{state}_{\hasmmach}$.
This similarity gives us a huge benefit, the reusability of the compiler ($\compcertx$) 
that we have used in our per-CPU layer interfaces. 

The program in the bottom level of our per-thread local layer interfaces also relies on 
the abstract layer definition $\TSched$,
which contains the same domain on the primitive set with that of $\CSched$ (\ie, see Definition~\ref{definition:chapter:conlink:domain-csched-tsched}.)

\begin{definition}[Primitive Domain of $\CSched$ and $\TSched$]
\label{definition:chapter:conlink:domain-csched-tsched}
\begin{tabular}{P{0.95\textwidth}}
$\CSched[\_, \_] = (\Layer_{\CSched}, \_ , \_) \rightarrow \TSched[\_, \_] = (\Layer_{\TSched}, \_ , \_) \rightarrow$\\
$\forall id.\ (id, \_) \in \Layer_{\CSched}  \leftrightarrow (id, \_) \in \Layer_{\TSched} $\\
\end{tabular}
\end{definition}

With the stated constraint and the refinement theorems that we have provided,
thy $\TSched$  layer on $\hasmmach$ provides the same functionality with 
the $\CSched$  layer on $\lasmmach$ but hides
the transitions for other threads that run on the same CPU. 
For example, the evaluation rule for $\yield$ is
\begin{mathpar}
\inferrule[Yield Rule]
{l' = \listconsoppo{(tid\ YIELD)}{l}\\
l'' = \listappoppo{(\oracle_{\codeinmath{tid}}\ tid \ l')}{l'}\\
l''' = \listconsoppo{(tid\ YIELDBACK)}{l''}}
{\PLayer{\TSched}{tid}{\oracle_{tid}}(\Layer)(\yield)
 \vdash_{\hasm}  \sstepr{\spec_{\yield}}{args}{\regs, m, adt, l}{\textit{res}\cup \{\}}{\regs,  m, adt, l'''}}
\end{mathpar}
which implies that updating the shared log from $l$ to $l'''$ with using the appropriate environmental context for 
the thread, $\oracle_{\codeinmath{tid}}$.
The rule, however, modifies any parts of its private (local) state at all, 
which differs from the $\yield$ rule in the program with layer $\CSched$ on $\lasmmach$ (\ie, the rule performs context switching between the threads on the same CPU.)

\jieung{AbsRelT??}


%
%To hide them, 
%we introduced multiple intermediate languages 
%that has a transition rules of those primitives explicitly, and the 
%the layer that are used as an argument of 
%intermediate languages ($\TLink$) is introduced which does not contain those scheduling primitives in it. 
%
%These differences are also applied to 
%other primitives in $\CSched$ and $\TSched$.
%Some primitives in $\CSched$ accesses CPU-local data, but some of them 
%are not always easy to divide them clearly for each thread. 
%Pages in the memory, for example, 
%can be dynamicaly allocated, thus 
%each thread does not know exactly 
%which pages are associated with the thread statically
%(\ie, it is possible for us to assume that each thread statically 
%reserve the number of pages as well as 
%the actual page number, but this is far from usual design principal.)
%With the given concrete layer definitions for 
%all $\CSched$, $\TLink$, as well as $\TSched$,
%providing the detailed 
%way to divide the CPU-local private states into multiple thread local private states 
%may be possible. 
%We, however, divide this process 
%as two parts, the connection between machine models based on the configurations and abstract relations, 
%and introduce the instances of them to link the concrete CPU local machine with 
%the thread local machine. 
%To make it general as much as possible, 
%we have defined two abstract relations, 
%which is $AbsRelC$ and $AbsRelT$ in the figure with will work 
%with the arbitrary layers for each machines as well as the thread configuration $C$. 
%
%
%



%This section illustrates each step of this framework, by providing 
%the relevant $\coq$ definitions with highlighting the necessary parts.
%Since the all our definitions are already highly related to the definitions in $\compcertx$, 
%providing the detailed formal rule for the whole intermediate machines is obsolete,
%which differs from machine models in our multicore linking framework. 



%
%This model is corresponding to the machine in Figure~\ref{fig:chapter:conlink:threadlinking} (1). 
%The machine only contains only one register set and one private abstract data in its state.
%The layer definitely captures the execution of the whole thread set of CPU $c$ 
%and does not support thread-local reasoning.



%
%The top-most component in our 
%
%also has to include those two primitives. 
%Formally, 
%when the layer $\TSched[tid, \oracle_{tid}] = (\Layer_{\TSched}, \ignorechar, \ignorechar)$, which is parameterized by the focused thread identifier ($tid$) and the proper environmental context ($\oracle_{tid}$), 
%$\Layer_{\TSched}$ has to satisfy
%$(\threadyieldfunc, \ignorechar) \in \Layer_{\TSched}$ as well as 
%$(\threadsleepfunc, \ignorechar) \in \Layer_{\TSched}$. 
%However one of purpose of our framework is building a different behavior for those 
%primitives in those two layers, which gives 
%the no-op like scheduling primitives for the top level layer with this thread linking framework,
%$\TSched[tid, \oracle_{tid}]$, by hiding other threads' state changes in the CPU while 
%those transitions manipulate the associated values for other threads in the CPU private state
%in $\CSched[cid, \oracle_{cid}]$.
%The $\TLink$ layer is for the intermediate machine models. 
%
%We first explain the process of multithreaded linking without detailed implementation in Section~\ref{chapter:linking:subsec:multithreaded-linking-overview}.
%The later sections show the detailed information about our
%implementation, which involves lots of 
%

%
%This multithreaded linking has similarities with the multicore linking.  
%The CPU local machine (parameterized by the layer - $\CSched$ - that contains the proper software schedulers for our mulitthreaded linking framework)
%can be viewed  as a deterministic multithreaded machine with one CPU (behavior of other threads on other CPUs are already described in the environmental context for other CPUs.)
%Since the environmental context for CPUs already hides the non-determinism, 
%this multithreaded machine is analogous to the oracle machine for multicore linking in Section~\ref{chapter:linking:subsec:multicore-machine-model-with-hardware-scheduler}. 
%It however does not have the clearly divided private states for each thread,
%since it is the machine model for us to build multiple local layer interfaces on it.
%The CPU local machine that uses $\CSched[cid, \oracle_{cid}]$  as its parameter 
%is working with the state that is a pair of CPU local state (but shared by all threads on the CPU) and 
%a global log;
%%
%%uses the state which is defined as
%%a tuple, 
%%$(rs, m, a, l)$,
%%which represents 
%%a register set, a memory, an abstract state, and a log.
%%Among them, a register set, a memory, and an abstract state are private datum for CPU local machines. 
%thus scheduling primitives, including $\threadyieldfunc$ and $\threadsleepfunc$,
%modifies the contents in the CPU local state during their transitions. 
%Especially, they requires the change in the register values,  kernel contexts,
%by saving the current thread's register value 
%and restoring the register values for the next scheduled thread. 
%
%In this sense, introducing the new machine model that decouples a single CPU local state into
%the collection of thread local states, while providing the same behaviors for all transitions 
%including context switching, which is $\easmmach$ in Figure~\ref{fig:chapter:linking:multithreaded-linking-structure}.
%
%
%%In this sense, the first step to introduced the per-thread layer machine, 
%%is dividing a single CPU machine into multithreaded machine models,
%which implies that the state of the machine is 
%a set of private states a single shared log
%A set of private states is a partial map from a thread id to its private state designed as a tuple that contains 
%a register set, a memory, as well as an abstract state. 
%Since the layer contains multiple private datum, we also add the flag for currently-running thread $curid$ in the state. 
%By doing this, we can also resolve one challenge in our thread-local machine, which is to replace an assembly style 
%context switch with no-op like operation (challenge number (1) in the previuos section)
%The layer contains per-thread register sets and, thus, the register values do not need to update along the change of the currently-running thread id. By changing the thread id alone, the layer knows which thread-local private data should be 
%used during the current evaluation.
%Now each thread can use its own private data for its evaluation, but that is not sufficient at all. 
%In fact, scheduling switches in this layer has a similar meaning with the ones in the lower layer, 
%$\CSched$ (\ie, the $\yield$) but with different context switching styles.
%Ideally, we would like to reason about each thread execution 
%independently, and later formally combine the reasoning to obtain a global
%property for the full set of threads on the same CPU.
%So, we need a machine model that gives semantics to
%a partially-composed set of threads to support this.



%
%
%
%
%To introduce a thread local interface while keeping a capability of 
%using the same compiler with the CPU local interface ($\compcertx$, 
%thread local layers also has to use the same state definition and the machine model 
%with the CPU local interface, which is $(rs, m, a, l)$.
%The private state of it, however, 
%is only related to the thread. 
%In this sense, 
%behaviors of scheduling primitives remain those thread local states as they are in their machines. 
%The only element that can be changed  in the thread local machine during the scheduling primitives 
%is the shared log $l$. 








%
%
%
%Therefore,  a new layer (\cf Figure~\ref{fig:chapter:conlink:threadlinking} (3)) has to be introduced such that other 
%threads' operations can be modeled as input strategies to the layer interface. 
%Here, we introduced a new kind of environment context, $\oracle_{\codeinmath{thrd}}$, which contains the strategy the environmental threads and is the key to support  thread-local reasoning.
%Formally, with a CPU-local layer  $\PLayer{L}{c}{\oracle_{\codeinmath{cpu}}}$,  we a multithreaded machine model 
%with  the whole thread set ($\fullthreadset$) running over CPU $c$.
%For this purpose, we introduce 
% \emph{multithreaded} layer $\PLayer{\TLink}{\threadset}{{\oracle_{\codeinmath{thrd}}}} := (\Layer_{\TLink},
% \Rely_{\TLink}, \Guard_{\TLink})$,
%which is 
%parameterized over an active thread set $\threadset \subseteq \fullthreadset$.
%The rely condition $\Rely_{\TLink}$ defines a set of acceptable thread contexts
%$\oracle_{\TLink}$ and the guarantee condition $\Guard_{\TLink}$ specifies the events generated by active threads. 
%Since our machine model does not allow
%preemption, $\oracle_{\codeinmath{thread}}$ will only be queried during the execution of scheduling primitives, 
%which have two kinds
%of behaviors  depending on whether the \emph{target
%thread} is active or not.
%
%Considering an execution in Fig.~\ref{fig:chapter:conlink:threadlinking} (3) with active thread set
%$\threadset = \{0\}$, whenever an execution switches (by $\yield$ or $\sleep$) 
%to a thread outside of $\threadset$ (\ie, the yellow $\yield$),
%it takes environmental steps (\ie, notated as arrows), repeatedly appending the 
%events returned by the environment context $\oracle$ and the thread
%context $\oracle_{\codeinmath{thrd}}$ to the log until a $\yield$
%event indicates that control switches back to an active thread.
%
%%
%%This layer is already a thread local because it only captures the behavior of one thread.
%However, the strategy query in this layer follows small-step style, and this is insufficient to build thread-local layer interface because we do not want to query multiple times for a single yield call. 
%Therefore, we introduce another  layer to merge those multiple strategy queries into a single big-query (\cf Figure~\ref{fig:chapter:conlink:threadlinking} (4)). 
%Finally, the last thing to do is to connect the machine state of thread-local layers to our general concurrent layer interface, which has the form of $(\regs, m, a, l)$.
%Therefore, we introduced the last layer (Figure~\ref{fig:chapter:conlink:threadlinking} (5)) that will become a base to build our multithreaded layers.
%
%
%





%Since the layer contains multiple private datum, we also add the flag for currently-running thread $curid$ in the state. 
%By doing this, we can also resolve one challenge in our thread-local machine, which is to replace an assembly style 
%context switch with no-op like operation (challenge number (1) in the previuos section)
%The layer contains per-thread register sets and, thus, the register values do not need to update along the change of the currently-running thread id. By changing the thread id alone, the layer knows which thread-local private data should be 
%used during the current evaluation.
%Now each thread can use its own private data for its evaluation, but that is not sufficient at all. 
%In fact, scheduling switches in this layer has a similar meaning with the ones in the lower layer, 
%$\CSched$ (\ie, the $\yield$) but with different context switching styles.
%Ideally, we would like to reason about each thread execution 
%independently, and later formally combine the reasoning to obtain a global
%property for the full set of threads on the same CPU.
%So, we need a machine model that gives semantics to
%a partially-composed set of threads to support this.



