\section{Overview}
\label{chapter:certikos:sec:overview}

\begin{figure}
\includegraphics[width=\textwidth]{figs/certikos/sysarch}
\caption{System Architecture of $\certikos$}
\label{fig:chapter:certikos:system-architecture-of-certikos}
\jieung{Can I use it? And, I need to delete FIFOBBQ in this figure. Need to direct line from Trap and Syscall to VM Monitor?}
\end{figure}

$\certikos$ was initially developed in the context of a sizeable DARPA-funded research project,
and its role is running high-assurance operating systems on a military land vehicle with an Intel Core i7 machine. 
It supports concurrency with fine-grained locking as well as a hypervisor
to run guest OSes on top of it, 
at most one guest OS per one CPU. 
In the real example with the land vehicle, 
we run 6 Ubuntu Linux systems as guests (one each on the first six cores), 
and  they are responsible for running  their one RADL (The Robot Architecture Definition Language) nodes (with the fixed hardware configuration such as accesses to radar, modem, GPS, \etc). 
Few simple device drivers (\eg, interrupt controllers, serial and keyboard devices)  are in parts of $\certikos$,
but other complex devices are also serviced via either a user level or a directly passed through (via IOMMU) to various guest Linux VMs. 
This $\certikos$ features strong isolation support by running different RADL nodes in different VMs. 
Thanks to this strong isolation property, corruptions or Attacks on one VM does not affect other VMs; 
thus attackers cannot break into all systems even though they take control of parts of the entire systems.

$\certikos$ also contain various shared objects protected by 
the low-level shared object that provides protection, spinlock modules.
It provides two lock implementations, Ticket and MCS, with the same high-level interface for them. 
Based on that, , 
multiple shared objects are implemented, which include
sleep queues (SleepQ - for condition variable), message queues (MsgQ - for warking up on a thread on another CPU), memory management modules (Container, PMM, VMM), an IPC module with supporting synchronizations via CV (condition variables) , \etc.  
It also contains per-core services including per-core scheduler (that can communicate with MsgQ and SleepQ), process management module,
virtual machine monitor (VM). 
Each thread also contains its private states, TCB (thread control block, thread management, context, and so on. 

With this operating system, 
What we want to prove is the correctness of our operating system's behavior with any context programs run on top of it.
We have built the operating systems in a layered structure to provide the highest level layer interface which hides most details of OS implementation and
works as a transition machine for the user programs that preserves the behavior of our operating systems.  
In this sense, providing this layer interface requires us 
to prove the contextual refinement between the $\certikos$ implementation with the bare machine (Intel Core i7) and 
the top-most layer for our OS layers ($\codeinmath{TSyscall}$ - the layer that defines system calls as its primitives).
Informally, we wants to prove that
\begin{quote}
"Any context program $\codeinmath{CTXT}$ running on the $\codeinmath{TSyscall}$ layer (the top-most layer of $\certikos$) on $\hasm$, a machine model for a per-thread layer interface, with the valid thread ID ($tid$)
and a valid environmental context ($\oracle_{\codeinmath{thrd}})$ 
contextually refines the context program $\codeinmath{CTXT}$ plus $\certikos$ kernel code running on $\codeinmath{MBoot}$ layer (the bottom-most layer of $\certikos$) with a hardware step semantics describing a non-deterministic $\intelmachine$ multicore machine model."
\end{quote}
Formally (with the notations in the previous chapters), the theorem we want to provide as the top-level theorem for $\certikos$ is 
 \begin{center}
\begin{tabular}{c}
$\forall\ tid\  \codeinmath{CTXT},$\\
$\semwmachine{\codeinmath{MBoot}}{\codeinmath{CertiKOS} \oplus \codeinmath{Ctxt}}{\codeinmath{mach}_{\codeinmath{x86}}} \refines_{R_{\codeinmath{certikos}}} \semwmachine{\codeinmath{TSyscall}[tid, \oracle_{\codeinmath{tid}}]}{\codeinmath{Ctxt}}{{\codeinmath{mach}_{\hasm}}}$\\
\end{tabular}
\end{center}
which involves not only contextual refinement proofs between multiple local layers but also contextual refinement proofs between
different machine models and focused sets\footnote{We omit explanations about some notations in here for simplicity, but later sections in this chapter explain them}

In order to  prove that, we first decompose the full $\certikos$ as two parts, 
per-thread $\certikos$ ($\certikos_{\codeinmath{td}}$) and per-CPU $\certikos$ ($\certikos_{\codeinmath{cpu}}$)
Then, we can prove that those two program parts satisfy contextual refinement property by using the proofs for each layer, 
along with our multilayer linking library described in Section~\ref{chapter:ccal:subsec:linking}. 
Then, we can link those two proofs with the multithreaded linking described in Section~\ref{chapter:linking:sec:multithreaded-linking}. 
This gives us a theorem for the contextual refinement of the entire $\certikos$ kernel. 
However, it is still insufficient to argue that this is an end-to-end theorem because its bottom-level machine model is a per-CPU local layer machine, 
which is a sequential-like machine based on the machine model of 
$\compcertx$.  % (the "mboot" layer).
 Therefore, we propagate the contextual refinement theorem all the way down to the non-deterministic $\intelmachine$ 
 multicore machine by using our certified multicore linking library discussed in Section~\ref{chapter:linking:sec:multicore-linking}.
%%%%%%%%%%%%%%%%%%%%%%%%%%%%%%%%%%%%%%%%%%%%%%%%
\begin{figure}
\includegraphics[width=\textwidth, page=1]{figs/certikos/concurrent_linking}
\caption{Idea of Dividing $\certikos$ with Formal Definitions} 
\label{fig:chapter:certikos:idea-of-dividing-certikos-with-formal-def}
\end{figure}
%%%%%%%%%%%%%%%%%%%%%%%%%%%%%%%%%%%%
All those steps are described in Figure~\ref{fig:chapter:certikos:idea-of-dividing-certikos-with-formal-def},
which are; (1) contextual refinement for per-CPU layers of $\certikos$; (2) contextual refinement for per-thread layers of $\certikos$;
(3) ontextual refinement for multithreaded linking; (4) contextual refinement for multicore layers.
Each section of this chapter explains each four component.
\jieung{Need to change the bottom-most layer in Chapter 2 (MCS Lock) as MBoot instead of MMCSBoot for the consistency with this chapter}
%
%\begin{figure}
%\includegraphics[width=\textwidth, page=11]{figs/certikos/concurrent_linking}
%\caption{Idea of Dividing $\certikos$}
%\label{fig:chapter:certikos:idea-of-dividing-certikos}
%\jieung{Resize it!!}
%\end{figure}
