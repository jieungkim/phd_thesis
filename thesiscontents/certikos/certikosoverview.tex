\section{Overview}
\label{chapter:certikos:sec:overview}

\begin{figure}
\includegraphics[width=\textwidth]{figs/certikos/sysarch}
\caption{System Architecture of $\certikos$.}
\label{fig:chapter:certikos:system-architecture-of-certikos}
\jieung{Can I use it? And, I need to delete FIFOBBQ in this figure. Need to direct line from Trap and Syscall to VM Monitor?}
\end{figure}


$\certikos$ was initially developed in the context of a sizeable DARPA-funded research project, 
and its role involves running high-assurance operating systems on a military land vehicle with an Intel Core i7 machine. 
It supports concurrency with fine-grained locking as well as a hypervisor to run guest OSes on top of it, 
with a maximum of one guest OS per one CPU. 
In the real-world example with a land vehicle, we run six Ubuntu Linux systems as guests (one each on the first six cores). 
These are responsible for running their one Robot Architecture Definition Language (RADL) 
nodes (with a fixed hardware configuration including accesses to radar, modem, and GPS). 
A few simple device drivers (\eg, interrupt controllers, serial and keyboard devices) are part of $\certikos$, though other complex devices are also serviced via either the user level or being directly passed through (via IOMMU) to various guest Linux VMs. 
This $\certikos$ features strong isolation support by running different RADL nodes in different VMs. 
Due to this strong isolation property, corruptions or attacks on one VM would not affect other VMs; therefore, attackers cannot break into all systems, even if they take control of parts of them.


$\certikos$ also contains various shared objects protected by the low-level shared object that provides protection--spinlock modules.
It provides two lock implementations, Ticket and $\mcsname$, with the same high-level interface for both of them. 
As such, multiple shared objects are implemented, including sleep queues (SleepQ, which is for a condition variable), 
message queues (MsgQ, which is for working up on a thread on another CPU), 
memory-management modules (Container, PMM, VMM), and an IPC module with supporting synchronizations via condition variables, among others. 
It also contains a per-core services including a per-core scheduler (that can communicate with MsgQ and SleepQ), 
a process-management module, and a virtual machine monitor (VM). 
Each thread also contains its private states, thread control block (TCB), thread management, and context, among other items.


With this operating system, we aim to prove the correctness of our operating system’s behavior with any context programs run on top of it. 
We built the operating systems in a layered structure to provide the highest-level layer interface that hides most details of OS implementation and works as a transition machine for user programs, thus preserving the behavior of our operating system.


In this sense, providing this layer interface requires us 
to prove a contextual refinement between our $\certikos$ implementation with a bare machine (Intel Core i7) and 
the top-most layer of our OS layers ($\codeinmath{TSyscall}$, the layer that defines system calls as its primitives).
Informally, we aim to prove that:
\begin{quote}
"Any context program $\codeinmath{CTXT}$ running on the $\codeinmath{TSyscall}$ layer (the top-most layer of $\certikos$) on $\hasm$, a machine model for a per-thread layer interface, with the valid thread ID ($tid$)
and a valid environmental context ($\oracle_{\codeinmath{thrd}})$ 
contextually refines the context program $\codeinmath{CTXT}$ plus $\certikos$ kernel code running on the $\codeinmath{MBoot}$ layer (the bottom-most layer of $\certikos$) with a hardware step semantics describing a nondeterministic $\intelmachine$ multicore machine model."
\end{quote}
Formally (with the notations in the previous chapters), the theorem we want to provide as the top-level theorem for $\certikos$ is as follows:
 \begin{center}
\begin{tabular}{c}
$\forall\ tid\  \codeinmath{CTXT},$\\
$\semwmachine{\codeinmath{MBoot}}{\codeinmath{CertiKOS} \oplus \codeinmath{Ctxt}}{\codeinmath{mach}_{\codeinmath{x86}}} \refines_{R_{\codeinmath{certikos}}} \semwmachine{\codeinmath{TSyscall}[tid, \oracle_{\codeinmath{tid}}]}{\codeinmath{Ctxt}}{{\codeinmath{mach}_{\hasm}}}$\\
\end{tabular}
\end{center}
This involves not only contextual refinement proofs between multiple local layers but also contextual refinement proofs between
different machine models and focused sets\newfootnote{We omit explanations about some notations in here for simplicity, but later sections in this chapter explain them.}.

To  prove this, we first decompose the full $\certikos$ into two parts--per-thread $\certikos$ ($\certikos_{\codeinmath{td}}$) and per-CPU $\certikos$ ($\certikos_{\codeinmath{cpu}}$).
Then, we can prove that those two program parts satisfy contextual refinement theorems by using proofs for all layers,
along with our multilayer linking library described in Section~\ref{chapter:ccal:subsec:linking}. 
Next, we can link those two proofs with the multithreaded linking described in Section~\ref{chapter:linking:sec:multithreaded-linking}. 
This provides a theorem for the contextual refinement of the entire $\certikos$ kernel. 
However, it remains insufficient to argue that this is an end-to-end theorem because its bottom-level machine model is still a per-CPU local-layer machine, which is a sequential-like machine based on the machine model of $\compcertx$. 
Therefore, we propagate the contextual refinement theorem down to the non-deterministic $\intelmachine$ multicore machine by using our certified multicore linking library (discussed in Section~\ref{chapter:linking:sec:multicore-linking}).
%%%%%%%%%%%%%%%%%%%%%%%%%%%%%%%%%%%%%%%%%%%%%%%%
\begin{figure}
\includegraphics[width=\textwidth, page=1]{figs/certikos/concurrent_linking}
\caption{Concept of Dividing $\certikos$ with Formal Definitions.} 
\label{fig:chapter:certikos:idea-of-dividing-certikos-with-formal-def}
\end{figure}
%%%%%%%%%%%%%%%%%%%%%%%%%%%%%%%%%%%%
These steps are described in Figure~\ref{fig:chapter:certikos:idea-of-dividing-certikos-with-formal-def}, and include: 
(1) contextual refinement for per-CPU layers of $\certikos$; 
(2) contextual refinement for per-thread layers of $\certikos$; 
(3) contextual refinement for multithreaded linking; and 
(4) contextual refinement for multicore layers. Each section of this chapter explains an individual component.
