\section{Per-thread Modules}
\label{chapter:certikos:sec:per-thread-modules}


Per-thread modules are built on
the bottom-most layer of our thread local layer interfaces, $\PHThrd$.
In this layer, all scheduling primitives are defined as an identity function for a private state (\ie, they change nothing in the thread's private state),
and only updates a shared log as a result of their evaluation. 
For example, the specification for the  function ``yield''  is
\lstinputlisting[language = C]{source_code/certikos/thread_yield.v}
which only updates a global log, and does not save and restore kernel contexts at all when compared to the thread yield  specification
in Section~\ref{chapter:certikos:subsec:process-management-module}.
This implies that all scheduling primitives in this per-thread modules always return its evaluation control to the thread  that calls those primitives while
encapsulating other threads' observable behaviors in a global log.
Due to the restrictions, we mentioned in Section~\ref{chapter:linking:sec:multithreaded-linking},
building per-thread stacks is not possible with the current toolkit. 
This restriction makes us build a few layers in our per-thread modules,
so we only include small pieces of our operating system in them.



\subsection{IPC Module}
\label{chapter:certikos:subsec:ipc-module}

$\certikos$ supports a double-copy inter-process communication (IPC) protocol.
Senders and receivers of IPC use the dynamically allocated IPC channel for their communication. 
When a sender wants to send the message, 
it copies the message in its memory to the mailbox in the kernel (the designated kernel memory pages for IPC) associated with the channel ID ($chanid$).
When the channel associated with the channel ID is busy, the sender goes into  sleep mode by itself until the other user of the channel
(\ie, the receiver for the communication) wakes up the sender and then completes the message copy again. 
On the other hand, the receiver copies the values in the mailbox to its memory and wakes up the sender to be scheduled for the next communication. 
Because IPC channels are shared by all processes in the system, 
we assign one lock instance per each channel in our IPC implementation.
 \begin{figure}
\begin{center}
\lstinputlisting[language = C]{source_code/certikos/thread_ipc_send.v}
(a) IPC Send Spec.
\lstinputlisting[language = C]{source_code/certikos/thread_ipc_receive.v} 
(a) IPC Receive Spec.
\end{center}
\caption{Top-level IPC Specifications (Written in $\coq$).}
\label{fig:chapter:certikos:ipc-specs}
\end{figure}
The top-level specifications for both IPC send and receive functions are defined in Figure~\ref{fig:chapter:certikos:ipc-specs}
even though lots of details are hidden. 


\subsection{Trap-Handler Module}
\label{chapter:certikos:subsec:trap-handler-module}

Our trap-handler module provides the top-level API--the $\TSyscall$ layer--of our operating system to users.  
The module begins with the layer that contains a list of primitives to retrieve the system call arguments and return values
by following the convention.
When we trap into the kernel, the register $\codeinmath{EAX}$ has a system call number, and other registers contain system call arguments in order of $\codeinmath{EBX}$, $\codeinmath{ECX}$, $\codeinmath{EDX}$, $\codeinmath{ESI}$, and $\codeinmath{EDI}$. 
When $\certikos$ returns to the user after the system call, we use $\codeinmath{EAX}$ for the error number and 
return values in other registers with the order of 
 $\codeinmath{EBX}$, $\codeinmath{ECX}$, $\codeinmath{EDX}$, $\codeinmath{ESI}$, and $\codeinmath{EDI}$. 
Based on them, 
we built a list of system call handlers with the guarantee of safety condition:
``the semantics of each system call handler does not get stuck.'' 
Our top-most layer also implements the system call dispatcher that facilitates the proper system call handler based on the system call number in the $\codeinmath{EAX}$ register.
The $\TSyscall$ layer also contains the page-fault handler as its interface.

\subsection{Connecting All Per-thread Layers}
\label{chapter:certikos:subsec:connecting-all-thread-layers}

We have introduced seven layers for all those modules in our per-thread layers, from the $\PHThrd$ layer to the 
$\TSyscall$ layer.
Similar to the per-CPU modules, they  can be linked together via the composition rules described in Section~\ref{chapter:ccal:subsec:linking},
thus forming a single contextual refinement theorem, which is defined in Theorem~\ref{theorem:chapter:certikos:per-thread-contextual-refinement},
corresponding to  number (2) in Figure~\ref{fig:chapter:certikos:idea-of-dividing-certikos-with-formal-def}.
\begin{theorem}[Contextual Refinement for Per-thread Layers]
\label{theorem:chapter:certikos:per-thread-contextual-refinement}
Let's assume that $tid$ is a valid thread ID, $\oracle_{\codeinmath{thrd}}$ and  $\oracle'_{\codeinmath{thrd}}$ are environmental contexts for 
$\PHThrd$  and $\TSyscall$ layers, respectively, and $\codeinmath{CTXT}$ is an any-context program. Then, we can say that
\begin{center}
$\semwmachine{\PLayer{\PHThrd}{tid}{\oracle_{\codeinmath{thrd}}}}{\certikos_{\codeinmath{td}} \oplus \codeinmath{Ctxt}}{\hasmmach} \refines_{R_{\certikos_\codeinmath{td}}} \semwmachine{\PLayer{\TSyscall}{tid}{\oracle'_{\codeinmath{thrd}}}}{\codeinmath{Ctxt}}{\hasmmach}$
\end{center}
when ${R_{\certikos_\codeinmath{td}}}$ is a refinement relation between those two layers (the refinement relation can be achieved by 
connecting all the refinement relations  between seven per-thread layers).
\end{theorem}


%This theorem shows that "any context program CTXT running on the tsyscall layer contextually refines the context program CTXT plus per_thread_mC2 running on phthrd (the bottom-most layer interface of per-thread layers) with HAsm". HAsm is described in Section 4-2 in this document; it can use CompCertX to propagate the C proofs down to the assembly level (HAsm), and it has the power to dynamically allocate the proper initial state for each thread. To link the proofs for all layers together, we use the LinkSourceTemplate.v library for source code in all layers, and the HAsmLinkTemplate.v library (both libraries are mentioned in Section 4 of this document) for proofs in all layers.
%
%\begin{lstlisting}[language=C]
%  Theorem Per_Thread_CertiKOS_correct:
%    forall (s: stencil) (CTXT: LAsm.module) kernel combined_program user_program
%           (builtin_idents_norepet_prf: CompCertBuiltins.BuiltinIdentsNorepet),
%      CertiKOS.per_thread_certikos = OK kernel ->
%      make_program s (CTXT <@$\oplus$@> kernel) (PHThread.phthread  <@$\oplus$@>  L64) = OK combined_program ->
%      make_program s CTXT (TSysCall.tsyscall  <@$\oplus$@>  L64) = OK user_program ->
%      inhabited
%        (backward_simulation
%           (HAsm.semantics (TSysCall.tsyscall  <@$\oplus$@>  L64) user_program)
%           (HAsm.semantics (PHThread.phthread <@$\oplus$@> L64) combined_program)).
%\end{lstlisting}