\section{Multicore Linking for CertiKOS}
\label{chapter:certikos:sec:multicore-linking-for-certikos}

Multicore linking in CertiKOS facilitates all the intermediate languages
and proofs in Section~\ref{chapter:linking:sec:multicore-linking}.
Figure~\ref{fig:chapter:conlink:intermediate-languages-and-their-relationsihps-for-multicore-linking} and 
Section~\ref{chapter:linking:sec:multicore-linking} show all definitions and 
proofs that connect all those things. 
However, to build the local layer interface, 
Instantiating the hardware configuration, 
use the concrete definition $L$, which will be the 
bottom layer of our local layer interface, 
and combine those intermediate languages and proofs with $\compcert$ semantic and backward simulation 
forms to connect them rigorously. 


\subsection{Hardware-configuration Instance}
\label{chapter:certikos:subsec:hardware-configuration-instance}

The hardware configuration in Section~\ref{chapter:linking:subsec:hardware-configuration} works as a 
template for users to instantiate all intermediate machine models in multicore linking. 
To show the instance for that configuration,
we again briefly show the state definition of $\compcertx$
as well as the lowest layer in $\certikos$, which includes TCB functions that we mentioned in Section~\ref{chapter:mcslock:sec:verification}.
Those functions are 
necessary for supporting extended features that $\compcert$ cannot support. 
For example, 
fetch-and-increase or compare-and-swap functions are atomic operations that are crucial for building a 
concurrent object by providing the lowest-level atomicity. 
However, they are not supported by $\compcert$'s instructions.
Our bottom (TCB) layer, $\mmcsbootfull$, works as a parameter of a machine model for $\compcertx$ 
and provides those  primitives to give  extra transition rules on the state of $\compcertx$.
%
%\begin{figure}
%\begin{lstlisting}[language=C]
%Inductive val: Type :=
%  | Vundef: val
%  | Vint: int -> val
%  | Vlong: int64 -> val
%  | Vfloat: float -> val
%  | Vptr: block -> int -> val.
%
%Definition regset := Pregmap.t val.


%\end{lstlisting}
%\caption{$\compcertx$ State Definition.}
%\label{fig:chapter:certikos:compcertx-state-definition}
%\end{figure}

As discussed in Section~\ref{chapter:linking:sec:multithreaded-linking} and Section~\ref{chapter:certikos:sec:multithreaded-linking-for-certikos},
the state definition for a machine model of  $\compcertx$ ($\lasmmach$) is
\begin{lstlisting}
Inductive state <@$\cdots$@> : Type :=
  | State: regset -> (mem: mwd LDADTOps)-> state.
\end{lstlisting}
which is a pair consisting of  a register and a memory. 
The state is similar to that of $\compcert$ except the extension of the memory type, which is associated with 
the abstract data type  (notated as $LDADTOps$).
Thanks to the similarity between this state definition and the state definition of $\compcert$, 
$\compcertx$ facilitates many parts of the verified compiler of $\compcert$ with some modifications, 
and it is applicable to our concurrent program verification (because 
we built all layers based on our sequential-like layer interfaces).
The memory in the $\compcertx$   is also a pair of two elements--a memory (a map from an identifier to a value) and an abstract datum. 
%\begin{lstlisting}
%Definition <@$\pi$@>_mem {D}: mwd D -> mem := fst.
%Definition <@$\pi$@>_data {D}: mwd D -> ldata_type D := snd.
%\end{lstlisting}
The first element of the memory is the same as the memory in the $\compcert$~\cite{leroy08},
 which is a map from its index (a pair of a block ID and an offset) to the value alongside with its type information about the size of the value. 
An abstract datum gives us huge freedom for the state definition by allowing us to add multiple fields.
To use the existing tools as much as possible, 
we also include our global log in the abstract data (as we have seen in Chapter~\ref{chapter:mcs-lock}).
Thus, the \lstinline$(mem: mwd LDATAOps)$ can be interpreted as a tuple consisting of 
\lstinline$mem$, \lstinline$RData$ (an abstract data), and a set of logs for shared objects (notated as \lstinline$multi_log$ in
Chapter~\ref{chapter:mcs-lock}.)


\begin{figure}
\begin{lstlisting}[language=C]
Definition mboot_passthrough : compatlayer (cdata RData) :=
  <@$\cdots$@>
  <@$\oplus$@> get_CPU_ID <@$\mapsto$@> gensem get_CPU_ID_spec
  <@$\oplus$@> acquire_shared <@$\mapsto$@> primcall_acquire_shared_compatsem acquire_shared0_spec0
  <@$\oplus$@> atomic_FAI <@$\mapsto$@> primcall_atomic_FAI_compatsem atomic_FAI_spec
  <@$\cdots$@>
\end{lstlisting}
\caption{Part of $\mmcsbootfull$ Layer Interface: the bottom layer interface of $\certikos$.}
\label{fig:chapter:certikos:bottom-layer-interface}
\end{figure}

\begin{figure}
\begin{lstlisting}[language=C]
Record RData := mkRData {
  CPU_ID: Z;  // current CPU_ID
  HP: flatmem;  // we model the memory from 1G to 3G as heap            
  CR3: globalpointer;  // abstract of CR3, stores the pointer to page table
  // set of logs for shared resources - we divide the global log to multiple
  // logs for each shared objects for simplicity   
  multi_log: MultiLogPool;     
 <@$\cdots$@>
}.
\end{lstlisting}
\caption{Abstract Data for  $\mmcsbootfull$ Layer Interface.}
\label{fig:chapter:certikos:abstract-data-for-bottom-layer}
\end{figure}

Figure~\ref{chapter:certikos:subsec:hardware-configuration-instance}
 and 
  Figure~\ref{fig:chapter:certikos:abstract-data-for-bottom-layer}
  show the layer definition 
  and the definition of the abstract data 
  in the bottom layer of our per-CPU layer interface, $\mmcsbootfull$.
%
%For all layers, the primitive definitions in the layer $\Layer$ (when $\PLayer{L}{\ignorechar}{\ignorechar} = (\Layer, \ignorechar, \ignorechar)$) contains a set of primitives 
%that can be called by the context programs.
%The layer definition for the bottom layer is in Figure~\ref{chapter:certikos:subsec:hardware-configuration-instance},
% contains a set of primitives,
%which is basically a partial map from a primitive ID ($\primitiveid$) to the specifications ($\spec$).
%Each layer also use its abstract data, and the abstract data definition for the bottom layer is in Figure~\ref{fig:chapter:certikos:abstract-data-for-bottom-layer}.
As mentioned, 
we have provided an atomic fetch-and-increase as an atomic operation in the layer,
and it will update the log in the abstract data ($\codeinmath{multi\_log}$), which is 
a part of our state definition. 
Our hardware-configuration instance is closely related to those definitions
because they are parts of the lowest machine model 
in our certified concurrent abstraction layers.
For example, atomic operations in the configuration (Figure~\ref{fig:chapter:conlinkg:abstract-hardware-configuration-and-semantics})
need to include $\codeinmath{atomic\_FAI}$  
in and the local evaluation rule in the atomic statement in Figure~\ref{fig:chapter:conlink:hardware-local-step-transition-rules}  must be
consistent with the specification of the primitive.
The private state can be instantiated with 
the state definition in the machine model of $\compcertx$--a pair two elements which consist of a register set 
and a memory (\ie, the extended memory with  abstract states in it)--and the shared piece is a list of bytes, which represents 
the snapshot of the shared memory via $\push$/$\pull$ operations. 

To instantiate the definition $\mcevent$ in Figure~\ref{chapter:certikos:subsec:hardware-configuration-instance},
we define ``\lstinline$OtherEvent$'' as follows:
\begin{lstlisting}[language=C, deletekeywords={for}]
// The instance that implements <@${\color{red}  \mcevent}$@>
Inductive OtherEvent := | OINC_TICKET (n: nat) <@$\cdots$@> 
\end{lstlisting}
The definition maps each atomic operation in layer $\mmcsbootfull$ into each single event in our hardware configuration.
For example, \lstinline$OINC_TICKET (n: nat)$ corresponds to the fetch-and-increase primitive,

In addition, \lstinline$OtherEvent_2_ident$ is an instance of the $\atomiceventidentkwd$ function, which is the function
that gets the event and returns the corresponding identifier (the primitive identifier) that is related to the event.
\begin{lstlisting}[language=C]
// The instance that implements <@${\color{red}  \atomiceventidentkwd}$@>
Function OtherEvent_2_ident (e: OtherEvent) := 
match e with | OINC_TICKET _ => atomic_FAI <@$\cdots$@> end.
\end{lstlisting}
The \lstinline$core_set$ is defined as a full CPU set ($\coreset$), and 
the set contains all CPU numbers in it (from zero to seven because $\certikos$ runs on eight cores.)
The last element is the identifier for 
the scheduler, which should not be in the \lstinline$core_set$. 
We designated eight for the scheduler ID.
Those configurations are all included in the 
\lstinline$hdsetting$ class, and the instance of the type class
is in Figure~\ref{fig:chapter:ceritkos:instances-of-abstract-hardware-setting}.
\begin{figure}
\begin{lstlisting}[language=C, deletekeywords={int}]
Global Instance hdseting: HardWaredSetting := {
  private_state := state; // <@$\color{red} \privatestate$@>
  shared_piece := list Integers.Byte.int; // <@$\color{red} \sharedpiece$@>
  atomic_event := OtherEvent;  // <@$\color{red} \atomicevent$@>
  atomic_event_ident := OtherEvent_2_ident; // <@$\color{red} \atomiceventidentkwd$@>
  core_set := cpu_set; // <@$\color{red} \coreset$@>
  sched_id := 8  // <@$\color{red} \schedid$@>
}.
\end{lstlisting}
\caption{The Hardware-setting Definition for $\certikos$ Multicore Linking.}
\label{fig:chapter:ceritkos:instances-of-abstract-hardware-setting}
\end{figure}
All values are instances of 
hardware-configuration definitions ($\hardwaresetting$) in Figure~\ref{fig:chapter:conlinkg:abstract-hardware-configuration-and-semantics}. 

\begin{figure}
\begin{lstlisting}[language=C]
Inductive command_predicate (ge: genv): mstate -> command -> Prop :=
<@$\cdots$@>
| command_prim_FAI: <@$\cdots$@>
  rs Asm.PC = Vptr b Int.zero -> Genv.find_funct_ptr ge b = Some (External ef) ->
  match ef with  | EF_external eid _ => if peq eid atomic_FAI then True else False
  <@$\cdots$@>
  command_predicate ge (State rs m) (ATOMIC lid atomic_FAI) 
<@$\cdots$@>
\end{lstlisting}
\begin{center}
(a) The Definition of Program Counter for \lstinline$atomic_FAI$.
\end{center}
\begin{lstlisting}[language=C, deletekeywords={int, unsigned}]
 Inductive get_shared_sem (ge: genv) 
   : mstate -> list Integers.Byte.int -> mstate -> Prop:=
   | get_shared_intro: 
    <@$\cdots$@>
    Genv.find_symbol ge id = Some b -> id2size (Int.unsigned index) = Some (size, id)
    -> Mem.loadbytes m b 0 size = Some (ByteList l)
    <@$\cdots$@>
     let a_res := a {multi_log: ZMap.set z (MultiDef (TEVENT (CPU_ID a)
       (TSHARED (OMEME l)) :: ml0)) (multi_log a)} in
    get_shared_sem ge (State (mem:= mwd LDATAOps) rs (m, a)) l 
       (State (rs_res#RA <- Vundef#PC <- (rs RA)) (m, a_res)).
\end{lstlisting}
\begin{center}
(b) The Transition Rule Definition for the $\acqsharedcmdkwd$ Command.
\end{center}
\begin{lstlisting}[language=C]
Inductive atomic_exec (ge: genv):
  Z -> Z -> mstate -> Log -> mstate -> OtherEvent -> Prop :=
  | atomic_step_external:
  <@$\cdots$@>
  atomic_exec ge cid id (State rs m) sl (State rs_res m_res) e
  <@$\cdots$@>
\end{lstlisting}
\begin{center}
(b) The  Transition-rule Definition for the $\atomiccmdkwd$ Command.
\end{center}
\caption{Program Counter and Transition Rules in $\certikos$ Multicore Linking (Details Are Ignored).}
\label{fig:chapter:certikos:definitions-for-hw-abstract-semantics}
\end{figure}

All our abstract languages in Section~\ref{chapter:linking:sec:multicore-linking} 
also rely on the abstract transition rules ($\hardwaresemantics$) for 
$\privatecmdkwd$, $\atomiccmdkwd$, $\acqsharedcmdkwd$, and $\relsharedcmdkwd$ commands, and the special auxiliary function to get a program counter ($\programcounterkwd$) from the current state.
Similar to the previous definitions, 
it also facilitates the layer definition ($\codeinmath{mboot}$), 
and machine model of $\compcertx$, $\lasmmach$.
For example, the core part of the program counter command definition for the fetch-and-increase function call is defined in Figure~\ref{fig:chapter:certikos:definitions-for-hw-abstract-semantics} (a), which is associated with the case when the current function pointer is pointing out the \lstinline$atomic_FAI$. 
In the rule, \lstinline$(ge :genv)$ stands for the global environment that contains the 
global information such as function names--global variables of the whole programs--which 
is necessary for us to divide the full program into multiple components without keeping the information 
about the entire program context.


Similarly, all abstract transition rules   ($\hardwaresemantics$)  in  Figure~\ref{fig:chapter:conlinkg:abstract-hardware-configuration-and-semantics}.
are defined 
by slightly modifying the step relation in $\compcertx$ as well as using the case analysis for the primitive identifiers. 
For instance, the transition rule for the $\acqsharedcmdkwd$ is defined in Figure~\ref{fig:chapter:certikos:definitions-for-hw-abstract-semantics} (b), which updates the program counter in the private state
and updates the shared log (note that the shared log is a part of our abstract state in our implementation).
Similarly, the atomic execution rule is in Figure~\ref{fig:chapter:conlinkg:abstract-hardware-configuration-and-semantics} (c),
which gets a global environment (\lstinline$ge$), a current CPU ID (\lstinline$cid$),  an identifier for the atomic primitive
(\lstinline$id$),  a local state (\lstinline$(State rs m)$),
and a shared log (\lstinline$sl$) and returns a newly updated local state (\lstinline$(State rs_res m_res)$) as well as an event (\lstinline$e$) generated during the evaluation of the primitive.
This type is identical to the type of $\atomicstepkwd$ in $\hardwaresemantics$ of Figure~\ref{fig:chapter:certikos:hardware-local-step-transition-rules}.

\begin{figure}
\begin{lstlisting}[language=C]
Variables (ge: genv) (sten: stencil) (M: module).
Context {Hmakege: make_globalenv (module_ops:= LAsm.module_ops) 
                                 (mkp_ops:= make_program_ops) 
                                 sten M (mboot <@$\oplus$@> L64) = ret ge}.

Global Program Instance hdsem : HardSemantics (hdset:= hdseting) := {
  PC := command_predicate ge; // <@$\programcounterkwd$@>
  private_step := private_exec ge; // <@$\privatestepkwd$@>
  get_shared := get_shared_sem ge; // <@$\getsharedstepkwd$@>
  set_shared := set_shared_sem ge; // <@$\setsharedstepkwd$@>
  atomic_step := atomic_exec ge // <@$\atomicstepkwd$@>
}.
\end{lstlisting}
\caption{The Hardware Setting Definition for $\certikos$ Multicore Linking.}
\label{fig:chapter:certikos:hardware-local-step-transition-rules}
\end{figure}

With similar transition rules for $\relsharedcmdkwd$ and $\privatecmdkwd$,
Figure~\ref{fig:chapter:conlinkg:abstract-hardware-configuration-and-semantics}
shows the instance of $\hardwaresemantics$, which is defined in Figure~\ref{fig:chapter:certikos:hardware-local-step-transition-rules}. 
We also prove basic properties of those transition rules in Figure~\ref{fig:chapter:conlink:properties-of-abstract-hardware-semantics-for-multicore-linking}.
Most of them are straightforward because our instances use  definitions in the machine model of $\compcertx$,
which already contains  deterministic properties. 

Note that we are using an assembly model  in $\compcertx$ for $\intelmachine$; thus, it implies that 
our instance for this hardware configuration is   part of a practical $\intelmachine$ multicore machine,
that has a subset of its instructions (but sufficiently rich for building operating systems) with a small TCB, 
and follows the sequentially consistent evaluation order for the memory consistency. 
However, it does not need to be specified by $\intelmachine$. 
All those hardware configurations can be instantiated with other definitions, such as ARM assembly machine for $\compcertx$, 
or even another machine model that is not entirely related to $\compcertx$ if users can prove the desired property of them.



%
%\begin{lstlisting}
%  Inductive OtherEvent :=
%  (* Events used for the ticket-lock primitives: *)
%  | OINC_TICKET (n: nat)
%  | OINC_NOW
%  | OGET_NOW
%  | OINIT
%  | OHOLD_LOCK
%  | OPAGE_COPY (b: PageBuffer)
%  .
%
%  Function OtherEvent_2_ident (e: OtherEvent) :=
%    match e with
%      | OINC_TICKET _ => atomic_FAI
%      | OINC_NOW => log_incr
%      | OGET_NOW => log_get
%      | OINIT => log_init
%      | OHOLD_LOCK => log_hold
%      | OPAGE_COPY _ => page_copy
%    end.
%
%  Global Instance hdseting: HardWaredSetting :=
%    {
%      private_state := mstate;
%      shared_piece := list Integers.Byte.int;
%      atomic_event := OtherEvent;
%      atomic_event_ident := OtherEvent_2_ident;
%      core_set := cpu_set;
%      sched_id := 9
%    }.
%\end{lstlisting}



%\begin{lstlisting}[language=C]
%    Inductive get_shared_sem (ge: genv) : 
%        mstate -> list Integers.Byte.int -> mstate -> Prop:=
%    | get_shared_intro: 
%        forall rs sig m a index ofs id l b size ml0 z,
%          Genv.find_symbol ge id = Some b
%          -> id2size (Int.unsigned index) = Some (size, id)
%          -> Mem.loadbytes m b 0 size = Some (ByteList l)
%          -> sig = mksignature (AST.Tint::AST.Tint::nil) None cc_default
%          -> extcall_arguments rs m sig (Vint index :: Vint ofs ::nil) ->
%          forall (Hindex: index2Z (Int.unsigned index) (Int.unsigned ofs) = Some z)
%            (Hlog: ZMap.get z (multi_log a) = MultiDef ml0),
%            let rs' := (undef_regs (CR ZF :: CR CF :: CR PF :: CR SF :: CR OF
%                                       :: IR EAX :: RA :: nil)
%                                   (undef_regs (List.map preg_of destroyed_at_call) rs)) in
%            let a' := a {multi_log: ZMap.set z (MultiDef (TEVENT (CPU_ID a) (TSHARED (OMEME l)) :: ml0))
%                                             (multi_log a)} in
%            get_shared_sem ge (Asm.State (mem:= mwd LDATAOps) rs (m, a)) l 
%                           (Asm.State (rs'#RA <- Vundef#Asm.PC <- (rs RA)) (m, a')).
%\end{lstlisting}



%\begin{lstlisting}[language=C]
%  Inductive step (ge: genv): state -> trace -> state -> Prop :=
%    | exec_step_internal:
%        forall b ofs f i rs m rs' m',
%        rs PC = Vptr b ofs ->
%        Genv.find_funct_ptr ge b = Some (Internal f) ->
%        find_instr (Int.unsigned ofs) f.(fn_code) = Some i ->
%        exec_instr ge f i rs m = Next rs' m' ->
%        step ge (State rs m) E0 (State rs' m')
%  | exec_step_builtin:
%      forall b ofs f ef args res rs m t vl rs' m',
%      rs PC = Vptr b ofs ->
%      Genv.find_funct_ptr ge b = Some (Internal f) ->
%      find_instr (Int.unsigned ofs) f.(fn_code) = Some (asm_instruction (Pbuiltin ef args res)) ->
%      external_call' (fun _ => True) ef ge (map rs args) m t vl m' ->
%      rs' = nextinstr_nf 
%             (set_regs res vl
%               (undef_regs (map preg_of (destroyed_by_builtin ef)) rs)) ->
%(** [CertiKOS:test-compcert-disable-extcall-as-builtin] We need
%to disallow the use of external function calls (EF_external) as
%builtins. *)
%      forall BUILTIN_ENABLED: match ef with
%                                | EF_external _ _ => False
%                                | _ => True
%                              end,
%(** [CompCertX:test-compcert-wt-builtin] We need to prove that registers updated by builtins are
%    of the same type as the return type of the builtin. *)
%      forall BUILTIN_WT: 
%               subtype_list (proj_sig_res' (ef_sig ef)) (map typ_of_preg res) = true,
%      step ge (State rs m) t (State rs' m')
%  | exec_step_annot:
%      forall b ofs f ef args rs m vargs t v m',
%      rs PC = Vptr b ofs ->
%      Genv.find_funct_ptr ge b = Some (Internal f) ->
%      find_instr (Int.unsigned ofs) f.(fn_code) = Some (asm_instruction (Pannot ef args)) ->
%      annot_arguments rs m args vargs ->
%      external_call' (fun _ => True) ef ge vargs m t v m' ->
%      forall BUILTIN_ENABLED: match ef with
%                                | EF_external _ _ => False
%                                | _ => True
%                              end,
%      step ge (State rs m) t
%           (State (nextinstr rs) m')
%    | exec_step_external:
%        forall b ef args res rs m t rs' m',
%        rs PC = Vptr b Int.zero ->
%        Genv.find_funct_ptr ge b = Some (External ef) ->
%        (* XXX maybe flip those two to keep the same order as before *)
%        extcall_arguments rs m (ef_sig ef) args ->
%        external_call' (fun _ => True) ef ge args m t res m' ->
%        (** [CompCertX:test-compcert-undef-destroyed-by-call] We erase non-callee-save registers. *)
%      (** [CompCertX:test-compcert-ra-vundef] We need to erase the value of RA,
%      which is actually popped away from the stack in reality. *)
%      rs' = (set_regs (loc_external_result (ef_sig ef)) res (undef_regs (CR ZF :: CR CF :: CR PF :: CR SF :: CR OF :: nil) (undef_regs (map preg_of destroyed_at_call) rs))) #PC <- (rs RA) #RA <- Vundef ->
%        forall STACK:
%        forall b o, rs ESP = Vptr b o ->
%                    (Ple (Genv.genv_next ge) b /\ Plt b (Mem.nextblock m)),
%  (** [CompCertX:test-compcert-protect-stack-arg] The following
%      NOT_VUNDEF conditions will allow to later parameterize the
%      per-function/module semantics of back-end languages over the stack
%      pointer and the return address. *)
%        forall SP_NOT_VUNDEF: rs ESP <> Vundef,
%        forall RA_NOT_VUNDEF: rs RA <> Vundef,
%        step ge (State rs m) t (State rs' m')
%    | exec_step_prim_call:
%        forall b ef rs m t rs' m',
%          rs PC = Vptr b Int.zero ->
%          Genv.find_funct_ptr ge b = Some (External ef) ->
%          primitive_call ef ge rs m t rs' m' ->
%          step ge (State rs m) t (State rs' m').
%
%  (** The low-level LAsm machine use the same initial states as the
%    regular Compcert Asm, however high-level, non-main thread machines
%    use a different entry point. We use the following class as a
%    parameter so that we can account for this. *)
%
%  Class AsmInitialState :=
%    {
%      asm_init_state {F V}: AST.program F V -> state -> Prop;
%
%      asm_init_state_determ {F V} (p : AST.program F V):
%        forall s1 s2, asm_init_state p s1 -> asm_init_state p s2 -> s1 = s2
%    }.
%
%  (** Compared to the original CompCert [Asm.initial_state], this
%    version leaves RA to be [Vundef] rather than set it to [Vzero].
%    This makes it easier to treat the initial state of the main thread
%    and that of other threads uniformly, since threads which start as
%    a "return from yield" have their RA undefined. *)
%
%  Inductive initial_state {F V} (p: AST.program F V): state -> Prop :=
%    | initial_state_intro m0:
%        Genv.init_mem p = Some m0 ->
%        let rs0 :=
%          (Pregmap.init Vundef)
%          # PC <- (symbol_offset (Genv.globalenv p) (prog_main p) Int.zero)
%          # ESP <- Vzero in
%        initial_state p (State rs0 m0).
%
%  Global Program Instance asm_init_default : AsmInitialState | 10 :=
%    {
%      asm_init_state F V := initial_state
%    }.
%  Next Obligation.
%    destruct H; subst rs0.
%    destruct H0; subst rs0.
%    congruence.
%  Qed.
%
%  Context `{asm_init : AsmInitialState}.
%
%  Definition final_state (s : state) (i : int) : Prop :=
%    False.
%
%  Definition semantics (p: program) :=
%    Smallstep.Semantics step (asm_init_state p) final_state (Genv.globalenv p).
%\end{lstlisting}
%
%\begin{figure}
%\begin{lstlisting}[language=C]
%Function get_last_event (l: MultiLogType) :=
%  match l with
%  | MultiUndef => None
%  | MultiDef nil => Some OINIT
%  | MultiDef (e:: l) =>
%    match e with
%    | TEVENT _ e0 =>
%      match e0 with
%      | TTICKET e1 =>
%        match e1 with
%        | INC_TICKET n => Some (OINC_TICKET n)
%        | INC_NOW => Some OINC_NOW       
%        | GET_NOW => Some OGET_NOW
%        | HOLD_LOCK => Some OHOLD_LOCK
%        | _ => None
%        end
%      | TSHARED (OBUFFERE b) => Some (OPAGE_COPY b)
%      | _ => None
%      end
%    end
%  end.
%\end{lstlisting} 
%\end{figure}



%\begin{figure}
%
%\begin{lstlisting}[language=C]
%  Local Open Scope Z_scope.
%
%  Definition genv := Genv.t fundef unit.
%    
%  // to determine PC from the state 
%  Inductive command_predicate (ge: genv): mstate -> command -> Prop :=
%  | acq_prim_call:
%      forall b ef rs m lid index ofs sig,
%        rs Asm.PC = Vptr b Int.zero ->
%        Genv.find_funct_ptr ge b = Some (External ef) ->
%        sig = mksignature (AST.Tint::AST.Tint::nil) None cc_default ->
%        extcall_arguments rs m sig (Vint index :: Vint ofs :: nil) ->
%        index2Z (Int.unsigned index) (Int.unsigned ofs) = Some lid ->
%        match ef with
%          | EF_external eid _ => 
%            if peq eid acquire_shared then True
%            else False
%          | _ => False
%        end ->
%        command_predicate ge (Asm.State rs m) (ACQ_SHARED lid)
%
%  | rel_prim_call:
%      forall b ef rs m lid index ofs sig,
%        rs Asm.PC = Vptr b Int.zero ->
%        Genv.find_funct_ptr ge b = Some (External ef) ->
%        sig = mksignature (AST.Tint::AST.Tint::nil) None cc_default ->
%        extcall_arguments rs m sig (Vint index :: Vint ofs :: nil) ->
%        index2Z (Int.unsigned index) (Int.unsigned ofs) = Some lid ->
%        match ef with
%          | EF_external eid _ => 
%            if peq eid release_shared then True
%            else False
%          | _ => False
%        end ->
%        command_predicate ge (Asm.State rs m) (REL_SHARED lid)
%\end{lstlisting}
%\end{figure}

%\begin{figure}
%\begin{lstlisting}[language=C]
%  | command_prim_atomic:
%      forall b ef rs m lid index ofs sig eid,
%        rs Asm.PC = Vptr b Int.zero ->
%        Genv.find_funct_ptr ge b = Some (External ef) ->
%        atomic_id eid ->
%        eid <> page_copy ->
%        match ef with
%          | EF_external eid' _ =>
%            if peq eid eid' then True
%            else False
%          | _ => False
%        end ->
%        sig = mksignature (AST.Tint::AST.Tint::nil) None cc_default ->
%        extcall_arguments rs m sig (Vint index :: Vint ofs :: nil) ->
%        index2Z (Int.unsigned index) (Int.unsigned ofs) = Some lid ->
%        command_predicate ge (Asm.State rs m) (ATOMIC lid eid)
%
%  | command_prim_page_copy:
%      forall b ef rs m lid cv count from sig,
%        rs Asm.PC = Vptr b Int.zero ->
%        Genv.find_funct_ptr ge b = Some (External ef) ->
%        match ef with
%          | EF_external eid _ => 
%            if peq eid page_copy then True
%            else False
%          | _ => False
%        end ->
%        sig = mksignature (AST.Tint::AST.Tint::AST.Tint::nil) None cc_default ->
%        extcall_arguments rs m sig (Vint cv :: Vint count :: Vint from :: nil) ->
%        index2Z ID_SC (slp_q_id (Int.unsigned cv) 0) = Some lid ->
%        command_predicate ge (Asm.State rs m) (ATOMIC lid page_copy)
%\end{lstlisting}
%\end{figure}
%
%\begin{figure}

%\begin{lstlisting}[language=C]
%  | command_prim_FAI:
%      forall b ef rs m lid index ofs sig bound,
%        rs Asm.PC = Vptr b Int.zero ->
%        Genv.find_funct_ptr ge b = Some (External ef) ->
%        match ef with
%          | EF_external eid _ => 
%            if peq eid atomic_FAI then True
%            else False
%          | _ => False
%        end ->
%        sig = mksignature (AST.Tint::AST.Tint::AST.Tint::nil) None cc_default ->
%        extcall_arguments rs m sig (Vint bound :: Vint index :: Vint ofs :: nil) ->
%        index2Z (Int.unsigned index) (Int.unsigned ofs) = Some lid ->
%        command_predicate ge (Asm.State rs m) (ATOMIC lid atomic_FAI)
%
%  | command_internal:
%      forall b ofs f rs m,
%        rs Asm.PC = Vptr b ofs ->
%        Genv.find_funct_ptr ge b = Some (Internal f) ->
%        command_predicate ge (Asm.State rs m) PRIVATE
%
%  | command_prim_private:
%      forall b ef rs m,
%        rs Asm.PC = Vptr b Int.zero ->
%        Genv.find_funct_ptr ge b = Some (External ef) ->
%        match ef with
%          | EF_external eid _ => 
%            private_id eid
%          | _ => True
%        end ->
%        command_predicate ge (Asm.State rs m) PRIVATE.
%\end{lstlisting}
%\end{figure}

%
%\begin{figure}
%\begin{lstlisting}[language=C]
%  Global Instance hdseting: HardWaredSetting :=
%    {
%      private_state := mstate;
%      shared_piece := list Integers.Byte.int;
%      atomic_event := OtherEvent;
%      atomic_event_ident := OtherEvent_2_ident;
%      core_set := cpu_set;
%      sched_id := 9
%    }.
%  Proof.
%    - intros.
%      eapply list_eq_dec. intros.
%      eapply Byte.eq_dec.
%    - repeat decide equality.
%      eapply Int.eq_dec.
%    - trivial.
%  Defined.
%\end{lstlisting}
%\end{figure}
%
%\begin{figure}
%
%\begin{lstlisting}[language=C]
%  // private execution - similar to Asm
%  Inductive private_exec (ge: genv): Z -> mstate -> mstate -> Prop :=
%    | private_step_internal:
%        forall b ofs f i rs (m m': mwd LDATAOps) rs' cid,
%        rs Asm.PC = Vptr b ofs ->
%        Genv.find_funct_ptr ge b = Some (Internal f) ->
%        find_instr (Int.unsigned ofs) f.(fn_code) = Some i ->
%        exec_instr ge f i rs m = Next rs' m' ->
%        CPU_ID (snd m) = cid ->
%        private_exec ge cid (Asm.State rs m) (Asm.State rs' m')
%
%  | private_step_builtin:
%      forall b ofs f ef args res rs (m m': mwd LDATAOps) t vl rs' cid,
%      rs Asm.PC = Vptr b ofs ->
%      Genv.find_funct_ptr ge b = Some (Internal f) ->
%      find_instr (Int.unsigned ofs) f.(fn_code) = Some (asm_instruction (Pbuiltin ef args res)) ->
%      external_call' (fun _ => True) ef ge (map rs args) m t vl m' ->
%      rs' = nextinstr_nf 
%             (set_regs res vl
%               (undef_regs (map preg_of (destroyed_by_builtin ef)) rs)) ->
%      CPU_ID (snd m) = cid ->
%     //  [CertiKOS:test-compcert-disable-extcall-as-builtin] We need
%     // to disallow the use of external function calls (EF_external) as
%     //  builtins. 
%      forall BUILTIN_ENABLED: match ef with
%                                | EF_external _ _ => False
%                                | _ => True
%                              end,
%    // [CompCertX:test-compcert-wt-builtin] We need to prove that registers updated by builtins are
%    // of the same type as the return type of the builtin. 
%      forall BUILTIN_WT: 
%               subtype_list (proj_sig_res' (ef_sig ef)) (map typ_of_preg res) = true,
%        private_exec ge cid (Asm.State rs m) (Asm.State rs' m')
%\end{lstlisting}
%\end{figure}
%
%\begin{figure}
%
%\begin{lstlisting}[language=C]
%  | private_step_annot:
%      forall b ofs f ef args rs (m m': mwd LDATAOps) vargs t v cid,
%      rs Asm.PC = Vptr b ofs ->
%      Genv.find_funct_ptr ge b = Some (Internal f) ->
%      find_instr (Int.unsigned ofs) f.(fn_code) = Some (asm_instruction (Pannot ef args)) ->
%      annot_arguments rs m args vargs ->
%      external_call' (fun _ => True) ef ge vargs m t v m' ->
%      CPU_ID (snd m) = cid ->
%      forall BUILTIN_ENABLED: match ef with
%                                | EF_external _ _ => False
%                                | _ => True
%                              end,
%      private_exec ge cid (Asm.State rs m)
%           (Asm.State (nextinstr rs) m')
%
%    | private_step_external:
%        forall b ef args res rs (m m': mwd LDATAOps) t rs' cid,
%        rs Asm.PC = Vptr b Int.zero ->
%        Genv.find_funct_ptr ge b = Some (External ef) ->
%        extcall_arguments rs m (ef_sig ef) args ->
%        external_call' (fun _ => True) ef ge args m t res m' ->
%        rs' = (set_regs (loc_external_result (ef_sig ef)) res 
%                        (undef_regs (CR ZF :: CR CF :: CR PF :: CR SF :: CR OF :: nil) 
%                                    (undef_regs (map preg_of destroyed_at_call) rs))) 
%                #Asm.PC <- (rs RA) #RA <- Vundef ->
%        CPU_ID (snd m) = cid ->
%        forall STACK:
%        forall b o, rs ESP = Vptr b o ->
%                    (Ple (Genv.genv_next ge) b /\ Plt b (Mem.nextblock m)),
%        forall SP_NOT_VUNDEF: rs ESP <> Vundef,
%        forall RA_NOT_VUNDEF: rs RA <> Vundef,
%        private_exec ge cid (Asm.State rs m) (Asm.State rs' m')
%
%    | private_step_prim_call:
%        forall b ef rs (m m': mwd LDATAOps) t rs' cid,
%          rs Asm.PC = Vptr b Int.zero ->
%          Genv.find_funct_ptr ge b = Some (External ef) ->
%          primitive_call ef ge rs m t rs' m' ->
%          CPU_ID (snd m) = cid ->
%          private_exec ge cid (Asm.State rs m) (Asm.State rs' m').
%\end{lstlisting}
%\end{figure}
%
%\begin{figure}
%\begin{lstlisting}[language=C]
%    // shared executions - get and set    
%    Inductive get_shared_sem (ge: genv) : mstate -> list Integers.Byte.int -> mstate -> Prop:=
%    | get_shared_intro: 
%        forall rs sig m a index ofs id l b size ml0 z,
%          Genv.find_symbol ge id = Some b
%          -> id2size (Int.unsigned index) = Some (size, id)
%          -> Mem.loadbytes m b 0 size = Some (ByteList l)
%          -> sig = mksignature (AST.Tint::AST.Tint::nil) None cc_default
%          -> extcall_arguments rs m sig (Vint index :: Vint ofs ::nil) ->
%          forall (Hindex: index2Z (Int.unsigned index) (Int.unsigned ofs) = Some z)
%            (Hlog: ZMap.get z (multi_log a) = MultiDef ml0),
%            let rs' := (undef_regs (CR ZF :: CR CF :: CR PF :: CR SF :: CR OF
%                                       :: IR EAX :: RA :: nil)
%                                   (undef_regs (List.map preg_of destroyed_at_call) rs)) in
%            let a' := a {multi_log: ZMap.set z (MultiDef (TEVENT (CPU_ID a) (TSHARED (OMEME l)) :: ml0))
%                                             (multi_log a)} in
%            get_shared_sem ge (Asm.State (mem:= mwd LDATAOps) rs (m, a)) l 
%                           (Asm.State (rs'#RA <- Vundef#Asm.PC <- (rs RA)) (m, a')).
%\end{lstlisting}
%\end{figure}
%
%\begin{figure}
%\begin{lstlisting}[language=C]
%    Inductive set_shared_sem (ge: genv) : mstate -> option (list Integers.Byte.int) -> mstate -> Prop:=
%    | set_shared_sem_intro: 
%        forall rs sig m m' a index ofs id l b size ml0 z,
%          Genv.find_symbol ge id = Some b
%          -> id2size (Int.unsigned index) = Some (size, id)
%          -> match l with
%               | Some l' => Mem.storebytes m b 0 (ByteList l') = Some m'
%               | _ => m' = m
%             end
%          -> sig = mksignature (AST.Tint::AST.Tint::nil) None cc_default
%          -> extcall_arguments rs m sig (Vint index :: Vint ofs :: nil) ->
%          forall (Hindex: index2Z (Int.unsigned index) (Int.unsigned ofs) = Some z)
%                 (Hlog: ZMap.get z (multi_log a) = MultiDef ml0),
%            let rs' := (undef_regs (CR ZF :: CR CF :: CR PF :: CR SF :: CR OF
%                                       :: IR EAX :: RA :: nil)
%                                   (undef_regs (List.map preg_of destroyed_at_call) rs)) in
%            let a' := a {multi_log: ZMap.set z (MultiDef (TEVENT (CPU_ID a) (TSHARED OPULL) :: ml0))
%                                             (multi_log a)} in
%            set_shared_sem ge (Asm.State (mem:= mwd LDATAOps) rs (m, a)) l 
%                           (Asm.State (rs'#RA <- Vundef#Asm.PC <- (rs RA)) (m', a')).
%\end{lstlisting}
%\end{figure}

%\begin{figure}
%\begin{lstlisting}[language=C]
%    // atomic executions 
%    Inductive atomic_exec (ge: genv): Z -> Z -> mstate -> Log -> mstate -> OtherEvent -> Prop :=
%    | atomic_step_external:
%        forall b ef args res rs (m m': mwd LDATAOps) t rs' cid e id sl,
%        rs Asm.PC = Vptr b Int.zero ->
%        Genv.find_funct_ptr ge b = Some (External ef) ->
%        extcall_arguments rs m (ef_sig ef) args ->
%        external_call' (fun _ => True) ef ge args m t res m' ->
%        rs' = (set_regs (loc_external_result (ef_sig ef)) res (undef_regs (CR ZF :: CR CF :: CR PF :: CR SF :: CR OF :: nil) (undef_regs (map preg_of destroyed_at_call) rs))) #Asm.PC <- (rs RA) #RA <- Vundef ->
%        get_last_event (ZMap.get id (multi_log (snd m'))) = Some e ->
%        forall STACK:
%        forall b o, rs ESP = Vptr b o ->
%                    (Ple (Genv.genv_next ge) b /\ Plt b (Mem.nextblock m)),
%        forall SP_NOT_VUNDEF: rs ESP <> Vundef,
%        forall RA_NOT_VUNDEF: rs RA <> Vundef,
%          atomic_exec ge cid id (Asm.State rs m) sl (Asm.State rs' m') e
%
%    | atomic_step_prim_call:
%        forall b ef rs (m m': mwd LDATAOps) t rs' cid  e id sl,
%          rs Asm.PC = Vptr b Int.zero ->
%          Genv.find_funct_ptr ge b = Some (External ef) ->
%          primitive_call ef ge rs m t rs' m' ->
%          get_last_event (ZMap.get id (multi_log (snd m'))) = Some e ->
%          atomic_exec ge cid id (Asm.State rs m) sl (Asm.State rs' m') e.
%\end{lstlisting}
%\end{figure}




%
%
%\begin{figure}
%\noindent\fbox{abstract hardware setting, abstract command definitions ($\hardwaresetting$):}
%$$
%\begin{array}{lllr}
%\privatestate &:& \toptype & \mbox{(private state)}\\
%\sharedpiece &:& \toptype  & \mbox{(shared state: }\forall (s_1 \ s_2: \sharedpiece), \set{s_1 = s_2} + \set{s_1 \neq s_2}\mbox{)} \\
%\atomicevent &:& \toptype & \mbox{(atomic event: }\forall (e_1 \ e_2: \atomicevent), \set{e_1 = e_2} + \set{e_1 \neq e_2}\mbox{)} \\
%\atomiceventidentkwd & : & \atomicevent \rightarrow \primitiveid  & \mbox{(getting identifier of an atomic event)} \\
%\coreset & : & \set{\ztype} & \mbox{(set of cores - full core set for the multicore machine)}\\
%\schedid & : & \ztype & \mbox{(logical hardware scheduler CPU ID: } \schedid \notin \coreset\mbox{)}\\
%\end{array}
%$$
%$$
%\begin{array}{lll}
%\command &:=& \privatecmdkwd ~\vert~ \atomiccmdkwd\langle id : \ztype, e:\primitiveid\rangle\\
%&& \vert~ \acqsharedcmdkwd\langle id : \ztype\rangle~\vert~\relsharedcmdkwd\langle id : \ztype\rangle\\
%\mcevent &:=& \yieldevkwd\langle from : \ztype\rangle~\vert~ \yieldbackevkwd\langle to : \ztype\rangle\\
%&& \vert~ \acqevkwd\langle from : \ztype, id : \ztype\rangle\\
%&& \vert~\relevkwd\langle from :\ztype, id : \ztype, d:\sharedpiece\rangle\\
%&&\vert~\atomicevkwd\langle from : \ztype, id : \ztype, e:\atomicevent\rangle\\
%\mclog & := & \listconstructorkwd\ \mcevent\\
%\mcoracle{S : \set{\ztype}}{log : \toptype}{ret: \toptype} & : &  \ztype \rightarrow log \rightarrow ret \\
%\ogetkwd_{[S: \set{\ztype}, log : \toptype, ret : \toptype]} & : &   log \rightarrow \mcoracle{S}{log}{ret} \rightarrow ret \\
%\end{array}
%$$
%\noindent\fbox{own state:}
%$$
%\begin{array}{llll}
%\ownstatekwd & := & \ofreekwd\langle s : \optioncmd~\sharedpiece\rangle~\vert~\ownkwd\langle i : \ztype\rangle \\
%\end{array}
%$$
%\noindent\fbox{abstract hardware semantics ($\hardwaresemantics$) :}
%$$
%\begin{array}{llll}
%\programcounterkwd & : &  \privatestate \rightarrow \command \rightarrow \mcprop\\
%\privatestepkwd & : & \ztype  \rightarrow \privatestate \rightarrow \privatestate \rightarrow \mcprop \\
%\getsharedstepkwd & : & \privatestate \rightarrow \sharedpiece \rightarrow \privatestate \rightarrow  \rightarrow \mcprop \\ 
%\setsharedstepkwd & :  & \privatestate \rightarrow \optioncmd{\privatestate}  \rightarrow \privatestate \rightarrow \mcprop\\
%\atomicstepkwd & : &  \ztype \rightarrow \ztype  \rightarrow \privatestate \rightarrow \mclog \rightarrow \privatestate \rightarrow \atomicevent \rightarrow \mcprop  \\
%\end{array}
%$$
%\caption{Abstract Hardware Configuration and Semantics.}
%\label{fig:chapter:conlinkg:abstract-hardware-configuration-and-semantics}
%\end{figure}

\subsection{Machine Model Instances for Multicore Linking}
\label{chapter:certikos:subsec:intermediate-machine-instantiation}



\begin{figure}
\begin{lstlisting}[language=C]
// the concrete instance of <@$\color{red}\hardwarestepkwd$@>
Definition hw_step_aux :=  @hardware_step hdseting hdsem current_CPU_ID.
// extends [hw_step_aux] to the rule with global environment  
Inductive hw_step_aux_ge : genv -> hstate -> trace -> hstate -> Prop :=
  | hw_step_aux_ge_intro : 
    forall s1 t s2, hw_step_aux s1 t s2 -> hw_step_aux_ge ge s1 t s2.
\end{lstlisting}
\begin{center}
(a) Transition Rules with Hardware-configuration for Multicore Machine Model
\end{center}
\begin{lstlisting}[language=C, deletekeywords={int}]    
// initial state, which initiates all private states as well as a global log
Inductive hwstep_initial_state (p: AST.program fundef unit): 
  (hstate (hdset := hdseting)) -> Prop := 
  | initial_hwstep_state_intro: 
    forall (m0: mwd LDATAOps), Genv.init_mem p = Some m0 ->
      let ge := Genv.globalenv p in
      let rs0 := (Pregmap.init Vundef) 
           # PC <- (symbol_offset ge p.(prog_main) Int.zero) # ESP <- Vzero in
      hwstep_initial_state p (HState current_CPU_ID (pinit (B := core_set)
        (Asm.State rs0 m0)) nil).
// final state of the machine model 
Definition hwstep_final_state (s : hstate (hdset := hdseting)) (i : int) := False.
// The complete machine model for <@$\color{red}\hardwarestepkwd$@> with proper initial and final states  
Definition hwstep_semantics (p: program) :=
  Smallstep.Semantics hw_step_aux_ge (hwstep_initial_state p) 
    hwstep_final_state (Genv.globalenv p).
\end{lstlisting}
\begin{center}
(b) Multicore Machine Model (\lstinline$hwstep_semantics$)
\end{center}
\begin{lstlisting}[language=C]
Definition single_step_aux := @single_step <@$\cdots$@>  // the concrete instance of <@$\color{red}\singlestepkwd$@>
// extends [single_step_aux] to the rule with global environment  
Inductive single_step_aux_ge : genv -> single_state -> trace -> 
  single_state -> Prop :=
  | single_step_aux_ge_intro : 
    forall s1 t s2, single_step_aux s1 t s2 -> single_step_aux_ge ge s1 t s2.
\end{lstlisting}
\begin{center}
(c) Transition Rules with Hwardware-configuration for Single-core Machine Model
\end{center}
\begin{lstlisting}[language=C, deletekeywords={int}]    
// The complete machine model for <@$\color{red}\singlestepkwd$@> with proper initial and final states
Definition single_semantics (p: program) :=
  Smallstep.Semantics single_step_aux_ge (single_initial_state p) 
    single_final_state (Genv.globalenv p).
\end{lstlisting}
\begin{center}
(d) Single-core Machine Model (\lstinline$single_semantics$)
\end{center}
\caption{Intermediate Machine Model Instances for Multicore Linking in $\certikos$.}
\label{fig:chapter:certikos:multicore-machine-model-instances}
\end{figure}


Our intermediate machine models in Section~\ref{chapter:linking:sec:multicore-linking} are abstract languages, but our ultimate goal is to connect them with per-CPU local layers. 
This requires us to connect those abstract machine models with $\lasmmach$--the machine model of $\compcertx$--by using backward-simulation proofs in $\compcert$~\cite{leroy06} (\ie, it is also used in the layer linking of local layer interfaces  with $\compcertx$).
To make  this possible,
we provide  full concrete definitions for those abstract languages
by following the \lstinline$Smallstep$ library of $\compcert$.
Figure~\ref{fig:chapter:certikos:multicore-machine-model-instances}
shows two instances, which are the concrete definition 
of two machine models among the machine models in Section~\ref{chapter:linking:sec:multicore-linking}.
The first example is the base of our multicore-linking, $\intelmachine$ multicore machine model that corresponds to $\hardwarestepkwd$ in Section~\ref{chapter:linking:subsec:multicore-machine-model}.
Providing the concrete definition first requires 
us to connect the abstract machine definition in Figure~\ref{fig:chapter:conlink:multicore-machine-syntax-and-semantics}
with concrete definitions for the hardware setting and  hardware semantics in Figure~\ref{fig:chapter:conlinkg:abstract-hardware-configuration-and-semantics}, \lstinline$hdseting$ (see Figure~\ref{fig:chapter:ceritkos:instances-of-abstract-hardware-setting}) and \lstinline$hdsem$ (see Figure~\ref{fig:chapter:certikos:hardware-local-step-transition-rules}). 
Figure~\ref{fig:chapter:certikos:multicore-machine-model-instances} (a) shows 
the code that defines how we combine them.
Figure~\ref{fig:chapter:certikos:multicore-machine-model-instances} (b) 
also shows how we define the full semantics by using the  definition in Figure~\ref{fig:chapter:certikos:multicore-machine-model-instances} (a) as well as the \lstinline$Smallstep$ library of $\compcert$  for the $\intelmachine$ multicore machine model, which is the machine model 
where we run $\certikos$. 
Defining initial and final states is necessary for using the library, and Figure~\ref{fig:chapter:certikos:multicore-machine-model-instances} (b)  also provides those definitions.
For simplicity, we assume that the program does not have a final state, which we think it is not a big assumption 
on the verification of operating systems.

Figure~\ref{fig:chapter:certikos:multicore-machine-model-instances} (c) and (d) provides an another example--how 
we instantiate single core semantics--which corresponds to the abstract machine in Section~\ref{chapter:linking:subsec:single-core-machine-model}.
We omit the details, but they are defined in a similar manner to those definitions in Figure~\ref{fig:chapter:certikos:multicore-machine-model-instances} (a) and (b). 
Other abstract machine models 
in Section~\ref{chapter:linking:sec:multicore-linking} also can be defined in
 the same way. 
We also prove that all machine models 
are receptive--producing at most a single event per one evaluation--as well as deterministic, 
except for the multicore machine model because it is a non-deterministic machine model by nature.
For example, 
we show that
our single-core machine model satisfies the following three properties:
\begin{lstlisting}[language=C]
Lemma si_sem_single_events  (pl: program): single_events (single_semantics pl).
Lemma si_sem_receptive (pl: program): receptive (single_semantics pl).
Lemma si_sem_determinate (pl: program): determinate (single_semantics pl).
\end{lstlisting}
which are also defined in  the \lstinline$Smallstep$ library of $\compcert$ and
necessary for us to use 
the forward-to-backward simulation template in $\compcert$. 

\jieung{I can always add more texts in here}
%
% hdseting hdsem current_CPU_ID.
%
%
%%We provide those instances in this section. 
%We show how we instantiate each intermediate language 
%in this section. All of them follows the exactly same form, 
%and need to provide the evidence about the receptiveness and the single event property (each step will generate at most a single event of $\compcert$ trace). 
%To use the forward simulation implies the backward simulation technique~\cite{leroy06},
%providing the deterministic behavior of those languages 
%are necessary. 
%In this sense, 
%we have also proven those properties 
%for all languages, except the multicore machine mode, which is non-deterministic. 
%The remaining parts of this section 
%show those instances.
%\jieung{I can remove some of the definitions below (or remove most of them...}
%\jieung{Mention about the False final state}

%\begin{itemize}[leftmargin=*]
%\item Generic context definitions and hardware semantics instance with contexts for all languages we defined.
%\end{itemize}
%\begin{lstlisting}[language=C]
%Variables (ge: genv) (sten: stencil) (M: module).
%Context {Hmakege: make_globalenv (module_ops:= LAsm.module_ops)
%               (mkp_ops:= make_program_ops) 
%               sten M (mboot <@$\oplus$@> L64) = ret ge}.        
%    
%Local Obligation Tactic := intros.
%    
%Definition hdsem_instance := @hdsem mem memory_model_ops Hmem 
%  Hmwd real_params_ops oracle_ops0 oracle_ops big_ops 
%  builtin_idents_norepet_prf ge sten M Hmakege.    
%\end{lstlisting}

%\begin{itemize}[leftmargin=*]
%\item Generic context definitions and hardware semantics instance with contexts for all languages we defined.
%\end{itemize}
%\begin{lstlisting}[language=C]
%Variables (ge: genv) (sten: stencil) (M: module).
%Context {Hmakege: make_globalenv (module_ops:= LAsm.module_ops)
%               (mkp_ops:= make_program_ops) 
%               sten M (mboot <@$\oplus$@> L64) = ret ge}.        
%    
%Local Obligation Tactic := intros.
%    
%Definition hdsem_instance := @hdsem <@$\cdots$@>
%\end{lstlisting}


%\subsubsection{Multicore Machine Model}


%\begin{itemize}[leftmargin=*]
%\item Properties of hardware step semantics (proofs are omitted)
%\end{itemize}
%\begin{lstlisting}[language=C]    
%Lemma hwstep_semantics_single_events: forall p, single_events (hwstep_semantics p).
%Lemma hwstep_semantics_receptive (pl: program):  receptive (hwstep_semantics pl).
%\end{lstlisting}
%
%
%\subsubsection{Multicore Machine Model with Hardware Oracle}
%
%Final state definitions for all languages from here is same with the final state definition for Multicore Machine Model. In this sense, 
%we have omitted them in our definitions. 
%
%\begin{itemize}[leftmargin=*]
%\item Hardware step with hardware oracle transition rule instance
%\end{itemize}
%\begin{lstlisting}[language=C]
%Definition oracle_step_aux :=
%  @oracle_step zset_op hdseting op_general hdsem_instance pmap hw_oracle.
%
%Inductive oracle_step_aux_ge : genv -> state -> trace -> state -> Prop :=
%  | oracle_step_aux_ge_intro : 
%    forall s1 t s2, oracle_step_aux s1 t s2 -> oracle_step_aux_ge ge s1 t s2.
%\end{lstlisting}
%
%\begin{itemize}[leftmargin=*]
%\item Hardware step with hardware oracle semantics definitions (including initial state definition)
%\end{itemize}
%\begin{lstlisting}[language=C]
%Inductive oracle_initial_state (p: AST.program fundef unit): 
%  (state (hdset := hdseting)) -> Prop := 
%  | initial_oracle_state_intro: 
%    forall (m0: mwd LDATAOps),
%      Genv.init_mem p = Some m0 ->
%      let ge := Genv.globalenv p in
%      let rs0 :=
%        (Pregmap.init Vundef)
%        # Asm.PC <- (symbol_offset ge p.(prog_main) Int.zero)
%        # ESP <- Vzero in
%      oracle_initial_state p (State current_CPU_ID 
%        (pset current_CPU_ID (LState (Asm.State rs0 m0) true)
%        (pinit (B := core_set) (LState (Asm.State rs0 m0) false))) nil).
%
%Definition oracle_final_state (s : state (hdset := hdseting)) (i : int)
%  : Prop := False.
%      
%Definition oracle_semantics (p: program) :=
%  Smallstep.Semantics oracle_step_aux_ge (oracle_initial_state p) 
%    oracle_final_state (Genv.globalenv p). 
%\end{lstlisting}
%
%\begin{itemize}[leftmargin=*]
%\item Properties of hardware step with hardware oracle semantics (proofs are omitted)
%\end{itemize}
%\begin{lstlisting}[language=C]
%Lemma oracle_semantics_single_events: forall p, single_events (oracle_semantics p).
%Lemma oracle_semantics_receptive (pl: program): receptive (oracle_semantics pl).
%Lemma oracle_semantics_determinate (pl: program): determinate (oracle_semantics pl).
%\end{lstlisting}
%
%\subsubsection{Environmental Step Machine}
%
%Properties that we have proven from here for all machines are same with the properties for  hardware step with hardware oracle semantics,
%\textit{single$\_$events}, \textit{receptive}, \textit{determinate}. 
%In this sense, we omitted them from this environmental step machine.
%
%\begin{itemize}[leftmargin=*]
%\item Environmental step transition rule instance. Since they can run an arbitrary number of cores with it, 
%the instance of this machine still has two parameters, $\codeinmath{core}$ (a focused set) and $\codeinmath{o}$ (an environmental context for 
%the focused set)
%\end{itemize}
%\begin{lstlisting}[language=C]
%Variable cores : ZSet.
%Variable o : GeneralOracleType.
%Definition env_step_aux:=
%  @env_step zset_op hdseting op_general hdsem_instance pmap cores o.
%
%Inductive env_step_aux_ge : genv -> (estate (A:= cores))  -> trace -> (estate (A:= cores)) -> Prop :=
%  | env_step_aux_ge_intro : 
%    forall s1 t s2, env_step_aux s1 t s2 -> env_step_aux_ge ge s1 t s2.
%\end{lstlisting}
%
%\begin{itemize}[leftmargin=*]
%\item Environmental step  semantics definitions (including initial state definition)
%\end{itemize}
%\begin{lstlisting}[language=C]
%Inductive env_initial_state (p: AST.program fundef unit): 
%  (estate (A := cores) (hdset := hdseting)) -> Prop :=
%  | initial_env_state_intro: 
%    forall (m0: mwd LDATAOps),
%       Genv.init_mem p = Some m0 ->
%       let ge := Genv.globalenv p in
%       let rs0 :=
%         (Pregmap.init Vundef)
%         # Asm.PC <- (symbol_offset ge p.(prog_main) Int.zero)
%         # ESP <- Vzero in
%       env_initial_state p (EState current_CPU_ID 
%         (pset current_CPU_ID (LState (Asm.State rs0 m0) true)
%         (pinit (B := cores) (LState (Asm.State rs0 m0) false))) nil).
%
%Definition env_semantics (p: program) :=
%  Smallstep.Semantics env_step_aux_ge (env_initial_state p) 
%    env_final_state (Genv.globalenv p).    
%\end{lstlisting}
%
%\subsubsection{Single Step Machine}    
%\begin{itemize}[leftmargin=*]
%\item Environmental step transition rule instance.
%\end{itemize}
%\begin{lstlisting}[language=C]
%Definition single_step_aux :=
%  @single_step zset_op hdseting op_general hdsem_instance 
%  current_CPU_ID single_oracle.
%
%Inductive single_step_aux_ge : genv -> single_state -> trace -> 
%  single_state -> Prop :=
%  | single_step_aux_ge_intro : 
%    forall s1 t s2, single_step_aux s1 t s2 -> single_step_aux_ge ge s1 t s2.
%\end{lstlisting}
%
%
%\begin{itemize}[leftmargin=*]
%\item Single step semantics definitions (including initial state definition)
%\end{itemize}
%\begin{lstlisting}[language=C]
%Inductive single_initial_state (p: AST.program fundef unit): 
%  (single_state (hdset := hdseting)) -> Prop :=
%  | initial_single_state_intro: 
%    forall (m0: mwd LDATAOps),
%      Genv.init_mem p = Some m0 ->
%      let ge := Genv.globalenv p in
%      let rs0 :=
%        (Pregmap.init Vundef)
%        # Asm.PC <- (symbol_offset ge p.(prog_main) Int.zero)
%        # ESP <- Vzero in
%      single_initial_state p (SState current_CPU_ID 
%        (LState (Asm.State rs0 m0) true) nil).
%
%Definition single_semantics (p: program) :=
%  Smallstep.Semantics single_step_aux_ge (single_initial_state p) 
%    single_final_state (Genv.globalenv p).
%\end{lstlisting}
%
%%\subsubsection{Big Step Machine}
%%From here to Big 2 step semantics,
%%our model requires fairness assumptions about hardware scheduler, which is notated as $\codeinmath{fair}$ in the semantics definitions.
%%\begin{itemize}[leftmargin=*]
%%\item Big step transition rule instance.
%%\end{itemize}
%%\begin{lstlisting}[language=C]
%%Definition single_big_step_aux :=
%%  @single_big_step zset_op hdseting op_general hdsem_instance 
%%  fair current_CPU_ID single_oracle.
%%
%%Inductive single_big_step_aux_ge : genv -> single_state -> trace -> 
%%  single_state -> Prop :=
%%  | single_big_step_aux_ge_intro : 
%%    forall s1 t s2, single_big_step_aux s1 t s2 -> single_big_step_aux_ge ge s1 t s2.
%%\end{lstlisting}
%
%
%%\begin{itemize}[leftmargin=*]
%%\item Big step semantics definitions (including initial state definition)
%%\end{itemize}
%%\begin{lstlisting}[language=C]
%%Inductive single_big_initial_state (p: AST.program fundef unit): 
%%  (single_state (hdset := hdseting)) -> Prop :=
%%  | initial_big_state_intro: 
%%    forall (m0: mwd LDATAOps),
%%      Genv.init_mem p = Some m0 ->
%%      let ge := Genv.globalenv p in
%%      let rs0 :=
%%        (Pregmap.init Vundef)
%%         # Asm.PC <- (symbol_offset ge p.(prog_main) Int.zero)
%%         # ESP <- Vzero in
%%      single_big_initial_state p (SState current_CPU_ID 
%%        (LState (Asm.State rs0 m0) true) nil).
%%      
%%Definition single_big_semantics (p: program) :=
%%  Smallstep.Semantics single_big_step_aux_ge (single_big_initial_state p) 
%%    single_big_final_state (Genv.globalenv p).
%%\end{lstlisting}
%%
%%\subsubsection{Big 2 Step Machine}
%%\begin{itemize}[leftmargin=*]
%%\item Big two step transition rule instance.
%%\end{itemize}
%%\begin{lstlisting}[language=C]
%%Definition single_big2_step_aux :=
%%  @single_big2_step zset_op hdseting op_general hdsem_instance 
%%   fair current_CPU_ID single_oracle.
%%
%%Inductive single_big2_step_aux_ge : genv -> rstate -> trace -> rstate -> Prop :=
%% | single_big2_step_aux_ge_intro : 
%%   forall s1 t s2, single_big2_step_aux s1 t s2 -> single_big2_step_aux_ge ge s1 t s2.
%%\end{lstlisting}
%
%
%%\begin{itemize}[leftmargin=*]
%%\item Big two step semantics definitions (including initial state definition)
%%\end{itemize}
%%\begin{lstlisting}[language=C]
%%Inductive single_big2_initial_state (p: AST.program fundef unit): 
%%  (rstate (hdset := hdseting)) -> Prop :=
%%  | initial_big2_state_intro: 
%%    forall (m0: mwd LDATAOps),
%%      Genv.init_mem p = Some m0 ->
%%      let ge := Genv.globalenv p in
%%      let rs0 :=
%%        (Pregmap.init Vundef)
%%        # Asm.PC <- (symbol_offset ge p.(prog_main) Int.zero)
%%        # ESP <- Vzero in
%%      single_big2_initial_state p (RState (Asm.State rs0 m0) nil).
%%
%%Definition single_big2_semantics (p: program) :=
%%  Smallstep.Semantics single_big2_step_aux_ge (single_big2_initial_state p) 
%%    single_big2_final_state (Genv.globalenv p).
%%\end{lstlisting}
%
%%\subsubsection{Split Step Machine}
%%\begin{itemize}[leftmargin=*]
%%\item Split step transition rule instance.
%%\end{itemize}
%%\begin{lstlisting}[language=C]
%%Definition single_split_step_aux :=
%%  @single_split_step zset_op hdseting op_general hdsem_instance 
%%    fair current_CPU_ID single_oracle.
%%
%%Inductive single_split_step_aux_ge : genv -> srstate -> trace -> srstate -> Prop :=
%%  | single_split_step_aux_ge_intro : 
%%  forall s1 t s2, single_split_step_aux s1 t s2 -> single_split_step_aux_ge ge s1 t s2.
%%\end{lstlisting}
%%
%%
%%\begin{itemize}[leftmargin=*]
%%\item Split step semantics definitions (including initial state definition)
%%\end{itemize}
%%\begin{lstlisting}[language=C]
%%Inductive single_split_initial_state (p: AST.program fundef unit): 
%%  (srstate (hdset := hdseting)) -> Prop :=
%%  | initial_split_state_intro: 
%%    forall (m0: mwd LDATAOps),
%%      Genv.init_mem p = Some m0 ->
%%      let ge := Genv.globalenv p in
%%      let rs0 :=
%%        (Pregmap.init Vundef)
%%        # Asm.PC <- (symbol_offset ge p.(prog_main) Int.zero)
%%        # ESP <- Vzero in
%%     single_split_initial_state p (SRState (Asm.State rs0 m0) nil nil).
%%          
%%Definition single_split_semantics (p: program) :=
%%  Smallstep.Semantics single_split_step_aux_ge (single_split_initial_state p) 
%%    single_split_final_state (Genv.globalenv p).
%%\end{lstlisting}
%%
%%\subsubsection{Reorder Step Machine}
%%\begin{itemize}[leftmargin=*]
%%\item Reorder step transition rule instance. Reorder step relies on different kinds of environmental context. In this sense, 
%%it is parameterized by environmental contexts.
%%\end{itemize}
%%\begin{lstlisting}[language=C]
%%Variable reorder_o : ReorderOracleType.
%%
%%Definition single_reorder_step_aux :=
%%  @single_reorder_step zset_op hdseting op_reorder hdsem_instance 
%%  current_CPU_ID reorder_o.
%%  
%%Inductive single_reorder_step_aux_ge : genv -> rstate -> trace -> rstate -> Prop :=
%%  | single_reorder_step_aux_ge_intro : 
%%    forall s1 t s2, single_reorder_step_aux s1 t s2 -> 
%%      single_reorder_step_aux_ge ge s1 t s2.  
%%\end{lstlisting}
%
%%
%%\begin{itemize}[leftmargin=*]
%%\item Reorder step semantics definitions (including initial state definition)
%%\end{itemize}
%%\begin{lstlisting}[language=C]
%%Inductive single_reorder_initial_state (p: AST.program fundef unit): 
%%  (rstate (hdset := hdseting)) -> Prop :=
%%  | initial_reorder_state_intro: 
%%    forall (m0: mwd LDATAOps),
%%      Genv.init_mem p = Some m0 ->
%%      let ge := Genv.globalenv p in
%%      let rs0 :=
%%        (Pregmap.init Vundef)
%%        # Asm.PC <- (symbol_offset ge p.(prog_main) Int.zero)
%%        # ESP <- Vzero in
%%      single_reorder_initial_state p (RState (Asm.State rs0 m0) nil).
%%
%%Definition single_reorder_semantics (p: program) :=
%%  Smallstep.Semantics single_reorder_step_aux_ge (single_reorder_initial_state p) 
%%    single_reorder_final_state (Genv.globalenv p).
%%\end{lstlisting}
%
%\subsubsection{Separate Step Machine}
%\begin{itemize}[leftmargin=*]
%\item Separate step transition rule instance.
%\end{itemize}
%\begin{lstlisting}[language=C]
%Definition single_separate_step_aux :=
%  @single_separate_step
%  hdseting separate_log_len zset_op op_sep hdsem_instance
%  current_CPU_ID sep_oracle.
%    
%Inductive single_separate_step_aux_ge : genv -> sp_state -> trace -> 
%  sp_state -> Prop :=
%   | single_separate_step_aux_ge_intro : 
%     forall s1 t s2, single_separate_step_aux s1 t s2 -> 
%       single_separate_step_aux_ge ge s1 t s2.
%\end{lstlisting}
%
%\begin{itemize}[leftmargin=*]
%\item Separate step semantics definitions (including initial state definition)
%\end{itemize}
%\begin{lstlisting}[language=C]
%Inductive single_separate_initial_state (p: AST.program fundef unit): 
%  (sp_state (hdset := hdseting)) -> Prop :=
%  | initial_separate_state_intro: 
%    forall (m0: mwd LDATAOps),
%      Genv.init_mem p = Some m0 ->
%      let ge := Genv.globalenv p in
%      let rs0 :=
%        (Pregmap.init Vundef)
%         # Asm.PC <- (symbol_offset ge p.(prog_main) Int.zero)
%         # ESP <- Vzero in
%      single_separate_initial_state p (SPState (Asm.State rs0 m0) (ZMap.init nil)).
%
%Definition single_separate_semantics (p: program) :=
%  Smallstep.Semantics single_separate_step_aux_ge (single_separate_initial_state p) 
%    single_separate_final_state (Genv.globalenv p).
%\end{lstlisting}



\subsection{Connect Layers on Multicore Machine and CPU-Local Machine}
\label{chapter:certikos:subsec:connect-multicore}

All refinement proofs in Section~\ref{chapter:linking:sec:multicore-linking} are also applicable 
to those instances. 
All those proofs requires the additional refinement proofs for between adjacent step relations, but they can 
use the corresponding theorems in Section~\ref{chapter:linking:sec:multicore-linking}. 
For example, the refinement proof between an environmental step machine with a single core and the environmental step machine with full core set is 
as follows:
\begin{lstlisting}[language=C]
Lemma one_step_env_refines_env_concrete:
  forall st t st' st0
    (Hone: env_step_single_aux_ge' ge st t st')
    (Hmatch: match_eestate_link st st0),
    (exists st0',
      (plus env_step_full_aux_ge') ge st0 t st0' /\
        match_eestate_link st' st0') \/
      ((state_measure (s_set current_CPU_ID) st' < state_measure (s_set current_CPU_ID) st)%nat /\ t = E0 /\
        match_eestate_link st' st0).
Proof.
  <@$\cdots$@>
   eapply single_env_step_refines_full_env_step in Hone; eauto.
   <@$\cdots$@>
Qed.
\end{lstlisting}
, which uses the proof in Section~\ref{chapter:linking:subsec:concurrent-machine-model}.
Using those theorems, 
we prove the backward simulation proofs (in $\compcert$ style) 
\begin{lstlisting}[language=C]
Theorem cl_backward_simulation:
  forall (s: stencil) (CTXT: LAsm.module) (ph: AST.program fundef unit)
           (Hmakep: make_program (module_ops:= LAsm.module_ops) s CTXT 
           (mboot <@$\oplus$@> L64) = OK ph),
  backward_simulation
   (env_semantics (s_set current_CPU_ID) single_oracle
   (Hmakege := make_program_globalenv (make_program_ops := make_program_ops)
      _ _ _ _ Hmakep) (Genv.globalenv ph) s CTXT ph)
   (env_semantics core_set hw_oracle
   (Hmakege := make_program_globalenv (make_program_ops := make_program_ops) 
     _ _ _ _ Hmakep) (Genv.globalenv ph) s CTXT ph).
\end{lstlisting}

The top level theorem for multicore linking uses the backward simulation proofs 
for each steps that we have described in Section~\ref{chapter:linking:sec:multicore-linking}.
The top level theorem is in Figure~\ref{fig:chapter:certikos:top-level-multicore-theorem}.

\begin{figure}
\begin{lstlisting}[language=C]
Theorem concurrent_backward_simulation:
  forall (s: stencil) (CTXT: LAsm.module) (ph: AST.program fundef unit)
    (Hmakep: make_program (module_ops:= LAsm.module_ops) s CTXT (mboot <@$\oplus$@> L64) = OK ph),
    backward_simulation
    (LAsm.semantics (lcfg_ops := LC (mboot <@$\oplus$@> L64)) ph)          
    (hwstep_semantics 
      (Hmakege := make_program_globalenv (make_program_ops := make_program_ops) 
      _ _ _ _ Hmakep) (Genv.globalenv ph) s CTXT ph).
\end{lstlisting}
\caption{Top Level Multicore Linking Theorem}
\label{fig:chapter:certikos:top-level-multicore-theorem}
\end{figure}

The proof, of course has several assumptions. 
\jieung{need to mention assumptions}

