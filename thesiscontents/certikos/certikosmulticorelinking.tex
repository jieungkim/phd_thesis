\section{Multicore Linking for CertiKOS}
\label{chapter:certikos:sec:multicore-linking-for-certikos}

Multicore linking in CertiKOS facilitates all the intermediate languages
and proofs in Section~\ref{chapter:linking:sec:multicore-linking}.
Figure~\ref{fig:chapter:conlink:intermediate-languages-and-their-relationsihps-for-multicore-linking} and 
Section~\ref{chapter:linking:sec:multicore-linking} show all definitions and 
proofs that connect all those things. 
However, to build the local layer interface, 
Instantiating the hardware configuration, 
use the concrete definition $L$, which will be the 
bottom layer of our local layer interface, 
and combine those intermediate languages and proofs with $\compcert$ semantic and backward simulation 
forms to connect them rigorously. 

\subsection{Hardware Configuration Instance}
\label{chapter:certikos:subsec:hardware-configuration-instance}

The hardware configuration in Section~\ref{chapter:linking:subsec:hardware-configuration} works as a 
template for users to instantiate all intermediate machine models in multicore linking. 
To show the instance for that configuration,
we first briefly show the state definition of $\compcertx$
as well as the lowest layer in $\certikos$, which includes TCB functions such as the functions in Section~\ref{chapter:mcslock:sec:verification}.
Those TCB functions are 
necessary for supporting extended features that $\compcert$ cannot support. 
For example, 
Fetch-and-Increase or Compare-and-Swap functions are atomic operations that are crucial to build a 
concurrent object by providing the lowest-level atomicity. 
They, however, are not supported by $\compcert$'s instructions.
Our bottom (TCB) layer, the parameter of the machine model for $\compcertx$ 
provides the primitive (which gives the extra transition rules for the state of $\compcertx$) . 
%
%\begin{figure}
%\begin{lstlisting}[language=Caml]
%Inductive val: Type :=
%  | Vundef: val
%  | Vint: int -> val
%  | Vlong: int64 -> val
%  | Vfloat: float -> val
%  | Vptr: block -> int -> val.
%
%Definition regset := Pregmap.t val.


%\end{lstlisting}
%\caption{$\compcertx$ State Definition.}
%\label{fig:chapter:certikos:compcertx-state-definition}
%\end{figure}

The $\compcertx$ state definition is
\begin{lstlisting}
Inductive state <@$\cdots$@> : Type :=
  | State: regset -> mem -> state.
\end{lstlisting}
The $\compcertx$ state is defined as a pair, 
a register and a memory. 
The memory in the $\compcertx$ is also a pair of two elements, 
a memory (a map from a identifier to a value) and an abstract data. 
The abstract data can contain 
multiple fields in it. 
To use the existing tools as much as possible, 
we also include our global log in the abstract data (as we have seen in Chapter~\ref{chapter:mcs-lock}.
\jieung{Need to add more descriptions about our state definition}

\begin{figure}
\begin{lstlisting}[language=Caml]
Definition mboot_passthrough : compatlayer (cdata RData) :=
  <@$\cdots$@>
  <@$\oplus$@> get_CPU_ID <@$\mapsto$@> gensem get_CPU_ID_spec
  <@$\oplus$@>  acquire_shared <@$\mapsto$@> primcall_acquire_shared_compatsem acquire_shared0_spec0
  <@$\oplus$@>  atomic_FAI <@$\mapsto$@> primcall_atomic_FAI_compatsem atomic_FAI_spec
  <@$\cdots$@>
\end{lstlisting}
\caption{Bottom Layer Interface}
\label{fig:chapter:certikos:bottom-layer-interface}
\end{figure}

\begin{figure}
\begin{lstlisting}[language=Caml]
Record RData :=
  mkRData {
     // current CPU_ID
      CPU_ID: Z;
     // we model the memory from 1G to 3G as heap            
      HP: flatmem;     
      // abstract of CR3, stores the pointer to page table
      CR3: globalpointer;
      // set of logs for shared resources - we divide the global log to multiple
      // logs for each shared objects for simplicity   
      multi_log: MultiLogPool;     
     <@$\cdots$@>
     }.
\end{lstlisting}
\caption{Abstract Data for TCB Layer}
\label{fig:chapter:certikos:abstract-data-for-bottom-layer}
\end{figure}

The layer definition for the bottom layer is in Figure~\ref{chapter:certikos:subsec:hardware-configuration-instance},
which is basically a partial map from a primitive ID ($\primitiveid$) to the specifications ($\spec$).
and the abstract data for the bottom layer is in Figure~\ref{fig:chapter:certikos:abstract-data-for-bottom-layer}.
As mentioned, 
we have provided atomic fetch and increase as an atomic operation in the layer 
and it will update the log in the abstract data ($\codeinmath{multi\_log}$), which is 
a part of our state definition. 

Our hardware configuration instance is closely related to those definitions. 
Atomic operations in the configuration (Figure~\ref{fig:chapter:conlinkg:abstract-hardware-configuration-and-semantics}), for example,
need to include $\codeinmath{atomic\_FAI}$  
in it and the local evaluation rule in the atomic statement in Figure~\ref{fig:chapter:conlink:hardware-local-step-transition-rules}  has to 
consistent with the specification of the primitive.
%
%\begin{lstlisting}
%  Inductive OtherEvent :=
%  (* Events used for the ticket-lock primitives: *)
%  | OINC_TICKET (n: nat)
%  | OINC_NOW
%  | OGET_NOW
%  | OINIT
%  | OHOLD_LOCK
%  | OPAGE_COPY (b: PageBuffer)
%  .
%
%  Function OtherEvent_2_ident (e: OtherEvent) :=
%    match e with
%      | OINC_TICKET _ => atomic_FAI
%      | OINC_NOW => log_incr
%      | OGET_NOW => log_get
%      | OINIT => log_init
%      | OHOLD_LOCK => log_hold
%      | OPAGE_COPY _ => page_copy
%    end.
%
%  Global Instance hdseting: HardWaredSetting :=
%    {
%      private_state := mstate;
%      shared_piece := list Integers.Byte.int;
%      atomic_event := OtherEvent;
%      atomic_event_ident := OtherEvent_2_ident;
%      core_set := cpu_set;
%      sched_id := 9
%    }.
%\end{lstlisting}

With those things, 
we have to instance all the definitions in Figure~\ref{chapter:certikos:subsec:hardware-configuration-instance}.

The private state can be instantiated with 
the state definition in the machine model of $\compcertx$,
and the shared piece is a list of byte, which represents 
the snapshot of the shared memory via $\push$/$\pull$ operations. 
Atomic events are defined above, 
which is mapped with the atomic operation that the bottom layer of our interface provide.
For example, \lstinline$OINC_TICKET (n: nat)$ corresponds to the fetch and increase primitive,
and it instantiates the definition $\mcevent$ in Figure~\ref{chapter:certikos:subsec:hardware-configuration-instance}.
\begin{lstlisting}[language=Caml]
  Inductive OtherEvent :=
  (* Events used for the ticket-lock primitives: *)
  | OINC_TICKET (n: nat)
  <@$\cdots$@>
\end{lstlisting}
In addition, \lstinline$OtherEvent_2_ident$ is an instance for $\atomiceventidentkwd$ function, which is the function
that gets the event and returns the corresponding identifier (the primitive identifier) that is related to the event.
\begin{lstlisting}[language=Caml]
  Function OtherEvent_2_ident (e: OtherEvent) :=
    match e with
      | OINC_TICKET _ => atomic_FAI
      <@$\cdots$@>
    end.
\end{lstlisting}
And, the \lstinline$core_set$ is defined as a full CPU set ($\coreset$), and 
the set contains the 8 CPUs in it from 0 to 7.
The last element is the identifier for 
the scheduler, which should not in the \lstinline$cpu_set$. 
we designated 8 for the scheduler ID.
Those configurations are all included in the 
\lstinline$hdsetting$ class, and the instance of the type class
is as follows:
\begin{lstlisting}[language=Caml]
Global Instance hdseting: HardWaredSetting := {
  private_state := mstate;
  shared_piece := list Integers.Byte.int;
  atomic_event := OtherEvent;
  atomic_event_ident := OtherEvent_2_ident;
  core_set := cpu_set;
  sched_id := 8 }.
\end{lstlisting}
, which is mapped with 
the proper atomic operations, number of CPUs (allows 8 CPUs), 
as well as the state definition. 


%\begin{lstlisting}[language=Caml]
%  Inductive step (ge: genv): state -> trace -> state -> Prop :=
%    | exec_step_internal:
%        forall b ofs f i rs m rs' m',
%        rs PC = Vptr b ofs ->
%        Genv.find_funct_ptr ge b = Some (Internal f) ->
%        find_instr (Int.unsigned ofs) f.(fn_code) = Some i ->
%        exec_instr ge f i rs m = Next rs' m' ->
%        step ge (State rs m) E0 (State rs' m')
%  | exec_step_builtin:
%      forall b ofs f ef args res rs m t vl rs' m',
%      rs PC = Vptr b ofs ->
%      Genv.find_funct_ptr ge b = Some (Internal f) ->
%      find_instr (Int.unsigned ofs) f.(fn_code) = Some (asm_instruction (Pbuiltin ef args res)) ->
%      external_call' (fun _ => True) ef ge (map rs args) m t vl m' ->
%      rs' = nextinstr_nf 
%             (set_regs res vl
%               (undef_regs (map preg_of (destroyed_by_builtin ef)) rs)) ->
%(** [CertiKOS:test-compcert-disable-extcall-as-builtin] We need
%to disallow the use of external function calls (EF_external) as
%builtins. *)
%      forall BUILTIN_ENABLED: match ef with
%                                | EF_external _ _ => False
%                                | _ => True
%                              end,
%(** [CompCertX:test-compcert-wt-builtin] We need to prove that registers updated by builtins are
%    of the same type as the return type of the builtin. *)
%      forall BUILTIN_WT: 
%               subtype_list (proj_sig_res' (ef_sig ef)) (map typ_of_preg res) = true,
%      step ge (State rs m) t (State rs' m')
%  | exec_step_annot:
%      forall b ofs f ef args rs m vargs t v m',
%      rs PC = Vptr b ofs ->
%      Genv.find_funct_ptr ge b = Some (Internal f) ->
%      find_instr (Int.unsigned ofs) f.(fn_code) = Some (asm_instruction (Pannot ef args)) ->
%      annot_arguments rs m args vargs ->
%      external_call' (fun _ => True) ef ge vargs m t v m' ->
%      forall BUILTIN_ENABLED: match ef with
%                                | EF_external _ _ => False
%                                | _ => True
%                              end,
%      step ge (State rs m) t
%           (State (nextinstr rs) m')
%    | exec_step_external:
%        forall b ef args res rs m t rs' m',
%        rs PC = Vptr b Int.zero ->
%        Genv.find_funct_ptr ge b = Some (External ef) ->
%        (* XXX maybe flip those two to keep the same order as before *)
%        extcall_arguments rs m (ef_sig ef) args ->
%        external_call' (fun _ => True) ef ge args m t res m' ->
%        (** [CompCertX:test-compcert-undef-destroyed-by-call] We erase non-callee-save registers. *)
%      (** [CompCertX:test-compcert-ra-vundef] We need to erase the value of RA,
%      which is actually popped away from the stack in reality. *)
%      rs' = (set_regs (loc_external_result (ef_sig ef)) res (undef_regs (CR ZF :: CR CF :: CR PF :: CR SF :: CR OF :: nil) (undef_regs (map preg_of destroyed_at_call) rs))) #PC <- (rs RA) #RA <- Vundef ->
%        forall STACK:
%        forall b o, rs ESP = Vptr b o ->
%                    (Ple (Genv.genv_next ge) b /\ Plt b (Mem.nextblock m)),
%  (** [CompCertX:test-compcert-protect-stack-arg] The following
%      NOT_VUNDEF conditions will allow to later parameterize the
%      per-function/module semantics of back-end languages over the stack
%      pointer and the return address. *)
%        forall SP_NOT_VUNDEF: rs ESP <> Vundef,
%        forall RA_NOT_VUNDEF: rs RA <> Vundef,
%        step ge (State rs m) t (State rs' m')
%    | exec_step_prim_call:
%        forall b ef rs m t rs' m',
%          rs PC = Vptr b Int.zero ->
%          Genv.find_funct_ptr ge b = Some (External ef) ->
%          primitive_call ef ge rs m t rs' m' ->
%          step ge (State rs m) t (State rs' m').
%
%  (** The low-level LAsm machine use the same initial states as the
%    regular Compcert Asm, however high-level, non-main thread machines
%    use a different entry point. We use the following class as a
%    parameter so that we can account for this. *)
%
%  Class AsmInitialState :=
%    {
%      asm_init_state {F V}: AST.program F V -> state -> Prop;
%
%      asm_init_state_determ {F V} (p : AST.program F V):
%        forall s1 s2, asm_init_state p s1 -> asm_init_state p s2 -> s1 = s2
%    }.
%
%  (** Compared to the original CompCert [Asm.initial_state], this
%    version leaves RA to be [Vundef] rather than set it to [Vzero].
%    This makes it easier to treat the initial state of the main thread
%    and that of other threads uniformly, since threads which start as
%    a "return from yield" have their RA undefined. *)
%
%  Inductive initial_state {F V} (p: AST.program F V): state -> Prop :=
%    | initial_state_intro m0:
%        Genv.init_mem p = Some m0 ->
%        let rs0 :=
%          (Pregmap.init Vundef)
%          # PC <- (symbol_offset (Genv.globalenv p) (prog_main p) Int.zero)
%          # ESP <- Vzero in
%        initial_state p (State rs0 m0).
%
%  Global Program Instance asm_init_default : AsmInitialState | 10 :=
%    {
%      asm_init_state F V := initial_state
%    }.
%  Next Obligation.
%    destruct H; subst rs0.
%    destruct H0; subst rs0.
%    congruence.
%  Qed.
%
%  Context `{asm_init : AsmInitialState}.
%
%  Definition final_state (s : state) (i : int) : Prop :=
%    False.
%
%  Definition semantics (p: program) :=
%    Smallstep.Semantics step (asm_init_state p) final_state (Genv.globalenv p).
%\end{lstlisting}
%
%\begin{figure}
%\begin{lstlisting}[language=Caml]
%Function get_last_event (l: MultiLogType) :=
%  match l with
%  | MultiUndef => None
%  | MultiDef nil => Some OINIT
%  | MultiDef (e:: l) =>
%    match e with
%    | TEVENT _ e0 =>
%      match e0 with
%      | TTICKET e1 =>
%        match e1 with
%        | INC_TICKET n => Some (OINC_TICKET n)
%        | INC_NOW => Some OINC_NOW       
%        | GET_NOW => Some OGET_NOW
%        | HOLD_LOCK => Some OHOLD_LOCK
%        | _ => None
%        end
%      | TSHARED (OBUFFERE b) => Some (OPAGE_COPY b)
%      | _ => None
%      end
%    end
%  end.
%\end{lstlisting} 
%\end{figure}



%\begin{figure}
%
%\begin{lstlisting}[language=Caml]
%  Local Open Scope Z_scope.
%
%  Definition genv := Genv.t fundef unit.
%    
%  // to determine PC from the state 
%  Inductive command_predicate (ge: genv): mstate -> command -> Prop :=
%  | acq_prim_call:
%      forall b ef rs m lid index ofs sig,
%        rs Asm.PC = Vptr b Int.zero ->
%        Genv.find_funct_ptr ge b = Some (External ef) ->
%        sig = mksignature (AST.Tint::AST.Tint::nil) None cc_default ->
%        extcall_arguments rs m sig (Vint index :: Vint ofs :: nil) ->
%        index2Z (Int.unsigned index) (Int.unsigned ofs) = Some lid ->
%        match ef with
%          | EF_external eid _ => 
%            if peq eid acquire_shared then True
%            else False
%          | _ => False
%        end ->
%        command_predicate ge (Asm.State rs m) (ACQ_SHARED lid)
%
%  | rel_prim_call:
%      forall b ef rs m lid index ofs sig,
%        rs Asm.PC = Vptr b Int.zero ->
%        Genv.find_funct_ptr ge b = Some (External ef) ->
%        sig = mksignature (AST.Tint::AST.Tint::nil) None cc_default ->
%        extcall_arguments rs m sig (Vint index :: Vint ofs :: nil) ->
%        index2Z (Int.unsigned index) (Int.unsigned ofs) = Some lid ->
%        match ef with
%          | EF_external eid _ => 
%            if peq eid release_shared then True
%            else False
%          | _ => False
%        end ->
%        command_predicate ge (Asm.State rs m) (REL_SHARED lid)
%\end{lstlisting}
%\end{figure}

%\begin{figure}
%\begin{lstlisting}[language=Caml]
%  | command_prim_atomic:
%      forall b ef rs m lid index ofs sig eid,
%        rs Asm.PC = Vptr b Int.zero ->
%        Genv.find_funct_ptr ge b = Some (External ef) ->
%        atomic_id eid ->
%        eid <> page_copy ->
%        match ef with
%          | EF_external eid' _ =>
%            if peq eid eid' then True
%            else False
%          | _ => False
%        end ->
%        sig = mksignature (AST.Tint::AST.Tint::nil) None cc_default ->
%        extcall_arguments rs m sig (Vint index :: Vint ofs :: nil) ->
%        index2Z (Int.unsigned index) (Int.unsigned ofs) = Some lid ->
%        command_predicate ge (Asm.State rs m) (ATOMIC lid eid)
%
%  | command_prim_page_copy:
%      forall b ef rs m lid cv count from sig,
%        rs Asm.PC = Vptr b Int.zero ->
%        Genv.find_funct_ptr ge b = Some (External ef) ->
%        match ef with
%          | EF_external eid _ => 
%            if peq eid page_copy then True
%            else False
%          | _ => False
%        end ->
%        sig = mksignature (AST.Tint::AST.Tint::AST.Tint::nil) None cc_default ->
%        extcall_arguments rs m sig (Vint cv :: Vint count :: Vint from :: nil) ->
%        index2Z ID_SC (slp_q_id (Int.unsigned cv) 0) = Some lid ->
%        command_predicate ge (Asm.State rs m) (ATOMIC lid page_copy)
%\end{lstlisting}
%\end{figure}
%
%\begin{figure}

%\begin{lstlisting}[language=Caml]
%  | command_prim_FAI:
%      forall b ef rs m lid index ofs sig bound,
%        rs Asm.PC = Vptr b Int.zero ->
%        Genv.find_funct_ptr ge b = Some (External ef) ->
%        match ef with
%          | EF_external eid _ => 
%            if peq eid atomic_FAI then True
%            else False
%          | _ => False
%        end ->
%        sig = mksignature (AST.Tint::AST.Tint::AST.Tint::nil) None cc_default ->
%        extcall_arguments rs m sig (Vint bound :: Vint index :: Vint ofs :: nil) ->
%        index2Z (Int.unsigned index) (Int.unsigned ofs) = Some lid ->
%        command_predicate ge (Asm.State rs m) (ATOMIC lid atomic_FAI)
%
%  | command_internal:
%      forall b ofs f rs m,
%        rs Asm.PC = Vptr b ofs ->
%        Genv.find_funct_ptr ge b = Some (Internal f) ->
%        command_predicate ge (Asm.State rs m) PRIVATE
%
%  | command_prim_private:
%      forall b ef rs m,
%        rs Asm.PC = Vptr b Int.zero ->
%        Genv.find_funct_ptr ge b = Some (External ef) ->
%        match ef with
%          | EF_external eid _ => 
%            private_id eid
%          | _ => True
%        end ->
%        command_predicate ge (Asm.State rs m) PRIVATE.
%\end{lstlisting}
%\end{figure}

%
%\begin{figure}
%\begin{lstlisting}[language=Caml]
%  Global Instance hdseting: HardWaredSetting :=
%    {
%      private_state := mstate;
%      shared_piece := list Integers.Byte.int;
%      atomic_event := OtherEvent;
%      atomic_event_ident := OtherEvent_2_ident;
%      core_set := cpu_set;
%      sched_id := 9
%    }.
%  Proof.
%    - intros.
%      eapply list_eq_dec. intros.
%      eapply Byte.eq_dec.
%    - repeat decide equality.
%      eapply Int.eq_dec.
%    - trivial.
%  Defined.
%\end{lstlisting}
%\end{figure}
%
%\begin{figure}
%
%\begin{lstlisting}[language=Caml]
%  // private execution - similar to Asm
%  Inductive private_exec (ge: genv): Z -> mstate -> mstate -> Prop :=
%    | private_step_internal:
%        forall b ofs f i rs (m m': mwd LDATAOps) rs' cid,
%        rs Asm.PC = Vptr b ofs ->
%        Genv.find_funct_ptr ge b = Some (Internal f) ->
%        find_instr (Int.unsigned ofs) f.(fn_code) = Some i ->
%        exec_instr ge f i rs m = Next rs' m' ->
%        CPU_ID (snd m) = cid ->
%        private_exec ge cid (Asm.State rs m) (Asm.State rs' m')
%
%  | private_step_builtin:
%      forall b ofs f ef args res rs (m m': mwd LDATAOps) t vl rs' cid,
%      rs Asm.PC = Vptr b ofs ->
%      Genv.find_funct_ptr ge b = Some (Internal f) ->
%      find_instr (Int.unsigned ofs) f.(fn_code) = Some (asm_instruction (Pbuiltin ef args res)) ->
%      external_call' (fun _ => True) ef ge (map rs args) m t vl m' ->
%      rs' = nextinstr_nf 
%             (set_regs res vl
%               (undef_regs (map preg_of (destroyed_by_builtin ef)) rs)) ->
%      CPU_ID (snd m) = cid ->
%     //  [CertiKOS:test-compcert-disable-extcall-as-builtin] We need
%     // to disallow the use of external function calls (EF_external) as
%     //  builtins. 
%      forall BUILTIN_ENABLED: match ef with
%                                | EF_external _ _ => False
%                                | _ => True
%                              end,
%    // [CompCertX:test-compcert-wt-builtin] We need to prove that registers updated by builtins are
%    // of the same type as the return type of the builtin. 
%      forall BUILTIN_WT: 
%               subtype_list (proj_sig_res' (ef_sig ef)) (map typ_of_preg res) = true,
%        private_exec ge cid (Asm.State rs m) (Asm.State rs' m')
%\end{lstlisting}
%\end{figure}
%
%\begin{figure}
%
%\begin{lstlisting}[language=Caml]
%  | private_step_annot:
%      forall b ofs f ef args rs (m m': mwd LDATAOps) vargs t v cid,
%      rs Asm.PC = Vptr b ofs ->
%      Genv.find_funct_ptr ge b = Some (Internal f) ->
%      find_instr (Int.unsigned ofs) f.(fn_code) = Some (asm_instruction (Pannot ef args)) ->
%      annot_arguments rs m args vargs ->
%      external_call' (fun _ => True) ef ge vargs m t v m' ->
%      CPU_ID (snd m) = cid ->
%      forall BUILTIN_ENABLED: match ef with
%                                | EF_external _ _ => False
%                                | _ => True
%                              end,
%      private_exec ge cid (Asm.State rs m)
%           (Asm.State (nextinstr rs) m')
%
%    | private_step_external:
%        forall b ef args res rs (m m': mwd LDATAOps) t rs' cid,
%        rs Asm.PC = Vptr b Int.zero ->
%        Genv.find_funct_ptr ge b = Some (External ef) ->
%        extcall_arguments rs m (ef_sig ef) args ->
%        external_call' (fun _ => True) ef ge args m t res m' ->
%        rs' = (set_regs (loc_external_result (ef_sig ef)) res 
%                        (undef_regs (CR ZF :: CR CF :: CR PF :: CR SF :: CR OF :: nil) 
%                                    (undef_regs (map preg_of destroyed_at_call) rs))) 
%                #Asm.PC <- (rs RA) #RA <- Vundef ->
%        CPU_ID (snd m) = cid ->
%        forall STACK:
%        forall b o, rs ESP = Vptr b o ->
%                    (Ple (Genv.genv_next ge) b /\ Plt b (Mem.nextblock m)),
%        forall SP_NOT_VUNDEF: rs ESP <> Vundef,
%        forall RA_NOT_VUNDEF: rs RA <> Vundef,
%        private_exec ge cid (Asm.State rs m) (Asm.State rs' m')
%
%    | private_step_prim_call:
%        forall b ef rs (m m': mwd LDATAOps) t rs' cid,
%          rs Asm.PC = Vptr b Int.zero ->
%          Genv.find_funct_ptr ge b = Some (External ef) ->
%          primitive_call ef ge rs m t rs' m' ->
%          CPU_ID (snd m) = cid ->
%          private_exec ge cid (Asm.State rs m) (Asm.State rs' m').
%\end{lstlisting}
%\end{figure}
%
%\begin{figure}
%\begin{lstlisting}[language=Caml]
%    // shared executions - get and set    
%    Inductive get_shared_sem (ge: genv) : mstate -> list Integers.Byte.int -> mstate -> Prop:=
%    | get_shared_intro: 
%        forall rs sig m a index ofs id l b size ml0 z,
%          Genv.find_symbol ge id = Some b
%          -> id2size (Int.unsigned index) = Some (size, id)
%          -> Mem.loadbytes m b 0 size = Some (ByteList l)
%          -> sig = mksignature (AST.Tint::AST.Tint::nil) None cc_default
%          -> extcall_arguments rs m sig (Vint index :: Vint ofs ::nil) ->
%          forall (Hindex: index2Z (Int.unsigned index) (Int.unsigned ofs) = Some z)
%            (Hlog: ZMap.get z (multi_log a) = MultiDef ml0),
%            let rs' := (undef_regs (CR ZF :: CR CF :: CR PF :: CR SF :: CR OF
%                                       :: IR EAX :: RA :: nil)
%                                   (undef_regs (List.map preg_of destroyed_at_call) rs)) in
%            let a' := a {multi_log: ZMap.set z (MultiDef (TEVENT (CPU_ID a) (TSHARED (OMEME l)) :: ml0))
%                                             (multi_log a)} in
%            get_shared_sem ge (Asm.State (mem:= mwd LDATAOps) rs (m, a)) l 
%                           (Asm.State (rs'#RA <- Vundef#Asm.PC <- (rs RA)) (m, a')).
%\end{lstlisting}
%\end{figure}
%
%\begin{figure}
%\begin{lstlisting}[language=Caml]
%    Inductive set_shared_sem (ge: genv) : mstate -> option (list Integers.Byte.int) -> mstate -> Prop:=
%    | set_shared_sem_intro: 
%        forall rs sig m m' a index ofs id l b size ml0 z,
%          Genv.find_symbol ge id = Some b
%          -> id2size (Int.unsigned index) = Some (size, id)
%          -> match l with
%               | Some l' => Mem.storebytes m b 0 (ByteList l') = Some m'
%               | _ => m' = m
%             end
%          -> sig = mksignature (AST.Tint::AST.Tint::nil) None cc_default
%          -> extcall_arguments rs m sig (Vint index :: Vint ofs :: nil) ->
%          forall (Hindex: index2Z (Int.unsigned index) (Int.unsigned ofs) = Some z)
%                 (Hlog: ZMap.get z (multi_log a) = MultiDef ml0),
%            let rs' := (undef_regs (CR ZF :: CR CF :: CR PF :: CR SF :: CR OF
%                                       :: IR EAX :: RA :: nil)
%                                   (undef_regs (List.map preg_of destroyed_at_call) rs)) in
%            let a' := a {multi_log: ZMap.set z (MultiDef (TEVENT (CPU_ID a) (TSHARED OPULL) :: ml0))
%                                             (multi_log a)} in
%            set_shared_sem ge (Asm.State (mem:= mwd LDATAOps) rs (m, a)) l 
%                           (Asm.State (rs'#RA <- Vundef#Asm.PC <- (rs RA)) (m', a')).
%\end{lstlisting}
%\end{figure}

%\begin{figure}
%\begin{lstlisting}[language=Caml]
%    // atomic executions 
%    Inductive atomic_exec (ge: genv): Z -> Z -> mstate -> Log -> mstate -> OtherEvent -> Prop :=
%    | atomic_step_external:
%        forall b ef args res rs (m m': mwd LDATAOps) t rs' cid e id sl,
%        rs Asm.PC = Vptr b Int.zero ->
%        Genv.find_funct_ptr ge b = Some (External ef) ->
%        extcall_arguments rs m (ef_sig ef) args ->
%        external_call' (fun _ => True) ef ge args m t res m' ->
%        rs' = (set_regs (loc_external_result (ef_sig ef)) res (undef_regs (CR ZF :: CR CF :: CR PF :: CR SF :: CR OF :: nil) (undef_regs (map preg_of destroyed_at_call) rs))) #Asm.PC <- (rs RA) #RA <- Vundef ->
%        get_last_event (ZMap.get id (multi_log (snd m'))) = Some e ->
%        forall STACK:
%        forall b o, rs ESP = Vptr b o ->
%                    (Ple (Genv.genv_next ge) b /\ Plt b (Mem.nextblock m)),
%        forall SP_NOT_VUNDEF: rs ESP <> Vundef,
%        forall RA_NOT_VUNDEF: rs RA <> Vundef,
%          atomic_exec ge cid id (Asm.State rs m) sl (Asm.State rs' m') e
%
%    | atomic_step_prim_call:
%        forall b ef rs (m m': mwd LDATAOps) t rs' cid  e id sl,
%          rs Asm.PC = Vptr b Int.zero ->
%          Genv.find_funct_ptr ge b = Some (External ef) ->
%          primitive_call ef ge rs m t rs' m' ->
%          get_last_event (ZMap.get id (multi_log (snd m'))) = Some e ->
%          atomic_exec ge cid id (Asm.State rs m) sl (Asm.State rs' m') e.
%\end{lstlisting}
%\end{figure}

All our intermediate languages 
also relies on the abstract transition rules for 
private, shared, release, and atomic steps, and the special auxiliary function, $\programcounterkwd$.
Similar to the previous definitions, 
it also facilitate the layer definition (the $\codeinmath{mboot}$ layer), 
and the $\compcertx$ definition.

For example, the program counter command
for fetch-and-increase function call is
\begin{lstlisting}[language=Caml]
Inductive command_predicate (ge: genv): mstate -> command -> Prop :=
<@$\cdots$@>
| command_prim_FAI:
  <@$\cdots$@>
  rs Asm.PC = Vptr b Int.zero -> Genv.find_funct_ptr ge b = Some (External ef) ->
  match ef with  | EF_external eid _ => if peq eid atomic_FAI then True else False
  <@$\cdots$@>
  command_predicate ge (Asm.State rs m) (ATOMIC lid atomic_FAI)
<@$\cdots$@>
\end{lstlisting}
which shows the case when the current function pointer is pointing out the \lstinline$atomic_FAI$
In the rule, \lstinline$(ge :genv)$ stands for the global environment that contains the 
global information such as the function names, global variables of the whole programs.
\jieung{Need additional explanation}

Similarly, all abstract hardware semantics in  Figure~\ref{fig:chapter:conlinkg:abstract-hardware-configuration-and-semantics}.
are defined 
by slightly modifying the step relation in $\compcertx$ 
and by the case analysis for the primitive identifiers. 
For instance, the evaluation relation for get shared is
\begin{lstlisting}[language=Caml]
 Inductive get_shared_sem (ge: genv) 
   : mstate -> list Integers.Byte.int -> mstate -> Prop:=
   | get_shared_intro: 
    <@$\cdots$@>
    Genv.find_symbol ge id = Some b -> id2size (Int.unsigned index) = Some (size, id)
    -> Mem.loadbytes m b 0 size = Some (ByteList l)
    <@$\cdots$@>
     let a' := a {multi_log: ZMap.set z (MultiDef (TEVENT (CPU_ID a)
       (TSHARED (OMEME l)) :: ml0)) (multi_log a)} in
    get_shared_sem ge (Asm.State (mem:= mwd LDATAOps) rs (m, a)) l 
       (Asm.State (rs'#RA <- Vundef#Asm.PC <- (rs RA)) (m, a')).
\end{lstlisting}
%\begin{lstlisting}[language=Caml]
%    Inductive get_shared_sem (ge: genv) : 
%        mstate -> list Integers.Byte.int -> mstate -> Prop:=
%    | get_shared_intro: 
%        forall rs sig m a index ofs id l b size ml0 z,
%          Genv.find_symbol ge id = Some b
%          -> id2size (Int.unsigned index) = Some (size, id)
%          -> Mem.loadbytes m b 0 size = Some (ByteList l)
%          -> sig = mksignature (AST.Tint::AST.Tint::nil) None cc_default
%          -> extcall_arguments rs m sig (Vint index :: Vint ofs ::nil) ->
%          forall (Hindex: index2Z (Int.unsigned index) (Int.unsigned ofs) = Some z)
%            (Hlog: ZMap.get z (multi_log a) = MultiDef ml0),
%            let rs' := (undef_regs (CR ZF :: CR CF :: CR PF :: CR SF :: CR OF
%                                       :: IR EAX :: RA :: nil)
%                                   (undef_regs (List.map preg_of destroyed_at_call) rs)) in
%            let a' := a {multi_log: ZMap.set z (MultiDef (TEVENT (CPU_ID a) (TSHARED (OMEME l)) :: ml0))
%                                             (multi_log a)} in
%            get_shared_sem ge (Asm.State (mem:= mwd LDATAOps) rs (m, a)) l 
%                           (Asm.State (rs'#RA <- Vundef#Asm.PC <- (rs RA)) (m, a')).
%\end{lstlisting}
which will only update the PC for the private state,
and update the global log in the abstract data.
Similarly, the atomic execution rule is defined as follows:
\begin{lstlisting}[language=Caml]
Inductive atomic_exec (ge: genv):
  Z -> Z -> mstate -> Log -> mstate -> OtherEvent -> Prop :=
  | atomic_step_external:
  <@$\cdots$@>
  atomic_exec ge cid id (Asm.State rs m) sl (Asm.State rs' m') e
  <@$\cdots$@>
\end{lstlisting}

The last piece of the hardware configuration then can be instantiated as in Figure~\ref{fig:chapter:certikos:hardware-local-step-transition-rules},
which contains all concrete definitions for the rules of ``abstract hardware semantics'' in  Figure~\ref{fig:chapter:conlinkg:abstract-hardware-configuration-and-semantics}.
We collect all those things in the \lstinline$HardSemantics$ class that is parameterized by the hardware setting that we defined above. 
\begin{figure}
\begin{lstlisting}[language=Caml]
      
      Variables (ge: genv) (sten: stencil) (M: module).
      Context {Hmakege: make_globalenv (module_ops:= LAsm.module_ops) (mkp_ops:= make_program_ops) 
                                       sten M (mboot <@$\oplus$@> L64) = ret ge}.
      
      Local Obligation Tactic := intros.
      
      Global Program Instance hdsem : HardSemantics (hdset:= hdseting)
        :=
          {
            PC := command_predicate ge;
            private_step := private_exec ge;
            get_shared := get_shared_sem ge;
            set_shared := set_shared_sem ge;
            atomic_step := atomic_exec ge
          }.
\end{lstlisting}
\caption{Instance for the Hardware Local Transition Rules.}
\label{fig:chapter:certikos:hardware-local-step-transition-rules}
\end{figure}
We also need the proofs for the property of hardware semantics, which is defined in Figure~\ref{fig:chapter:conlink:properties-of-abstract-hardware-semantics-for-multicore-linking}. 
Most of them are straightforward, because our instances are using the definitions in $\compcertx$,
which already contains the deterministic properties. 

Note that we are using the assembly model in $\compcertx$ for $\intelmachine$, which implies that 
our instance for this hardware configuration is 
for the $\intelmachine$ multicore machine,
that has a subset of its instructions (but sufficiently rich for building operating systems) with a small TCB, 
and follows the sequentially consistent evaluation order for the memory consistency. 

It, however, does not need to be specified by $\intelmachine$. 
All  those hardware configurations can be instantiated with other definitions, such as ARM assembly machine for $\compcertx$, 
or even other machine model that is not entirely related to $\compcertx$ if users can prove the desired property of it. 

\subsection{Intermediate Machine Instantiation}
\label{chapter:certikos:subsec:intermediate-machine-instantiation}

Our intermediate machine models are abstract languages, but 
our goal is connecting them with backward simulation proofs in $\compcert$~\jieung{need to cite back-end paper}
For that purpose, 
we need to instantiate those abstract languages to make them fit into 
the semantics definition in $\compcert$.
They are related to
providing instances 
for the abstract definitions that intermediate languages relies on. 
We provide those instances in this section. 

\subsubsection{Multicore Machine Model}

\begin{lstlisting}[language=Caml]
    Variables (ge: genv) (sten: stencil) (M: module).
    Context {Hmakege: make_globalenv (module_ops:= LAsm.module_ops)
                  (mkp_ops:= make_program_ops) 
                  sten M (mboot <@$\oplus$@> L64) = ret ge}.        
    
    Local Obligation Tactic := intros.
    
    Definition hdsem_instance := @hdsem mem memory_model_ops Hmem 
      Hmwd real_params_ops oracle_ops0 oracle_ops big_ops 
      builtin_idents_norepet_prf ge sten M Hmakege.
    
    // now, define a concrete semantics for single core machine
    Definition hw_step_aux :=
      @hardware_step hdseting hdsem_instance pmap current_CPU_ID.
      
    Inductive hw_step_aux_ge : genv -> hstate -> trace -> hstate -> Prop :=
    | hw_step_aux_ge_intro : 
        forall s t s',
            hw_step_aux s t s' -> hw_step_aux_ge ge s t s'.
    
    Inductive hwstep_initial_state (p: AST.program fundef unit): 
      (hstate (hdset := hdseting)) -> Prop := 
    | initial_hwstep_state_intro: 
        forall (m0: mwd LDATAOps),
          Genv.init_mem p = Some m0 ->
          let ge := Genv.globalenv p in
          let rs0 :=
              (Pregmap.init Vundef)
                # Asm.PC <- (symbol_offset ge p.(prog_main) Int.zero)
                # ESP <- Vzero in
          hwstep_initial_state p (HState current_CPU_ID (pinit (B := core_set)
            (Asm.State rs0 m0)) nil).

    Definition hwstep_final_state (s : hstate (hdset := hdseting)) (i : int) : Prop :=
      False.
      
         
    Definition hwstep_semantics (p: program) :=
      Smallstep.Semantics hw_step_aux_ge (hwstep_initial_state p) 
                          hwstep_final_state (Genv.globalenv p).
    
\end{lstlisting}

\subsubsection{Multicore Machine Model with Hardware Oracle}

\begin{lstlisting}[language=Caml]
   
    Variables (ge: genv) (sten: stencil) (M: module).
    Context {Hmakege: make_globalenv (module_ops:= LAsm.module_ops) (mkp_ops:= make_program_ops) 
                                     sten M (mboot <@$\oplus$@> L64) = ret ge}.
    
    Local Obligation Tactic := intros.
    
    Definition hdsem_instance := @hdsem mem memory_model_ops Hmem Hmwd real_params_ops oracle_ops0
                                        oracle_ops big_ops builtin_idents_norepet_prf ge sten M Hmakege.

    // now, define a concrete semantics for single core machine
    Definition oracle_step_aux :=
      @oracle_step zset_op hdseting op_general hdsem_instance pmap hw_oracle.

    Inductive oracle_step_aux_ge : genv -> state -> trace -> state -> Prop :=
    | oracle_step_aux_ge_intro : 
        forall s t s',
          oracle_step_aux s t s' -> oracle_step_aux_ge ge s t s'.

    Inductive oracle_initial_state (p: AST.program fundef unit): 
      (state (hdset := hdseting)) -> Prop := 
    | initial_oracle_state_intro: 
        forall (m0: mwd LDATAOps),
          Genv.init_mem p = Some m0 ->
          let ge := Genv.globalenv p in
          let rs0 :=
              (Pregmap.init Vundef)
                # Asm.PC <- (symbol_offset ge p.(prog_main) Int.zero)
                # ESP <- Vzero in
          oracle_initial_state p (State current_CPU_ID 
                                        (pset current_CPU_ID (LState (Asm.State rs0 m0) true)
                                              (pinit (B := core_set) (LState (Asm.State rs0 m0) false)))
                                        nil).

    Definition oracle_final_state (s : state (hdset := hdseting)) (i : int) : Prop :=
      False.
      
    Definition oracle_semantics (p: program) :=
      Smallstep.Semantics oracle_step_aux_ge (oracle_initial_state p) 
                          oracle_final_state (Genv.globalenv p). 
\end{lstlisting}



\begin{lstlisting}[language=Caml]
    Variable cores : ZSet.
    Variable o : GeneralOracleType.
    Variables (ge: genv) (sten: stencil) (M: module).
    Context {Hmakege: make_globalenv (module_ops:= LAsm.module_ops) (mkp_ops:= make_program_ops) 
                                     sten M (mboot <@$\oplus$@> L64) = ret ge}.
    
    Local Obligation Tactic := intros.
    
    Definition hdsem_instance := @hdsem mem memory_model_ops Hmem Hmwd real_params_ops oracle_ops0
                                        oracle_ops big_ops builtin_idents_norepet_prf ge sten M Hmakege.
    
    // now, define a concrete semantics for single core machine 
    Definition env_step_aux:=
      @env_step zset_op hdseting op_general hdsem_instance pmap cores o.
    
    Inductive env_step_aux_ge : genv -> (estate (A:= cores))  -> trace -> (estate (A:= cores)) -> Prop :=
    | env_step_aux_ge_intro : 
        forall s t s',
          env_step_aux s t s' -> env_step_aux_ge ge s t s'.
    
    Inductive env_initial_state (p: AST.program fundef unit): 
      (estate (A := cores) (hdset := hdseting)) -> Prop :=
    | initial_env_state_intro: 
        forall (m0: mwd LDATAOps),
          Genv.init_mem p = Some m0 ->
          let ge := Genv.globalenv p in
          let rs0 :=
              (Pregmap.init Vundef)
                # Asm.PC <- (symbol_offset ge p.(prog_main) Int.zero)
                # ESP <- Vzero in
          env_initial_state p (EState current_CPU_ID 
                                      (pset current_CPU_ID (LState (Asm.State rs0 m0) true)
                                            (pinit (B := cores) (LState (Asm.State rs0 m0) false)))
                                      nil).
    

    Definition env_final_state (s : estate (A:=cores) (hdset := hdseting)) (i : int) : Prop :=
      False.
      

    Definition env_semantics (p: program) :=
      Smallstep.Semantics env_step_aux_ge (env_initial_state p) 
                          env_final_state (Genv.globalenv p).
    
\end{lstlisting}



\begin{lstlisting}[language=Caml]

    Variables (ge: genv) (sten: stencil) (M: module).
    Context {Hmakege: make_globalenv (module_ops:= LAsm.module_ops) (mkp_ops:= make_program_ops) 
                                     sten M (mboot <@$\oplus$@> L64) = ret ge}.

    Local Obligation Tactic := intros.
    
    Definition hdsem_instance := @hdsem mem memory_model_ops Hmem Hmwd real_params_ops oracle_ops0
                                        oracle_ops big_ops builtin_idents_norepet_prf ge sten M Hmakege.

    // now, define a concrete semantics for reorder  machine 
    Definition single_step_aux :=
      @single_step zset_op hdseting op_general hdsem_instance 
                   current_CPU_ID single_oracle.

    Inductive single_step_aux_ge : genv -> single_state -> trace -> single_state -> Prop :=
    | single_step_aux_ge_intro : 
        forall s t s',
          single_step_aux s t s' -> single_step_aux_ge ge s t s'.

    Inductive single_initial_state (p: AST.program fundef unit): 
      (single_state (hdset := hdseting)) -> Prop :=
    | initial_single_state_intro: 
        forall (m0: mwd LDATAOps),
          Genv.init_mem p = Some m0 ->
          let ge := Genv.globalenv p in
          let rs0 :=
              (Pregmap.init Vundef)
                # Asm.PC <- (symbol_offset ge p.(prog_main) Int.zero)
                # ESP <- Vzero in
          single_initial_state p (SState current_CPU_ID (LState (Asm.State rs0 m0) true) nil).

    Definition single_final_state (s : single_state (hdset := hdseting)) (i : int) : Prop :=
      False.
      
    Definition single_semantics (p: program) :=
      Smallstep.Semantics single_step_aux_ge (single_initial_state p) 
                          single_final_state (Genv.globalenv p).

\end{lstlisting}


\begin{lstlisting}[language=Caml]
\end{lstlisting}

\begin{lstlisting}[language=Caml]
\end{lstlisting}

\begin{lstlisting}[language=Caml]
\end{lstlisting}

\begin{lstlisting}[language=Caml]
\end{lstlisting}

\begin{lstlisting}[language=Caml]
\end{lstlisting}


\begin{lstlisting}[language=Caml]
\end{lstlisting}


\clearpage
\subsection{Connect Layers on Multicore Machine and CPU-Local Machine}
\label{chapter:certikos:subsec:connect-multicore}

\begin{lstlisting}[language=Caml]
    Variables (ge: genv) (sten: stencil) (M: module).
    Context {Hmakege: make_globalenv (module_ops:= LAsm.module_ops) (mkp_ops:= make_program_ops) 
                                       sten M (mboot <@$\oplus$@>  L64) = ret ge}.

    Definition hdsem_instance := @hdsem mem memory_model_ops Hmem Hmwd real_params_ops oracle_ops0
                                        oracle_ops big_ops builtin_idents_norepet_prf ge sten M Hmakege.

    Definition oracle_step_aux_ge' :=
      @oracle_step_aux_ge mem memory_model_ops Hmem Hmwd
                          real_params_ops oracle_ops0 oracle_ops big_ops
                          builtin_idents_norepet_prf pmap zset_op mc_oracle
                          ge sten M Hmakege.

    Definition hw_step_aux_ge' :=
      @hw_step_aux_ge mem memory_model_ops Hmem Hmwd
                      real_params_ops oracle_ops0 oracle_ops big_ops
                      builtin_idents_norepet_prf pmap
                      ge sten M Hmakege.
                      
    Definition match_state_link := match_state (hdset := hdseting) (pmap := pmap) current_CPU_ID.
    
    Hint Unfold match_state_link.
    Existing Instance op_general.                      
                      
\end{lstlisting}

\begin{lstlisting}[language=Caml]


    // backward refinement
    Lemma one_step_hw_refines_oracle_concrete:
      forall s s0 s' t
             (Hone: hw_step_aux_ge' ge s t s')
             (Hmatch: match_state_link s0 s),
      exists s0',
        plus oracle_step_aux_ge' ge s0 t s0'
        /\ match_state_link s0' s'.
    Proof.

  Theorem cl_backward_simulation:
    forall (s: stencil) (CTXT: LAsm.module) (ph: AST.program fundef unit)
           (Hmakep: make_program (module_ops:= LAsm.module_ops) s CTXT (mboot <@$\oplus$@> L64) = OK ph),
      backward_simulation
        (oracle_semantics 
           (Hmakege := make_program_globalenv (make_program_ops := make_program_ops) _ _ _ _ Hmakep)
           (Genv.globalenv ph) s CTXT ph)
        (hwstep_semantics 
           (Hmakege := make_program_globalenv (make_program_ops := make_program_ops) _ _ _ _ Hmakep)
           (Genv.globalenv ph) s CTXT ph).
  Proof.
\end{lstlisting}

Connecting all proofs for mluticore linking is then possible by using the result of previous sections.