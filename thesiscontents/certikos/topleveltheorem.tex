\section{Top-level Theorem in CertiKOS}
\label{chapter:certikos:sec:top-level-theorem}


Using all results described in this chapter, 
we finally show the top-level single theorem for the correctness of $\certikos$, 
described in Theorem~\ref{theorem:chapter:certikos:contextual-refinement-theorem-for-certikos}. 

\begin{theorem}[Contextual Refinement Theorem for $\certikos$]
\label{theorem:chapter:certikos:contextual-refinement-theorem-for-certikos}
Let's assume that $tid$ is a valid thread ID, $\oracle_{tid}$ is an
environmental context for thread $tid$, $\codeinmath{certikos}$ is a implementation of $\certikos$ with C and assembly, and $\codeinmath{Ctxt}$ is a
 context program that runs on top of the $\certikos$. 
 Then, we want to prove that 
 \begin{center}
\begin{tabular}{c}
$\semwmachine{\mmcsbootfull}{\codeinmath{CertiKOS} \oplus \codeinmath{Ctxt}}{\codeinmath{mach}_{\codeinmath{x86mc}}} \refines_{R_{\codeinmath{certikos}}} \semwmachine{\PLayer{\TSyscall}{tid}{\oracle_{tid}}}{\codeinmath{Ctxt}}{\hasmmach}$\\
\end{tabular}
\end{center}
when $R_{\codeinmath{certikos}}$ is a refinement relation between those two different layers on different machine models (\ie, connecting all refinement relations of multicore linking, per-CPU layers, multithreaded linking, and per-thread layers).
\end{theorem}

The proof of this theorem is 
straightforward with 
the lemmas that we have stated in previous sections.
It still requires multiple assumptions that we discussed 
in Section~\ref{chapter:linking:sec:multicore-linking} and Section~\ref{chapter:linking:sec:multithreaded-linking}. 
Providing a better framework than the current version remains our future work. 

%
%\begin{proof}
%First, we assume that 
%
%\end{proof}

%In $\coq$, the theorem is stated as follows:
%
%\begin{lstlisting}[language=C]
%  Theorem CertiKOS_correct:
%    forall (s: stencil) (CTXT: LAsm.module) k1 k2 p combined_program user_program,
%      genv_next s = init_nb ->
%      CertiKOS.per_thread_certikos = OK k1 ->
%      CertiKOS.per_cpu_certikos = OK k2 ->
%      forall Hmp: make_program s ((CTXT <@$\oplus$@> k1) <@$\oplus$@> k2) (MBoot.mboot <@$\oplus$@> L64) = OK combined_program,
%      make_program s (CTXT <@$\oplus$@> k1) (PHThread.phthread <@$\oplus$@> L64) = OK p ->
%      make_program s CTXT (TSysCall.tsyscall <@$\oplus$@> L64) = OK user_program ->
%      noglobvar p ->
%      inhabited
%        (backward_simulation
%           (HAsm.semantics (TSysCall.tsyscall <@$\oplus$@> L64) user_program)
%           (HWStepSemImpl.hwstep_semantics 
%             (Hmakege := make_program_globalenv _ _ _ _ Hmp)
%             (Genv.globalenv combined_program) s
%             ((CTXT <@$\oplus$@> k1) <@$\oplus$@> k2)
%             combined_program)).
%\end{lstlisting}
%and the proof is exactly same with the 

%\begin{figure}
%\includegraphics[width=\textwidth]{figs/certikos/layer_diagram}
%\caption{Layer Diagram for $\certikos$}
%\label{fig:chapter:certikos:layer-diagram-for-certikos}
%\jieung{Let's slightly modify it}
%\end{figure}
%
%
%With the all ingredients in this chapter, 
%connecting all proof pieces in $\certikos$ is available;
%thus we provide a single correctness theorem for $\certikos$
%as well as connecting proofs
%for the theorem.


\section{Evaluation}


We provide some statistics of $\certikos$ in this section. 
\begin{table}
\begin{center}
\renewcommand{\arraystretch}{1.1}
\setlength{\tabcolsep}{0.3em}
\begin{tabular}{|c|c||c|c|}
\hline
Components  & LOC & Components  & LOC \\
\hline
\hline
Hardware semantics & 750 &
Machine models & 2,600 \\ 
\hline
Refinement proofs & 3,300 & Top-level theorem & 100\\ 
\hline
\end{tabular}
\end{center}
\caption{Statistics for Multicore Linking in $\certikos$}
\label{table:multicore-evaluation}
\hrulefill
\end{table}
Table~\ref{table:multicore-evaluation} shows LOC information of $\certikos$ multicore linking.
Instantiating hardware configurations and auxiliary definitions requires 750 lines of code. 
By using them, providing concrete machine models is straightforward.
However, due to multiple numbers of intermediate machine models and  definitions that we have to provide for those machine models to fit them into $\compcert$ style semantics, the process requires 2,600 lines of code, which implies that 260 lines of code per one machine model.
However, those definitions are quite similar to others, so providing them requires a small amount of time. 
Lines of code information to provide refinement proofs between the program on $\mmcsbootfull$ with those intermediate languages has a similar manner to that of giving those concrete machine models. 
It requires us to write 3,300 lines of code, which implies that 300 lines of code per one for providing refinement proofs between two intermediate languages. 
Most of those lines of code are to adjust generic proofs in our framework to make them fit into the $\compcert$ style backward simulation proof. 
Those parts also contain similar proofs from others, so there is a room for improvement by providing a way to simplify or automatically generate those redundant parts. 
Connecting all proofs requires us to write only 100 lines of code because the only thing that we have to do is combining all refinement proofs that we have already written. 
\begin{table}
\begin{center}
\renewcommand{\arraystretch}{1.1}
\setlength{\tabcolsep}{0.3em}
\begin{tabular}{|c|c||c|c|}
\hline
Components  & LOC & Components  & LOC \\
\hline
\hline
Intermediate layer & 7,000 &
Connecting per-thread layer & 6,700 \\ 
\hline
Connecting per-CPU layer & 23,000 & Top-level theorem & 190\\ 
\hline
\end{tabular}
\end{center}
\caption{Statistics for Multithreaded Linking in $\certikos$}
\label{table:multithreaded-evaluation}
\hrulefill
\end{table}
Table~\ref{table:multithreaded-evaluation} shows LOC information of $\certikos$ multithreaded linking.
Instantiating all auxiliary functions as well as defining new specifications and memory accessor definitions is a tedious task since
we have to write all specifications for all primitives in $\PHBThrd$. 
It requires us to write  7,000 LOC. 
Connecting this  intermediate layer with the bottom layer in per-thread interfaces ($\PHThrd$) and 
connecting this with the top layer in per-CPU interfaces ($\PBThrd$) require us 
to provide concrete evidence for $\AbsRelT$ and $\AbsRelC$ per each primitive, respectively.
These steps require a huge amount of proofs, and we write 6,700  and 23,000 lines of code for them. 
However,  we can assume that the effort per each primitive is reasonable because 
 we have approximately 65 primitives in $\PHBThrd$, which implies that  $\PBThrd$ and $\PHThrd$ contains around 57 primitives including yield and sleep functions.

\begin{table}
\begin{center}
\renewcommand{\arraystretch}{1.1}
\setlength{\tabcolsep}{0.3em}
\begin{tabular}{|c|c|c|c|c|c|c|}
\hline
 & \makecell{ C \& Asm \\Source} & Specification & \makecell{Invariant \\ Proof} & \makecell{C \& Asm \\Proof} & \makecell{Simulation \\ Proof} \\
%\cline{3-4}
% & Source & Specification & Invariant Proof & Proof & Proof  \\
\hline
Ticket lock & 74 & 615 & 1,080 & 1,173 & 2,296 \\
\hline
Sequential queue & 377 & 554 & 748 & 2,821& 3,647 \\
\hline
Shared queue &  20 & 107 & 190 & 171& 419\\
\hline
Yield/Sleep/Wakeup & 62 & 153 & 166 & 1,724 & 2,042 \\
\hline
Queuing lock & 112 & 255 & 992 & 328 & 464\\
\hline
\end{tabular}
\newline
\end{center}
\begin{flushright}
* Starvation proof for Ticket lock included: 980\\
\end{flushright}
\caption{Statistics for Implemented Components in $\certikos$}
\label{table:certikos-evaluation}
\hrulefill
\end{table}
Table~\ref{table:certikos-evaluation} presents  the statistics for some other  $\certikos$ components.
 As for lock implementations, their source code contains not only the code of the associated functions but also the data structures and their initialization. In addition to the top-level interface, the specification contains all the specifications used in the intermediate layers.

For the ticket lock case, the simulation proof column includes progress property proofs in addition to the correctness proof. 
The ticket lock implementation usually requires less than 10 lines of code, but the proof size is huge when compared to the implementation.
The gap between the underlying C implementation and the high-level specification of the locks also contributes to the large proof size for these components. For example, intermediate specifications of a ticket lock use an unbounded integer for the ticket field, while the implementation uses a binary integer which wraps back to zero. Similarly, the queue is represented as a logical list in the specification, while it is implemented as a doubly linked list.

