
\subsection{Machine Model Instances for Multicore Linking}
\label{chapter:certikos:subsec:intermediate-machine-instantiation}



\begin{figure}
\begin{lstlisting}[language=C]
// the concrete instance of <@$\color{red}\hardwarestepkwd$@>
Definition hw_step_aux :=  @hardware_step hdseting hdsem current_CPU_ID.
// extends [hw_step_aux] to the rule with global environment  
Inductive hw_step_aux_ge : genv -> hstate -> trace -> hstate -> Prop :=
  | hw_step_aux_ge_intro : 
    forall s1 t s2, hw_step_aux s1 t s2 -> hw_step_aux_ge ge s1 t s2.
\end{lstlisting}
\begin{center}
(a) Transition Rules with Hardware-configuration for Multicore Machine Model
\end{center}
\begin{lstlisting}[language=C, deletekeywords={int}]    
// initial state, which initiates all private states as well as a global log
Inductive hwstep_initial_state (p: AST.program fundef unit): 
  (hstate (hdset := hdseting)) -> Prop := 
  | initial_hwstep_state_intro: 
    forall (m0: mwd LDATAOps), Genv.init_mem p = Some m0 ->
      let ge := Genv.globalenv p in
      let rs0 := (Pregmap.init Vundef) 
           # PC <- (symbol_offset ge p.(prog_main) Int.zero) # ESP <- Vzero in
      hwstep_initial_state p (HState current_CPU_ID (pinit (B := core_set)
        (Asm.State rs0 m0)) nil).
// final state of the machine model 
Definition hwstep_final_state (s : hstate (hdset := hdseting)) (i : int) := False.
// The complete machine model for <@$\color{red}\hardwarestepkwd$@> with proper initial and final states  
Definition hwstep_semantics (p: program) :=
  Smallstep.Semantics hw_step_aux_ge (hwstep_initial_state p) 
    hwstep_final_state (Genv.globalenv p).
\end{lstlisting}
\begin{center}
(b) Multicore Machine Model (\lstinline$hwstep_semantics$)
\end{center}
\begin{lstlisting}[language=C]
Definition single_step_aux := @single_step <@$\cdots$@>  // the concrete instance of <@$\color{red}\singlestepkwd$@>
// extends [single_step_aux] to the rule with global environment  
Inductive single_step_aux_ge : genv -> single_state -> trace -> 
  single_state -> Prop :=
  | single_step_aux_ge_intro : 
    forall s1 t s2, single_step_aux s1 t s2 -> single_step_aux_ge ge s1 t s2.
\end{lstlisting}
\begin{center}
(c) Transition Rules with Hwardware-configuration for Single-core Machine Model
\end{center}
\begin{lstlisting}[language=C, deletekeywords={int}]    
// The complete machine model for <@$\color{red}\singlestepkwd$@> with proper initial and final states
Definition single_semantics (p: program) :=
  Smallstep.Semantics single_step_aux_ge (single_initial_state p) 
    single_final_state (Genv.globalenv p).
\end{lstlisting}
\begin{center}
(d) Single-core Machine Model (\lstinline$single_semantics$)
\end{center}
\caption{Intermediate Machine Model Instances for Multicore Linking in $\certikos$.}
\label{fig:chapter:certikos:multicore-machine-model-instances}
\end{figure}


Our intermediate machine models in Section~\ref{chapter:linking:sec:multicore-linking} are abstract languages, but our ultimate goal is to connect them with per-CPU local layers. 
This requires us to connect those abstract machine models with $\lasmmach$--the machine model of $\compcertx$--by using backward-simulation proofs in $\compcert$~\cite{leroy06} (\ie, it is also used in the layer linking of local layer interfaces  with $\compcertx$).
To make  this possible,
we provide  full concrete definitions for those abstract languages
by following the \lstinline$Smallstep$ library of $\compcert$.
Figure~\ref{fig:chapter:certikos:multicore-machine-model-instances}
shows two instances, which are the concrete definition 
of two machine models among the machine models in Section~\ref{chapter:linking:sec:multicore-linking}.
The first example is the base of our multicore-linking, $\intelmachine$ multicore machine model that corresponds to $\hardwarestepkwd$ in Section~\ref{chapter:linking:subsec:multicore-machine-model}.
Providing the concrete definition first requires 
us to connect the abstract machine definition in Figure~\ref{fig:chapter:conlink:multicore-machine-syntax-and-semantics}
with concrete definitions for the hardware setting and  hardware semantics in Figure~\ref{fig:chapter:conlinkg:abstract-hardware-configuration-and-semantics}, \lstinline$hdseting$ (see Figure~\ref{fig:chapter:ceritkos:instances-of-abstract-hardware-setting}) and \lstinline$hdsem$ (see Figure~\ref{fig:chapter:certikos:hardware-local-step-transition-rules}). 
Figure~\ref{fig:chapter:certikos:multicore-machine-model-instances} (a) shows 
the code that defines how we combine them.
Figure~\ref{fig:chapter:certikos:multicore-machine-model-instances} (b) 
also shows how we define the full semantics by using the  definition in Figure~\ref{fig:chapter:certikos:multicore-machine-model-instances} (a) as well as the \lstinline$Smallstep$ library of $\compcert$  for the $\intelmachine$ multicore machine model, which is the machine model 
where we run $\certikos$. 
Defining initial and final states is necessary for using the library, and Figure~\ref{fig:chapter:certikos:multicore-machine-model-instances} (b)  also provides those definitions.
For simplicity, we assume that the program does not have a final state, which we think it is not a big assumption 
on the verification of operating systems.

Figure~\ref{fig:chapter:certikos:multicore-machine-model-instances} (c) and (d) provides an another example--how 
we instantiate single core semantics--which corresponds to the abstract machine in Section~\ref{chapter:linking:subsec:single-core-machine-model}.
We omit the details, but they are defined in a similar manner to those definitions in Figure~\ref{fig:chapter:certikos:multicore-machine-model-instances} (a) and (b). 
Other abstract machine models 
in Section~\ref{chapter:linking:sec:multicore-linking} also can be defined in
 the same way. 
We also prove that all machine models 
are receptive--producing at most a single event per one evaluation--as well as deterministic, 
except for the multicore machine model because it is a non-deterministic machine model by nature.
For example, 
we show that
our single-core machine model satisfies the following three properties:
\begin{lstlisting}[language=C]
Lemma si_sem_single_events  (pl: program): single_events (single_semantics pl).
Lemma si_sem_receptive (pl: program): receptive (single_semantics pl).
Lemma si_sem_determinate (pl: program): determinate (single_semantics pl).
\end{lstlisting}
which are also defined in  the \lstinline$Smallstep$ library of $\compcert$ and
necessary for us to use 
the forward-to-backward simulation template in $\compcert$. 

\jieung{I can always add more texts in here}
%
% hdseting hdsem current_CPU_ID.
%
%
%%We provide those instances in this section. 
%We show how we instantiate each intermediate language 
%in this section. All of them follows the exactly same form, 
%and need to provide the evidence about the receptiveness and the single event property (each step will generate at most a single event of $\compcert$ trace). 
%To use the forward simulation implies the backward simulation technique~\cite{leroy06},
%providing the deterministic behavior of those languages 
%are necessary. 
%In this sense, 
%we have also proven those properties 
%for all languages, except the multicore machine mode, which is non-deterministic. 
%The remaining parts of this section 
%show those instances.
%\jieung{I can remove some of the definitions below (or remove most of them...}
%\jieung{Mention about the False final state}

%\begin{itemize}[leftmargin=*]
%\item Generic context definitions and hardware semantics instance with contexts for all languages we defined.
%\end{itemize}
%\begin{lstlisting}[language=C]
%Variables (ge: genv) (sten: stencil) (M: module).
%Context {Hmakege: make_globalenv (module_ops:= LAsm.module_ops)
%               (mkp_ops:= make_program_ops) 
%               sten M (mboot <@$\oplus$@> L64) = ret ge}.        
%    
%Local Obligation Tactic := intros.
%    
%Definition hdsem_instance := @hdsem mem memory_model_ops Hmem 
%  Hmwd real_params_ops oracle_ops0 oracle_ops big_ops 
%  builtin_idents_norepet_prf ge sten M Hmakege.    
%\end{lstlisting}

%\begin{itemize}[leftmargin=*]
%\item Generic context definitions and hardware semantics instance with contexts for all languages we defined.
%\end{itemize}
%\begin{lstlisting}[language=C]
%Variables (ge: genv) (sten: stencil) (M: module).
%Context {Hmakege: make_globalenv (module_ops:= LAsm.module_ops)
%               (mkp_ops:= make_program_ops) 
%               sten M (mboot <@$\oplus$@> L64) = ret ge}.        
%    
%Local Obligation Tactic := intros.
%    
%Definition hdsem_instance := @hdsem <@$\cdots$@>
%\end{lstlisting}


%\subsubsection{Multicore Machine Model}


%\begin{itemize}[leftmargin=*]
%\item Properties of hardware step semantics (proofs are omitted)
%\end{itemize}
%\begin{lstlisting}[language=C]    
%Lemma hwstep_semantics_single_events: forall p, single_events (hwstep_semantics p).
%Lemma hwstep_semantics_receptive (pl: program):  receptive (hwstep_semantics pl).
%\end{lstlisting}
%
%
%\subsubsection{Multicore Machine Model with Hardware Oracle}
%
%Final state definitions for all languages from here is same with the final state definition for Multicore Machine Model. In this sense, 
%we have omitted them in our definitions. 
%
%\begin{itemize}[leftmargin=*]
%\item Hardware step with hardware oracle transition rule instance
%\end{itemize}
%\begin{lstlisting}[language=C]
%Definition oracle_step_aux :=
%  @oracle_step zset_op hdseting op_general hdsem_instance pmap hw_oracle.
%
%Inductive oracle_step_aux_ge : genv -> state -> trace -> state -> Prop :=
%  | oracle_step_aux_ge_intro : 
%    forall s1 t s2, oracle_step_aux s1 t s2 -> oracle_step_aux_ge ge s1 t s2.
%\end{lstlisting}
%
%\begin{itemize}[leftmargin=*]
%\item Hardware step with hardware oracle semantics definitions (including initial state definition)
%\end{itemize}
%\begin{lstlisting}[language=C]
%Inductive oracle_initial_state (p: AST.program fundef unit): 
%  (state (hdset := hdseting)) -> Prop := 
%  | initial_oracle_state_intro: 
%    forall (m0: mwd LDATAOps),
%      Genv.init_mem p = Some m0 ->
%      let ge := Genv.globalenv p in
%      let rs0 :=
%        (Pregmap.init Vundef)
%        # Asm.PC <- (symbol_offset ge p.(prog_main) Int.zero)
%        # ESP <- Vzero in
%      oracle_initial_state p (State current_CPU_ID 
%        (pset current_CPU_ID (LState (Asm.State rs0 m0) true)
%        (pinit (B := core_set) (LState (Asm.State rs0 m0) false))) nil).
%
%Definition oracle_final_state (s : state (hdset := hdseting)) (i : int)
%  : Prop := False.
%      
%Definition oracle_semantics (p: program) :=
%  Smallstep.Semantics oracle_step_aux_ge (oracle_initial_state p) 
%    oracle_final_state (Genv.globalenv p). 
%\end{lstlisting}
%
%\begin{itemize}[leftmargin=*]
%\item Properties of hardware step with hardware oracle semantics (proofs are omitted)
%\end{itemize}
%\begin{lstlisting}[language=C]
%Lemma oracle_semantics_single_events: forall p, single_events (oracle_semantics p).
%Lemma oracle_semantics_receptive (pl: program): receptive (oracle_semantics pl).
%Lemma oracle_semantics_determinate (pl: program): determinate (oracle_semantics pl).
%\end{lstlisting}
%
%\subsubsection{Environmental Step Machine}
%
%Properties that we have proven from here for all machines are same with the properties for  hardware step with hardware oracle semantics,
%\textit{single$\_$events}, \textit{receptive}, \textit{determinate}. 
%In this sense, we omitted them from this environmental step machine.
%
%\begin{itemize}[leftmargin=*]
%\item Environmental step transition rule instance. Since they can run an arbitrary number of cores with it, 
%the instance of this machine still has two parameters, $\codeinmath{core}$ (a focused set) and $\codeinmath{o}$ (an environmental context for 
%the focused set)
%\end{itemize}
%\begin{lstlisting}[language=C]
%Variable cores : ZSet.
%Variable o : GeneralOracleType.
%Definition env_step_aux:=
%  @env_step zset_op hdseting op_general hdsem_instance pmap cores o.
%
%Inductive env_step_aux_ge : genv -> (estate (A:= cores))  -> trace -> (estate (A:= cores)) -> Prop :=
%  | env_step_aux_ge_intro : 
%    forall s1 t s2, env_step_aux s1 t s2 -> env_step_aux_ge ge s1 t s2.
%\end{lstlisting}
%
%\begin{itemize}[leftmargin=*]
%\item Environmental step  semantics definitions (including initial state definition)
%\end{itemize}
%\begin{lstlisting}[language=C]
%Inductive env_initial_state (p: AST.program fundef unit): 
%  (estate (A := cores) (hdset := hdseting)) -> Prop :=
%  | initial_env_state_intro: 
%    forall (m0: mwd LDATAOps),
%       Genv.init_mem p = Some m0 ->
%       let ge := Genv.globalenv p in
%       let rs0 :=
%         (Pregmap.init Vundef)
%         # Asm.PC <- (symbol_offset ge p.(prog_main) Int.zero)
%         # ESP <- Vzero in
%       env_initial_state p (EState current_CPU_ID 
%         (pset current_CPU_ID (LState (Asm.State rs0 m0) true)
%         (pinit (B := cores) (LState (Asm.State rs0 m0) false))) nil).
%
%Definition env_semantics (p: program) :=
%  Smallstep.Semantics env_step_aux_ge (env_initial_state p) 
%    env_final_state (Genv.globalenv p).    
%\end{lstlisting}
%
%\subsubsection{Single Step Machine}    
%\begin{itemize}[leftmargin=*]
%\item Environmental step transition rule instance.
%\end{itemize}
%\begin{lstlisting}[language=C]
%Definition single_step_aux :=
%  @single_step zset_op hdseting op_general hdsem_instance 
%  current_CPU_ID single_oracle.
%
%Inductive single_step_aux_ge : genv -> single_state -> trace -> 
%  single_state -> Prop :=
%  | single_step_aux_ge_intro : 
%    forall s1 t s2, single_step_aux s1 t s2 -> single_step_aux_ge ge s1 t s2.
%\end{lstlisting}
%
%
%\begin{itemize}[leftmargin=*]
%\item Single step semantics definitions (including initial state definition)
%\end{itemize}
%\begin{lstlisting}[language=C]
%Inductive single_initial_state (p: AST.program fundef unit): 
%  (single_state (hdset := hdseting)) -> Prop :=
%  | initial_single_state_intro: 
%    forall (m0: mwd LDATAOps),
%      Genv.init_mem p = Some m0 ->
%      let ge := Genv.globalenv p in
%      let rs0 :=
%        (Pregmap.init Vundef)
%        # Asm.PC <- (symbol_offset ge p.(prog_main) Int.zero)
%        # ESP <- Vzero in
%      single_initial_state p (SState current_CPU_ID 
%        (LState (Asm.State rs0 m0) true) nil).
%
%Definition single_semantics (p: program) :=
%  Smallstep.Semantics single_step_aux_ge (single_initial_state p) 
%    single_final_state (Genv.globalenv p).
%\end{lstlisting}
%
%%\subsubsection{Big Step Machine}
%%From here to Big 2 step semantics,
%%our model requires fairness assumptions about hardware scheduler, which is notated as $\codeinmath{fair}$ in the semantics definitions.
%%\begin{itemize}[leftmargin=*]
%%\item Big step transition rule instance.
%%\end{itemize}
%%\begin{lstlisting}[language=C]
%%Definition single_big_step_aux :=
%%  @single_big_step zset_op hdseting op_general hdsem_instance 
%%  fair current_CPU_ID single_oracle.
%%
%%Inductive single_big_step_aux_ge : genv -> single_state -> trace -> 
%%  single_state -> Prop :=
%%  | single_big_step_aux_ge_intro : 
%%    forall s1 t s2, single_big_step_aux s1 t s2 -> single_big_step_aux_ge ge s1 t s2.
%%\end{lstlisting}
%
%
%%\begin{itemize}[leftmargin=*]
%%\item Big step semantics definitions (including initial state definition)
%%\end{itemize}
%%\begin{lstlisting}[language=C]
%%Inductive single_big_initial_state (p: AST.program fundef unit): 
%%  (single_state (hdset := hdseting)) -> Prop :=
%%  | initial_big_state_intro: 
%%    forall (m0: mwd LDATAOps),
%%      Genv.init_mem p = Some m0 ->
%%      let ge := Genv.globalenv p in
%%      let rs0 :=
%%        (Pregmap.init Vundef)
%%         # Asm.PC <- (symbol_offset ge p.(prog_main) Int.zero)
%%         # ESP <- Vzero in
%%      single_big_initial_state p (SState current_CPU_ID 
%%        (LState (Asm.State rs0 m0) true) nil).
%%      
%%Definition single_big_semantics (p: program) :=
%%  Smallstep.Semantics single_big_step_aux_ge (single_big_initial_state p) 
%%    single_big_final_state (Genv.globalenv p).
%%\end{lstlisting}
%%
%%\subsubsection{Big 2 Step Machine}
%%\begin{itemize}[leftmargin=*]
%%\item Big two step transition rule instance.
%%\end{itemize}
%%\begin{lstlisting}[language=C]
%%Definition single_big2_step_aux :=
%%  @single_big2_step zset_op hdseting op_general hdsem_instance 
%%   fair current_CPU_ID single_oracle.
%%
%%Inductive single_big2_step_aux_ge : genv -> rstate -> trace -> rstate -> Prop :=
%% | single_big2_step_aux_ge_intro : 
%%   forall s1 t s2, single_big2_step_aux s1 t s2 -> single_big2_step_aux_ge ge s1 t s2.
%%\end{lstlisting}
%
%
%%\begin{itemize}[leftmargin=*]
%%\item Big two step semantics definitions (including initial state definition)
%%\end{itemize}
%%\begin{lstlisting}[language=C]
%%Inductive single_big2_initial_state (p: AST.program fundef unit): 
%%  (rstate (hdset := hdseting)) -> Prop :=
%%  | initial_big2_state_intro: 
%%    forall (m0: mwd LDATAOps),
%%      Genv.init_mem p = Some m0 ->
%%      let ge := Genv.globalenv p in
%%      let rs0 :=
%%        (Pregmap.init Vundef)
%%        # Asm.PC <- (symbol_offset ge p.(prog_main) Int.zero)
%%        # ESP <- Vzero in
%%      single_big2_initial_state p (RState (Asm.State rs0 m0) nil).
%%
%%Definition single_big2_semantics (p: program) :=
%%  Smallstep.Semantics single_big2_step_aux_ge (single_big2_initial_state p) 
%%    single_big2_final_state (Genv.globalenv p).
%%\end{lstlisting}
%
%%\subsubsection{Split Step Machine}
%%\begin{itemize}[leftmargin=*]
%%\item Split step transition rule instance.
%%\end{itemize}
%%\begin{lstlisting}[language=C]
%%Definition single_split_step_aux :=
%%  @single_split_step zset_op hdseting op_general hdsem_instance 
%%    fair current_CPU_ID single_oracle.
%%
%%Inductive single_split_step_aux_ge : genv -> srstate -> trace -> srstate -> Prop :=
%%  | single_split_step_aux_ge_intro : 
%%  forall s1 t s2, single_split_step_aux s1 t s2 -> single_split_step_aux_ge ge s1 t s2.
%%\end{lstlisting}
%%
%%
%%\begin{itemize}[leftmargin=*]
%%\item Split step semantics definitions (including initial state definition)
%%\end{itemize}
%%\begin{lstlisting}[language=C]
%%Inductive single_split_initial_state (p: AST.program fundef unit): 
%%  (srstate (hdset := hdseting)) -> Prop :=
%%  | initial_split_state_intro: 
%%    forall (m0: mwd LDATAOps),
%%      Genv.init_mem p = Some m0 ->
%%      let ge := Genv.globalenv p in
%%      let rs0 :=
%%        (Pregmap.init Vundef)
%%        # Asm.PC <- (symbol_offset ge p.(prog_main) Int.zero)
%%        # ESP <- Vzero in
%%     single_split_initial_state p (SRState (Asm.State rs0 m0) nil nil).
%%          
%%Definition single_split_semantics (p: program) :=
%%  Smallstep.Semantics single_split_step_aux_ge (single_split_initial_state p) 
%%    single_split_final_state (Genv.globalenv p).
%%\end{lstlisting}
%%
%%\subsubsection{Reorder Step Machine}
%%\begin{itemize}[leftmargin=*]
%%\item Reorder step transition rule instance. Reorder step relies on different kinds of environmental context. In this sense, 
%%it is parameterized by environmental contexts.
%%\end{itemize}
%%\begin{lstlisting}[language=C]
%%Variable reorder_o : ReorderOracleType.
%%
%%Definition single_reorder_step_aux :=
%%  @single_reorder_step zset_op hdseting op_reorder hdsem_instance 
%%  current_CPU_ID reorder_o.
%%  
%%Inductive single_reorder_step_aux_ge : genv -> rstate -> trace -> rstate -> Prop :=
%%  | single_reorder_step_aux_ge_intro : 
%%    forall s1 t s2, single_reorder_step_aux s1 t s2 -> 
%%      single_reorder_step_aux_ge ge s1 t s2.  
%%\end{lstlisting}
%
%%
%%\begin{itemize}[leftmargin=*]
%%\item Reorder step semantics definitions (including initial state definition)
%%\end{itemize}
%%\begin{lstlisting}[language=C]
%%Inductive single_reorder_initial_state (p: AST.program fundef unit): 
%%  (rstate (hdset := hdseting)) -> Prop :=
%%  | initial_reorder_state_intro: 
%%    forall (m0: mwd LDATAOps),
%%      Genv.init_mem p = Some m0 ->
%%      let ge := Genv.globalenv p in
%%      let rs0 :=
%%        (Pregmap.init Vundef)
%%        # Asm.PC <- (symbol_offset ge p.(prog_main) Int.zero)
%%        # ESP <- Vzero in
%%      single_reorder_initial_state p (RState (Asm.State rs0 m0) nil).
%%
%%Definition single_reorder_semantics (p: program) :=
%%  Smallstep.Semantics single_reorder_step_aux_ge (single_reorder_initial_state p) 
%%    single_reorder_final_state (Genv.globalenv p).
%%\end{lstlisting}
%
%\subsubsection{Separate Step Machine}
%\begin{itemize}[leftmargin=*]
%\item Separate step transition rule instance.
%\end{itemize}
%\begin{lstlisting}[language=C]
%Definition single_separate_step_aux :=
%  @single_separate_step
%  hdseting separate_log_len zset_op op_sep hdsem_instance
%  current_CPU_ID sep_oracle.
%    
%Inductive single_separate_step_aux_ge : genv -> sp_state -> trace -> 
%  sp_state -> Prop :=
%   | single_separate_step_aux_ge_intro : 
%     forall s1 t s2, single_separate_step_aux s1 t s2 -> 
%       single_separate_step_aux_ge ge s1 t s2.
%\end{lstlisting}
%
%\begin{itemize}[leftmargin=*]
%\item Separate step semantics definitions (including initial state definition)
%\end{itemize}
%\begin{lstlisting}[language=C]
%Inductive single_separate_initial_state (p: AST.program fundef unit): 
%  (sp_state (hdset := hdseting)) -> Prop :=
%  | initial_separate_state_intro: 
%    forall (m0: mwd LDATAOps),
%      Genv.init_mem p = Some m0 ->
%      let ge := Genv.globalenv p in
%      let rs0 :=
%        (Pregmap.init Vundef)
%         # Asm.PC <- (symbol_offset ge p.(prog_main) Int.zero)
%         # ESP <- Vzero in
%      single_separate_initial_state p (SPState (Asm.State rs0 m0) (ZMap.init nil)).
%
%Definition single_separate_semantics (p: program) :=
%  Smallstep.Semantics single_separate_step_aux_ge (single_separate_initial_state p) 
%    single_separate_final_state (Genv.globalenv p).
%\end{lstlisting}
