
\subsection{Connect Layers on Multicore Machine and CPU-Local Machine}
\label{chapter:certikos:subsec:connect-multicore}

All refinement proofs in Section~\ref{chapter:linking:sec:multicore-linking} are also applicable 
to those instances. 
All those proofs requires the additional refinement proofs for between adjacent step relations, but they can 
use the corresponding theorems in Section~\ref{chapter:linking:sec:multicore-linking}. 
For example, the refinement proof between an environmental step machine with a single core and the environmental step machine with full core set is 
as follows:
\begin{lstlisting}[language=C]
Lemma one_step_env_refines_env_concrete:
  forall st_h1 t st_h2 st_l1
    (Hone: env_step_single_aux_ge_p ge st_h1 t st_h2)
    (Hmatch: match_eestate_link st_h1 st_l1),
    (exists st_l2,
      (plus env_step_full_aux_ge_p) ge st_l1 t st_l2 /\
        match_eestate_link st_h2 st_l2) \/
      ((state_measure (s_set current_CPU_ID) st_h2 < state_measure (s_set current_CPU_ID) st_h1)%nat /\ t = E0 /\
        match_eestate_link st_h2 st_l1).
Proof.
  <@$\cdots$@>
   eapply single_env_step_refines_full_env_step in Hone; eauto.
   <@$\cdots$@>
Qed.
\end{lstlisting}
, which uses the proof in Section~\ref{chapter:linking:subsec:concurrent-machine-model}.
Using those theorems, 
we prove the backward simulation proofs (in $\compcert$ style) 
\begin{lstlisting}[language=C]
Theorem cl_backward_simulation:
  forall (s: stencil) (CTXT: LAsm.module) (ph: AST.program fundef unit)
           (Hmakep: make_program (module_ops:= LAsm.module_ops) s CTXT 
           (mboot <@$\oplus$@> L64) = OK ph),
  backward_simulation
   (env_semantics (s_set current_CPU_ID) single_oracle
   (Hmakege := make_program_globalenv (make_program_ops := make_program_ops)
      _ _ _ _ Hmakep) (Genv.globalenv ph) s CTXT ph)
   (env_semantics core_set hw_oracle
   (Hmakege := make_program_globalenv (make_program_ops := make_program_ops) 
     _ _ _ _ Hmakep) (Genv.globalenv ph) s CTXT ph).
\end{lstlisting}

The top level theorem for multicore linking uses the backward simulation proofs 
for each steps that we have described in Section~\ref{chapter:linking:sec:multicore-linking}.
The top level theorem is in Theorem~\ref{theorem:chapter:certikos:contextual-refinement-theorem-in-certikos-multicore-linking}.
\begin{theorem}[Multicore Linking Theorem for $\certikos$]
\label{theorem:chapter:certikos:contextual-refinement-theorem-in-certikos-multicore-linking}
Let's assume that $cid$ is a valid CPU ID, $\oracle_{cpu}$ is an
environmental context for thread $cid$, and $\codeinmath{Ctxt}$ is a
 context program that runs on top of the $certikos$. 
 Then, we want to prove that 
 \begin{center}
\begin{tabular}{c}
$\semwmachine{\codeinmath{MBoot}}{\codeinmath{Ctxt}}{\codeinmath{mach}_{\codeinmath{x86mc}}} \refines_{R_{\codeinmath{mclink}}} \semwmachine{\codeinmath{MBoot}[cid, \oracle_{cpu}]}{\codeinmath{Ctxt}}{{\codeinmath{mach}_{\lasm}}}$\\
\end{tabular}
\end{center}
when ${R_{\certikos_\codeinmath{mclink}}}$ is a refinement relation for multicore linking.
\end{theorem}
%The top level theorem is in Figure~\ref{fig:chapter:certikos:top-level-multicore-theorem}.
%
%\begin{figure}
%\begin{lstlisting}[language=C]
%Theorem concurrent_backward_simulation:
%  forall (s: stencil) (CTXT: LAsm.module) (ph: AST.program fundef unit)
%    (Hmakep: make_program (module_ops:= LAsm.module_ops) s CTXT (mboot <@$\oplus$@> L64) = OK ph),
%    backward_simulation
%    (LAsm.semantics (lcfg_ops := LC (mboot <@$\oplus$@> L64)) ph)          
%    (hwstep_semantics 
%      (Hmakege := make_program_globalenv (make_program_ops := make_program_ops) 
%      _ _ _ _ Hmakep) (Genv.globalenv ph) s CTXT ph).
%\end{lstlisting}
%\caption{Top Level Multicore Linking Theorem}
%\label{fig:chapter:certikos:top-level-multicore-theorem}
%\end{figure}



\jieung{need to mention assumptions - The proof, of course has several assumptions. }
\jieung{OK. Change $\codeinmath{Mach}_{xxx}$ in Multicore Linking Framework as the form that does not have $Mach$.
This for the instance. Then, we add $Mach$ in this section to distinguish the abstract definition for intermediate languages and the concrete definitions} 
