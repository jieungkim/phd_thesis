\section{Certified Abstraction Layers}
\label{chapter:ccal:sec:cal}


Building certified software in a layered approach was already addressed in the previous work by Gu \etal~\cite{deepspec},
but that work does not address handling concurrent programs.
However, the approach, \textbf{C}ertified \textbf{A}bstraction \textbf{L}ayers ($\calname$), works as a basis of our concurrent layer framework.
It enables us to build and verify the (sequential) system by dividing it into modular layers, 
proving the correctness of its every layer independently, and connecting the proofs of all layers in the system to show the correctness of the system's end-to-end behavior via \textit{contextual refinement} between layers. 
It only supports sequential program verification, but works as a basis of our \textbf{C}ertified \textbf{C}oncurrent \textbf{A}bstraction \textbf{L}ayers ($\ccalname$).

$\calname$ is a predicate over two layers with a context program, 
``$\ltyp{L_{\codeinmath{low}}}{R}{\codeinmath{M}_\codeinmath{high}}{L_{\codeinmath{high}}}$'' 
and a mechanized proof object (written in $\coq$) for the predicate where $L_{\codeinmath{low}}$ and $L_{\codeinmath{high}}$ are state machines, $\codeinmath{M}_\codeinmath{high}$ is a program written in C and/or Assembly, $R$ is a relation between states of two layers.
Informally, the predicate implies that 
a layer $L_{\codeinmath{low}}$ contextually refines  a layer $L_{\codeinmath{high}}$ with a relation $R$
 if each state transition made by $L_{\codeinmath{high}}$ based on a context program \code{Ctxt} 
% (when \code{Prog} = $\codeinmath{M}_{\codeinmath{high}} \oplus \codeinmath{Ctxt}$, where $\oplus$ computes the union 
% of implementation modules and contexts) 
 corresponds to (with a relation $R$)  a sequence of 
 state transitions by $L_{\codeinmath{low}}$ with $\codeinmath{M}_{\codeinmath{high}}$ and  \code{Ctxt} together.
 Formally, it is defined as a forward simulation~\cite{Lynch95,leroy09,Milner71,Park81} with a (simulation) relation $R$
 and  $\ltyp{L_{\codeinmath{low}}}{R}{\codeinmath{M}_{\codeinmath{high}} }{L_{\codeinmath{high}}}$ is stated as
 \begin{center}
$\forall\ \codeinmath{Ctxt},\ \sem{L_{\codeinmath{low}}}{\modulecompose{\codeinmath{M}_{\codeinmath{high}}}{\codeinmath{Ctxt}}} \refines_{R} \sem{L_{\codeinmath{high}}}{\codeinmath{Ctxt}},$
\end{center}
where $\refines_{R}$ is a refinement relation with $R$, 
$\modulecomposekwd$ computes the union 
 of implementation modules and contexts,
and $\sem{L}{\cdot}$ denotes a  behavior of the layer machine based on $L$.


Contextual refinement, an underlying technique of $\calname$, is powerful because layers can be verified only once individually, and they can be linked together.
 When a layer stack is extended with a new verified layer on top, 
  inter-layer contextual refinement proofs can be reused with an updated context to include a new layer continuously. 
For instance, if we add a new layer $L_{\codeinmath{top}}$ on top of verified layers $L_{\codeinmath{mid}}$ and $L_{\codeinmath{btm}}$, 
we need one new proof that shows $L_{\codeinmath{mid}}$ contextually refines $L_{\codeinmath{top}}$, but we can reuse the proof for $L_{\codeinmath{btm}} \refines_{R} L_{\codeinmath{mid}}$ without requiring any modification to the proof thanks to the characteristic of the contextual refinement. After the proof of $L_{\codeinmath{mid}} \refines_{{R}} L_{\codeinmath{top}}$, we are automatically guaranteed that $L_{\codeinmath{btm}}$ contextually refines all the way up to $L_{\codeinmath{top}}$ as follows:
\begin{center}
$
\forall\ \codeinmath{Ctxt},\ \sem{L_{\codeinmath{btm}}}{\modulecompose{\codeinmath{M}_{\codeinmath{mid}}}{(\modulecompose{\codeinmath{M}_{\codeinmath{top}}}{\codeinmath{Ctxt})}}} \refines_{R_{(\codeinmath{mid}, \codeinmath{low})}}  \sem{L_{\codeinmath{mid}}}{\modulecompose{\codeinmath{M}_{\codeinmath{top}}}{\codeinmath{Ctxt}}} \refines_{R_{(\codeinmath{top}, \codeinmath{mid})}} \sem{L_{\codeinmath{top}}}{\codeinmath{Ctxt}}. 
$
\end{center}
where $R_{(\codeinmath{mid}, \codeinmath{low})}$ and $R_{(\codeinmath{top}, \codeinmath{mid})}$ are relations between states of $ L_{\codeinmath{btm}}$ and $L_{\codeinmath{mid}}$ as well as
 $ L_{\codeinmath{mid}}$ and $L_{\codeinmath{top}}$, respectively, 
and $\relcomposekwd$ is an operator that combines two refinement relations and is used for representing  
 ${R}_{total} =\relcompose{R_{(\codeinmath{mid}, \codeinmath{low})}}{{R_{(\codeinmath{top}, \codeinmath{mid})}}}$ here.



Each layer in $\calname$ is internally composed of  C implementations, specifications, and proofs.
To develop a layer $L_k$, 
the developer writes 
1) a source code written in C and/or Assembly; 
2) high-level and  low-level specifications written in $\coq$, which specify how the code changes an abstract state and  a memory, respectively; 
3) an auto-generated AST ($\clight$) of the code by using a tool in $\compcert$,
and the developer writes three proofs: 
1) $p_k$, a proof that the generated AST refines the low-level specification; 
2) $r_k$, a proof that the low-level specification refines the high-level specification; and 
3) $R_{k-1,k}$, a proof using $p_k$ and $r_k$ to verify that $L_{k-1}$ contextually refines $L_{k}$. 
The proofs $p_k$ and $r_k$ guarantee that the C code (\ie , its verified $\coq$ representation) is correct as defined by the specifications.
 With the contextual refinement proof $R_{k-1, k}$, we are assured that the C code in $L_k$ never uses the code in $L_{k-1}$ in an undefined way; 
 calls to C functions in $L_{k-1}$ always return defined results to $L_k$; and variables 
 used and allocated in each layer have their memory locations and are safely accessed.
 This process will remain the same after we extend $\calname$ to support concurrency.
