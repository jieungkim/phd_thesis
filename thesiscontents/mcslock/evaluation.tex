
\section{Evaluation}
\label{chapter:mcslock:sec:evaluation}

\begin{figure}
\begin{minipage}{\linewidth}
\noindent
\begin{multicols}{2}
\lstinputlisting[numbers = left, language = C]{source_code/mcslock/palloc_example.c}
\lstinputlisting[numbers = left]{source_code/mcslock/sharedeventtype.v}
\end{multicols}
\end{minipage}
\caption{$\pallocfunc$ Example.}
\label{fig:chapter:mcslock:palloc-example}
\end{figure}

\subsection{Clients}

The verified $\mcsname$ lock code is used by multiple clients in the $\certikos$ (in Chapter~\ref{chapter:certikos})
system. To be practical the design should require as little extra work
as possible compared to verifying non-concurrent programs, both to
better match the programmer's mental model, and to allow code-reuse
from the earlier, single-processor version of $\certikos$.

For this reason, we don't want our machine model to generate an event
for every single memory access to shared memory. Instead we use what
we call a \emph{push/pull memory model} in Section~\ref{chapter:ccal:sec:interface-calculus}.
To recall it,
 A CPU that wants to access shared memory first generates a ``$\pull$''
event, which declares that that CPU now owns a particular block of
memory. After it is done it generates a ``$\push$'' event, which
publishes the CPU's local view of memory to the rest of the system. In
this way, individual memory reads and writes are treated by the same
standard operational semantics as in sequential programs, but the
state of the shared memory can still be replayed from the log.  The
$\push$/$\pull$ operations are logical (generate no machine code) but
because the replay function is undefined if two different CPUs try to
pull at the same time, they force the programmer to prove that
programs are well-synchronized and race-free. Like we did for atomic
memory operations, we extend the machine model at the lowest layer by
adding logical primitives, e.g. $\releaseshared$ which takes a
memory block identifier as an argument and adds a
$\mcsomeme{\codeinmath{(l:list Integers.Byte.int)}}$ event to the log, where the byte list is a
copy of the contents of the shared memory block when the primitive was
called.

When we use $\acquirereleaselock$ we
need a lock to make sure that only one CPU pulls, so we begin
by defining combined functions $\acquirelockfunc$ which
takes the lock (with a bound of 10) and then pulls, and
$\releaselockfunc$ which pushes and then releases the
lock. The specification is similar to $\mcspasshlockspec$,
except it appends \emph{two} events.

Similar to Section~\ref{chapter:mcslock:sec:liveness-atomicity}, logs for different
layers can use different types of pull and push events.
Figure~\ref{fig:chapter:mcslock:palloc-example} (right) shows the events for the
$\pallocfunc$ function (which uses a lock to protect the page
allocation table). The lowest layer in the palloc-verification adds
$\mcsOMEME$ events, while higher layers instead add
($\mcsoateev{(a:\ \mcsalloctabletype)}$) events, where the relation between logs
uses the same relation as between raw memory and abstract
$\mcsalloctabletype$ data. Therefore, we write wrapper functions
$\acquirereleaselockATspec$, where the
implementation just calls $\acquirereleaselock$
with the particular memory block that contains the allocation table,
but the specification adds an $\mcsOATEev$ event.
This refinement step, which changes the log replay function to compute
allocation tables instead of byte lists, is
specific to the $\pallocfunc$ function.

\begin{figure}

\lstinputlisting{source_code/mcslock/release_lock_AT_spec_short.v}
\lstinputlisting{source_code/mcslock/palloc_spec_short.v}

\caption{$\pallocfunc$ Specification.}
\label{fig:chapter:mcslock:palloc-spec}
\end{figure}

We can then ascribe a low-level functional specification
$\pallocpspec$ to the $\pallocfunc$ function. As shown
in Figure~\ref{fig:chapter:mcslock:palloc-spec}, this is decomposed into three parts, the
acquire/release lock, and the specification for the critical
section. The critical section spec is exactly the same in a sequential
program: it does not modify the log, but instead only affects the $\mcsalloctable$ field in the abstract data.

Then in a final, pure refinement step, we ascribe a high-level atomic
specification $\lpallocspec$ to the $\pallocfunc$
function. In this layer we no longer have any lock-related events at
all, a call to $\pallocfunc$ appends a single
$\mcsopalloce$ event to the log. This is when we see the
proof obligations related to liveness of the locks.
Specifically, in order to prove the downwards refinement, we need to
show that the call to $\pallocpspec$ doesn't return
$\None$, so we need to show that $\HCalLock\codeinmath{ l'}$ is
defined, so in particular the bound counter must not hit zero.
By expanding out the definitions, we see that
$\pallocpspec$ takes a log $\codeinmath{l}$ as its initial global log
and generates the result,
$\rellock\codeinmath{::}\mcsoateev{(\mcsalloctable\ adt)}\codeinmath{::} \mcstshared{\mcsopullev}\codeinmath{::}\waitlock{10}\codeinmath{::l}.$
The initial bound is 10, and there are two shared memory events, so the
count never goes lower than 8. If a function modified more than one
memory block there would be additional push- and pull-events, which
could be handled by a larger initial bound.

Like all kernel-mode primitives in $\certikos$, the $\pallocfunc$ function is
total: if its preconditions are satisfied it always returns. So
when verifying it, we show that all loops inside the critical section
terminate. Through the machinery of bound numbers, this guarantee is
propagated to the the while-loops inside the lock implementation:
because all functions terminate, they can know that other CPUs will
make progress and add more events to the log, and because of the
bound number, they cannot add push/pull events forever. On the other
hand, the framework completely abstract away how long time (in microseconds) elapses
between any two events in the log.

\subsection{Code reuse} 
The same
$\acquirereleaselock$ specifications can be
used for all clients of the lock. The only proofs that need to be done
for a given client is the refinement into abstracted primitives like
$\releaselockATspec$ (easy if we already have a sequential
proof for the critical section), and the refinement proof for the
atomic primitive like $\lpallocspec$ (which is very
short). We never need to duplicate the thousands of lines of proof
related to the lock algorithm itself.

\subsection{Using more than one lock}

The layers approach is particularly nice when verifying code that uses more than one
lock. To avoid deadlock, all functions must acquire the locks in the
same order, and to prove the correctness the ordering must be
included in the program invariant. We \emph{could} do such a
verification in a single layer, by having a single log with different
events for the two locks, with the replay function being undefined if
the events are out of order. But the layers approach provides a
better way. Once we have ascribed an atomic specification to
$\pallocfunc$, as above, all higher layers can use it
freely without even knowing that the $\pallocfunc$ implementation
involves a lock (Note that the lock is not re-exported from the
$\pallocfunc$ layer, and if it was the proof of the atomic
specification would not go through.)  For example, some function in a
higher layer could acquire a lock, allocate a page, and release the
lock; in such an example the the order of the layers provides an order
on the locks implicitly.

\subsection{Proof Effort}


As an evaluation, we do not count the total lines of code in $\coq$ for our entire 
$\mcsname$ Lock module due to the two following reasons. First, our $\mcsname$ Lock implementation 
is a part of $\certikos$ (in Chapter~\ref{chapter:certikos}). Therefore, our $\mcsname$ Lock module also contains several definitions 
and proofs that are totally irrelevant to $\mcsname$ Lock verification. 
This implies that counting the total lines of code for $\mcsname$ Lock module has a 
high possibility of misinterpretation due to the lines of code for those definitions and proofs.
Second, we intensively use contextual refinement approach to 
build the whole system rather than focusing on verifying the correctness and 
liveness of $\mcsname$ Lock. Therefore, our proof efforts are mainly focus on proving 
$\mcsname$ Lock that is able to be easily combined with multiple client codes 
rather than the efficient lock verification itself.  

Among the whole proofs, the most challenging parts are the proofs for starvation 
freedom theorems like Theorem~\ref{thm:chapter:mcslock:mcs_wait_lock_exist}, 
and the functional correctness proofs for $\mcsacquire$ 
and $\mcsrelease$ functions
in Section~\ref{chapter:mcslock:subsec:atomicoperation}.
The total lines of codes for starvation freedom is 2.5K lines, 0.6K lines for specifications, 
and 1.9k lines for proofs. This is because of the subtlety of those proofs. 
To prove the starvation freedom theorems and show the evidence of loop termination,
lots of lemmas are required to express
state changes by replaying the log. 
When $\QSCalLock\codeinmath{(l)=}\Some\codeinmath{(c1,c2,b,q,s,t)}$
and $\codeinmath{q=}\nil$, 
for instance, the mechanized proof for $\codeinmath{s=}\emptyset$ 
and $\codeinmath{t=}\nil$ is necessary. It looks trivial in the hand-written proofs, 
but requires multiple lines of codes in the mechanized proof. 

The total lines of codes for the low-level functional correctness
of two main C functions, $\mcsacquire$ and $\mcsrelease$, are 3.2K lines,  
0.7K lines for specifications, and 2.5K lines for proofs.
It is much bigger than other code correctness proofs for while-loops in $\certikos$, which we will
discuss in Chapter~\ref{chapter:certikos},
because these loops do not have any explicit decreasing value.
One another big part in our $\mcsname$ Lock proofs is the proofs for 
Theorem~\ref{thm:chapter:mcslock:machine-state-refinement} and the lines of code for this part is 
approximately 5K lines. The log replay function ($\calmcslock$) always 
return the whole $\mcsname$ Lock values ($\lockabsloc$) related 
to the  $\codeinmath{mcs\_lock}$ structure defined in Figure~\ref{fig:chapter:mcslock:mcs_lock}. 
In this sense, we always have to give the exact values for all memory 
chunks and prove the correspondence between the memory and the abstract 
data even the event associated with reading values (\eg, $\getnext$).
Hence, those proofs contain a lot of duplicate proofs for the memory access. 
However, they are quite straightforward and easy to produce. 
On top of that, we strongly believe 
that they can be easily reduced by introducing mechanized user-defined tactics later. 

As can be seen from these line counts, proofs about concurrent programs
have a huge ratio of lines of proof to lines of C code.
If we tried to directly verify shared objects that use locks to 
perform more complex operations, like thread scheduling
and inter-process communication, a monolithic proof  
would become much bigger than the current one, and would be quite
unmanageable. The modular lock specification is essential here.

By contrast, the proofs for them in $\certikos$ are quite tractable, 
because the proofs for the locks are modular, re-usable, and can 
be combined with other client-part proofs like we have briefly 
mentioned earlier in this Section.
Therefore, we believe that our approach is a promising way to 
show the correctness of large systems that use shared objects with mutex protection. 

