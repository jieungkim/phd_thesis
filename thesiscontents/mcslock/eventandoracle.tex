\section{Events, Logs, and  Contexts for MCS Lock}
\label{chapter:mcslock:sec:eventlogandoracle}

As discussed in Chapter~\ref{chapter:ccal}, 
modeling a shared state among CPUs is 
closely related to defining a set of events, a concurrent context, and a replay function that are suitable for
the state.
%%%%%%%%%%%%%%%%%%%%%%%%%%%%%%
\begin{figure}
\begin{minipage}{\linewidth}
\lstinputlisting[numbers = left, language = C]{source_code/mcslock/lockeventtype.v}
\end{minipage}
\caption{Event set for $\mcsname$ Lock.}
\label{fig:chapter:mcslock:lock_event_type}
\end{figure}
%%%%%%%%%%%%%%%%%%%%%%%%%%%%%%
Again, an \emph{event} is any action which has observable consequences for
other CPUs. Each specification must define events for all the points
in the program where it reads or writes to shared memory (but not for
accesses to thread-local memory). The \emph{log} is a list of
events, representing all actions that have happened in the system
since it began running. Actions from different CPUs are interleaved in
the list.
When we write a specification we can choose the set of events, as long
as it is fine-grained enough to capture all scheduling interleavings
that may happen.
Figure~\ref{fig:chapter:mcslock:lock_event_type} shows the event definition used to
model lock acquire and release. They correspond to the part of the $\mcsname$ lock source code in Figure~\ref{fig:chapter:mcslock:mcs_lock}
and acquiring/releasing the lock after we show starvation freedom. 


As discussed in Section~\ref{chapter:ccal:sec:interface-calculus},  we again assume that the
machine is sequentially consistent (all CPUs see a single linear log). 
This rules out 
Even with this assumption,
verifying the $\mcsname$ algorithm is not easy (the other proofs we are aware
of assuming sequential consistency too), so we leave weak memory models
to future work.


\begin{itemize}

\item \textbf{${\swaptail{bound}{success}}$} event is for the
operations from line 5 to 7 in Figure~\ref{fig:chapter:mcslock:mcs_lock} and takes
two arguments. The second argument is a boolean flag indicating
whether the previous ``last'' value of $\mcsname$ lock was $\invalidmcsval$,
which means it records whether the if-statement at line 9 took the fast path or not.
The first argument is the \emph{bound number}, which is a key idea in
our development. Every client that invokes $\mcsacquire$ has
to promise a bound for the critical section. This number
does not influence the compiled code in any way, but the
\emph{specification} says that it is invalid to hold the critical
section for longer than that (\cf,  the counter $c_1$ in
Section~\ref{chapter:mcslock:sec:representation-ghost}).
This bound number enables \emph{local} reasoning about liveness.
For the thread waiting for acquiring or releasing a lock,
its wait time can be estimated based on other threads' bound number. For the lock holder, it has to guarantee
to exit the critical section within its own bound. 
Thus, by showing that each thread follows this protocol,
we can derive the liveness property for the whole system.

To be precise, the bound number is a limit on the number of events
that can get appended to the log, see the counter $\codeinmath{c1}$ in
Section~\ref{chapter:mcslock:sec:representation-ghost}.
Every CPU adds at least
one event every time it "does something", \eg, each loop iteration in $\mcsrelease$ appends a $\getnext$
event, so
as we will see in
Section~\ref{chapter:mcslock:sec:liveness-atomicity} this suffices to give a bound of the number of loop iterations in the lock acquire function. In the following, we often speak of
``number of operations'', which does not mean single CPU instructions,
but instead whatever operation is represented by particular events.



\item \textbf{$\setnext{prev\_last}$} event corresponds to the code at line 10. 
the $\codeinmath{prev\_last}$ represents the $\codeinmath{prev\_id}$ in the code.

\item\textbf{$\getbusy{busy}$} event shows the busy waiting in the acquire lock function.
The first argument will be true when the last value is equal to the current CPU ID that calls the primitive which generates this event.
It will be false when the last value is not equal to the current CPU ID that calls the primitive.
\end{itemize}

Next, other threes are enough to represent release lock in Figure~\ref{fig:chapter:mcslock:mcs_lock}.

\begin{itemize}

\item \textbf{$\castail{busy}$} represents the atomic expression at line 21  in Figure~\ref{fig:chapter:mcslock:mcs_lock}. 
In addition, the ``busy'' corresponds to the result of the $\CAS$ operation in Figure~\ref{fig:chapter:mcslock:mcs_lock}.

\item  \textbf{$\getnext$} corresponds to the primitive that try to get the next value of the current CPU's node and abstracts busy waiting in release lock function.

\item  \textbf{$\setbusy$} represents the last three lines in $\mcsrelease$.
\end{itemize}

Those six events are used to show the functional correctness of
an $\mcsname$ Lock. However, for clients that use the $\mcsname$ Lock to build shared
objects they expose too much implementation details.
In Section~\ref{chapter:mcslock:sec:liveness-atomicity} we will prove linearizability and
starvation freedom,  to replace them
with just two events.


\begin{itemize} 
\item \textbf{$\waitlock{bound}$} corresponds to lock acquire. The ``bound'' number in here is exactly same with the ``bound'' number in $\swaptail{bound}{success}$ event.

\item \textbf{$\rellock$} corresponds to the lock release.
\end{itemize}

In addition to the above eight events, which are generated by the lock
acquire and release functions, the clients of the lock will also
generate events while they are in the critical section. Mutex locks in
CertiKOS are used to protect blocks of shared memory, so we call the
events generated by the client code \textbf{shared memory events}. The
final specification we prove will entail that a shared memory event
from CPU $i$ can only happen in the interval between a lock acquire
event for $i$ and a lock release event for $i$, which is how we
express the mutual exclusion property.