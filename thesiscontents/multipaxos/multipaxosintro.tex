
Distributed systems are notoriously complex due to the many possible interleavings of their coarsely-connected 
instances as well as the possibility of errors in both  those instances and the network environment. 
For these reasons, verification of distributed systems is desirable to remove the possibility of bugs and guarantee their safety and correctness. 
However, much current verification work still requires a great deal of effort and sometimes has limitations.
Paxos~\cite{paxos}, proposed by Lamport, is an old distributed consensus protocol but still in use. 
%Based on Paxos, 
%we propose the Write-Once Register (WOR) which consists of multiple 


%
%A WOR exposes a simple, data-centric API: clients can capture, write, and read it. Applications can use a sequence or a set of WORs to obtain properties such as durability, concur- rency control, and failure atomicity. By hiding the logic for distributed coordination underneath a data-centric API, the WOR abstraction enables easy, incremental, and extensible implementation and verification of applications built above it. We present the design, implementation, and verification of a system called WormSpace that provides developers with an address space of WORs, implementing each WOR via a Paxos instance. We describe three applications built over WormSpace: a flexible, efficient Multi-Paxos implementa- tion; a shared log implementation with lower append latency than the state-of-the-art; and a fault-tolerant transaction co- ordinator that uses an optimal number of round-trips. We show that these applications are simple, easy to verify, and match or surpass the performance of monolithic implementa- tions. We use a modular layered verification approach to link the verification proofs for the applications, the WormSpace system, and a verified OS to produce an end-to-end verified distributed system stack from the application to the OS.
%
%
%
%We present a verification approach that uses \textit{write-witness-passing}, which is simple but novel in distributed system verification. It is a scalable, reusable, and extensible approach that can be directly linked with the low-level implementations of distributed protocols through contextual refinement. Write-witness-passing can capture the common behaviors of many distributed protocols, and provides both a simple way of understanding the protocols as well as an easy methodology for verifying them.
%
%To demonstrate how write-witnesses work, we verify the functional correctness and safety of Paxos, one of the most famous consensus protocols. We implement the key routines of Paxos in C, and use Coq to verify both the functional correctness of the implementation as well as the safety properties of the protocol within less than 4 person-months. We also describe how we can apply our approach to other distributed protocols to illustrate its generality.
