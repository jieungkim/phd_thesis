
Our verification approach has been inspired by the $\certikos$ project,
but the applicability of our framework in Chapter~\ref{chapter:ccal} is not restricted to operating system verification, but general enough
to build and verify other interesting system software. 
Among them, we choose to build a distributed system as another example to demonstrate our 
verification framework.
Distributed systems are notoriously complex due to  many possible interleavings of their coarsely-connected 
instances as well as the possibility of errors in both those instances and the network environment. 
For these reasons, verification of distributed systems is desirable to remove the possibility of bugs and guarantee their safety and correctness. 
However, much current verification work still requires a great deal of effort and sometimes has limitations.

Even without interleavings and% errors, 
modern distributed applications are not built on top of one single distributed protocol (or system), but with multiple protocols.
They are composed of distributed systems that provide replication, 
logging, and transactions to achieve properties such as durability, concurrency control, failure atomicity, high availability, and strong consistency.
While the high-level APIs and implementations of such systems vary widely, 
the properties provided by them and the protocols used to provide the properties are similar. 
Still, developers re-implement the system (such as Paxos~\cite{paxos} and 
Two-Phase Commit (2PC)~\cite{2PC}) in slightly different ways to obtain different APIs and performance 
characteristics despite the difficulty of implementing distributed systems correctly.

To handle those issues, we propose Write-Once Registers ($\WOR$) as an abstraction for building and verifying distributed systems. 
A $\WOR$ exposes a simple, data-centric API: 
clients can capture, write, and read it. 
Applications can use a sequence or a set of $\WOR$s to obtain properties such as durability, concurrency control, and failure atomicity. 
By hiding the logic for distributed coordination underneath a data-centric API, the $\WOR$ abstraction enables easy, incremental, 
and extensible implementation and verification of applications built above it. 
Accordingly, we present the design, implementation, and verification of a system called $\wormspace$ 
that provides developers with an address space of $\WOR$s, implementing each $\WOR$ via a $\paxos$ instance,
which is the most popular consensus protocol that is still in use. 

