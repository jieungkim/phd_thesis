\section{WormSpace Verification}
\label{sec:formal_verification}

%\jieung{I do not need to expose all the details (including specifications for send and receive functions / log replay functions, and so on. 
%But, if I have enough time to do that, I'll add more details in here}

$\wormspace$ acts as a foundation for verifying distributed systems. 
We verify $\wormspace$ once and reuse its proof for verifying three applications (\ie, WormPaxos, WormLog, and WormTX)
built on top while hiding the complexity of distributed protocol verification.     
To do so, we extend the Certified Concurrent Abstraction Layer ($\ccalname$)  in Chapter~\ref{chapter:ccal}, 
modeling an asynchronous network of distributed nodes to verify $\wormspace$. 
We apply $\ccalname$ beyond a single system verification for the first time and link the proof of $\wormspace$, 
$\wormspace$ applications and a verified OS. 

\subsection{Layer Structure for Verification}

$\wormspace$ consists of two separate stacks of verification layers, the client library (17 layers) and the wormserver (2 layers), over a common set of base functionalities (5 layers). While the number of layers may seem excessive, it matches a conventional software stack designed for modularity: each layer is a C component implementing some interface. A simplified layer diagram is shown in Figure~\ref{fig:chapter:multipaxos:layerdiagram}.
\footnote{A set of base layers offer common functionalities to the client library and the wormserver stacks: 
the bottom layer provides an interface to the TCB, including network communication functions and a small number of system calls, 
while the data layer implements various data structures over the trusted primitives.
Above, the server stack includes layers for the $\paxos$ acceptor logic and the WormServer code. 
The client stack includes layers for the $\paxos$ proposer logic, a wormclient layer that issues individual $\paxos$ proposals, 
the $\WOR$ abstraction, the WOS abstraction, and the $\wormspace$ API.
Applications such as WormPaxos and WormLog are built on top of the client library.}
\begin{figure}
\centering
\includegraphics[width=0.7\textwidth]{figs/multipaxos/layer_diagram.pdf}
\caption{Layer Diagram: client and server stacks are combined as a distributed system in the ghost layer and the distributed nature is invisible from the $\WOR$ layer.}
\label{fig:chapter:multipaxos:layerdiagram}
\end{figure}
Both stacks share a common set of base layers: 
the bottom layer provides an interface to the trusted computing base (TCB), 
including network communication functions and a small number of system calls. Above this bottom layer, 
we introduce a data layer which implements various data structures over the trusted primitives. 
Above the data layer, the client and server stacks diverge. 
The server stack includes $\paxos$ acceptor layers and the wormserver code above it.
The client stack includes layers for $\paxos$ proposer logic and a wormclient layer that issues individual $\paxos$ proposals.

The ghost layer horizontally composes the two stacks and proves properties across multiple wormservers and clients.
The ghost layer includes a global state transition system that can reason all concurrent client and server interactions based on a network model. 
Safety properties of $\paxos$ are proved in this layer.
The contextual refinement proof between the ghost layer and the composition of wormserver and wormclient provides a powerful guarantee 
for  the layers built on top of the ghost layer. 
Any layer that the ghost layer contextually refines is guaranteed to be correct with respect to both client and server layers. 
It is guaranteed that any concurrent behaviors of distributed nodes using the client and server layers are correct. 
Verified distributed protocol properties hold in higher layers while complex proofs are encapsulated in the ghost layer.

Providing functional correctness for the high-level API of $\wormspace$ as well as its applications
above the ghost layer is as easy as verifying a sequential program.
For example, the top-level specification for a write in $\wormspace$ is simply translating the global address to a segment address and offset and 
passing the captureID (cid) to call the lower-level write which is already proved safe under concurrent distributed accesses:
\begin{lstlisting}
Function WormSpace_write (addr: Z) (val: Payload) (node_id: Z) (adt: EnvVars) : option (EnvVars * Z) :=
    let segment:= addr / WOS_SIZE in 
    let offset := addr mod WOS_SIZE in
    WOR_write segment offset val cid adt.
\end{lstlisting}
We verify the $\WOR$ abstraction, the WOS abstraction, and the WormSpace API.
The client stack can be extended to applications such as WormPaxos, WormLog, and WormTX. 
%In the rest of this section we detail our verification effort.

\subsection{Network Model}
\label{subsec:network_modeling}

It is possible that treating distributed systems as concurrent programs with one shared resource, network, which is
a logically linearized sequence of network operations (\ie, we call it the global network log);
thus we employ $/ccalname$ (in Section~\ref{chapter:ccal:subsec:concurrent-layer-with-environment})
to model a real-world network and to prove distributed properties about the system
by providing non-block network read and write primitives in our TCB. 
Among them, two basic primitives, $\codeinmath{send\_msg}$ and $\codeinmath{recv\_msg}$, 
manipulate the modeled network state. 
Using those two primitives enables each distributed node  extracts its local interaction with the network from the log, 
and then the log is used to reason about the interaction between nodes.

In our model, we depart from single-node concurrency verification by modeling the network as unreliable (but non-Byzantine). 
In our model, $\codeinmath{send\_msg}$ simply creates a $\codeinmath{send}$ event in the log, 
while $\codeinmath{recv\_msg}$ creates either $\codeinmath{timeout}$ (this models dropped packet) or $\codeinmath{recv}$ 
events in an arbitrary future location (this models packet delays) than the$\codeinmath{send}$ event in the log. 
In between a pair of $\codeinmath{send}$ and $\codeinmath{timeout/recv}$, 
any other nodes can freely record their operations (this models packet reordering). 
A \textsc{recv} after a \textsc{send} does not necessarily mean that the $\codeinmath{recv}$ event received the value sent by this $\codeinmath{send}$. 
The actual value can be a duplicate message from a previous send (this models duplicate packets).

However, network communication patterns can be complicated when a client interacts with multiple wormservers in a one-to-many request pattern.
$\wormspace$ code is designed to tolerate an unreliable (but non-Byzantine) network. 
For example, the code always receives packets in a loop to filter out duplicate packets and waits for timeout for lost packets. 
Such communication patterns become more complicated when a client interacts with multiple wormservers in a one-to-many request pattern.
Preserving such complexity is necessary to model the real network and to verify what the C code precisely does,
but when it comes to proving the global property of the system, some of the details can be abstracted out.
Therefore, we facilitate the  \textit{log-lift} pattern in $\ccalname$~(Section~\ref{chapter:ccal:subsec:local-layer-interface}) 
to simplify network patterns in our software stack. 
Accordingly, we create a network log layer with simpler semantics, and prove that the original log refines the simplified log, 
which coalesces broadcasts and receptions into singleton events and eliminates duplicates simplifying global property proofs.
\begin{figure}
\begin{center}
\includegraphics[width=0.98\textwidth]{figs/multipaxos/wormspace_network_refine}
\end{center}
\caption{Network Refinement in $\wormspace$}
\label{fig:chapter:multipaxos:network-refinement}
\end{figure}
Figure~\ref{fig:chapter:multipaxos:network-refinement} illustrates the reduction process.
Acceptors in our implementation is a simple event-driven code, which receives a single packet ($\codeinmath{ARECV}$) and 
sends the proper message ($\codeinmath{ASEND}$, followed by the $paxos$ algorithm~\ref{fig:chapter:multipaxos:paxos-pseudocode}, 
based on the received packet.
In this sense, 
network reduction is not necessary for acceptors' parts. 
Proposers, on the other hand, require multiple network reductions as described in the figure. 
Context programs on the bottom layer in our verification stack do not have any restrictions in using send and receive primitives.
Building layers for proposers,  however, 
restricts the pattern of the network update method 
via introducing broadcast send (from (a) to (b) in the figure), 
filtering unnecessary received messages due to the asynchronous  network model (from (b) to (c) in the figure),
and mark footprints packets to indicate the start and the end of each round of $\paxos$ protocol (from (c) to (d) in the figure).
Each reduction refines multiple network packets into one single packet with preserving the proper interleaving with other nodes in the system. 
For example, 
introducing broadcast send reduces multiple send events ($\codeinmath{PSEND}$) into one single broadcasting send event ($\codeinmath{PBSEND}$),
and filtering out unnecessary receive packets reduces multiple receive events ($\codeinmath{PRECV}$ and/or $\codeinmath{PTIME\_OUT}$) 
to the one single receive event ($\codeinmath{PBRECV}$ and/or $\codeinmath{PBTIME\_OUT}$), which triggers the local state change of the proposer,
by removing all unnecessary receive events. 
These reductions dramatically simplify network patterns in our model as well as the specifications for clients. 


\subsection{Proving Global Properties}
\label{subsec:safety_verification}

\begin{figure}
\lstinputlisting[language=C]{source_code/multipaxos/abswrite.v}
\caption{A Simplified Log Construction Function Example: it logs local and network events of a node to the network log and calls the log replay function to check state changes.}
\label{fig:chapter:multipaxos:spec}
\end{figure}


Network simplifications also give considerable benefits regarding the global property proof of our $\wormspace$ (Theorem~\ref{theorem:chapter:multipaxos:immutability}) by reducing the complexity of 
shared state (network) changes via proposers.
The basic idea to provide the global property theorem is defining the way to construct the whole states of $\wormspace$ (the global state transition system) using the shared resource, the network, and prove the invariant, the safety theorem of our $\wormspace$, with the network as well as the construction function.  
In this sense, this global transition system has a similarity with log replay functions to calculate the shared state status of 
multicore/multithreaded environments discussed in Chapter~\ref{chapter:ccal} and Chapter~\ref{chapter:mcs-lock}.

The global state transition system that we define in the ghost layer (the ghost layer in Figure~\ref{fig:chapter:multipaxos:layerdiagram}) 
models a distributed system with multiple concurrent $\paxos$ clients and acceptors from the viewpoint of the global network to enable the distributed protocol verification. 
It includes (network) log construction functions, a (network) log replay function, and a global state.
The log construction function models how each client/server operation affects the network; 
it governs the communication pattern of each node in the network log to define the $\paxos$ protocol. 
The network log replay function constructs the global state, 
which is a snapshot of the entire distributed system state or a combination of $\paxos$-related states in all nodes,
 by interpreting network events in the network log. Log construction and replay functions are derived from wormclient and wormserver specifications and their refinement relations for the derivation are verified. 
Log construction functions interact with the network log and the global state to introduce new network events in the network log. 
To record local state changes of a node which do not involve network operations, ghost messages are written 
to the network log. 
Log construction functions use the log replay function to learn and use state changes incurred by other concurrent nodes and itself (Figure~\ref{fig:chapter:multipaxos:spec}).


The log replay function by itself can replay all behaviors and state changes of a distributed system step by step from the global network log. Based on this capability we prove the $\paxos$-based safety/immutability property of $\wormspace$:
\begin{theorem}[Immutability]
\label{theorem:chapter:multipaxos:immutability}
 Once a value is written to a $\WOR$, the value in the $\WOR$ never changes.
 \end{theorem}
To prove Theorem~\ref{theorem:chapter:multipaxos:immutability}, we prove the following key lemma: 
\begin{lemma}
\label{lemma:chapter:multipaxos:first-lemma}
 Given a valid network log $\ell$, if there exists a $\paxos$ round $n$ where a value $v$ is successfully written to a $\WOR$ $r$, any following write to $r$ in $\paxos$ rounds $n' > n$ in the log $\ell$
can only attempt to write $v'$ which satisfies $v' = v$.
\end{lemma}
The valid network log is the log that preserves verified invariants such as communication patterns derived from log construction functions.
Lemma~\ref{lemma:chapter:multipaxos:first-lemma} is proved by induction on writes in the log using other supporting lemmas: \eg,, $n'$ is unique and is monotonically increasing, the $\paxos$-phase-1a/capture at round $n'$ on $r$ returns the written value $v$, \etc

This safety property also can be propagated to  applications built on top of our $\wormspace$.
the verification of WormPaxos, WormLog, and WormTX does not have to reason about complex 
$\paxos$ safety proofs to 
guarantee that each $\WOR$ in the system satisfies 
the immutability property.
our safety proof is freely guaranteed in their proofs 
Because the ghost layer encapsulates the distributed nature of $\wormspace$.
In detail, the verification of any additional distributed protocols above $\wormspace$ reuses 
the same network model, and our contextual refinement guarantees the 
immutability with the same network log, the global transition system, and invariant about the network log.
Theorem~\ref{theorem:chapter:multipaxos:immutability-contextual} shows it.
\begin{theorem}[Immutability with Contextual Refinement]
\label{theorem:chapter:multipaxos:immutability-contextual}
Let's assume that $\codeinmath{nid}$ is a node identifier in the distributed system (either one of acceptors or proposers),
$L_{\codeinmath{Ghost}}$ is a ghost layer that all $\wormspace$ related transitions are illustrated with global transition systems, 
$\oracle_{\codeinmath{wp}}$ is an environmental context for the network communication in $\wormspace$.
Then for any programs $\codeinmath{Ctxt}$ runs with the ghost layer (notated as
$\sem{L_{\codeinmath{Ghost}}[\codeinmath{nid}, \oracle_{\codeinmath{wp}}}{Ctxt}$),
once a value is written to a $\WOR$, the value in the $\WOR$ never changes.
 \end{theorem}

\subsection{Top-Level Theorem, Reusability, Linking, and Limitations}
\label{chapter:multipaxos:top-level-theorem-reusability-linking-limitations}

The top-level theorem that we prove for $\wormspace$ is, 
\begin{theorem}[Contextual Refinement of $\wormspace$]
\label{theorem:chapter:multipaxos:wormspace}
Let's assume that $\codeinmath{nid}$ is a node identifier in the distributed system (either one of acceptors or proposers),
$\oracle_{wp}$ is a low-level network environmental context for the network (with no reductions at all) and 
$\oracle_{wp}'$ is a high-level network environmental context for the network (with reductions for broadcasting send calls and the
receive filtering),
 Then the $\wormspace$ implementation $\codeinmath{M}_{\codeinmath{wp}}$ which is built on top of the lowest layer $L_{\codeinmath{TCB}}$ satisfies 
 the following contextual refinement property
$\ltyp{\PLayer{L_{\codeinmath{TCB}}}{nid}{\oracle_{\codeinmath{wp}}}}{R}{\codeinmath{M}_{\codeinmath{wp}}}{\PLayer{L_{\codeinmath{WORMSPACE}}}{\codeinmath{nid}}{\oracle_{wp}'}}$
\end{theorem}
The contextual refinement proof between all adjacent layers is used as lemmas to guarantee the correctness of the entire code. 
As mentioned, 
Theorem~\ref{theorem:chapter:multipaxos:wormspace} also guarantees that the verified $\paxos$ properties in the ghost layer 
hold for the $\wormspace$ implementation.
The top-level theorems that we prove for WormPaxos, WormLog, and WormTX are in the same format.
For example, the functional correctness theorem for WormPaxos is as follows:
\begin{theorem}
\label{theorem:chapter:multipaxos:wompaxos}
Let's assume that $\codeinmath{nid}$ is a node identifier in the distributed system (either one of acceptors or proposers),
$\oracle_{wp}$ is a low-level network environmental context for the network (with no reductions at all) and 
$\oracle_{wp}'$ is a high-level network environmental context for the network (with reductions for broadcasting send calls and the
receive filtering),
 Then the $\wormspace$ implementation $\codeinmath{M}_{\codeinmath{wp}}$, which is built on top of the lowest layer $L_{\codeinmath{TCB}}$,  and 
 the WormPaxs implementation $\codeinmath{M}_{\codeinmath{mp}}$, which is built by using the API of our $\wormspace$ provided by the $L_{\codeinmath{WORMSPACE}}$ layer, satisfies the following property:
 $\ltyp{\PLayer{L_{\codeinmath{TCB}}}{\codeinmath{nid}}{\oracle_{\codeinmath{wp}}}}{R}{\modulecompose{\codeinmath{M}_{\codeinmath{wp}}}{\codeinmath{M}_{\codeinmath{wp}}}}{\PLayer{L_{\codeinmath{WORMPAXOS}}}{\codeinmath{nid}}{\oracle_{wp}'}}$.
\end{theorem}
By reusing Theorem~\ref{theorem:chapter:multipaxos:wormspace} and Theorem~\ref{theorem:chapter:multipaxos:immutability-contextual},
applications are guaranteed to be correct with respect to all layers of $\wormspace$ and
 to encapsulate verified $\paxos$ safety properties (in Theorem~\ref{theorem:chapter:multipaxos:immutability}.)
 Similarly, Theorem~\ref{theorem:chapter:multipaxos:wormspace} can be reused to verify any systems to guarantee $\WOR$ semantics, 
if we use $\wormspace$ as a building block. 
	
%\jieung{Do we need to mention about linking with CertiKOS in here?}	
	
The extensibility of $\wormspace$ verification to applications is difficult for other verified systems~\cite{ironfleet, hyperkernel} to achieve. 
Especially, it is unnatural and difficult to support contextual refinement, 
which is based on high-order logic, when the verification tool is based on an SMT solver or first-order logic (e.g., Dafny~\cite{dafny} 
and Z3~\cite{moura08}). 
However, the proofs also have limitations. 
Even with the simplified network log in our $\wormspace$, additional safety proofs for applications (such as linearizability in WormPaxos) 
requires other treatments. 
It, however, depends on multiple $\WOR$s as well as the communication between the clients that use the $\wormspace$ interface. 
We illustrate our new and robust methodology for this purpose in the next Chapter.

%\jieung{adding the full layer diagram?}

%\jieung{Don't we need to add evaluations. My thesis is not focusing on evaluations or performances at all}
