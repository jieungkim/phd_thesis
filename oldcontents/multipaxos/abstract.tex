Distributed systems are notoriously complex due to the many possible interleavings of their coarsely-connected instances as well as the possibility of errors in both  those instances and the network environment. For these reasons, verification of distributed systems is desirable to remove the possibility of bugs and guarantee their safety and correctness. However, much current verification work still requires a great deal of effort and sometimes has limitations.

We present a verification approach that uses \textit{write-witness-passing}, which is simple but novel in distributed system verification. It is a scalable, reusable, and extensible approach that can be directly linked with the low-level implementations of distributed protocols through contextual refinement. Write-witness-passing can capture the common behaviors of many distributed protocols, and provides both a simple way of understanding the protocols as well as an easy methodology for verifying them.

To demonstrate how write-witnesses work, we verify the functional correctness and safety of Paxos, one of the most famous consensus protocols. We implement the key routines of Paxos in C, and use Coq to verify both the functional correctness of the implementation as well as the safety properties of the protocol within less than 4 person-months. We also describe how we can apply our approach to other distributed protocols to illustrate its generality.
