\section{Evaluation}
\label{sec:evaluation}

\para{Proof Effort}

Our verification of Paxos can be divided into two main parts: functional correctness and safety.
All of the proofs of functional correctness took approximately 1.5 person-months, in large part thanks to the power of the CCAL framework.
The safety proofs using write-witness-passing were completed in 2 person-months with a small amount of trial-and-error.

\para{Comparison with non-witness-passing proof}

Our initial attempt to prove the safety of Paxos did not use witnesses and instead went by induction on the round number or the network log.
We quickly found that this strategy created large, intractable proofs due to the difficulty of trying to reconstruct global state
from only partial local information.
Adding in write-witness-passing required only minor changes to our specifications, but made many of the proofs significantly easier.
Because we never completed the proofs without write-witness-passing, we cannot directly compare how long each method took;
however, before using witnesses we struggled to make any significant progress, and afterwards progress was steady.

\para{Hand-written Proof vs. Mechanized Proof}

The overall structure of our Paxos safety proof is similar to some hand-written proofs.
However, these proofs often proceed by induction on the round numbers or network transitions,
which, as stated earlier, we found to be difficult in a mechanized proof.
The nature of machine-checked proofs means that every corner-case and detail must be accounted for,
and where a hand-written proof might gloss over some seemingly trivial fact, the full proof can turn out
to be surprisingly involved.
The reward for this extra work is a more rigorous proof that can be linked to the actual implementation,
thus providing an even stronger assurance of correctness.
