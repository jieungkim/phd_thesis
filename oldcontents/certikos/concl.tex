\section{Conclusion}
\label{sec:concl}

We have presented a novel extensible architecture for building
certified concurrent OS kernels that have not only an efficient
assembly implementation but also machine-checkable contextual
correctness proofs.  OS kernels developed using our layered
methodology also come with a clean, rigorous, and layered
specification of all kernel components.  We show that building
certified concurrent kernels is not only feasible but also
quite practical. 
%can also be done quite economically.
\ignore{
Traditional OS kernels use a hardware-enforced ``red line'' to isolate
the behaviors of user programs and to protect the integrity of the
kernel code.
}
Our layered approach to certified concurrent kernels replaces the
hardware-enforced ``red line'' with a large number of abstraction
layers enforced via formal specification and proofs. We believe this
will open up a whole new dimension of research efforts toward building
truly reliable, secure, and extensible system software.  

\ignore{
While we initially pursued layered decomposition to reduce the cost of
verification, we later found that such decomposition is also critical
for managing the complexity of specification.  Decomposing a kernel
into many smaller, well-specified components also makes it possible to
aggressively apply proof automation.  Indeed, the majority of our
development effort was spent on layer definitions; once all the layers
are in place, the verification of source-level code and the layer
refinement proofs were done quickly by using shared Coq tactic
libraries.
}

\ignore{
}

%\vspace*{-5pt}
\section*{Acknowledgments}
We would like to acknowledge the contribution of many former and
current team members on various CertiKOS-related projects at Yale,
especially Jérémie Koenig, Tahina Ramananandro, Shu-Chun Weng, Liang
Gu, Mengqi Liu, Quentin Carbonneaux, Jan Hoffmann, Hernán Vanzetto,
Bryan Ford, Haozhong Zhang, Yu Guo, and Joshua Lockerman. We also want
to thank our shepherd Gernot Heiser and anonymous referees for helpful
feedbacks that improved this paper significantly.  This research is
based on work supported in part by NSF grants 1065451, 1521523, and
1319671 and DARPA grants FA8750-12-2-0293, FA8750-16-2-0274, and
FA8750-15-C-0082.  Hao Chen's work is also supported in part by China
Scholarship Council. The U.S. Government is authorized to reproduce
and distribute reprints for Governmental purposes notwithstanding any
copyright notation thereon. The views and conclusions contained herein
are those of the authors and should not be interpreted as necessarily
representing the official policies or endorsements, either expressed
or implied, of DARPA or the U.S. Government.

