\iflongabs
Operating System (OS) kernels form the backbone of all system
software. They can have a significant impact on the resilience
and security of today's computers. Recent efforts
have demonstrated the feasibility of building formal proofs of
functional correctness for simple general-purpose kernels, but they
have ignored the important issues of concurrency, which include not
just user- and I/O concurrency on a single core, but also
multicore parallelism with fine-grained locking.  Many 
researchers believe that building a certified concurrent kernel is
very challenging, and even if it is possible, its cost would far
exceed that of verifying a single-core sequential kernel.

In this paper, we 
\else
Complete formal verification of a non-trivial concurrent OS kernel is
widely considered a grand challenge.  We 
\fi
present a novel compositional approach for building certified
concurrent OS kernels. Concurrency allows interleaved execution of
kernel/user modules across different layers of abstraction. Each such
layer can have a different set of observable events. We insist on
formally specifying these layers and their observable events, and then
verifying each kernel module at its proper abstraction level. To
support certified linking with other CPUs or threads, we prove a
strong contextual refinement property for every kernel function, which
states that the implementation of each such function will behave like
its specification under any kernel/user context with any valid
interleaving. We have successfully developed a practical concurrent OS
kernel and verified its (contextual) functional correctness in Coq.
Our certified kernel is written in 6500 lines of C and x86 assembly
and runs on stock x86 multicore machines. To our knowledge, this is
the first proof of functional correctness of a complete,
general-purpose concurrent OS kernel with fine-grained locking.


